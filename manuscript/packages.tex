%%%% Font et typos %%%%%%%%%%%
\usepackage[utf8]{inputenc}
\usepackage[T1]{fontenc}

\usepackage[square,sort&compress,sectionbib]{natbib}	
%\bibpunct{\textcolor{black}{[}}{\textcolor{black}{]}}{\textcolor{black}{,}}{a}{}{\textcolor{black}{;}}
\usepackage{chapterbib}
		\renewcommand{\bibsection}{\chapter*{Références}}	

\usepackage[french]{babel}
\usepackage{amsmath,amsfonts,amssymb}
\usepackage{lmodern}
\usepackage{ae,aecompl}	 % Utilisation des fontes vectorielles modernes
\usepackage[upright]{fourier}

\usepackage{textgreek}
\usepackage{underscore}
\usepackage{blindtext}
%%%% Allure g�n�rale du document %%%%%%%%%%%
\usepackage{enumerate}
\usepackage{enumitem}
\usepackage[section]{placeins}	% Place un FloatBarrier � chaque nouvelle section
\usepackage{epigraph}
\usepackage[font=footnotesize]{caption}
\usepackage[francais,nohints]{minitoc}		% Mini table des mati�res
	\setcounter{minitocdepth}{2}	% Mini-toc d�taill�s (section/sous-section)
\usepackage[notbib]{tocbibind}		% Ajoute les Tables des Mati�res/Figures/Tableaux � la table des mati�res
\usepackage{setspace}
\onehalfspacing
\usepackage[raggedright]{titlesec}

\addtolength{\skip\footins}{2pc plus 5pt}

\newcommand\blfootnote[1]{%
  \begingroup
  \renewcommand\thefootnote{}\footnote{#1}%
  \addtocounter{footnote}{-1}%
  \endgroup}
  
%%%% Tableaux %%%%%%%%%%%
\usepackage{multirow}
\usepackage{booktabs}
\usepackage{colortbl}
\usepackage{tabularx}
\usepackage{multirow}
\usepackage{threeparttable}
\usepackage[table]{xcolor}
\definecolor{Gray}{gray}{0.9}
{\rowcolors{3}{white}{Gray}
\usepackage{stackengine}
\newcommand\xrowht[2][0]{\addstackgap[.5\dimexpr#2\relax]{\vphantom{#1}}}
\usepackage{array}
\newcolumntype{M}[1]{>{\centering\arraybackslash}m{#1}}
\usepackage{etoolbox}
	\appto\TPTnoteSettings{\footnotesize}
	\AtBeginEnvironment{tabular}{\footnotesize}
\addto\captionsfrench{\def\tablename{{\textsc{Tableau}}}}	% Renomme 'table' en 'tableau'

%%%% El�ments graphiques %%%%%%%%%%%                 
\usepackage{graphicx}			% Permet l'inclusion d'images
\usepackage{subcaption}
\usepackage{floatrow}
\usepackage{pdfpages}
\usepackage{rotating}
\usepackage{pgfplots}
	\usepgfplotslibrary{groupplots}
\usepackage{tikz}
	\usetikzlibrary{backgrounds,automata}
	\pgfplotsset{width=7cm,compat=1.3}
	\tikzset{every picture/.style={execute at begin picture={
   		\shorthandoff{:;!?};}
	}}
	\pgfplotsset{every linear axis/.append style={
		/pgf/number format/.cd,
		use comma,
		1000 sep={\,},
	}}
\usepackage{eso-pic}
\usepackage{import}
\usepackage{chngcntr}
\counterwithout{figure}{chapter}
\counterwithout{table}{chapter}

\usepackage[normalem]{ulem}

%%%% Navigation dans le document %%%%%%%%%%%      
\usepackage[pdftex,pdfborder={0 0 0},
			colorlinks=true,
			linkcolor=blue,
			citecolor=black,
			pagebackref=true,
			]{hyperref}	% Cr��ra automatiquement les liens internes au PDF

% \Autoref is for the beginning of the sentence
\let\orgautoref\autoref
\providecommand{\Autoref}{%
\def\figureautorefname{Figure}%
\def\tableautorefname{Tableau}%
\def\subfigureautorefname{Figure}%
\orgautoref}
% \autoref is used inside the sentence to produce Fig., and Eq. for figures, subfigures, and equations
\renewcommand{\autoref}{%
\def\figureautorefname{Fig.}%
\def\tableautorefname{Tableau}%
\def\subfigureautorefname{Fig.}%
\orgautoref}

\usepackage{nameref}

\newcommand{\fullref}[1]{\autoref{#1}. \nameref{#1}}

\usepackage{tocloft}
\newlength{\mylen}
\renewcommand*\cftfigpresnum{Figure~}
\settowidth{\mylen}{\cftfigpresnum\cftfigaftersnum}
\addtolength{\cftfignumwidth}{\mylen}

\renewcommand*\cfttabpresnum{Tableau~}
\settowidth{\mylen}{\cfttabpresnum\cfttabaftersnum}
\addtolength{\cfttabnumwidth}{\mylen}

%% PACKAGES CHARGES EN DERNIER  %%%%%%%%%%%      
	             
\usepackage[top=2.5cm, bottom=2cm, left=2.5cm, right=2.5cm,
			headheight=15pt]{geometry}

\usepackage{fancyhdr}			% Ent�te et pieds de page (apr�s geometry)
	\pagestyle{fancy}		% Indique que le style de la page sera justement fancy
	\lfoot[\thepage]{} 		% gauche du pied de page
	\cfoot{} 			% milieu du pied de page
	\rfoot[]{\thepage} 		% droite du pied de page
	\fancyhead[RO, LE] {}	
	
\usepackage[acronym, nonumberlist, nomain, nogroupskip,nopostdot]{glossaries}
\usepackage{glossary-mcols}
\renewcommand{\glossarypreamble}{\small}
\renewcommand{\glsnamefont}[1]{\textnormal{#1}}

	\makeglossaries
	\loadglsentries{glossary.tex}			% Utilisation d'un fichier externe pour la définition des entr�es (glossary.tex)		
