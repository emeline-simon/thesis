%% Copyright (C) 2017-2021 Emeline Simon
%%
%% The current owner of this work is Emeline Simon
%% <contact at emeline.simon@gmail.com>.
%%
%% This is 04_material.tex the fourth chapter for my PhD Thesis.
%%
%%%%%%%%%%%%%%%%%%%%%%%%%%%%%%%%%%%%%%%%%%

\chapter{Matériels}
	\minitoc
	\newpage

%%%%%%%%%%%%%%%%%%%%%%%%%%%%%%%%%%%%%%%%%%%%%%%%%%%%%%%%%%%%%%%%%%%%%%%%%%%%%%%%%%%%%%%%%%%%%%%%%%%%%%%%%%%%%%%%%%%%%%%%%%%%%%%%%%%%%%%%%%%%%%%%%%%%%%%%%%%%%%%%%%%%%%%%

% 1 - ETHIQUE
			
	\section{Déclaration d'éthique}

Cette étude s'est conformée aux directives éthiques de la déclaration d'Helsinki de 1975 et a été approuvée par l'un des comités et instituts suivants selon les souches virales cliniques concernées : le Comité consultatif national d'éthique pour les sciences de la vie et de la santé en France (déclaration numéro DC-2008-531, C. Gondeau) ; le Comité national d'éthique du Cameroun (numéro 199/CNE/SE/2011, R. Njouom) et le Ministère de la Santé du Cameroun (numéro 631- 01.12) ; le \textit{Bioethics Committee of the Cantacuzino National Medical-Military Institute of Research and Development} de Roumanie (numéro 16/CEE, G. Oprisan). Un consentement éclairé écrit a été obtenu de tous les patients ou de leurs familles. Aucune information sur les patients n'est disponible au laboratoire, hormis leur sexe, leur âge et la raison de la résection chirurgicale.


% 2 - ISOLATS CLINIQUES
			
	\section{Isolats cliniques et souches prototypiques du VHC}

Les ARN viraux extraits du sérum de patients souffrant d'hépatite C chronique et atteints à des degrés variables de micro et macro-stéatose ont été fournis par le Dr. C. Gondeau (Institut de Recherche en Biothérapie, Montpellier, France) \citep{RN812} à l’exception d’une souche 3a (GenBank \
\#LN897692.1) obtenue du Dr. Penelope Mavromara (\textit{Hellenic Pasteur Institute, Athens, Greece}). Les degrés de stéatose ont été évalués par microscopie optique sur coupes colorées par estimation semi-quantitative d'un pourcentage d'hépatocytes contenant des vacuoles graisseuses par rapport à l'ensemble des hépatocytes. La séquence de la protéine Core mature de l'isolat S311 de sous-type 3a s'est avérée identique à celle de la souche grecque que le laboratoire avait précédemment utilisée pour produire un virus intergénotypique Jad/C3aG. Le résidu en position 178 (Leu), au sein du peptide signal C-terminal de la protéine Core immature exprimée par Jad/C3aG est une Phe dans l'isolat 311, différence que nous avons considéré négligeable puisque le peptide signal est clivé dans la protéine Core mature fonctionnelle. Ce virus, nouvellement désigné Jad/C3a-311, a été utilisé dans ce projet pour refléter les caractéristiques de Core de l'isolat S311. La séquence codant Core de l’isolat de sous-type 4a (désignée 4aR, ENA \#ERZ655054) a été isolée d’un donneur de sang séro-positif pour le VHC et fournie par le Dr. Gabriela Oprisan (\textit{Cantacuzino National Medical-Military Institute of Research and Development, Titu Maiorescu University,} Bucharest, \textit{Romania}). La séquence codant Core de l’isolat de sous-type 4f (désignée 4fC, ENA \#ERZ672786) a été isolée d’un patient souffrant d’un CHC et fournie par le Dr. Richard Njouom (Centre Pasteur du Cameroun, Yaoundé, \textit{Cameroon}). Les séquences 4aR et 4fC ont été déterminées et publiées dans une étude précédente de notre laboratoire \citep{RN808}. La séquence codant Core de la souche prototypique H77 de sous-type 1a est issue de l’ADNc de longueur génomique (1aH77, GenBank \#AF011751) \citep{RN310}, généreusement fourni par le Dr. Robert Purcell (\textit{National Institute of Health}, Bethesda, \textit{United States}). Les séquences codant Core des souches prototypiques JFH-1 (GenBank \#AB047639) et J6 (GenBank \#D00944) de sous-type 2a proviennent des ADNc de longueur génomique, respectivement pJFH1 et pJc1-2EI3, généreusement fournis par les Drs. T. Wakita (\textit{National Institute of Infectious Diseases,} Tokyo, \textit{Japan}) \citep{RN124} et R. Bartenschlager (\textit{University of Heidelberg,} Heidelberg, \textit{Germany}) \citep{RN813}.


% 3 - PLASMIDES

	\section{Plasmides}
			
		\subsubsection{Les plasmides pJFH1 et pJad codant les ADNc viraux}

Le plasmide pJFH1 contient l’ADNc de la souche JFH-1 de sous-type 2a \citep{RN124} cloné en aval du promoteur de l’ARN polymérase du phage T7. L’ADNc d'un dérivé du virus JFH-1 hautement adapté à la culture cellulaire Huh-7.5 a été produit par le laboratoire du Dr. Annette Martin (Groupe Hepacivirus, Institut Pasteur, Paris) à partir des plasmides pJFH1 et pJFH1-2EI3 adapt (don du Dr. Ralf Bartenshlager) et nommé pJad \citep{RN811} (\autoref{fig:figM1}). Il contient deux substitutions nucléotidiques dans la séquence codant NS5A (codant les conversions d'acides aminés V2153A et V2240L) et une substitution dans la région codant NS5B (conversion V2941M) qui confèrent à la progénie virale un titre infectieux atteignant ~3 x 10\up{5} TCID50/mL \citep{RN342}.

	\begin{figureth}
	\centering
			\includegraphics[width=0.75\linewidth]{Figure_78.png}
		\caption[Carte du plasmide pJad.]{\textbf{Carte du plasmide pJad.} L'ADNc de longueur génomique de la souche Jad est inséré en aval du promoteur de l'ARN polymérase du phage T7, générant le plasmide pJad de 12 360 paires de bases (pb). Les mutations adaptatrices (Mut. adapt.) sont indiquées, ainsi que le site de restriction unique (\textit{Xba}I) utilisé à des fins de linéarisation de l'ADN avant la transcription \textit{in vitro}. Les sites de restriction (\textit{Age}I et \textit{Bsi}WI) utilisés pour la construction des ADNc recombinants intergénotypiques sont également indiqués. La carte a été conçue par S. Aicher à l’aide du logiciel CLC Main Workbench.}
				\label{fig:figM1}
	\end{figureth}
		\FloatBarrier

		\subsubsection{Les plasmides rapporteurs pJFH1-2EIL3, pJad-2EIL3 et dérivés}

Les plasmides pJad-2EIL3, pJFH1-2EIL3/GAA, pJad-2EIL3ΔEp7 étaient disponibles dans le laboratoire du Dr. Annette Martin (Groupe Hepacivirus, Institut Pasteur, Paris). Le plasmide pJad-2EIL3 est dérivé du pJad et contient un ADNc bicistronique du Jad, pour lequel les séquences codant C-NS2 sont insérées en aval de la région 5'NC du VHC et les séquences NS3-NS5B sont placées en aval d'une insertion de l’IRES de l'EMCV \citep{RN810}. De plus, l'IRES de l'EMCV est immédiatement suivi de la séquence rapportrice de la FLuc, de la séquence codant le peptide 2A du FMDV et de la séquence codant pour un monomère d’ubiquitine qui permet la libération de NS3 (\autoref{fig:figM2}). Le plasmide pJFH1-2EIL3/GAA code la substitution des codons du site actif de l’ARN polymérase NS5B (Gly-Asp-Asp) en codons non fonctionnels (Gly-Ala-Ala), et le plasmide pJad-2EIL3ΔEp7 contient une délétion des séquences codant les glycoprotéines d’enveloppe E1 et E2 et la viroporine p7.

	\begin{figureth}
	\centering
			\includegraphics[width=0.85\linewidth]{Figure_79.png}
		\caption[Carte du plasmide pJad-2EIL3.]{\textbf{Carte du plasmide pJad-2EIL3.} L'ADNc du génome du VHC est inséré en aval du promoteur de l'ARN polymérase T7. Les séquences codantes de FLuc, du peptide 2A du FMDV et du monomère d’ubiquitine sont insérées entre les séquences codantes de NS2 et NS3, générant un vecteur pJad-2EIL3 de 15 443 pb. Les sites de restriction (\textit{Mlu}I, \textit{Age}I et \textit{Not}I) utilisés pour la linéarisation de l’ADN avant la transcription \textit{in vitro} et le clonage sont indiqués. La carte a été conçue par S. Aicher à l’aide du logiciel CLC Main Workbench.}
				\label{fig:figM2}
	\end{figureth}
		\FloatBarrier


% 4 - Culture cellulaire

\section{Culture cellulaire}	

La lignée cellulaire d’hépatome humain Huh-7.5 (aimablement fournie par le Dr. Charles Rice, \textit{The Rockefeller University,} New York, \textit{United States}), hautement permissive à la réplication du VHC \citep{RN322}, a été utilisée comme modèle expérimental principal. Les cellules Huh-7.5 ont été cultivées en monocouche en milieu complet, \textit{i.e} milieu DMEM (\textit{Dulbecco’s Modified Eagle Medium}, Gibco, \#11574486) supplémenté avec 10\% de sérum de veau foetal inactivé (FBS, Gibco, \#10270-106), 1mM de pyruvate de sodium (Gibco, \#11360039), 100 U/mL de pénicilline, 100µg/mL de streptomycine (Gibco, \#11548876) et des acides aminés non essentiels (Gibco, \#11140035) et maintenues à 37°C sous une atmosphère de 5\% de CO\textsubscript{2}.

% 5 - ANTICORPS

\section{Anticorps}	

Les anticorps utilisés lors de cette étude sont répertoriés dans Tableaux \ref{tab:tabM1} et \ref{tab:tabM2}.

\begin{flushleft}
\begin{tabular}{|m{5cm}|M{5cm}|M{1.75cm}|M{2cm}|}
\hline
Anticorps &  Société & Référence & Dilution \\
\hline
Anticorps monoclonal anti-actine ß de souris (AC-15)\up{a} & Abcam (Cambridge, \textit{United Kingdom})    &ab6276&   1:5000\\
Anticorps monoclonal anti-ubiquitine de souris (P4D1)\up{a} &   Affymetrix eBioscience (San Diego, \textit{United States})  & 15217707   &1:2000\\
Anticorps polyclonal anti-PSMB5 de lapin\up{b} &Thermo FisherScientific (Waltham, \textit{United States}) & PA1-977&  1:200\\
Anticorps polyclonal anti-PLIN2 de mouton\up{b} &Don du Dr. John McLauchlan (\textit{MRC}, Glasgow, \textit{United Kingdom}) & -&  1:1000\\
 Anticorps monoclonal anti-Core JFH-1 de souris (1851)\up{a} &Santa Cruz Biotechnology (Dallas, \textit{United States}) & SC-58144&  1:1000\\
Anticorps monoclonal anti-Core Con1b de souris (C7-50)\up{a}&Abcam 
(Cambridge, \textit{United Kingdom})  & Ab2740&1:1000\\
Anticorps monoclonal anti-Core JFH-1 de souris (4F5)\up{a}& BioFront Technologies (\textit{Tallahassee, United States})  & HCV-4F5 &1:1000\\
 Anticorps monoclonal anti-Core JFH-1 de souris (3D11)\up{a}& BioFront Technologies (\textit{Tallahassee, United States})  & HCV-3D11&1:1000\\
Anticorps monoclonal anti-Core 1aH77 de souris (ACAP27)\up{a,b}& Don du Dr. Agata Budkowska (Institut Pasteur, Paris, France)
& -& WB : 1:3000   IF : 1:1000\\
 Anticorps polyclonal anti-Core de lapin (FL)\up{a,b}& Don du Dr. Athanasios Kakkanas (\textit{HPI, Athens, Greece})
& -& WB : 1:2000   IF : 1:2000\\
Anticorps monoclonal anti-NS3 JFH-1 de souris (2E3)\up{c} & BioFront Technologies (\textit{Tallahassee, United States})& HCV-2E3& 1:3000\\
Anticorps monoclonal anti-NS5A JFH-1 de souris (2F6)\up{b}& BioFront Technologies (\textit{Tallahassee, United States})& HCV-2F6& 1:2000\\
Anticorps monoclonal anti-NS5B JFH-1 souris (4B8)\up{a}& BioFront Technologies (\textit{Tallahassee, United States})& HCV-4B8 & 1:2000 \\
\hline
\end{tabular}
\end{flushleft}

\begin{tableth}
\caption[Liste des anticorps commerciaux et issus de dons utilisés lors de cette étude.]{\textbf{Liste des anticorps commerciaux et issus de dons utilisés lors de cette étude.} Les méthodes expérimentales qui reposent sur l’utilisation des anticorps sont précisées : révélation de protéines cellulaires ou virales par immuno-blot (a), immunofluorescence (b) ou détection chromogène des foyers infectieux pour déterminer le titre viral (c).}.
			\label{tab:tabM1}
\end{tableth}

\begin{flushleft}
\begin{tabular}{ |m{5cm}|M{5cm}|M{1.75cm}|M{2cm}|  }
\hline
Anticorps &  Société & Référence & Dilution\\
\hline
Anticorps anti-souris IgG (H\&L) de chèvre conjugué Dylight 680\up{a}& Li-Cor Biosciences 
(Lincoln, \textit{United States})& 926-68070& 1:10.000\\
Anticorps anti-souris IgG (H\&L) de chèvre conjugué Dylight 800\up{a}& Thermo Fisher Scientific 
(Waltham, \textit{United States})& SA5-35521& 1:10.000\\
Anticorps anti-lapin IgG (H\&L) de chèvre conjugué Dylight 800\up{a}& Thermo Fisher Scientific 
(Waltham, \textit{United States})& SA5-35571& 1:10.000\\
Anticorps anti-souris IgG (H\&L) de chèvre conjugué Alexa Fluor 488\up{b}& Thermo Fisher Scientific (Waltham, \textit{United States})& A-11029& 1:500\\
Anticorps anti-lapin IgG (H\&L) d’âne conjugué Alexa Fluor 555\up{b}& Thermo Fisher Scientific (Waltham, \textit{United States})& A-31572& 1:500\\
Anticorps anti-lapin IgG (H\&L) d’âne conjugué Alexa Fluor 647\up{b}& Thermo Fisher Scientific (Waltham, \textit{United States})& A-31673& 1:500 \\
Anticorps anti-mouton IgG (H\&L) d'âne conjugué Alexa Fluor 647\up{b}& Thermo Fisher Scientific (Waltham, \textit{United States})& A-21448& 1:500 \\
 \hline
\end{tabular}
\end{flushleft}

\begin{tableth}
\caption[Liste des anticorps commerciaux et issus de dons utilisés lors de cette étude (Suite).]{\textbf{Liste des anticorps commerciaux et issus de dons utilisés lors de cette étude (Suite).} Les méthodes expérimentales qui reposent sur l’utilisation des anticorps sont précisées : révélation de protéines cellulaires ou virales par immuno-blot (a), immunofluorescence (b) ou détection chromogène des foyers infectieux pour déterminer le titre viral (c).}.
			\label{tab:tabM2}
\end{tableth}
\clearpage

% 6 - KITS ET REACTIFS

\section{Enzymes, réactifs et kits commerciaux}	

Les enzymes, réactifs et les kits commerciaux utilisés lors de cette étude sont répertoriés dans les Tableaux \ref{tab:tabM3} et \ref{tab:tabM4}.

\begin{flushleft}
\begin{tabular}{ |m{8.5cm}|M{4cm}|M{2.5cm}|  }
\hline
Nom & Société & Référence \\
\hline
Countess Cell Counting Chamber Slides & Thermo Fisher Scientific & C10228 \\
MassRuler Express Forward DNA Ladder Mix & Fermentas & SM1283 \\
RNA Millenium Markers-Formamide & Thermo Fisher Scientific & 10351375 \\
Pwo Super Yield DNA Polymerase & Roche & 04743750001 \\
One Taq 2X Master Mix Polymerase & New England BioLabs & M0482S \\ 
TOPO TA cloning for sequencing & Thermo Fisher Scientific & 450071 \\
Zero Blunt TOPO PCR cloning for sequencing & Thermo Fisher Scientific & 450159 \\
Mung Bean Nuclease & New England BioLabs & M0250S \\
NucleoBond Xtra Midi Plus & MACHEREY-NAGEL & 740412.10 \\
QIAGEN Plasmid Mini Kit & Qiagen & 12123 \\
HiPure Plasmid Filter Midi Purification Kit & Thermo Fisher Scientific & K210014 \\
QIAquick PCR Purification Kit & Qiagen & 28104 \\
Shrimp Alkaline Phosphatase rSAP & New England BioLabs & M0371S \\
T7 RiboMAX Express Large Scale RNA Production System & Promega & P1320 \\
RQ1 RNase-Free DNase & Promega & M6101 \\
TaqMan Fast-virus 1 Step Master Mix & Applied Biosystems & 13498446 \\
Eukaryotic 18S rRNA Endogenous Control & Applied Biosystems & 10321085 \\
Oligo(dT)18 Primer 100μM & Thermo Fisher Scientific & SO131 \\
Random primers d(pN)6 1mM & ROCHE & 1103473100 \\
Mix dNTPs 100 mM & Eurobio & GAEPCR11-5C \\
SUPERSCRIPT II Reverse Transcriptase & Thermo Fisher Scientific & 1806401 \\
RNasin Ribonuclease inhibitor 20-40 u/μL, 10000U & Promega & N2115 \\
Rapid DNA Ligation Kit & ROCHE & 11635379001 \\
Phosphatase Inhibitor Cocktail Tablets & ROCHE & 04906845001 \\
NUPAGE NOVEX 4-12\% BIS TRIS GELS 1,5 mm 12 Puits & Thermo Fisher Scientific & NP0322BOX \\
NUPAGE NOVEX 4-12\% BIS TRIS GELS 1,5 mm 15 Puits & Thermo Fisher Scientific & NP0323BOX \\
NUPAGE NOVEX 12\% BIS TRIS GELS 1,5 mm 12 Puits & Thermo Fisher Scientific & NP0342BOX \\
NuPAGE LDS Sample Buffer (4X) & Thermo Fisher Scientific & NP0007 \\
NuPAGE MOPS SDS Running Buffer (20X) & Thermo Fisher Scientific & NP0001 \\
NuPAGE MES SDS Running Buffer (20X) & Thermo Fisher Scientific & NP0002 \\
NuPAGE Transfer Buffer (20X) & Thermo Fisher Scientific & NP0006 \\
NuPAGE Antioxidant & Thermo Fisher Scientific & NP0005 \\
Stripping Buffer & Euromedex & ST010 \\
Nitrocellulose Premium, 0.45μm & Amersham & 10600003 \\
SeeBlue Pre-Stained Standard & Thermo Fisher Scientific & LC5625 \\
Revert 700 Total Protein Stain & Li-Cor Biosciences & 926-11011 \\
\hline
\end{tabular}
\end{flushleft}

\begin{tableth}
\caption[Liste des enzymes, kits et réactifs commerciaux utilisés lors de cette étude.]{\textbf{Liste des enzymes, kits et réactifs commerciaux utilisés lors de cette étude.}}.
			\label{tab:tabM3}
\end{tableth}

\begin{flushleft}
\begin{tabular}{ |m{8.5cm}|M{4cm}|M{2.5cm}|  }
\hline
Nom & Société & Référence \\
\hline
Big Dye v.1.1 & Applied Biosystems & 4336799 \\
Big Dye v.3.1 & Applied Biosystems & 4337455 \\
Big Dye Terminator Buffer 5X & Applied Biosystems & 4336097 \\
Reporter Lysis 5X Buffer & Promega & E3971 \\
Luciferase Assay Reagent & Promega & E1500 \\
MG132 Proteasomal Inhibitor & Sigma & C-2211 \\
RNAzol RT & Sigma & R4533 \\
RNeasy Plus Mini & Qiagen & 74134 \\
RNase-free DNase I Set & Qiagen & 79254 \\
QiaAmp Viral RNA Mini & Qiagen & 52904 \\
Dynabeads mRNA Purification Kit & Thermo Fisher Scientific & 61006 \\
HS NGS Fragment Kit & Agilent & DNF-474-0500 \\
µ-Slide VI 0.5 Glass Bottom & Ibidi & 80607 \\
Nunc Optical MicroWell 96-well plates Coverglass Bottom & Thermo Fisher Scientific & 160376 \\
Ibidi Mounting Medium & Ibidi & 50001 \\
Goat Serum & Sigma & G9023 \\
Donkey Serum & Sigma & D9663 \\
ViewRNA Cell Plus Assay & Thermo Fisher Scientific & 88-19000 \\
BODIPY 558/568 C12 & Thermo Fisher Scientific & D3835 \\
LipidTOX Red Neutral Lipid Stain & Thermo Fisher Scientific & H34476\\
DAPI & Thermo Fisher Scientific & D1306 \\
ImmPRESS HRP Anti-Mouse IgG Polymer Detection Kit & Vector Laboratories & MP-7402 \\
DAB Substrate Kit & Vector Laboratories & SK-4100 \\
\hline
\end{tabular}
\end{flushleft}

\begin{tableth}
\caption[Liste des enzymes, kits et réactifs commerciaux utilisés lors de cette étude (Suite).]{\textbf{Liste des enzymes, kits et réactifs commerciaux utilisés lors de cette étude (Suite).}}.
\label{tab:tabM4}
\end{tableth}

\clearpage


% 7 - INSTRUMENTS

\section{Instruments}	

\begin{itemize}
  \item[$\bullet$] \textit{Countess II Automated Cell Counter}, Thermo Fisher Scientific (Groupe Hepacivirus, Institut Pasteur, Paris)
  \item[$\bullet$] \textit{7500 Fast Real-Time PCR System}, Applied Biosystems (Département de Virologie, Institut Pasteur, Paris)
  \item[$\bullet$] \textit{QuantStudio™ 7 Flex Real-Time PCR System}, Thermo Fisher Scientific (U5 Signalisation Antivirale, Institut Pasteur, Paris)
  \item[$\bullet$] \textit{LSM 700 Inverted Confocal Microscope}, Zeiss (Plateforme de Bio-imagerie Photonique, Institut Pasteur, Paris)
  \item[$\bullet$] \textit{JEM-1011 Transmission Electron Microscope}, JEOL (Faculté de Médecine, Université de Tours, Tours)
  \item[$\bullet$] \textit{EM UC7 Ultramicrotome}, Leica (Faculté de Médecine, Université de Tours, Tours)
  \item[$\bullet$] \textit{TriStar LB 942 Multimode Microplate Reader}, Berthold Technologies (Groupe Hepacivirus, Institut Pasteur, Paris)
  \item[$\bullet$] \textit{Odyssey CLx Imaging System}, Li-Cor (Département de Virologie, Institut Pasteur Paris)
  \item[$\bullet$] \textit{QIAcube HT}, Qiagen (Groupe Hepacivirus, Institut Pasteur, Paris)
  \item[$\bullet$] \textit{GenePulser XCell}, Bio-Rad (Groupe Hepacivirus, Institut Pasteur, Paris)
  \item[$\bullet$] \textit{Novex NuPAGE SDS-PAGE Gel System} Thermo Fisher Scientific  (Groupe Hepacivirus, Institut Pasteur, Paris)
  \item[$\bullet$] \textit{Veriti 96-Well Fast Thermal Cycler}, Applied Biosystems (Groupe Hepacivirus, Institut Pasteur, Paris)
  \item[$\bullet$] \textit{Eppendorf Mastercycler 5333 Thermal Cycler}, Eppendorf (Groupe Hepacivirus, Institut Pasteur, Paris)
  \item[$\bullet$] \textit{Nanodrop MySPEC Micro-volume Spectrophotometer}, Ozyme (Groupe Hepacivirus, Institut Pasteur, Paris)
  \item[$\bullet$] \textit{2100 Bioanalyzer}, Agilent (Plateforme Biomics, Institut Pasteur, Paris)
  \item[$\bullet$] DNB-G400, MGI (Plateforme Biomics, Institut Pasteur, Paris)
\end{itemize}