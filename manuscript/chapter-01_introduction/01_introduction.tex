%% Copyright (C) 2017-2021 Emeline Simon
%%
%% The current owner of this work is Emeline Simon
%% <contact at emeline.simon@gmail.com>.
%%
%% This is 01_introduction.tex the first chapter of my PhD Thesis.
%%
%%%%%%%%%%%%%%%%%%%%%%%%%%%%%%%%%%%%%%%%%%

\chapter{\'{E}tude bibliographique}
	\minitoc
	\newpage

%%%%%%%%%%%%%%%%%%%%%%%%%%%%%%%%%%%%%%%%%%%%%%%%%%%%%%%%%%%%%%%%%%%%%%%%%%%%%%%%%%%%%%%%%%%%%%%%%%%%%%%%%%%%%%%%%%%%%%%%%%%%%%%%%%%%%%%%%%%%%%%%%%%%%%%%%%%%%%%%%%%%%%%%
	
% 1 - LE FOIE HUMAIN

\section{Le foie humain}

Le foie est l’organe le plus volumineux du corps humain : une masse spongieuse brune rougeâtre d’environ 1,5kg et 30 cm de large, qui se situe au niveau de la cavité abdominale, sous le diaphragme. Il est le centre du métabolisme des nutriments et de l’élimination des déchets métaboliques, faisant de lui un des organes les plus vitaux. Son principal rôle est de contrôler le parcours et l’effet notoire des substances absorbés par le système digestif avant de les redistribuer au système circulatoire. \\ \\
\indent
Au cours de ce chapitre nous allons brièvement traiter de la structure et des fonctions principales du foie pour mieux comprendre les aspects physiologiques qui peuvent mener au développement de la stéatose, principale pathologie hépatique au coeur de cette étude, détaillée dans la \autoref{section:steatose}.

	\subsection{Anatomie, histologie et structure du foie}

\subsubsection{L'organe}
Le foie est divisé en deux lobes cunéiformes (droit et gauche) de taille et de forme inégale séparés par un ligament falciforme \citep{RN1237}. On peut également individualiser deux lobes mineurs situés sur la face viscérale du foie : le lobe caudé et le lobe carré. La vésicule biliaire est attachée au foie à la limite du lobe carré et du lobe hépatique droit (\autoref{fig:fig1}). Ils sont séparés par un sillon appelé le hile du foie : c’est à ce niveau que les vaisseaux sanguins pénètrent dans le foie et que passent les canaux biliaires majeurs. \\ \\
\indent
Le foie est relié à deux grands vaisseaux sanguins : l’artère hépatique et la veine porte. L'artère hépatique transporte le sang riche en oxygène de l'aorte, tandis que la veine porte transporte le sang riche en nutriments digérés par le tube digestif. En pénétrant dans le foie, ces vaisseaux sanguins se subdivisent en petits capillaires appelés sinusoïdes hépatiques, qui permettent l’apport des nutriments et de l’oxygène aux différentes cellules du foie. Parallèlement aux vaisseaux sanguins, le foie est parcouru par un grand nombre de canaux biliaires. Ils collectent la bile stockée dans la vésicule biliaire et la transportent vers le duodénum où elle sera utilisée pour la digestion.

	\begin{figureth}
	\centering
			\includegraphics[width=0.65\linewidth]{Figure_1.png}
		\caption[Le foie et les organes connexes]{\textbf{Le foie et les organes connexes.} Les deux lobes majeurs du foie (en brun) et les structures du système digestif connectées au système hépatique sont indiqués (Adapté de \citealt{RN1230}).}
				\label{fig:fig1}
	\end{figureth}
		\FloatBarrier
	

	\subsubsection{Structure du tissu hépatique}
Le tissu hépatique est composé de différents types cellulaires : les hépatocytes, les cellules épithéliales biliaires, les cellulaires stellaires, les cellules de Kupffer et les cellules endothéliales sinusoïdales. Chaque cellule hépatique a des fonctions uniques bien définies qui peuvent être liées à la structuration de l’organe, au métabolisme ou à la protection de l’organisme contre les pathogènes ou les substances xénobiotiques \citep[pour revue,][]{RN251}. \\ \\
\indent
Les hépatocytes sont des cellules cubiques hautement spécialisées, d’une taille allant de 20 à 30µm. Ils représentent la majorité de la masse hépatique (60\% des cellules constitutives du foie) et assurent la plupart des fonctions hépatiques liées au métabolisme. Les hépatocytes sont des cellules polarisées : c’est-à-dire qu’elles remplissent des fonctions distinctes selon l’orientation de leur membrane plasmique, basale ou apicale. Les hépatocytes peuvent contenir jusqu’à 8 jeux de chromosomes, ce qui faciliterait la régénération du foie. Les cellules stellaires jouent un rôle fondamental dans l’absorption et le stockage des vitamines liposolubles A et D. Des atteintes au foie peuvent activer et transformer les cellules stellaires en myofibroblastes, qui vont produire des composants de la matrice extracellulaire comme du collagène I, de la fibronectine et des protéoglycanes. Bien que ces éléments contribuent à la cicatrisation, une activation chronique des cellulaires stellaires représente un facteur important dans la progression des stades fibrotiques. Les cellules de Kupffer constituent la population de macrophages résidents du foie. Ils assurent la tolérance et l’élimination des corps étrangers constamment introduits par le système digestif. \\ \\
\indent
A l’échelle microscopique, les cellules hépatiques sont organisées sous forme de lobules, la plus petite unité fonctionnelle du foie \citep{RN1236}. Ce sont des structures hexagonales formées autour d’une veine centrale, dans lesquelles s’agencent des rangées d’hépatocytes, des sinusoïdes et des canalicules biliaires (\autoref{fig:fig2}\textcolor{blue}{.A}). Ces petites structures secondaires prennent naissance à partir du canal biliaire, de la veine portale et de l’artère hépatique qui forment la triade portale, à chaque angle du lobule. La structure des sinusoïdes est maintenue par les cellules endothéliales sinusoïdales qui, contrairement à l’endothélium des capillaires classiques, ont une membrane basale discontinue et des parois minces afin d’augmenter la perméabilité du tissu et les échanges entre le plasma sanguin et les hépatocytes. Une fine matrice de collagène appelée l’espace de Disse sépare les sinusoïdes des hépatocytes, dans lequel circulent les cellules stellaires et les cellules de Kupffer. Le sang riche en oxygène issu de l’artère hépatique est mélangé avec les nutriments dans la zone péri-portale, puis circule dans les sinusoïdes au contact des différentes cellules hépatiques avant d’être drainé par la veine centrale. Lors du passage du sang, les nutriments et les toxines sont extraits et métabolisés par les hépatocytes tandis que les cellules sanguines endommagées et les corps étrangers sont éliminés par les macrophages résidents. Les hépatocytes collectent les nutriments par les transporteurs présents sur la membrane basolatérale au contact des sinusoïdes (\autoref{fig:fig2}\textcolor{blue}{.B}). Ils produisent également des acides biliaires qui sont sécrétés par les transporteurs de la membrane apicale, au contact des canalicules biliaires. Des jonctions serrées imperméables permettent une séparation stricte entre la bile et le sang, qui peut être rompue en cas de dommages au foie.

	\begin{figureth}
	\centering
			\includegraphics[width=\linewidth]{Figure_2.png}
		\caption[Structure du tissu hépatique à l’échelle microscopique]{\textbf{Structure du tissu hépatique à l’échelle microscopique.} (A) Vues transversale et verticale de l’organisation d’un lobule hépatique (Adapté de \citealt{RN1230}). (B) Zoom sur une section du lobule représentant la structure des hépatocytes et leur implication dans les fonctions hépatiques (Adapté de \citealt{RN826}).}
				\label{fig:fig2}
	\end{figureth}
	\FloatBarrier

	\subsection{Fonctions hépatiques}

	\subsubsection{Action centrale dans le métabolisme énergétique}

L’organisme dépend de l’alimentation pour son approvisionnement en source d’énergie. Après le repas, les sucres et les graisses alimentaires sont absorbées par les entérocytes de l’intestin grêle et transférées vers le foie sous la forme de chylomicrons. Les nutriments qui arrivent de l'intestin dans le foie sont des molécules complexes qui nécessitent d’être transformées en formes utilisables par les cellules de l'organisme \citep[pour revue,][]{RN251}. Les sucres sont transformés en glucose qui est libéré dans la circulation sanguine vers les tissus extra-hépatiques. Les graisses sont transformées en acide gras et transportées dans le sang par les lipoprotéines. Les protéines sont converties en acides aminés et transférées aux muscles squelettiques. Le foie est responsable de 85-90\% de l’apport en protéines dans l’organisme. Comme la prise de nourriture est discontinue et entrecoupée de périodes plus ou moins longues de jeûne, l’organisme a besoin de stocker l’apport en nutriments excédentaire. Cette réserve d’énergie est conservée sous la forme de glycogène pour le glucose, de protéines pour les acides aminés et de triglycérides pour les acides gras. Le foie adapte son action selon le statut nutritionnel (stockage ou production d’énergie) qui dépend de l’équilibre entre la concentration sanguine de l’insuline et du glucagon \citep[pour revues,][]{RN827, RN1249}. Lors d’une forte demande en énergie par l’organisme, le foie va libérer de nouveau du glucose par la gluconéogénèse. En cas de carence en glycogène, le foie est capable de dégrader les triglycérides et les protéines et de métaboliser les acides gras qui peuvent fournir de l’énergie par la ß-oxydation ou les acides aminés qui peuvent servir de substrats pour les cycles de néoglucogénèse. Le foie est également responsable de la synthèse de nombreuses protéines du sérum, comme l’albumine, des facteurs de coagulation et des hormones stéroïdiennes.
	
	\subsubsection{Sécrétion de la bile}

Le foie assiste la digestion intestinale par la sécrétion de bile, un fluide jaune-verdâtre alcalin qui contient des sels biliaires, de la bilirubine (responsable de sa pigmentation) et des électrolytes. Les sels biliaires ont des propriétés émulsifiantes, aidant à l’absorption des acides gras par la muqueuse intestinale. Chaque jour, le foie sécrète environ 800 à 1000 mL de bile, qui sera directement déversée dans le duodénum ou stockée et concentrée dans la vésicule biliaire. La bile est également le moyen d'excrétion de certains principes actifs endogènes et exogènes.

	\subsubsection{Protection contre les substances xénobiotiques}

Le foie est souvent considéré comme un organe épurateur. Il constitue le premier site de passage du sang qui provient du tractus digestif et qui est donc porteur de nombreuses substances exogènes issues de l’alimentation. Celles-ci peuvent comporter différents pathogènes, qui seront reconnus et éliminés par les cellules du système immunitaire circulant dans le foie, ou des toxines qui seront inactivées par les cellules hépatiques. Le foie est également responsable de l’élimination de substances endogènes, tels que les déchets azotés issus du catabolisme. Les hépatocytes sont responsables de la dénaturation des molécules chimiques endogènes et exogènes, pour réduire leur toxicité et faciliter leur excrétion par les voies rénales ou intestinales. Alors que les substances hydrosolubles peuvent être directement éliminées par les reins, les substances lipophiles doivent être au préalable transformées par le foie. 

	\subsubsection{Système immunitaire}

Le foie est riche en glandes lymphatiques, ce qui assure un apport constant en cellules du système immunitaire. Après les macrophages de l’intestin, les cellules de Kupffer constituent la deuxième ligne de défense contre les bactéries alimentaires. Elles éliminent directement les bactéries par phagocytose ou production d’oxyde nitrique et sécrètent des cytokines inflammatoires qui recrutent les autres acteurs du système immunitaire. Le foie permet également de mettre en place un seuil de tolérance contre les antigènes issus de l’alimentation.

	\subsubsection{Stockage de minéraux et de vitamines}
	
Le foie est responsable du stockage de certaines vitamines et de minéraux telles que le fer sous la forme de ferritine, qui sera secrété lors des besoins de production d’hématies, le cuivre, la vitamine B12, la vitamine D, la vitamine A, la vitamine E et la vitamine K.

\clearpage
%%%%%%%%%%%%%%%%%%%%%%%%%%%%%%%%%%%%%%%%%%%%%%%%%%%%%%%%%%%%%%%%%%%%%%%%%%%%%%%%%%%%%%%%%%%%%%%%%%%%%%%%%%%%%%%%%%%%%%%%%%%%%%%%%%%%%%%%%%%%%%%%%%%%%%%%%%%%%%%%%%%%

% 2 - La stéatose hépatique

\section{La stéatose hépatique}
		\label{section:steatose}
		
La stéatose hépatique ou le syndrome du foie « gras » se caractérise par une accumulation de lipides dans les hépatocytes. Auparavant, la plupart des cas de stéatose étaient attribués à une consommation excessive d'alcool, dans le cadre de la maladie du foie gras alcoolique (AALD) (voir \autoref{section:etiologie}). En 1980, Ludwig et ses collègues ont décrit des patients d’âge moyen, présentant des résultats anormaux aux tests biochimiques du foie et des signatures histologiques identiques à l’hépatite alcoolique, c’est-à-dire, une stéatose modérée à sévère sans consommation apparente d’alcool \citep{RN847}. Cette cause aujourd’hui fréquente de stéatose chez les adultes a été baptisée maladie du foie gras non alcoolique (NAFLD), et a une prévalence mondiale estimée à 25\% \citep{RN846}. Les hépatites virale, en particulier l’hépatite C chronique, sont également des conditions médicales à risque dans le développement de la stéatose hépatique. Cette pathologie était traditionnellement considérée comme bénigne et réversible avec seule une petite proportion des patients évoluant vers une cirrhose avec un risque de développer un carcinome hépatocellulaire (CHC) (voir \autoref{section:histopathologie}). Toutefois, en raison de sa prévalence élevée, tant chez les sujets en surpoids, de poids normal ou chez les sujets maigres, elle est responsable de près de 2 millions de décès par an dans le monde et représente aujourd’hui la deuxième indication principale pour la transplantation hépatique aux États-Unis \citep{RN829}. Approximativement 2 milliards d’humains sont en surpoids ou obèses, 400 millions souffrent de diabète et 325 millions sont atteints d’hépatites virales. Avec l’accroissement de la population obèse et diabétique (qui s’est multipliée par 6 sur les quatres dernières décennies), l’augmentation préoccupante du taux d’enfants en surpoids et la stabilité du taux d’hépatites virales, les scientifiques prédisent que le fardeau mondial des pathologies hépatiques aiguës et chroniques risque d’augmenter de façon critique à l’avenir et les transplantations hépatiques deviendront une ressource inestimable.

\subsection{Étiologies}
		\label{section:etiologie}

Dans les pays occidentaux, la stéatose hépatique est majoritairement attribuée soit à une consommation excessive d'alcool, \textit{i.e.} l’AALD, soit au surpoids ou à l’obésité, \textit{i.e.} la NAFLD. L’excès de nutrition peut mener à une désensibilisation du foie à la sécrétion d’insuline. La résistance à l’insuline favorise la libération d’acides gras libres du tissu adipeux dans le sang et leur ré-absorption par le foie \citep{RN837}. L’excès d’alcool, quant à lui, inhibe la mobilisation des graisses par le simple changement dans l’équilibre d’oxydo-réduction par la détoxification récurrente de l’éthanol en acétate \citep{RN831}. Il n’existe pas de consensus sur ce qui représente une consommation « excessive » d’alcool d’après les différentes études scientifiques sur le sujet, mais le seuil de 210g/semaine pour les hommes et 140g/semaine pour les femmes sur une période minimale de 2 ans précédant le diagnostic histologique est généralement utilisé pour différencier l’AALD de la NAFLD \citep{RN866}. La progression vers la cirrhose est plus importante chez les patients atteints d’AALD que de NAFLD, en raison de l’accumulation d’éthanal, un produit toxique issu de l’oxydation partielle de l’éthanol, qui peut occasionner des lésions hépatiques. Globalement, 50\% de la mortalité liée aux cirrhoses est attribuable à l’AALD. La progression vers un stade avancé de fibrose ou de cirrhose est relativement lente pour les patients atteints de NAFLD et la plupart d’entre deux meurent de causes indépendantes comme des accidents cardiovasculaires. \\ \\
\indent
Dans les pays en développement, l’hépatite C chronique restent la cause majeure de développement de la stéatose hépatique. La primo-infection par le virus de l’hépatite C (VHC) est majoritairement asymptomatique mais le virus persiste dans 80\% des cas d’infection \citep{RN834}. La détection du VHC dans le sang au-delà de 6 mois signe l’évolution de l’hépatite C vers une forme chronique. Au fil des années, le virus provoque des atteintes hépatiques subtils mais cumulatifs et une stéatose hépatique apparaît dans 40 à 80\% des cas d’infection selon le génotype viral infectant (voir \autoref{section:classification}). Les mécanismes virologiques liés au développement de la stéatose ne sont pas encore établis, mais le VHC est connu pour interférer avec les processus d’homéostasie du métabolisme des lipides. Le stockage excessif des triglycérides dans les GL pourraient résulter de différents processus : (i) une augmentation de la synthèse des triglycérides, combinée à la biogénèse des GL, (ii) une diminution du catabolisme des lipides et de la lipolyse des GL (voir \autoref{section:formationgl}), et (iii) une altération de la sécrétion des lipoprotéines \citep[pour revue,][]{RN627}. L’hépatite C chronique est également une prédisposition majeure au développement de complications hépatiques, comme la cirrhose et le CHC, résultant en moyenne en 500.000 décès par an, tuant plus de personnes que le paludisme ou la tuberculose \citep{RN833}. En effet, les protéines virales interfèrent avec les voies de signalisation impliquées dans la survie cellulaire, la prolifération et la transformation \citep{RN838}. Après 25 à 30 ans d’infection, 15 à 35\% des patients développent une cirrhose et le risque d’atteindre un état de décompensation ou un cancer est plus important dans le cas de l’hépatite C que d'autres étiologies. Un quart des cancers du foie, la 4ème cause de décès par cancers dans le monde, est attribuable à une infection par le VHC, faisant de ce virus l’un des sept virus oncogènes chez l’Homme \citep{RN832}. L’évolution globale de la pathologie de l’hépatite C est résumée sur la \Autoref{fig:fig3}.

	\begin{figureth}
	\centering
			\includegraphics[width=\linewidth]{Figure_3.png}
		\caption[Évolution et stades pathologiques de l’hépatite C chronique]{\textbf{Évolution et stades pathologiques de l’hépatite C chronique.} Les mécanismes biologiques principaux impliqués dans la progression des différentes stades pathologiques sont encadrés en rouges. ROS : dérivés réactifs de l’oxygène (Adapté de \citealt{RN855}).}
				\label{fig:fig3}
	\end{figureth}
	\FloatBarrier

Par ailleurs, la stéatose hépatique peut également survenir dans une variété de conditions médicales ou être déclenchée par des médicaments ou des régimes alimentaires riches en omégas 6 et en fructose, qui stimulent la lipogénèse \textit{de novo}. Les populations hispaniques ou asiatiques seraient plus sensibles au syndrome métabolique du foie « gras » que leurs homologues occidentaux. Enfin, des polymorphismes génétiques dans les gènes \textit{PNPLA3} \citep{RN851} et \textit{TM6SF2} \citep{RN850}, impliqués dans un défaut de lipidation des lipoprotéines, sont également des facteurs de risque du développement de la stéatose.


\subsection{Histopathologie, diagnostic et traitements}
		\label{section:histopathologie}

Traditionnellement, une teneur en graisse hépatique dépassant 5\% du poids du foie est considérée comme la définition d’une stéatose hépatique. Cette définition est difficilement applicable dans le contexte clinique, de ce fait, le diagnostic de la stéatose est établi si > 5\% des hépatocytes contiennent des vacuoles graisseuses, plus communément appelées gouttelettes lipidiques (GL) \citep{RN1242}. La stéatose hépatique est classée en deux étiologies : la stéatose macro-vésiculaire et la stéatose micro-vésiculaire. La plupart des stéatoses sont de type macro-vésiculaire, dans laquelle une unique vacuole volumineuse occupe le cytoplasme de la cellule, déformant l’architecture cellulaire (\autoref{fig:fig4}\textcolor{blue}{.A}). La stéatose micro-vésiculaire est moins fréquente mais généralement plus grave et correspond à un fractionnement de la graisse en nombreuses petites GL (\autoref{fig:fig4}\textcolor{blue}{.B}). La sévérité de la stéatose est déterminée histologiquement par le pourcentage d'hépatocytes comportant une ou plusieurs vésicules graisseuses dans le cytoplasme \citep[pour revue,][]{RN1245}. On parle de stéatose légère en dessous de 20\%, stéatose modérée entre 20 et 50\% et stéatose sévère au-delà de 50\%. La stéatose comprend un spectre de lésions hépatiques allant de la simple stéatose, c’est-à-dire, uniquement des vacuoles graisseuses qui ne perturbent pas la fonction hépatique, à la stéatohépatite (SH), un état dans lequel la stéatose est associée à une inflammation lobulaire avec une infiltration mixte de neutrophiles, de lymphocytes et de macrophages (\autoref{fig:fig4}\textcolor{blue}{.C}). Le phénomène qui mène d’une simple accumulation de lipides considérée comme « inoffensive » à l’inflammation n’est pas entièrement comprise. Une théorie proposée en 1975 par l'hépatologiste Heribert Thaler établit que « la cause de la stéatose et non l’accumulation de graisses serait responsable d’une évolution vers la fibrose et la cirrhose », expliquant les différents degrés d’évolution selon l’origine de la pathologie \citep{RN1244}. Une autre hypothèse souligne que la capacité du foie à stocker les acides gras serait dépassée ce qui conduit à l’accumulation d’espèces lipidiques toxiques \citep{RN1246}. Ces métabolites induisent un stress hépatocellulaire et des lésions qui entraînent la mort cellulaire, responsable de l’inflammation. Dans ce contexte, la stéatose est souvent considérée comme une étape favorisant le développement de complications hépatiques graves. En effet, la destruction substantielle du tissu hépatique par nécrose résultant de l’inflammation peut déclencher une synthèse de matrice extracellulaire issue du processus de cicatrisation : la fibrose. Quand la fibrose s’étend et éventuellement remplace toute l’architecture du foie jusqu’à entraîner une congestion des nodules hépatiques, elle peut mener à une dégénérescence hépatique grave, la cirrhose avec un risque de développement de CHC de 2 à 5\% par an. Lorsque les défaillances hépatiques s’aggravent, la progression d’une cirrhose vers un état de décompensation implique des hémorragies dues à la formation de varices hépatiques et à l’hypertension du système veineux du foie.

	\begin{figureth}
	\centering
			\includegraphics[width=\linewidth]{Figure_4.png}
		\caption[Coupes histologiques du tissu hépatique présentant les différents types de stéatose]{\textbf{Coupes histologiques du tissu hépatique présentant les différents types de stéatose.} (A) La stéatose macro-vésiculaire est caractérisée par des hépatocytes présentant des GL volumineuses, arrondies et bien définies (souvent uniques) dans le cytoplasme et les noyaux sont déplacés à la périphérie. (B) La stéatose micro-vésiculaire est caractérisée par des hépatocytes comportant de nombreuses gouttelettes lipidiques et leurs noyaux restent centrés. (C) Stéatohépatite avec au centre des infiltrats inflammatoires (D'après \citealt{RN1239} et \citealt{RN1245}).}
				\label{fig:fig4}
	\end{figureth}
	\FloatBarrier

Les individus atteints de stéatose hépatique ne présentent souvent aucun symptôme perceptible et la pathologie n’est souvent détectée qu’au cours d’analyses sanguines de routine, par l’augmentation anormale des enzymes du foie de type transaminases (ASAT, ALAT) ou des biomarqueurs comme l’albumine et la bilirubine. L’analyse histologique des tissus par biopsie est le seul test largement accepté pour diagnostiquer et distinguer définitivement la stéatose des autres formes de pathologies hépatiques et pour évaluer la gravité de l’inflammation et de la fibrose qui en résulte. L’aspect invasif de cette méthode peut néanmoins entraîner des complications, en plus de ne pas nécessairement être représentative de la situation. En effet, l’échantillon de foie obtenu par biopsie hépatique ne représente que 1/50.000 de la masse totale du foie selon l’individu \citep{RN841}. Les limites inhérentes à la biopsie ont encouragé le développement de techniques non invasives. Initialement, l'échographie était utilisée car elle présentait une excellente sensibilité et spécificité pour les stéatoses modérées et sévères. La mesure de la teneur en triglycérides hépatiques et l'utilisation de la résonance magnétique et de l'élastographie ont également évolué au cours de la dernière décennie et sont devenues des méthodes prometteuses pour diagnostiquer et quantifier la stéatose et les stades fibrotiques dans le domaine de la recherche clinique. \\ \\
\indent
À l’heure actuelle, il n’existe pas de traitements approuvés pour soigner la stéatose, bien que certains soient en cours de développement pré-clinique ou en phase d’essais cliniques. Toutefois, la stéatose est une pathologie réversible, qui peut disparaître en éliminant la cause : une perte de poids et une alimentation plus adaptée, par exemple en instaurant un régime méditerranéen ou végétarien pauvre en calories dans le cas de la NAFLD \citep{RN836,RN844} ; un arrêt de la consommation d’alcool dans le cas de l’AALD ; une guérison dans le cas de l’hépatite C chronique (voir \autoref{section:elimination}). Cependant, si les stades plus avancés de la pathologie sont atteints, comme la cirrhose, l’élimination de la cause n’est généralement pas suffisante et la progression vers un état de décompensation ou un cancer reste un risque.  En effet, il est désormais admis que le risque de développer un CHC peut persister après guérison par thérapie antivirale chez les patients qui étaient atteints de formes graves d’hépatite C \citep{RN1076,RN1077}. À ce stade, les seules options thérapeutiques sont la transplantation hépatique ou la résection de la tumeur. L’espérance de vie sans transplantation hépatique est d’au maximum 3 ans dans le cas d’une cirrhose décompensée sous surveillance étroite et traitement médical lourd et de 18\% après 5 ans pour le CHC. La transplantation hépatique est la seconde transplantation d’organe la plus commune après celle du rein. Cependant, moins de 10\% de la demande en transplantations est honorée au vu du nombre actuel de donneurs \citep{RN839}. Malgré les mesures prises pour encourager les dons et développer les procédures de greffes d’urgence suite à des arrêts cardiaques, le nombre de transplants hépatiques reste très insuffisant dans les pays occidentaux et dans les pays en développement.

\clearpage
%%%%%%%%%%%%%%%%%%%%%%%%%%%%%%%%%%%%%%%%%%%%%%%%%%%%%%%%%%%%%%%%%%%%%%%%%%%%%%%%%%%%%%%%%%%%%%%%%%%%%%%%%%%%%%%%%%%%%%%%%%%%%%%%%%%%%%%%%%%%%%%%%%%%%%%%%%%%%%%%%%%%

% 3 - Les gouttelettes lipidiques

\section{Les gouttelettes lipidiques}	
		\label{section:gl}

Les être vivants déploient un flux constant d'énergie essentiel au maintien de leurs fonctions biologiques vitales. Les nouvelles sources d’énergie n’étant pas toujours disponibles, la capacité à stocker les carbohydrates, les lipides et les protéines dans les tissus est un mécanisme évolutif crucial à la survie. Les tissus sont capables de conditionner et de stocker les lipides excédentaires sous une forme inerte dans des organites intracellulaires spécialisés. Leur première description date du 19ème siècle lorsque des chercheurs identifient la présence de « gouttes d’huile » dans le cytoplasme d’embryons d’oursin en développement. Peu après, ces organites ont été reconnus comme un composant inerte de la plupart des cellules eucaryotes et appelées « liposomes ». Depuis l’invention des liposomes artificiels à la fin des années 1960, elles ont été renommées selon différentes appellations, en passant par « corps lipidiques », « adiposomes », « vacuoles lipidiques », « vésicules lipidiques » jusqu’à l’officialisation récente de GL comme nomenclature officielle \citep{RN871}. Longtemps perçues comme de simples inclusions cytoplasmiques stockant les lipides neutres, les GL sont apparues ces dernières années comme des organites dynamiques assurant des fonctions clés dans l’homéostasie lipidique et énergétique. Elles sont particulièrement importantes dans les tissus spécialisés dans le stockage de l'énergie ou le renouvellement des lipides, comme le tissu adipeux, le foie et l'intestin, mais elles sont également retrouvées dans les muscles squelettiques, le cortex surrénalien et les glandes mammaires. Bien que les GL soient des organites universels retrouvés dans la plupart des cellules eucaryotes, y compris chez les organismes végétaux, leur contenu, leur morphologie, leur composition, leur organisation spatiale et leurs fonctions physiologiques varient considérablement entre les types cellulaires et les tissus. Au cours de ce chapitre, nous détaillerons l’ensemble de ces points en nous centrant principalement sur les caractéristiques et les fonctions des GL propres aux hépatocytes.

	\subsection{Caractéristiques principales des gouttelettes lipidiques}
	
	\subsubsection{Structure, composition et distribution des gouttelettes lipidiques}
	
Les GL possèdent une architecture unique sous la forme d’un noyau hydrophobe composé de lipides neutres et entouré d’une monocouche phospholipidique décorée par un ensemble de protéines résidentes ou simplement localisées de façon transitoires (\autoref{fig:fig5}). Toutes les gouttelettes lipidiques présentent une organisation structurale similaire qui les distingue des autres organites. Le noyau central est principalement constitué de triacylglycérols (TAG) et d’esters de stérol et la monocouche superficielle est riche en phosphatidylcholine (PC), phosphatidylethanolamine (PE), phosphatidylinositol (PI), phosphatidylsérine (PS) et en lysophospholipides, ce qui diffère des bicouches phospholipidiques typiques des autres membranes intracellulaires \citep{RN872,RN873}.

	\begin{figureth}
	\centering
			\includegraphics[width=0.5\linewidth]{Figure_5.png}
		\caption[Structure et composition de la gouttelette lipidique]{\textbf{Structure et composition de la gouttelette lipidique.} Le noyau central des GL est composé principalement de lipides neutres, comme le triacylglycérol et les esters de stérol. La membrane des GL est une monocouche phospholipidique associée à des protéines adhérentes ou intégrées (Adapté de \citealt{RN645}).}
				\label{fig:fig5}
	\end{figureth}
	\FloatBarrier
	
Les premières protéines constitutives des GL découvertes appartiennent à la famille des périlipines (PLIN) \citep{RN875}. Cette famille de protéines de mammifères est constituée de 5 membres, PLIN1 à PLIN5, qui présentent un profil d’expression différent selon le type cellulaire et qui jouent de multiples rôles dans la biologie des GL \citep{RN874}. Depuis, le développement récent des technologies à haut débit a permis de révéler au cours d’études protéomiques une grande variété de protéines co-purifiées avec les GL, qui varient selon les types cellulaires, tissus et organismes \citep[pour revue,][]{RN878}. Au sein d’une cellule de mammifère prototypique, le protéome des GL est généralement composé de 100 à 150 protéines. Ces protéines sont adhérentes à la périphérie des GL ou intégrées au sein de la monocouche de phospholipides par le biais de domaines hydrophobes ou d’hélices amphipatiques. Le mécanisme de recrutement des protéines sera détaillé dans la \autoref{section:recrutement}. La composition lipidique et protéique des gouttelettes lipidiques est influencée par le rôle physiologique de la cellule d’origine, ce qui résulte en une panoplie de fonctions possibles assurées par les GL, décrites dans la \autoref{section:fonctions}. À titre d’exemple, les cellules stellaires contiennent des GL riches en esters de rétinol peu courants, en raison de leur rôle dans le stockage de la vitamine A. De plus, au sein d’une même cellule, la composition lipidique et protéique des GL évolue en fonction de leur état métabolique et de leur devenir (\autoref{fig:fig6}\textcolor{blue}{.A}), par exemple s’il s’agit d’une GL en cours d’expansion ou de dégradation (voir \autoref{section:formationgl}). En effet, certaines protéines, dont les périlipines, s’associent à différentes sous-populations de GL, soit pour promouvoir leur biogénèse et expansion pour PLIN3, soit pour les stabiliser, dans le cas de PLIN1 et PLIN2. \\ \\
\indent
En plus des variations dans la composition, le contenu et la morphologie des GL est très hétérogène entre les types cellulaires (\autoref{fig:fig6}\textcolor{blue}{.B}) \citep[pour revue,][]{RN876}. Le nombre de GL dans les cellules peut varier de 4 à 8 dans les levures à des centaines dans les cellules de mammifères en culture. Toutes les cellules eucaryotes sont capables de former des petites GL de l’ordre de 300 à 800nm lors de la biosynthèse, mais certaines cellules spécialisées dans le stockage des lipides peuvent présenter des GL supérieures à 1µm, comme les adipocytes ou les hépatocytes. Les adipocytes blancs présentent parfois même une unique GL dite « uniloculaire » allant jusqu’à 100µm de diamètre, remplissant le cytoplasme et repoussant mécaniquement le noyau à la périphérie. Une GL large présente un rapport surface/volume plus faible qu’un ensemble de petites GL ayant la même teneur totale en lipides neutres. Les grandes GL seraient donc plus adaptées au stockage à long terme des lipides neutres, en minimisant le besoin en phospholipides. Les petites GL en revanche, sont plus adaptées aux situations dynamiques nécessitant une libération et un stockage rapide de lipides lors des interactions avec d'autres organites \citep{RN877}. De plus, les GL peuvent présenter une distribution intracellulaire dispersée, groupée ou polarisée selon l’état métabolique de la cellule (\autoref{fig:fig6}\textcolor{blue}{.C}). Par exemple, les GL ont tendance à se regrouper et à s'étendre pendant la lipogenèse et à se redisperser en cas de besoins énergétiques. En effet, les changements de statut énergétique influencent hautement le devenir, la fonction et le positionnement des GL, démontrant la versatilité de cette organite et son rôle crucial dans l’homéostasie et la survie cellulaire.

	\begin{figureth}
	\centering
			\includegraphics[width=0.65\linewidth]{Figure_6.png}
		\caption[Diversité dans le contenu, la morphologie et la distribution intracellulaire des gouttelettes lipidiques]{\textbf{Diversité dans le contenu, la morphologie et la distribution intracellulaire des gouttelettes lipidiques.} (A) La composition du noyau lipidique et de l’enveloppe protéique des GL individuelles peut différer selon leur état métabolique et leur fonction, par exemple si les GL grossissent (+) ou rétrécissent (-). (B) Les GL peuvent avoir une taille uniforme, plus ou moins hétérogène ou géante, comme les GL uniloculaires retrouvées dans les adipocytes blancs de mammifères. (C) Les GL peuvent être dispersées, regroupées ou être localisées de manière polarisée (Adapté de \citealt{RN876}).}
				\label{fig:fig6}
	\end{figureth}
	\FloatBarrier
	
\subsubsection{Propriétés biophysiques des gouttelettes lipidiques}

Compte tenu de leur nature, les GL se comportent comme la phase dispersée d’une émulsion « huile dans l’eau », le cytosol étant le milieu aqueux des cellules, et suivent donc les principes biophysiques des émulsions \citep[pour revue,][]{RN608}. La plupart des émulsions sont thermodynamiquement instables, en raison de l’absence d’interactions cohésives entre les molécules de la phase dispersée et de la phase aqueuse. Une tension de surface est alors générée à l’interface entre les deux phases. La clé de la stabilité des émulsions est la présence de tensioactifs, qui minimisent la surface de contact entre les fluides non miscibles, ce qui diminue le coût énergétique. Pour les GL, la monocouche de phospholipides constitue le principal surfactant \citep{RN882}. Elle permet de fournir des interactions favorables avec la phase aqueuse grâce aux têtes polaires orientées vers le cytosol et avec la phase dispersée grâce aux chaînes acyles directement en contact avec le noyau lipidique. Les protéines présentes à la surface de la gouttelette lipidique peuvent également augmenter l'élasticité de l'interface, contribuant ainsi à la stabilité thermodynamique de l’organelle. Cette tension de surface est d’ailleurs responsable de la forme arrondie des gouttelettes lipidiques qui permet de minimiser la surface de lipides neutres potentiellement en contact avec le cytosol aqueux.


	\subsection{Cycle de formation et de dégradation des gouttelettes lipidiques}
	\label{section:formationgl}

	Les GL sont des organites dynamiques qui se forment, s’élargissent, se rétractent ou se dégradent en réponse aux changements d’état métabolique des cellules, par exemple s’il s’agit d’une période d’alimentation ou de jeûne avec une forte demande en besoins énergétiques. Notre compréhension actuelle du cycle de formation et de dégradation des GL est principalement dérivée de travaux sur les adipocytes et les mécanismes ou les acteurs protéiques impliqués dans toutes ces étapes peuvent différer dans d'autres types de cellules.

	\subsubsection{Biosynthèse des gouttelettes lipidiques}

Dans tous les organismes eucaryotes, la formation des GL est initiée par une accumulation de lipides neutres entre le feuillet interne et le feuillet externe du RE \citep[pour revue,][]{RN645}. La synthèse de ces lipides est catalysée par des enzymes impliquées dans la transformation des acides gras en excès ou dans la lipogénèse \textit{de novo} à partir de glycérol. L’étape finale de la synthèse des triacylglycérides (TAG) résulte de l’activité des acyl-CoA diacylglycérol 1 (DGAT1) et 2 (DGAT2)  et les esters de stérol sont le produit des acétyl-CoA acétyltransférase 1 (ACAT1) et 2 (ACAT2) et des stérol O-acyltransférase 1 (SOAT1) et 2 (SOAT2). À faible concentration, les lipides neutres sont solubilisés entre les feuillets de la bicouche lipidique et s’équilibrent à travers le réseau réticulé. Lorsqu’un seuil de concentration locale en lipides est atteint, une « lentille » lipidique se forme au sein de la bicouche (\autoref{fig:fig7}). Ces précurseurs sont généralement très petits (50 nm) et ont une durée de vie de l’ordre de quelques minutes, rendant difficile l’observation des sites de nucléation des GL en cellules vivantes. Dans des conditions de lipogénèse prolongée, la lentille lipidique s’agrandit et déforme la bicouche, jusqu’à ce qu’elle atteigne une taille et une tension de surface critique qui déclenche le bourgeonnement d’une GL « naissante » d’environ 100 à 300 nm. Pour maintenir l’homéostasie de la membrane lors de l’expansion de la lentille, la synthèse de lipides neutres est couplée à une synthèse \textit{de novo} de phospholipides grâce à l’activation de l’enzyme limitante phosphocholine cytidylyltransférase α (CCTα). Plusieurs études suggèrent que les principes physico-chimiques suffisent pour expliquer la formation et le bourgeonnement de la lentille, ces processus étant énergétiquement favorables pour réduire l’interaction des lipides neutres avec les autres composants membranaires. Or, des travaux récents ont identifié plusieurs complexes protéiques qui régulent et facilitent ce processus, au-delà de leur rôle d’enzyme catalysant la synthèse des lipides neutres ou des phospholipides, en induisant par exemple une courbure appropriée du feuillet phospholipidique externe et en contrôlant la direction du bourgeonnement \citep{RN549}. C’est le cas des protéines transmembranaires induisant le stockage des graisses 1 (FIT1) et 2 (FIT2) et des oligomères de seipine, qui sont capables de cibler les sites enrichis en lipides neutres et de réguler le bourgeonnement pour que les GL émergentes soient plus uniformes par rapport aux cellules dépourvues de ces protéines. Au cours de cette étape, les GL naissantes acquièrent un certain nombre de protéines, qui diffusent directement à partir du RE ou qui proviennent du cytosol et se délocalisent préférentiellement à leur surface. Seule la monocouche externe peut accueillir des protéines cytosoliques, comme la périlipine Pln1 chez la levure (orthologue de PLIN3 chez les mammifères), qui s’associe à la surface des GL naissantes lors des étapes très précoces de bourgeonnement. Cette asymétrie dans le recrutement des protéines entre les deux feuillets modifierait l’équilibre de la tension superficielle, ce qui déclencherait le bourgeonnement de la GL exclusivement vers le cytosol (\autoref{fig:fig7}\textcolor{blue}{, encadré}). Ches les eucaryotes supérieurs, une population de GL se détache et diffuse librement dans le cytosol selon un mécanisme de fission encore indéterminé.

	\begin{figureth}
	\centering
			\includegraphics[width=\linewidth]{Figure_7.png}
		\caption[Biosynthèse \textit{de novo} des gouttelettes lipidiques cytosoliques]{\textbf{Biosynthèse \textit{de novo} des gouttelettes lipidiques cytosoliques.} Les réactions successives de la lipogénèse \textit{de novo} sont illustrées dans l’encadré sur fond gris. G3P : glycérol-3-phosphate : GPAT : glycérol-3-phosphate acyltransférase ; LPA : acide lysophosphatidique ; AGPAT :  acylglycérolphosphate acyltransférase ; PA : acide phosphatidique ; PAP : acide phosphatidique phosphatase ; DAG : diacylglycérol ; DGAT : acyl-CoA diacylglycérol ; TAG ; triacylglycérol. (Adapté de \citealt{RN645}).}
				\label{fig:fig7}
	\end{figureth}
	\FloatBarrier
	
\subsubsection{Expansion des gouttelettes lipidiques}

En cas de forte période d’alimentation, une surcharge du processus de biosynthèse \textit{de novo} des GL peut être compensée par un élargissement des GL pré-existantes par synthèse locale en lipides neutres (\autoref{fig:fig8}\textcolor{blue}{.A}). Ce processus est médié par la présence d’enzymes comme les glycérol-3-phosphate acyltransférase (GPAT), les 1-acylglycérol-3-phosphate O-acyltrans- férase (AGPAT) ou DGAT2 qui catalysent les étapes successives de la synthèse des TAG directement à la surface des GL \citep[pour revue,][]{RN607}. De plus, les membres de la famille des acyl-CoA synthétases à longue chaîne (ASCL), qui convertissent les acides gras libres à longue chaîne en esters d'acyl-CoA, seraient également mobilisés pour la synthèse locale de lipides neutres. En l’absence de ces enzymes, les GL peuvent s’approvisionner en lipides neutres en interagissant de nouveau avec les membranes du RE. Le mécanisme de ré-attachement au RE sera explicité en détail dans la \autoref{section:recrutement}. De nouveau au contact avec le RE, les GL seront supplémentées en lipides neutres et en enzymes qui pourront se délocaliser préférentiellement à leur surface. Il est également possible que certaines enzymes impliquées dans la synthèse des lipides neutres soient recrutées directement à la surface des GL à partir du cytosol. \\ \\
\indent
Un mécanisme alternatif d’élargissement des GL repose sur la fusion homotypique de deux GL adjacentes (\autoref{fig:fig8}\textcolor{blue}{.B}). La fusion est initialement déclenchée par un contact entre les membranes adjacentes de deux GL qui, si l’épaisseur du film aqueux les séparant s’approche de quelques nanomètres, peuvent former un pore reliant directement les deux phases lipidiques neutres et favoriser leur mélange lent au fur et à mesure de l'expansion du pore. La fusion est un processus relativement long qui peut prendre jusqu'à 15 minutes et qui est fortement régulé par les protéines de l'hôte et par l'équilibre entre les lipides neutres et les phospholipides composant la monocouche de surface \citep{RN555}. Parmi les régulateurs connus du mécanisme de fusion des GL, on retrouve la famille des effecteurs induisant la mort cellulaire (CIDE), constituée de trois membres : CIDEA, CIDEB et CIDEC. Dans le tissu adipeux, CIDEA et CIDEC régulent conjointement le mécanisme de fusion des GL, tandis que CIDEB est constitutif du tissu hépatique \citep{RN556}. Les protéines CIDE partagent le même mécanisme d’action, à savoir la formation d’oligomères stables qui diffusent sur la surface des GL et qui sont capables d’interagir avec les protéines CIDE homologues localisées sur une autre GL. Cette interaction permet de relier physiquement deux GL adjacentes, ce qui provoque leur fusion \citep{RN557,RN558}. D’autres protéines favorisent le processus de fusion comme l’hydrolyse associé aux GL (LDAH) selon un mécanisme d’action encore inconnu \citep{RN560}. La coalescence spontanée des GL par simple principe physico-chimique peut également se produire mais ce phénomène est peu fréquent dans des conditions normales de croissance. En cas de stress membranaire où les besoins en phospholipides sont importants ou dans des cellules présentant un défaut de synthèse des phospholipides, les GL adjacentes fusionnent, ce qui diminue la quantité de phospholipides nécessaires pour couvrir densément la surface du noyau hydrophobe \citep{RN561}. Ainsi, la fusion des GL peut être un moyen de fournir des phospholipides membranaires pour d’autres processus cellulaires et de limiter le besoin en phospholipides pour la synthèse des GL.

	\begin{figureth}
	\centering
			\includegraphics[width=\linewidth]{Figure_8.png}
		\caption[Mécanismes d’expansion des gouttelettes lipidiques cytosoliques par synthèse locale ou par fusion]{\textbf{Mécanismes d’expansion des gouttelettes lipidiques cytosoliques par synthèse locale ou par fusion.} (A) La synthèse locale de lipides neutres peut être médiée par des enzymes de biosynthèse des TAG ou d’esters de stérol délocalisés à la surface des GL lors l’étape de la biosynthèse ou recrutées dans le cytosol. L’approvisionnement en lipides neutres et en enzymes peut également s’effectuer lors des contacts ultérieurs avec le RE, par la formation d’un pont entre les membranes facilité par le complexe ARF1/COPI. (B) La coalescence ou la fusion peut avoir lieu spontanément en cas de défaut dans la couverture phospholipidique ou peut être activement médiée par l’intermédiaire des complexes protéiques CIDE (Adapté de \citealt{RN627}).}
				\label{fig:fig8}
	\end{figureth}
	\FloatBarrier

\subsubsection{Dégradation des gouttelettes lipidiques}

La dégradation des GL a lieu lorsque des précurseurs lipidiques doivent être mobilisés comme carburant métabolique ou pour la synthèse de composants comme les membranes cellulaires. Comme pour la synthèse locale, ce processus peut s’effectuer par l’action de lipases qui agissent directement à la surface des GL et qui vont catalyser la digestion des lipides neutres : on parle de « lipolyse » (\autoref{fig:fig9}\textcolor{blue}{.A}). La première réaction est catalysée par la patatine contenant un domaine phospholipase 2 (PNPLA2), également connue sous le nom de lipase triglycéride adipeuse (ATGL), qui hydrolyse les TAG pour générer du diacylglycérol (DAG) et une molécule d’acide gras libre. Le DAG est ensuite clivé en monoacylglycérol (MAG) libérant une autre molécule d’acide gras libre par la lipase sensible aux hormones (HSL). Enfin, le MAG est clivé en glycérol et en acide gras libre par la monoglycéride lipase (MGL) \citep[pour revue,][]{RN562}. Contrairement à PNPLA2 et HSL qui hydrolysent des composés strictement hydrophobes, MGL cible les MAG qui sont davantage solubles et qui peuvent diffuser librement dans le cytosol. MGL ne requiert donc pas une localisation stricte à la surface des GL et la dernière étape de la digestion est généralement cytoplasmique. Les enzymes ATGL et MGL sont exprimées dans le foie, bien qu’à des niveaux plus faibles que dans les adipocytes, tandis que l’expression de l’enzyme HSL qui catalyse la deuxième réaction semble négligeable dans le tissu hépatique. Il n’est donc pas certain que ces enzymes constituent les principales enzymes de la lipolyse hépatique. En revanche, un nombre grandissant de témoignages pointent d’autres lipases qui pourraient avoir un rôle majeur dans la lipolyse hépatique, comme la patatine contenant un domaine phospholipase 3 (PNPLA3) et la triacylglycérol lipase hépatique (LIPC). Les phospholipides pourraient être éliminés en parallèle de la lipolyse par voie enzymatique, les rendant solubles dans l’eau. À l’appui de cette hypothèse, plusieurs membres de la famille des phospholipases A2 sont retrouvés à la surface des GL. Lorsque les lipases catalysent la digestion du noyau hydrophobe, la taille des GL diminue, ce qui génère un encombrement stérique des protéines présentes à leur surface \citep{RN564}. En général, les protéines qui ciblent les GL à partir du cytosol via une hélice amphipatique sont faiblement liées et sont susceptibles d’être détachées aisément et remobilisées dans d’autres processus cellulaires. À l’inverse, les protéines ancrées via un domaine plus hydrophobe sont davantage susceptibles d’être dégradées en parallèle du noyau lipidique. Elles sont reconnues et extraites sélectivement par la protéine de choc thermique 70 (HSP70), transférées aux lysosomes pour être dégradée par autophagie \citep{RN567}. \\ \\
\indent
Plus récemment, une forme de macro-autophagie appelée « lipophagie » a été découverte en 2009 et représente une voie catabolique importante dans la mobilisation rapide des lipides \citep{RN568}. La lipophagie ne peut recycler qu’une sous-population de GL de petites tailles produites après lipolyse, qui se prêtent à l’internalisation lipophagique \citep{RN880}. Au cours de ce processus, les petites GL sont recouvertes par une bicouche membranaire pour former un autophagosome, qui sera acheminé vers les lysosomes, avec lesquels ils fusionnent pour former un autolysosome (\autoref{fig:fig9}\textcolor{blue}{.B}). La dégradation lysosomal des lipides neutres est effectuée par la lipase acide (LIPA) en raison de son activité optimale au pH lysosomal de 4,5 - 5. Elle présente une large spécificité de substrat, hydrolysant le TAG, le DAG et les esters de stérol \citep[pour revue,][]{RN574}. En revanche, on ne sait pas si LIPA hydrolyse le MAG ou si celui-ci est libéré pour une digestion successive par la MGL dans le cytoplasme. En parallèle, les phospholipases et les peptidases lysosomales dégradent les phospholipides et les protéines associées au noyau hydrophobe. Une étude récente a confirmé l’importance du mécanisme de lipophagie dans la dégradation des GL dans le tissu hépatique à partir de cultures primaires d’hépatocytes et de cellules en lignées \citep{RN573}. Compte tenu du rôle crucial de la lipolyse et de la lipophagie dans l’homéostasie des lipides, de nombreuses études valorisent le rôle protecteur de ces voies cataboliques dans les pathologies hépatiques. À ce titre, divers traitements connus pour induire les voies d’autophagie conventionnelles sont actuellement en cours d’étude en tant que pistes thérapeutiques pour le traitement de la stéatose.

	\begin{figureth}
	\centering
			\includegraphics[width=0.80\linewidth]{Figure_9.png}
		\caption[Mécanisme de consommation des gouttelettes lipidiques]{\textbf{Mécanisme de consommation des gouttelettes lipidiques.} (A) Représentation schématique du mécanisme de la lipolyse. (B) Représentation schématique du mécanisme de l’autophagie des gouttelettes lipidiques ou « lipophagie ». Adapté de \citealt{RN608}).}
				\label{fig:fig9}
	\end{figureth}
	\FloatBarrier

	\subsection{Interface entre les gouttelettes lipidiques et les autres organites}
	\label{section:interface}

 Au cours de leur cycle de vie, les GL communiquent fréquemment avec les autres organites intracellulaires tels que le RE, l’appareil de Golgi, les mitochondries, les lysosomes, les peroxysomes, ainsi qu’avec les membranes plasmique et nucléaire \autoref{fig:fig10} \citep[pour revues,][]{RN578,RN579}. Une étude récente par imagerie multi-spectrale a montré qu’environ 85\% des GL sont physiquement connectées avec le RE dans les cellules de mammifères, soulignant le rôle important de cet organite dans la biogénèse des GL et dans l’apport de la plupart de leurs molécules constitutives \citep{RN580}. Pour interagir avec les autres organites, les GL se déplacent à travers le réseau microtubulaire selon un mouvement bidirectionnel qui dépend du recrutement de protéines motrices. \\ \\
 \indent
 Les contacts membranaires homo- ou hétéro-typiques sont reconnus comme une voie majeure de trafic intracellulaire de molécules entre les différents compartiments cellulaires \citep[pour revue,][]{RN583,RN584}. Dans le contexte des GL, ces interactions permettent d’approvisionner les différents compartiments cellulaires en lipides et en protéines, stockés et transportés par les GL. Par exemple, le repositionnement spatial des GL vers les mitochondries pendant l’état de lipolyse, permet d’optimiser la canalisation des acides gras libérés dans la ß-oxydation. La base moléculaire de ces interactions reste aujourd’hui mal comprise, mais plusieurs études mettent en lumière le rôle de complexes d’ancrage protéiques dans l'établissement d’une connexion physique entre les GL et les autres organites \citep{RN586}. Premièrement, la monocouche des GL s’associe à la membrane externe de la bicouche des organistes partenaires par un mécanisme d’hémi-fusion. Ce processus serait facilité par un complexe formé par la GTPase facteur 1 d'ADP-ribosylation (ARF1) et le coatomère COPI (ARF1/COPI) \citep{RN588,RN589} qui provoque le détachement des phospholipides de la monocouche sous la forme de très petites gouttelettes lipidiques appelées « nano-GL » (\autoref{fig:fig8}). Cet appauvrissement en phospholipides diminue la couverture surfactante du noyau hydrophobe ce qui entraîne une augmentation de la tension superficielle et favorise conséquemment le pontage entre la monocouche des GL et la membrane externe de l’organite. De plus, de nombreux membres de la famille Rab, les GTPases qui contrôlent le trafic vésiculaire \citep[pour revue,][]{RN590}, sont retrouvées à la surface des GL et favoriseraient la formation et l’expansion du pont entre les GL et l’organite cible, en interagissant avec la machinerie SNARE de fusion membranaire eucaryote \citep{RN592}. Ce pont permet la diffusion mono- ou bi-directionnelle passive de lipides, de métabolites, de protéines et d’ions entre les lumières des deux organites, sans nécessiter de transfert actif médié par des protéines \citep[pour revue,][]{RN593}. Un autre mécanisme d’ancrage pourrait s’effectuer par la dimérisation homotypique de protéines localisées à la surface à la fois des GL et des autres organites. Par exemple, les protéines AUP1 et CIDE seraient responsables des contacts homotypiques entre GL et de leur clusterisation \citep{RN557,RN596}.

	\begin{figureth}
	\centering
			\includegraphics[width=\linewidth]{Figure_10.png}
		\caption[Interface entre les gouttelettes lipidiques et les organites intracellulaires]{\textbf{Interface entre les gouttelettes lipidiques et les organites intracellulaires.} (Adapté de \citealt{RN869}).}
				\label{fig:fig10}
	\end{figureth}
	\FloatBarrier

Si la majorité des GL sont localisées dans le cytosol, certaines GL dites « luminales » ont été observées dans la lumière du RE et à l’intérieur du compartiment nucléaire \citep{RN598}. Les GL nucléaires ont été principalement observées dans des cellules d'origine hépatique et se formeraient par des invaginations de l’enveloppe nucléaire \citep{RN600} ou par le reflux des GL luminales vers le nucléoplasme \citep{RN639}. Les GL nucléaires ont été proposées comme plateforme de régulation de l’expression des gènes mais leur rôle n’est pas complètement établi (voir \autoref{section:fonctions}).

	\subsection{Mécanisme de recrutement des protéines à la surface des gouttelettes lipidiques}
	\label{section:recrutement}
	
	Les protéines associées physiquement aux GL se répartissent en deux catégories selon leur mécanisme d’ancrage : les protéines membranaires qui diffusent à partir du RE vers les GL appartiennent à la classe I ; les protéines solubles directement recrutées à la surface des GL à partir du cytosol définissent la classe II \citep[pour revue,][]{RN602}. Il est probable que des protéines provenant des mitochondries, de l’appareil de Golgi, des peroxysomes ou des lysosomes soient également recrutées à la surface des GL, par le biais des sites de contact établis entre les GL et ces autres organites cellulaires. Les protéines sont intégrées dans la monocouche phospholipidique des GL par le biais de divers mécanismes et motifs de liaison. Jusqu'à présent, les hélices amphipathiques et les domaines hydrophobes ont été identifiés comme les motifs les plus courants d’association stable à la surface des GL. Les protéines peuvent également s’associer indirectement et de façon réversible à la périphérie des GL en se liant à des protéines intégrées dans la membrane.

	\subsubsection{Les protéines de classe I : ciblage depuis le réticulum endoplasmique}
	
Les protéines de classe I sont initialement insérées dans la membrane du RE et se répartissent à travers le feuillet externe vers la monocouche délimitante de la GL par diffusion latérale (\autoref{fig:fig11}). Cette redistribution est susceptible de se produire au moment du bourgeonnement des GL lors de la synthèse \textit{de novo} où à des stades ultérieurs lorsque les GL matures forment de nouveaux contacts physiques avec le RE. Les domaines hydrophobes des protéines de classe I forment généralement des motifs en épingles à cheveux, qui s’insèrent à mi-chemin dans les bicouches lipidiques, les extrémités amino- et carboxy- terminales face au cytosol. Cette conformation monotonique, dépourvue de boucles luminales, permet à ces protéines d’être logées à la fois dans la bicouche lipidique du RE et dans la monocouche des GL \citep{RN604}. Cette topologie a été clairement démontrée pour les oléosines, une protéine constitutive des  GL végétales \citep{RN605}. On trouve des protéines aux fonctions diverses parmi les protéines de classe I associées aux GL, telles que des enzymes de biosynthèse des lipides neutres comme ASCL3, AGPAT3, GPAT4 et DGAT2 et des protéines impliquées dans la protéolyse dépendante de l'ubiquitine comme AUP1. \\ \\
\indent
Les mécanismes qui conduisent à l’accumulation des protéines de classe I au niveau des gouttelettes lipidiques sont encore obscurs, mais plusieurs auteurs supposent que cette redistribution dépendrait de l’état métabolique par (i) un recrutement actif des populations de protéines destinées aux GL ou (ii) d’une structure particulière ne permettant qu'à des sous-populations adéquates de protéines de traverser la monocouche des GL \citep[pour revue,][]{RN603}. Les propriétés biophysiques de la bicouche du RE et de la monocouche des GL créent probablement une première barrière de sélection qui contrôle le type de protéines capables de se répartir entre ces membranes \citep{RN608}. Par exemple, les GL ne peuvent pas accueillir les protéines avec des régions luminales hydrophiles car l'exposition de ces domaines dans le noyau neutre hydrophobe est énergétiquement défavorable. Inversement, les bicouches sont perméables aux molécules d'eau et une protéine constituée d’un domaine strictement hydrophobe sera plus favorablement recrutée sur les GL qui offrent un environnement exclusivement constitué de lipides neutres. De plus, la longueur des régions hydrophobes peut être critique pour diriger la délocalisation des protéines vers la surface des GL. Les oléosines végétales possèdent un domaine hydrophobe d’une longueur de 72 acides aminés \citep{RN609}, plus long que l’épaisseur limitante de la bicouche lipidique du RE d’environ 3 nm. Leur intégration dans la bicouche peut provoquer un stress membranaire à cause des mésappariements hydrophobes et la délocalisation vers la membrane des GL pourrait être un état plus favorable. Pour finir, il existe des éléments de séquence retrouvées dans les protéines de classe I qui facilitent la délocalisation vers les GL, requis initialement pour l’insertion dans la membrane du RE. À titre d’exemple, la topologie monotopique est favorisée par la présence d’un noeud de proline au centre de la région hydrophobe. La mutation de ce noeud de proline dans la séquence hydrophobe de AUP1 ou dans celle des oléosines retient les protéines au sein du feuillet externe du RE et restreint leur partitionnement vers les GL.

	\subsubsection{Les protéines de classe II : Ciblage depuis le cytosol}

Les protéines de classe II sont traduites dans le cytosol et sont recrutées directement à la surface des GL (\autoref{fig:fig11}). Elles ont pour la plupart des hélices amphipathiques dépliées en solution qui se replient en hélice lorsqu'elles interagissent avec une membrane phospholipidique. Elles peuvent également s'insérer dans la monocouche phospholipidique par le biais d'ancres lipidiques qui modifient la nature des acides gras comme la GTPase Rab18. D’autres s’associent en périphérie par liaison avec une protéine déjà intégrée dans la monocouche lipidique, comme la lipase HSL qui interagit avec PLIN1 ou la protéine de capside du virus de la dengue qui est recrutée par PLIN3. \\ \\
\indent
La monocouche des GL peut accueillir une densité ~13\% moins riche en phospholipides que le feuillet externe des bicouches lipidiques et est généralement moins statique. Des défauts dans le positionnement des phospholipides peuvent alors apparaître à la surface des GL, ce qui provoque une exposition transitoire des lipides neutres hydrophobes à la phase aqueuse (\autoref{fig:fig11}\textcolor{blue}{, encadré}). L’insertion des protéines de classe II peut alors être déclenchée au niveau de l’espace disponible entre les phospholipides afin de réduire la tension de surface \citep{RN617}. Si une pénurie en phospholipides constitue la force motrice de la liaison des protéines de classe II, certains mécanismes de régulation existent pour éviter le recrutement non sélectif. Par exemple, l’encombrement stérique de la surface des GL par des protéines « gardiennes », telles que celles de la famille des périlipines constituerait une barrière pour empêcher le recrutement d’autres protéines cytosoliques \citep{RN564}. Elles sont particulièrement importantes pour prévenir d’une dégradation indésirable des GL en bloquant l’accès aux lipases \citep{RN620}. En cas de changement d’état métabolique, les périlipines sont ciblées et dégradées par la machinerie d’autophagie médiée par les chaperones afin de libérer de la surface disponible pour recruter d’autres protéines, comme les enzymes responsables de la lipolyse ou de l’expansion des GL \citep{RN567}. Contrairement aux adipocytes, PLIN1 n'est pas exprimée dans les hépatocytes et PLIN2 s'associe moins fortement avec les GL, rendant ainsi la surface des GL plus permissive dans ce type cellulaire.

	\begin{figureth}
	\centering
			\includegraphics[width=\linewidth]{Figure_11.png}
		\caption[Mécanismes de recrutement des protéines de classes I et II à la surface des gouttelettes lipidiques]{\textbf{Mécanismes de recrutement des protéines de classes I et II à la surface des gouttelettes lipidiques.} Les protéines de classe I sont initialement insérées dans la membrane du RE selon une topologie monotopique en épingle à cheveux et diffusent latéralement ou sont recrutées activement à la surface des GL (vert). Les protéines de classe II sont recrutées à la surface GL à partir du cytosol en s’associant directement avec la monocouche phospholipidique via des hélices amphipathiques ou des ancres lipidiques, ou par des interactions protéine-protéine (bleu). (Adapté de \citealt{RN603}).}
				\label{fig:fig11}
	\end{figureth}
	\FloatBarrier
	
	\subsection{Fonctions des gouttelettes lipidiques}
	\label{section:fonctions}
	
Les GL sont des organites multi-tâches qui remplissent de nombreux rôles selon le type cellulaire et l’état physiologique de la cellule. Leur fonction primaire est de stocker les lipides lors des conditions de surplus de nutriments et de les re-mobiliser pour la production d'énergie ou pour la synthèse des phospholipides membranaires. Au-delà de leur rôle dans l’homéostasie énergétique, les GL assurent également des fonctions liées à la réponse généralisée face au stress lipotoxique ou oxydatif et servent de plateformes de repliement moléculaire \citep[pour revues,][]{RN625,RN645}. Les différentes fonctions communes aux GL seront décrites dans les paragraphes suivants et sont illustrées sur le schéma récapitulatif de la \Autoref{fig:fig12}. Le dysfonctionnement des gouttelettes lipidiques est lié au développement de nombreuses pathologies, telles que le diabète de type 2, l’insulino-résistance, l’athérosclérose et la stéatose hépatique \citep[pour revues,][]{RN626,RN627}.

	\subsubsection{Stockage des lipides et autres molécules lipophiliques}

Les GL sont responsables du stockage des acides gras en excès sous la forme de triacylglycérides ou d’esters de stérol. En cas de privation de nutriments, les acides gras stockés sont mobilisés par lipolyse ou lipophagie afin d’alimenter les processus métaboliques comme la ß-oxydation, la synthèse des lipoprotéines et autres médiateurs lipidiques. Lors de la croissance cellulaire, les acides gras stockés seront fortement mobilisés pour synthétiser les phospholipides nécessaires à l’extension des membranes plasmique ou nucléaire. À l’échelle de l’organisme, la sécrétion des lipoprotéines à partir des tissus adipeux ou hépatiques permet d’alimenter les autres tissus en substrat énergétique en cas d’insuffisance de nutriments. En fonction de l’organisme et du tissu, les gouttelettes lipidiques peuvent stocker d’autres molécules lipophiliques impliquées dans des fonctions diverses qui ne sont pas directement liées à la synthèse d’énergie, telles que des vitamines liposolubles, des précurseurs d’hormones stéroïdiennes ou des médiateurs de l’inflammation.
	
	\subsubsection{Neutralisation des composés lipotoxiques}

Malgré leur rôle essentiel dans les processus biologiques, une concentration élevée en acides gras  libres est toxique en raison de leur solubilité limitée et de leur potentiel de transformation en espèces bioactives hautement cytotoxiques qui peuvent entraîner la mort cellulaire \citep{RN629}. En effet, une augmentation locale de stérols libres perturbe l’intégrité et la rigidité des membranes cellulaires et peut directement agir comme détergent en induisant une lyse de la membrane. La conversion des stérols et des acides gras libres sous la forme relativement inerte et stable d’esters et de triacylglycérides et leur incorporation dans les GL protège les cellules des effets toxiques liés à une quantité excessive en lipides, plus communément définis par « lipotoxicité ». En parallèle des acide gras endogènes, les GL peuvent séquestrer une grande variété de molécules cytotoxiques présentes dans l’environnement extérieur. Les polluants exogènes étant généralement lipophiles (dérivés d’hydrocarbures), ils ont une forte pré-disposition à diffuser au travers des membranes cellulaires, à la manière des acides gras libres. L’accumulation de polluants exogènes dans les GL renforce l’idée que ces organites ont un rôle essentiel dans la préservation des autres organites cellulaires. La liste des espèces lipidiques toxiques séquestrées au sein des GL s’est récemment allongée, avec l’identification des polyprénols tels que les dolichols. Une concentration élevée de polluants exogènes dans les tissus adipeux des mammifères et des poissons est à présent reconnue comme un signe d’exposition chronique à des contaminants écologiques. Toutefois, les propriétés de séquestration des GL ne sont pas toujours bénéfiques : elles peuvent réduire l’efficacité thérapeutique, comme cela a été montré dans le contexte de la chimiothérapie avec le docétaxel, un médicament utilisé pour traiter le cancer du sein \citep{RN632}.

	\subsubsection{Réponse au stress oxydatif et à l’hypoxie}

Lors de l’utilisation des lipides comme substrat pour produire de l’adénosine triphosphate (ATP) par ß-oxydation, des dérivés réactifs de l’oxygène (ROS) sont générés le long de la chaîne de transport d’électrons de la mitochondrie. Les acides gras qui contiennent des chaînes polyinsaturées sont particulièrement sensibles à la peroxydation induite par une production excessive de ROS. La redistribution des acides gras polyinsaturés au sein des GL empêche leur incorporation dans les phospholipides, ce qui protège les parois membranaires des dommages oxydatifs. L’hypoxie favorise la lipogénèse en modulant l’expression des gènes impliqués dans l’absorption et la synthèse des acides gras. Afin de réduire l’effet lipotoxique lié à l’accumulation excessive de lipides sous condition hypoxique, la formation des GL est accrue.

	\subsubsection{Plateforme de stockage et de repliement des protéines}

La réponse protéique non repliée (UPR) a été initialement définie comme une réponse cellulaire adaptative qui s’active en cas de surcharge du RE dans la capacité de repliement des protéines \citep{RN635}. Les GL sont essentielles au maintien de l’homéostasie du RE en participant à la clairance des protéines aggrégées ou mal repliées qui s’accumulent dans la bicouche lipidique. Lors de leur formation, les GL servent alors de dépôt pour les protéines avant leur dégradation par la voie de dégradation des protéines associées au RE (ERAD). En lien avec la fonction de stockage des GL, il a été proposé que ces organites servent également au dépôt temporaire de protéines pour faire face à des situations de besoins aigus.

	\subsubsection{Régulation des gènes et entretien de l’ADN}

Dans les embryons de drosophile, les GL sont associées à des quantités importantes d'histones et peuvent être transférées vers les noyaux, ce qui favoriserait la structuration de la chromatine \citep{RN638}. Il a été également proposé que les GL nucléaires servent de plateforme de régulation de l’expression des gènes en transportant des facteurs de transcription. Par exemple, une étude a récemment mis en évidence une synthèse et une mobilisation de GL nucléaires transportant la protéine CCTα lors d’une augmentation importante en acides gras libres, ce qui active la synthèse de la PC selon un mécanisme rétro-actif \citep{RN639}.

	\begin{figureth}
	\centering
			\includegraphics[width=0.75\linewidth]{Figure_12.png}
		\caption[Diversité des fonctions communes aux gouttelettes lipidiques]{\textbf{Diversité des fonctions communes aux gouttelettes lipidiques.} Les fonctions essentielles des GL être classifiées en trois domaines qui concernent (i) l’homéostasie énergétique, où les GL servent de réservoirs pour diverses espèces lipidiques (ii) le stockage ou la dégradation des protéines et (iii) la réponse au stress cellulaire où les GL ont un rôle de protection contre les ROS et les composés lipotoxiques (Adapté de \citealt{RN579}).}
				\label{fig:fig12}
	\end{figureth}
	\FloatBarrier
	
	\subsubsection{Vers de nouvelles fonctions ?}
	
Récemment, le protéome des GL d’une lignée hépatocytaire a été établi \citep{RN508,RN641}. Il est dominé par des enzymes impliquées dans la régulation du métabolisme énergétique. Toutefois, on retrouve des protéines impliquées dans d’autres fonctions biologiques, comme le traffic vésiculaire, l’organisation du cytosquelette, la signalisation cellulaire, l’autophagie, les régulations transcriptionnelle et post-traductionnelle, renseignant sur de nombreuses fonctions potentiellement assurées par les GL au sein du tissu hépatique (\autoref{fig:fig13}).

	\begin{figureth}
	\centering
			\includegraphics[width=\linewidth]{Figure_13.png}
		\caption[Cartographie du protéome des gouttelettes lipidiques hépatocytaires]{\textbf{Cartographie du protéome des gouttelettes lipidiques hépatocytaires.} Les protéines associées aux GL sont regroupées en modules fonctionnels sur la base de l’analyse GO et des annotations fonctionnelles UNIPROT. Les lignes pleines représentent les interactions physiques au sein des modules fonctionnels et les lignes transparentes représentent les interactions entre les protéines de modules distincts. L’intensité de la couleur bleue au sein d’un noeud indique le degré de confiance avec lequel la protéine fait partie du module. Les facteurs entourés en rouge représentent les protéines qui ont été précédemment validées par études de co-localisation (Adapté de \citealt{RN641}).}
				\label{fig:fig13}
	\end{figureth}
	\FloatBarrier
	
	\subsection{Interface entre les gouttelettes lipidiques et les pathogènes humains}
	
L'implication des GL dans le cycle infectieux des virus a été documenté pour la première fois en 2007 avec le VHC \citep{RN488} où elles sont mobilisées pour le stockage de la protéine de capside et pour l’assemblage des particules virales (voir \autoref{section:cycle}). D'autres espèces virales, principalement des virus à ARN en raison de leur réplication cytoplasmique, dépendent également des GL au cours de leur cycle : le virus de la dengue (DENV), le rotavirus (RV), l’orthoréovirus et le virus de l’hépatite B (VHB) \citep[pour revue,][]{RN643} et plus récemment le coronavirus du syndrome respiratoire aigu sévère (SARS-CoV2) \citep{RN867}.

\clearpage

%%%%%%%%%%%%%%%%%%%%%%%%%%%%%%%%%%%%%%%%%%%%%%%%%%%%%%%%%%%%%%%%%%%%%%%%%%%%%%%%%%%%%%%%%%%%%%%%%%%%%%%%%%%%%%%%%%%%%%%%%%%%%%%%%%%%%%%%%%%%%%%%%%%%%%%%%%%%%%%%%%%%

% 4 - Généralités sur le VHC

\section{Généralités sur le virus de l'hépatite C}	

	\subsection{Identification et histoire naturelle}

L’identification du virus de l'hépatite C (VHC) n’a pas été une mince affaire et les très nombreuses tentatives utilisant les méthodes conventionnelles apparurent extraordinairement laborieuses et décevantes pendant les 15 premières années de recherche à son sujet \citep[pour revue,][]{RN127}. Le domaine d’étude des hépatites virales a débuté dans les années 1960 avec la distinction de deux hépatites virales majeures sur les plans clinique et immunologique, plus tard prouvées comme étant attribuables à l’infection par le virus de l'hépatite A (VHA) \citep{RN96} et par le virus de l'hépatite B (VHB) \citep{RN88}. Avec l’avancée des tests sérologiques, il a été montré que la plupart des cas de transmission d’hépatites par voie parentérale n’était pas due à ces virus, mais à un agent étiologique inconnu. La terminologie « hépatites non A, non B » fut alors introduite pour désigner ces hépatites, dont les agents responsables n’étaient pas identifiés mais qui apparaissaient sérologiquement distinctes des hépatites A et B. Les chercheurs se sont rapidement mobilisés afin d’identifier ce nouvel agent pathogène en utilisant les méthodes connues de l’époque (ultra-filtration, immunodiffusion, dosage immuno-enzymatique (ELISA), dosage radio-immunologique (RIA) mais sans succès malgré des années de travaux intensifs. L’émergence des approches moléculaires dans les années 1980 n’a pas non plus permis de caractériser l’agent pathogène, mais a permis de mettre en évidence sa proximité avec des familles virales connues (\textit{Flavirividae}, \textit{Togaviridae}, \textit{Hepadnaviridae} et \textit{Picornaviridae}) puis avec le virus de l'hépatite Delta (VHD) découvert en 1977 \citep{RN113}. Des essais pour cultiver l’agent pathogène \textit{in vitro} dans différents systèmes cellulaires ont également été menés mais jamais aucun cycle viral productif n’a pu s’établir (cet « exploit » n’a été accompli qu’en 2005 par l’équipe de T. Wakita à partir d’une souche dérivé d’un patient japonais \citep{RN124}). En parallèle, les essais de visualisation par microscopie électronique de l’éventuelle progénie virale circulant dans le sérum des patients n’ont pas été particulièrement instructifs. Les premières particules virales infectieuses du VHC dans le sérum n’ont été détectées que très récemment \citep{RN125} (voir \autoref{section:organisation}). \\ \\
\indent
À la fin des années 1970, le développement de l’immuno-criblage permettait une détection précise de la présence d’antigènes, mais l’efficacité de cette technique dépend grandement de la qualité et de la spécificité des anticorps employés. Dans le cas du VHC, aucun anticorps ciblant les composants viraux n’ont pu être purifiés à partir des sérums de patients malgré les tentatives répétées par de nombreux laboratoires de recherche \citep{RN116}, ce qui limitait alors grandement le recours à des méthodes basées sur des anticorps dans le cadre de l’identification de cet agent pathogène. En 1989, l’équipe du Dr. Houghton a décidé de recourir à une méthode particulièrement hasardeuse d’immuno-criblage en partant directement du sérum d’un patient souffrant d’hépatite chronique en guise de source d’anticorps anti-viraux. En utilisant ce sérum contre une librairie d’acide désoxyribonucléique complémentaire (ADNc) générée de manière aléatoire à partir du plasma de chimpanzés infectés, les scientifiques sont parvenus à mettre en évidence 6 clones positifs, malgré les difficultés techniques rencontrées et la quantité importante de contaminants. Un des clones positifs contenait un acide ribonucléique (ARN) simple brin de polarité positive et d’une longueur d’environ 10.000 nucléotides qui ne correspondait pas à un gène cellulaire et qui était spécifique des animaux infectés. Les tests qui suivirent le clonage et l’étude de cet ARN confirmaient son caractère immunogène et pathogène. Il fut ainsi caractérisé comme l’agent étiologique majeur des hépatites transmises par voie parentérale et nommé le VHC \citep{RN91}. La séquence du VHC demeurait toutefois incomplète jusqu’à la découverte de la région 3’X en 1996, un élément hautement conservé situé à l’extrémité du génome viral et indispensable à sa réplication autonome \citep{RN249}. Le prix Nobel de Médecine 2020 a été attribué aux Drs. Harvey J. Alter, Michael Houghton et Charles M. Rice pour leur contribution décisive à la découverte et à la caractérisation du VHC. Le comité Nobel a déclaré que ces trois scientifiques avaient « rendu possibles des méthodes de diagnostic sanguin et des solutions thérapeutiques qui ont permis de sauver des millions de vies ».

	\subsection{Classification et variabilité génétique}
	\label{section:classification}
	
La séquence du génome du VHC fut rapidement décrite et analysée suite à son identification et le virus fut classé dans la famille des \textit{Flaviviridae} puisqu’il présente une organisation génomique proche de celles des Pestivirus et des Flavivirus, deux genres appartenant à cette famille virale \citep{RN92}. Néanmoins, en raison de la faible homologie de séquences nucléotidique et protéique entre le VHC et les autres espèces virales connues, il devint le représentant prototype d’un nouveau genre, les Hepacivirus \citep{RN114}. Depuis, d’autres espèces virales infectant une grande variété d’hôtes ont été décrites et attribuées au genre Hepacivirus, c’est le cas du GB virus B (GBV-B) infectant les singes du Nouveau Monde \citep{RN119} et plus récemment des espèces infectant les rongeurs, chauve-souris, oiseaux, bovins, requins et chevaux \citep[pour revue,][]{RN100}. Un quatrième genre Pegivirus fut introduit à la famille des \textit{Flaviviridae} suite à la découverte de nouveaux virus hépatotropes humains, le GB virus A (GBV-A), GB virus C (GBV-C) et le GB virus D (GBV-D) relativement éloignés des Hepacivirus sur le plan phylogénétique \citep{RN121}.

	\begin{figureth}
	\centering
			\includegraphics[width=0.80\linewidth]{Figure_14.png}
		\caption[Arbre phylogénétique des membres représentatifs des 4 genres de la famille des \textit{Flaviviridae}]{\textbf{Arbre phylogénétique des membres représentatifs des 4 genres de la famille des \textit{Flaviviridae}.} L'analyse est basée sur la comparaison des séquences codant l'hélicase NS3. La longueur des branches est proportionnelle au nombre de substitutions par site. (D'après \citealt{RN100}).}
				\label{fig:fig14}
	\end{figureth}
		\FloatBarrier

Les \textit{Flaviviridae} sont des virus à ARN simple brin de polarité positive, ils ont par conséquent une dynamique d’évolution très forte et circulent sous la forme d’une distribution complexe de variants génétiques. Au sein d’une même espèce virale, ces variants peuvent être relativement éloignés sur le plan phylogénétique : c’est le fruit d’une longue évolution par la génération et le maintien continu des variants les plus adaptés à un environnement et à une population donnée au fil des années. La dynamique des populations de virus à ARN peut également s’observer à l’échelle de l’individu : la réplication rapide favorise l’apparition de mutations au cours d’une infection chronique ce qui aboutit à la formation d’un ensemble de variants génétiquement proches communément définit par l’appellation « quasi-espèces ». La mutagénèse spontanée des virus à ARN pose un grave problème de santé publique, en premier lieu sur le plan clinique en raison de l’émergence de mutations de résistance sous la pression de sélection exercée par les molécules antivirales, qui constitue la principale raison des échecs thérapeutiques (voir \autoref{section:elimination}), mais également sur le plan épidémiologique puisque les \textit{Flaviviridae} sont responsables d’épidémies émergentes ou ré-émergentes contemporaines.

	\subsubsection{Variabilité inter-individus : Génotypes et sous-types viraux}
	
Peu après la publication de la première séquence complète du génome du VHC, la forte hétérogénéité génétique entre les isolats cliniques étudiés par les laboratoires du monde entier fut rapidement remarquée. Face à l’accumulation de séquences et de terminologies diverses dans la littérature, l’élaboration d’un système universel pour classer les variants du VHC devint une priorité afin de simplifier les futures études épidémiologiques et phylogénétiques. Une première nomenclature uniforme fut proposée en 1994, reposant sur les similarités de séquence en nucléotides : les séquences analogues entre les variants furent regroupées sous le terme de « génotypes » et les groupes de séquences les plus proches au sein de ces génotypes furent désignés sous le terme de « sous-types » \citep{RN117}. Les génotypes et sous-types sont attribués aux séquences par ordre de découverte chronologique, par exemple le premier clone mis en évidence par Choo et ses collègues en 1989 appartient au génotype 1 et sous-type a. L’appartenance à un génotype et à un sous-type est déterminée par l’analyse de petites régions peu variables au sein du génome, souvent dans les extrémités 5' non codante (5'NC) et 3' non codante (3'NC) ou dans les gènes codant la protéine Core ou la protéine non structurale 5B (NS5B). Les séquences nucléotidiques des souches appartenant à des génotypes hétérologues diffèrent de 30 à 35\% et de plus de 15\% entre sous-types \citep{RN120}. En mai 2019, les bases de données publiques comptabilisent plus de 1300 séquences complètes du génome du VHC, classées en 8 génotypes et 90 sous-types reconnus \citep{RN267}. Malgré l’élargissement considérable des données relatives à la diversité du VHC, la classification basée sur les génotypes et sous-types reste très robuste aujourd’hui, bien qu’elle ne référence pas encore les rares recombinants naturels (voir paragraphe \nameref{section:recombinants}). Tous les génotypes sont responsables de maladies hépatiques, toutefois des études cliniques ont pu mettre en évidence des disparités dans la progression et la sévérité de la pathologie, ainsi que dans la sensibilité aux traitements selon l’origine génotypique de la souche infectante. En effet, les infections chroniques par les souches de génotype 3 seraient liées à une prévalence plus élevée de stéatose hépatique et par conséquent, à un risque accru de progression vers des complications hépatiques graves \citep{RN306,RN360} tandis que les infections chroniques par les souches de génotype 1b seraient associées à un risque plus important de CHC \citep{RN1072,RN1073}. Ces découvertes indiquent l’importance de coupler aux études cliniques des analyses plus fondamentales pour identifier des signatures moléculaires propres à certains génotypes, voire directement des résidus « à risque » au sein des séquences virales, et contribuer ainsi à la mise en place d’une cartographie globale permettant de prédire l’évolution de la pathologie hépatique chez les individus chroniquement infectés.

	\begin{figureth}
	\centering
			\includegraphics[width=0.80\linewidth]{Figure_15.png}
		\caption[Classification du virus de l'hépatite C en 8 génotypes majeurs et 90 sous-types]{\textbf{Classification du virus de l'hépatite C en 7 génotypes majeurs et 90 sous-types.} L'arbre phylogénétique repose sur l'analyse comparative des séquences des cadres de lecture ouverts des différentes souches du VHC. L'isolat unique de génotype 8 identifié en 2019 est manquant sur la figure. (D'après \citealt{RN1233}).}
				\label{fig:fig15}
	\end{figureth}
		\FloatBarrier
		
		\subsubsection{Variabilité intra-individus : Quasi-espèces}
		
La notion de « quasi-espèces » a été introduite par Eigen et Schuster en 1977 \citep{RN95} dans le contexte de leurs travaux sur la diversité des acides nucléiques née des erreurs de leur réplication autonome, dans le but d’établir un modèle mathématique théorisant le processus d’évolution des formes de vie primitives. Ce concept a rapidement été adopté en virologie avec la première étude sur le sujet démontrant que la progénie virale du bactériophage Qß n’était pas génétiquement homogène \citep{RN93}. Ce postulat constitue le premier témoignage expérimental du comportement des virus à ARN sous la forme de quasi-espèces. Depuis, cette caractéristique a été démontrée pour les représentants majeurs de groupes de virus à ARN humains, animaux ou végétaux en culture et \textit{in vivo}, parmi lesquels on retrouve le virus de l'immunodéficience humaine 1 (VIH-1), le virus de la grippe et le VHC \citep{RN106}. Les variants qui composent une quasi-espèce ont une séquence génomique qui ne diffère généralement pas plus de 1 à 3\%. L’existence de ces variants est le résultat de l’absence d’activité de relecture de la réplicase de ces virus, induisant une accumulation de mutations au fur et à mesure des cycles réplicatifs. \\ \\
\indent
Chez le VHC, le taux de mutation par cycle réplicatif est estimé à 3,5x10\up{-5} \citep{RN97}. Ces mutations apparaissent dans l’intégralité du génome du VHC bien qu’il existe des régions plus ou moins sujettes à une variabilité intrinsèque. En effet, les séquences du génome qui codent des fonctions virales essentielles (telles que la traduction et la réplication) ou à des domaines structuraux majeurs (les extrémités 3’NC et 5’NC et la capside) sont généralement plus conservées. Cela provient certainement du fait qu’une mutation dans ces domaines plus « sensibles » pourrait plus fréquemment aboutir à la production d’un génome déficient. À l’inverse, des régions considérées comme hypervariables peuvent atteindre des taux de mutation de l’ordre de 1x10\up{-3}, c’est le cas par exemple de certaines séquences codantes de E2 \citep{RN97}. Ainsi, la forte hétérogénéité génétique du VHC à l’échelle de l’individu résulte d’une combinaison de deux facteurs : l’infidélité de la réplicase virale couplée à un processus d’infection très dynamique avec des cycles de production et de clairance des particules virales estimés à 10\up{12} virions par jour et une demi-vie des virions de l’ordre de quelques heures \citep{RN109}. Au cours de l’infection chronique, l’ensemble des variants produits sont théoriquement neutres en terme de valeur sélective et évoluent autour d’une séquence moyenne stable dite « consensus ». Toutefois, sous une forte pression de sélection, la nature de la souche circulante dominante peut changer à tout instant pour une souche mineure pré-existante plus adaptée au nouvel environnement. Une conséquence de ce phénomène d’hétérogénéité permettrait ainsi au VHC d’échapper aux acteurs du système immunitaire, en particulier aux anticorps neutralisants \citep[pour revue,][]{RN87}, mais également de résister aux traitements antiviraux \citep[pour revue,][]{RN218} puisqu’une mutation de résistance pourrait pré-exister au sein de la population virale avant la prise en charge des patients sous thérapie. En plus de représenter un challenge majeur pour la vaccination, ce phénomène a des implications thérapeutiques importantes puisque l’efficacité des traitements antiviraux reposerait à la fois sur une utilisation précoce avant que la population virale ne se diversifie et sur un ciblage plus global de la distribution des variants plutôt que sur une entité génomique individuelle.

		\subsubsection{Les recombinants naturels}
		\label{section:recombinants}
		
La recombinaison désigne le mécanisme par lequel un génome comportant des portions chimériques d'origine phylogénétique distincte est formé. Des évènements de recombinaison ont été identifiés parmi les virus à ARN de polarités négative et positive, notamment chez plusieurs membres de la famille des \textit{Flaviviridae} du genre Flavivirus \citep{RN101}, Pestivirus \citep{RN89} et Hepacivirus \citep{RN123}. Ce processus joue un rôle important dans l’évolution des virus à ARN en générant de la variabilité génétique lors d’évènements d’échanges de séquences nucléotidiques entre des génomes hétérologues. Lors d’une réplication concomitante des deux génomes, la polymérase passe d'une matrice ARN à l’autre, tout en conservant le cadre de lecture, ce qui aboutit à la formation de brins hybrides complémentaires. Le mécanisme exact de l'échange de brin n'est pas connu mais pourrait être facilité par les temps de «  pause » de la réplicase lors de l'élongation du brin et la présence de structures secondaires pouvant déstabiliser la liaison à l’ARN \citep{RN94}. Dans le cadre du VHC, il a été montré que les cellules infectées deviennent réfractaires à une infection consécutive limitant fortement la possibilité d’avoir deux virus dans la même cellule \citep{RN141}. Toutefois, la recombinaison apparaît comme un processus important puisqu’elle serait présente chez 18\% des patients chroniquement infectés \citep{RN115}. Il est d’ailleurs très probable que le taux de recombinaison soit sous-estimé au sein des quasi-espèces virales, étant donné que la détection d’une discordance dans l’alignement phylogénétique du génome résultant repose sur le fait que les deux virus parentaux doivent différer suffisamment dans leurs séquences nucléotidiques. \\ \\
\indent
Ce phénomène serait moins fréquent pour les recombinants inter- et intra-génotypiques car il requiert que l’individu soit co-infecté par au moins deux souches de génotypes ou sous-types distincts. La prévalence des co-infections varie entre 1 et 20\% selon le groupe de populations \citep{RN147,RN99}. Les co-infections seraient favorisées par une activité insuffisante du système immunitaire, qui ne parviendrait pas à empêcher les évènements de réinfection. À ce jour, 14 recombinants naturels du VHC ont été identifiés dans la population mais un seul circule activement, notamment en Irlande \citep{RN148}, en Uzbekistan \citep{RN151}, en France \citep{RN154} et à Chypre \citep{RN155}. Il s’agit d’un virus recombinant intergénotypique codant les gènes structuraux d’une souche de sous-type 2k et les gènes non structuraux d’une souche de sous-type 1b, identifié pour la première fois à St Petersburg en 2002 \citep{RN102}. Des séquences appartenant au génotype 2 sont présentes dans la majorité des recombinants naturels décrits jusqu'à présent ce qui pourrait suggérer un rôle critique d’éléments propres au génotype 2 dans la stabilité et la fonctionnalité du génome recombinant résultant. Pour l’heure, il n’existe pas de méthode pour classifier les formes recombinantes du VHC qui nécessitent un séquençage complet du génome afin de définir les jonctions entre les génotypes et les sous-types \citep{RN126}. Toutefois, certains points « chauds » de recombinaison ont été identifiés dans les gènes codant NS2 et NS3 ainsi qu’à la jonction NS2/NS3 \citep{RN98}. Cet échange d’information génétique est un mécanisme clé pour la production de nouveaux génomes conférant des avantages sélectifs. La recombinaison peut agir comme un catalyseur de la résistance antivirale en combinant plusieurs mutations de résistance issues des virus parentaux. Ainsi, l’étude des mécanismes et des conséquences de la recombinaison est importante en raison de son potentiel à produire de nouvelles souches hybrides qui pourraient avoir de nouvelles propriétés pathogéniques, en particulier dans le cadre de la conception de vaccins vivants atténués multivalents \citep[pour revue,][]{RN268}.

	\subsection{Épidémiologie, prévalence et facteurs de risque}
			\label{section:epidemiologie}
	
	D’après les données épidémiologiques de l’Organisation Mondiale de la Santé (OMS), la séroprévalence globale du VHC est estimée à 1,6\% soit 115 millions d’individus. Il est à noter que ces chiffres prennent en compte les infections résolues spontanément et les succès thérapeutiques. Le nombre actuel d’individus chroniquement infectés est quant à lui estimé à 58 millions en 2021, ce qui correspond à une prévalence inférieure à 1\%. L’absence de données pour un nombre important de pays, en particulier les pays à faible revenus (où seulement 29\% d’entre eux renseignent des données), contraint à extrapoler la prévalence du VHC et la fréquence génotypique au niveau mondial à partir d’études cliniques et épidémiologiques publiées après 2013, restreintes à 100 pays. A l’échelle du globe, la prévalence du VHC est très hétérogène, avec 5 pays comptabilisant à eux seuls la moitié des cas totaux d’infections : la Chine, le Pakistan, l’Inde, l’Egypte et la Russie. La forte prévalence dans ces pays est attribuable à une transmission iatrogène du virus favorisée par l’utilisation de dispositifs médicaux contaminés, avant la découverte de l’agent pathogène et la mise en place des premiers diagnostics dans les années 1990. La source d’infection par le VHC a notamment bien été documentée en Egypte où lors d’une campagne anti-parasitaire massive dans les années 1960-1970 qui consistait en une succession d’injections intraveineuses avec du matériel n’ayant pas été systématiquement stérilisé a déclenché la dissémination du virus au sein de la population \citep{RN86}. Depuis les années 1990, la majorité des infections par le VHC survient chez les usagers de drogues par voie intraveineuse, une pratique qui constitue le facteur de risque principal dans les pays occidentaux (concerne 50 à 60\% des infections aigües) \citep{RN112}. 
	
	\begin{figureth}
	\centering
			\includegraphics[width=\linewidth]{Figure_16.png}
		\caption[Prévalence relative des génotypes 1 à 7 du virus de l’hépatite C à l’échelle mondiale]{\textbf{Prévalence relative des génotypes 1 à 7 du virus de l’hépatite C à l’échelle mondiale.} La répartition des différents génotypes du VHC déterminée dans 21 régions (codées par couleurs) est représentée sous la forme de diagrammes circulaires dont le diamètre est proportionnel au nombre d’individus infectés par région (Adapté de \citealt{RN111}).}
		\label{fig:fig16}
	\end{figureth}
		\FloatBarrier

Les génotypes 1 et 3 sont les plus prévalents au niveau mondial : ils comptabilisent respectivement 46\% et 22\% des cas totaux d’infection par le VHC \citep{RN111}. Leur dissémination massive aurait été engendrée par la distribution à l’échelle mondiale de produits sanguins contaminés par les sous-types 1a et 1b \citep{RN103} et par la forte association du sous-type 3a avec l’utilisation de drogues injectables \citep{RN107}. Les génotypes 2, 4 et 6 sont responsables d’une plus faible proportion d’infections, avec des taux respectifs de 13\%, 13\% et 2\% et le génotype 5 plus rare, a un taux d’infection inférieur à 1\%. Le génotype 7 n’a été retrouvé jusqu’à présent que dans quelques cas d’infection chez des patients originaires de la République Démocratique du Congo \citep{RN108,RN270}. Récemment, une équipe du Canada a isolé un nouveau génotype 8 à partir de patients infectés originaires de l’état de Punjab en Inde \citep{RN272}. La distribution des génotypes du VHC varie considérablement selon les zones géographiques, à l’exception du génotype 1 omniprésent à la surface du globe. Des souches considérés comme « endémiques » circulent depuis plusieurs siècles dans des secteurs plus restreints : le génotype 2 est principalement retrouvé en Afrique de l’Ouest, le génotype 3 en Asie du Sud, le génotype 4 en Afrique Centrale et au Moyen-Orient et le génotype 6 en Asie du Sud-Est (\autoref{fig:fig14}). L’expansion de la diversité génétique du VHC entre les continents aurait été influencée par les migrations de populations humaines au cours de l’histoire et plus particulièrement entre les zones ayant entretenu des fortes relations humaines, favorisant ainsi le transfert de variants. Une des périodes particulièrement propice à ces transferts fut lors des années 1700 à 1850 à l’apogée de l’expansion coloniale, où le commerce d’esclaves trans-atlantiques a été le moteur de la diffusion de souches de génotype 2 originaires d’Afrique de l’Ouest vers des zones plus éloignées, parfois sur un autre continent, comme en Amérique du Nord et en Europe \citep{RN105}. De la même manière, les migrations de populations venant des pays où le génotype 3 est dominant comme l’Inde ou le Pakistan a largement permis sa dissémination à la surface du globe \citep{RN122}. Globalement, le nombre total de personnes chroniquement infectées par le VHC décroît depuis 2007 bien qu’il y ait des variations majeures entre les zones géographiques. Ce phénomène s’explique par un taux de mortalité lié aux maladies du foie plus important que les nouvelles infections dans la plupart des pays, estimées à 1,5 millions de nouveaux cas en 2019. 

	\subsection{Élimination du virus de l'hépatite C en tant que problème de santé publique}
	\label{section:elimination}
	
Le récent développement et l’implémentation en clinique des agents antiviraux à action directe (AAD) constituent une véritable révolution thérapeutique pour l’hépatite C. Dans ce contexte, l’OMS a fixé l’objectif d’éliminer l’hépatite C en tant que problème de santé publique d’ici 2030, c’est-à-dire, réduire l’incidence des nouvelles infections de 90\% et la mortalité liée au VHC de 65\%. Atteindre cet objectif ambitieux repose sur la mise en place de politiques de santé et de plans d’action à l’échelle locale et nationale ainsi que la coopération de tous les acteurs de la santé publique, les chercheurs, les cliniciens et les entreprises pharmaceutiques. De nombreux pays occidentaux ne devraient pas y parvenir avant 2050 et cela paraît irréalisable pour les pays à faible et moyens revenus, souvent les plus touchés par l’hépatite C, qui commencent tout juste à aborder la question \citep[pour revue,][]{RN156}. Dans ce contexte, les défis majeurs sont : (i) l’identification et la prise en charge systématique de tous les porteurs sains du VHC, (ii) l’accès au dépistage et aux thérapies antivirales dans les pays aux ressources limitées, et (iii) le développement de nouvelles options thérapeutiques afin d’assurer un taux de guérison global de 100\% pour tous les génotypes et sous-types viraux \citep[pour revue,][]{RN128}. Combiner l’ensemble de ces efforts devraient aboutir à une réduction drastique de la problématique de l’hépatite C au niveau mondial, mais la question de l’éradication du VHC reste difficilement concevable sans le développement d’un vaccin préventif.

	\subsubsection{Dépistage, diagnostic et outils de surveillance}

Les approches traditionnelles de dépistage et de diagnostic du VHC reposent sur des ELISA visant à documenter l’exposition par détection de la présence d’anticorps dans le sérum ou le plasma des patients. Les tests sérologiques ne conviennent pas pour détecter les infections précoces et reposent sur une séroconversion qui intervient généralement 6 à 12 semaines après l'exposition \citep{RN157}. Ils ne permettent également pas de témoigner d’une infection active puisque les anticorps peuvent persister plusieurs années voir décennies chez les patients ayant résolus l’infection \citep{RN161}. En cas de dépistage sérologique positif, la détection de l’ARN viral par réaction en chaîne de la polymérase (PCR) en temps réel ou des antigènes viraux dans le sang sont donc indispensables pour repérer une réplication active du virus et identifier les individus nécessitant une prise en charge thérapeutique \citep{RN162}. Le suivi du niveau d’ARN ou des antigènes viraux ne permet pas de prédire l’évolution et l’issue de la pathologie mais peut servir d’indicateur de la réponse au traitement. La guérison se traduit par un niveau indétectable de l’ARN viral 12 semaines après la fin du traitement \citep{RN163}. Ces procédures s’effectuent majoritairement dans des laboratoires médicaux et autres infrastructures de santé, généralement chez les patients souffrant déjà d’atteintes hépatiques graves, qui ne constituent pas la majorité des porteurs du VHC. En effet, 80\% des individus infectés ignorent leur condition dans la plupart des pays  puisque l’infection aigüe ne s’accompagne généralement pas de symptômes apparents jusqu’aux stades plus avancés de la pathologie, ce qui facilite la transmission continue du VHC dans les populations à risques. Un dépistage de masse apparaît donc aujourd’hui indispensable pour renseigner tous les porteurs sains du VHC sur leur statut virologique afin de réduire le risque de transmission et de les prendre en charge avant une détérioration de leur état de santé. \\ \\
\indent
Dans la perspective d’améliorer le dépistage et l’accès au soin en particulier auprès des populations marginalisées (milieu carcéral, milieu rural, squats, migrants, sans-abris) ou dans les pays à ressources limitées disposant peu de centres cliniques spécialisés, un test rapide d'orientation diagnostic (TROD) a été conçu en 2010, permettant de doser en moins de 30 minutes les anticorps par simple prélèvement de sang par piqûre de doigt \citep{RN190}. Depuis 2017, ces petits dispositifs ont été perfectionnés afin de mesurer en parallèle les marqueurs de l’infection par un système automatisé d’extraction et amplification d’acide nucléique, fournissant un résultat en 100 minutes. Un diagnostic complet peut être ainsi réalisé en moins de 2 heures \citep{RN192}. Pour finir, le recours à la simple prévention par la sensibilisation des individus aux pratiques à risques, la formation des prestataires de soin sur les procédures de stérilisation et l’approvisionnement suffisant en aiguilles et seringues pour couvrir toutes les injections pourraient réduire l’incidence du VHC de 80\% \citep[pour revue,][]{RN193}.

	\subsubsection{Thérapies à l’interféron α et à la ribavirine (1989-2011)}
	
À partir de 1989, les patients atteints d’hépatites C chroniques étaient naturellement traités à l’interféron α (IFNα), un médiateur clé de la réponse antivirale innée, couramment utilisé pour d’autres types d’infections virales. Toutefois, le succès des monothérapies à l’IFNα pour l’hépatite C s’avéra très limité avec un taux de guérison de 12\% pour un traitement sur 6 mois et de 20\% lors d’une extension sur 12 mois \citep{RN136}. En 1998, l’addition de ribavirine, un agent antiviral à large spectre, en combinaison avec l’IFNα a permis d’augmenter le taux de guérison à 35-40\% \citep{RN199}. Jusqu’en 2011, la bithérapie reposant sur des injections pendant 24 à 48 semaines de PEG-IFNα (un dérivé comportant une liaison covalente à du polyéthylène-glycol (PEG) améliorant la demi-vie et l’absorption de la molécule) et de ribavirine constituait alors la norme thérapeutique pour l’hépatite C \citep{RN203,RN205}. Cette bithérapie permettait d’atteindre des taux de guérison globale de 90\% pour les infection aigües et de 55\% pour les infections chroniques \citep{RN207,RN209} mais la réponse au traitement variait fortement selon les caractéristiques de l’hôte (âge, genre, ethnicité, obésité, stade de la maladie, consommation d’alcool) et les caractéristiques virales telles que le génotype. En effet, 80\% des patients infectés par les souches de génotype 2 ou 3 éliminaient avec succès le virus, en revanche seulement 40\% des patients infectés par le le génotype 1, plus commun, parvenaient une clairance virale \citep{RN211}. L’utilisation de ces molécules pouvaient entraîner des effets secondaires sévères tels que des troubles hématologiques (anémie, neutropénie, thrombocytopénie), neuropsychiatriques (dépression) ou des maladies auto-immunes conduisant à l’arrêt prématuré du traitement chez environ 10\% des patients.


	\subsubsection{Thérapies antivirales à action directe (depuis 2011)}
	
Le développement successif des systèmes d’étude du VHC (décrits dans la \autoref{section:modele}) et la résolution des structures tridimensionnelles des protéines virales dotées d’activité enzymatique a énormément contribué au progrès de nos connaissances sur le cycle infectieux. Ces données ont guidé la conception des AAD ciblant de manière spécifique des processus clés du virus : la maturation de la polyprotéine virale par blocage du clivage par la protéine non structurale 3-4A (NS3-4A) (boceprevir, telapravir en 2011), la réplication virale par interférence avec la polymérase NS5B (sofosbuvir en 2013) et l’activité de la protéine non structurale 5A (NS5A) (daclatasvir, ledipasvir en 2014). En 2019, 17 AAD ciblant ces trois protéines virales sont approuvés dans les pays occidentaux \citep[pour revue,][]{RN212}. \\ \\
\indent
La communauté scientifique s’accorde sur le fait que les AAD doivent être combinés sous la forme de bi- ou tri-thérapie afin de limiter le risque d’émergence de mutations de résistance dans les protéines ciblées, puisqu’il a été montré qu’une unique substitution dans la séquence de la protéine ciblée serait suffisante pour conférer une résistance \textit{in vitro} \citep{RN213}. Dans les pays occidentaux, l’émergence des traitements basés sur une combinaison de trois AAD a permis d’atteindre des taux de guérison >90\% pour la majorité des génotypes. En plus d’être facilement administrables (voie orale), les tri-thérapies ont permis de limiter le recours à l’IFNα et la ribavirine particulièrement mal tolérés, ce qui a considérablement amélioré la qualité de vie des patients lors du traitement. L’action directe des AAD vis-à-vis du virus a également permis de réduire le traitement de 8 à 24 semaines par rapport aux précédentes thérapies. En ce qui concerne les infections aigües, des essais cliniques sont en cours afin d’optimiser les traitements sur des courtes durées, de 4 à 6 semaines, dans le but d’améliorer la rentabilité économique \citep[pour revue,][]{RN137}. En effet, le coût très élevé des thérapies antivirales freine son accès dans les pays aux ressources limitées, souvent les plus touchés par l’hépatite C et l’OMS estime que seulement 62\% de la population dépistée bénéficie de ces traitements. Les thérapies basé sur l’IFNα sont donc encore très largement répandues dans les pays où les AAD sont inabordables, en Europe de l’Est ou en Asie. Dans le cadre des stratégies locales d’élimination du VHC, des médicaments génériques doivent être établis afin d’étendre l'utilisation de ces traitements dans les pays en voie de développement. \\ \\
\indent
En dépit de l’optimisation de ces traitements, les études cliniques déclarent jusqu’à 15\% d’échecs thérapeutiques selon le groupe de patients. Les échecs thérapeutiques sont majoritairement liés à la présence de variants qui peuvent contenir des polymorphismes naturellement résistants dans les régions ciblées par les AAD, réduisant ainsi leur susceptibilité face à ces molécules \citep[pour revues,][]{RN218,RN129}. Les variants résistants les plus stables peuvent persister plusieurs années après l’échec du traitement ce qui pose un grave problème pour les thérapies de 2ème intention. Et ce problème ne s’amenuisera pas puisque l’utilisation généralisée des AAD va contribuer à augmenter la prévalence et la transmission des variants résistants. En France, parmi les patients en échec thérapeutique entre 2015 et 2018, 22.3\% d’entre eux étaient infectés par des souches de génotype 4 et parmi eux, on note une pré-dominance du sous-type 4r qui comporte fréquemment des mutations de résistance dans les régions NS3, NS5A et NS5B \citep{RN221}, rendant ce sous-type viral particulièrement difficile à traiter quelle que soit la combinaison d’AAD utilisée. D’autres études cliniques récentes relatent des taux sous-optimaux de clairance virale chez les patients infectés par des souches de sous-types 1e, 1g, 1h ou 1l \citep{RN225}. Ces sous-types inhabituels en Europe et aux Etats-Unis sont prévalents dans les régions africaines et représentent 15\% des cas de VHC dans le monde, laissant en conséquence plusieurs millions de personnes sans option thérapeutique. Ainsi, il est indispensable de continuer à développer de nouvelles combinaisons d’AAD, ciblant par exemple les autres protéines virales \citep{RN226,RN274,RN275} pour les patients en échec ou infectés par ces sous-types multi-résistants et de rendre systématique la détermination du génotype et du sous-type pour guider les indications thérapeutiques.

	\subsubsection{Stratégies vaccinales}

Aucun vaccin prophylactique n’est disponible contre l’infection par le VHC depuis la découverte du virus dans les années 1990. La conception d’un vaccin a été fortement contrainte par l’absence de souche cultivable jusqu’en 2005, qui rendait impossible le recours aux méthodes traditionnelles les plus prometteuses dans l’histoire de la vaccination. Les différentes stratégies vaccinales pour le VHC reposaient alors principalement sur des peptides ou protéines recombinantes, des vecteurs, des particules pseudo-virales ou des vaccins à ADN à différents stades de développement ou phases d’essais cliniques. Depuis l’émergence des premiers systèmes de culture, les vaccins basés sur des virus vivant atténué ou inactivé sont désormais envisageables pour l’hépatite C et ouvrent de nouvelles possibilités \citep[pour revues,][]{RN228,RN234,RN277}. \\ \\
\indent
Les barrières au développement d’un vaccin préventif pour l’hépatite C sont la grande variabilité génétique du VHC impliquant la mise en place d’une protection croisée entre les génotypes et la méconnaissance des corrélats de protection. L’immunité protectrice résiduelle chez les patients suite à la guérison par les thérapies antivirales est insuffisante et ils restent susceptibles aux réinfections et à la persistence virale \citep{RN276}. Toutefois, des études cliniques montrent qu’il y a une diminution de 80\% du risque de développer une chronicité dans le cadre des réinfections récurrentes, signifiant qu’une mémoire immunitaire partielle subsiste bien que non stérilisante. Ces données renforcent l’idée qu’une vaccination contre l’hépatite C serait réalisable, au minimum pour empêcher la pathologie chronique, en concentrant les efforts de recherche sur l’identification des corrélats de protection. Il a été mis en évidence que la clairance de l’infection virale serait liée à l’action concertée d’une expansion rapide et continue en lymphocytes T CD4+ et CD8+ effecteurs et mémoires intrahépatiques \citep{RN236} et à la production précoce d’anticorps neutralisants à large spectre \citep{RN87}. Dans ce contexte, le défi des futures stratégies vaccinales est de réussir à combiner l’induction d’une réponse humorale neutralisante ciblant les protéines structurales du virus et d’une réponse cellulaire T CD4+ et CD8+ dirigée contre les régions plus conservées du génome viral, comme les protéines non structurales NS3, NS4, NS5A et NS5B. Le choix d’un adjuvant approprié pour maintenir les réponses cellulaires T et B sera également crucial pour la protection. À ce jour, seuls deux candidats vaccins ont été testés lors d’essais cliniques de phase I. Après validation de leur innocuité et pouvoir immunogène, seul le vecteur poxviral atténué codant les protéines NS3 à NS5B du VHC (MVA-NSmut) \citep{RN241} conçu pour générer une immunité cellulaire, a atteint la phase II mais les résultats publiés récemment ont été décevants. Enfin, des progrès ont été faits pour développer un modèle animal permissif à l’infection et immuno-compétent afin de valider les stratégies vaccinales en phases pré-cliniques et limiter l’emploi des chimpanzés, jusqu’à présent utilisés pour tous les tests de vaccins (voir section \nameref{section:animaux}). Les connaissances accumulées sur ce sujet ces 20 dernières années laissent espérer que la mise sur le marché d’un vaccin au moins partiellement protecteur serait réalisable d’ici les 5 à 10 prochaines années. En diminuant le risque de persistance, ce vaccin sera déjà un grand pas pour réduire la charge mondiale des maladies hépatiques et la transmission du VHC \citep[pour revues,][]{RN212,RN244,RN245}.

\medskip
	\begin{figureth}
	\centering
			\includegraphics[width=0.90\linewidth]{Figure_17.png}
		\caption[Historique des thérapies antivirales sur les 30 dernières années depuis la découverte de l'agent étiologique et défis futurs pour éradiquer le virus de l'hépatite C]{\textbf{Historique des thérapies antivirales sur les 30 dernières années depuis la découverte de l'agent étiologique et défis futurs pour éradiquer le virus de l'hépatite C.} Figure composée à partir des données de la littérature.}
				\label{fig:fig17}
	\end{figureth}
		\FloatBarrier

\clearpage
%%%%%%%%%%%%%%%%%%%%%%%%%%%%%%%%%%%%%%%%%%%%%%%%%%%%%%%%%%%%%%%%%%%%%%%%%%%%%%%%%%%%%%%%%%%%%%%%%%%%%%%%%%%%%%%%%%%%%%%%%%%%%%%%%%%%%%%%%%%%%%%%%%%%%%%%%%%%%%%%%%%%

% 5 - Organisation moléculaire et structurale du virus de l'hépatite C

\section{Organisation moléculaire et structurale du virus de l'hépatite C}	
				\label{section:organisation}
				
	\subsection{Structure des particules virales}
						\label{section:particule}

La particule virale est constituée d’une copie du génome viral incorporée au sein d’une capside non-icosaédrique et d’une enveloppe lipoprotéique recouverte de lipides neutres (esters de cholestérol et triglycérides), des glycoprotéines virales et des apolipoprotéines cellulaires B (ApoB), C (ApoC) et E (ApoE) (\autoref{fig:fig18}). Contrairement aux Flavivirus qui codent une unique glycoprotéine E d’environ 50kDa à l’origine d’une enveloppe virale uniforme, le VHC possède deux glycoprotéines distinctes E1 et E2 \citep[pour revue,][]{RN279}. Ces protéines d’une longueur de 192 et 363 acides aminés sont constituées d’un ectodomaine N-terminal et d’un court domaine C-terminal transmembranaire d’environ 30 acides aminés et possèdent de nombreux sites de N-glycosylations. En effet, les glycoprotéines E1 et E2 possèdent respectivement 5 et 11 sites de N-glycosylations qui, en plus d’assurer le repliement correct des protéines, ont un rôle dans la protection contre la neutralisation \citep{RN396}. À la surface des particules virales, E1 et E2 s’associent sous la forme d’hétérodimère par un processus lent de réarrangement de leur domaine transmembranaire respectif. Puis, les hétérodimères E1-E2 adoptent une conformation complexe structurée en trimères d’hétérodimères stabilisées par des ponts disulfures, puis en 12 pentamères de trimères, selon de récentes prédictions bioinformatiques \citep{RN397,RN398}. À la surface des virions, la glycoprotéine E2 exposent des régions hypervariables (HVR1, HVR2 et IgVR) qui diffèrent jusqu’à 80\% entre les génotypes \citep{RN399}. Ces régions contiennent des épitopes très immunostimulateurs qui fonctionnent comme « leurre » immunologique et protègent les régions les plus conservées. L’action concertée des glycoprotéines E1 et E2 assure l’entrée virale dans la cellule-hôte par la reconnaissance et la fixation aux récepteurs cibles et la fusion avec la membrane des endosomes décrite dans la \autoref{section:tropisme} \citep{RN400}. L’association de l’enveloppe à des composés lipidiques confère à la particule dite «  lipovirale » une faible densité, de l’ordre de 1,06 à 1,10g/mL, comparable à celle des lipoprotéines de basse densité (LDL) ou de très basse densité (VLDL) \citep{RN283,RN284,RN285}. Les premières observations de l'ultrastructure des particules purifiées à partir de sérum de patients ou issues de cultures cellulaires infectées, par cryo-microscopie, tomographie et microscopie électronique à transmission (MET), ont mis en évidence le caractère sphérique des virions et l’hétérogénéité de la taille des particules allant de 40 à 100nm selon leur contenu lipidique \citep{RN281,RN282,RN125}. L’aspect pléomorphique des particules lipovirales, qui imitent la morphologie des lipoprotéines cellulaires, contribuerait à réduire l’immunogénicité des virions circulant dans le sang. De plus, les apolipoprotéines présentes à la surface des virions pourraient protéger de la neutralisation en masquant les épitopes viraux, ce qui diminue l’accessibilité des particules virales aux anticorps anti-E2. Contrairement aux Flavivirus apparentés qui ont une forme icosaédrique et une taille constante d’environ 50nm, les particules lipovirales du VHC ne présentent aucune symétrie apparente de leur enveloppe ou de leur capside, faisant d’elles des structures atypiques parmi les virus eucaryotes.

	\begin{figureth}
	\centering
			\includegraphics[width=\linewidth]{Figure_18.png}
		\caption[Organisation structurale de la particule lipovirale du virus de l'hépatite C]{\textbf{Organisation structurale de la particule lipovirale du virus de l'hépatite C.} ((A) Modèle théorique de la particule lipovirale illustrant les similarités de structure et de composition avec les lipoprotéines du sérum. Les glycoprotéines de surface E1 et E2 , les apolipoprotéines apoB, apoC-I, apoC-II et apoE et la nucléocapside composée de Core et de l’ARN viral sont indiquées (Adapté de \citealt{RN345}). (B) Micrographie électronique représentative des particules lipovirales capturées à l’aide d’anticorps anti-E2 partir du sérum de patients infectés par le VHC (D’après \citealt{RN125}).}
				\label{fig:fig18}
	\end{figureth}
		\FloatBarrier

			\subsubsection{Organisation génomique}

Le génome du VHC est un ARN monocaténaire de polarité positive de 9600 nucléotides. Il est constitué d'une unique phase ouverte de lecture (ORF) encadrée par des régions non codantes hautement structurées 5’NC et 3'NC (\autoref{fig:fig19}). Ce génome joue un rôle essentiel dans plusieurs étapes distinctes du cycle viral : (i) en tant qu’ARN messager (ARNm) pour la synthèse des protéines virales (partie détaillée dans le paragraphe \nameref{section:traduction}); (ii) en servant de matrice pour la synthèse du brin de polarité négative lors de la réplication génomique (partie détaillée dans le paragraphe \nameref{section:replication}) et (iii) en tant que copie d’ARN encapsidée dans les particules virales néosynthétisées (partie détaillée dans le paragraphe \nameref{section:assemblage}).

	\begin{figureth}
	\centering
			\includegraphics[width=\linewidth]{Figure_19.png}
		\caption[Organisation du génome du virus de l'hépatite C]{\textbf{Organisation du génome du virus de l'hépatite C.} Le génome du VHC est représenté avec les structures secondaires théoriques des régions 5’NC, 3’NC (en rouge) et de l’ORF (en bleu). Deux copies du micro-ARN miR-122 interagissant avec l’extrémité 5’ sont illustrées. L’emplacement de l’IRES, du codon initiateur AUG, du codon terminateur STOP, de la région variable (VR), de la région riche en polyU/C et du domaine 3’X sont indiqués (Adapté de \citealt{RN1231}).}
				\label{fig:fig19}
	\end{figureth}
		\FloatBarrier

L’ARN viral contient un ensemble de structures secondaires en tiges-boucles et pseudo-noeuds, présents à la fois dans les régions non codantes et dans l’ORF, qui contribuent à la stabilité du génome et aux interactions inter- et intra-moléculaires importantes sur le plan fonctionnel \citep[pour revues,][]{RN402,RN403}. En effet, ces structures sont impliquées dans les interactions ARN/ARN en cis et dans le recrutement des protéines virales et cellulaires lors des processus de traduction et de réplication génomique. La région 5'NC d'une longueur d’environ 340 nucléotides est constituée de 4 domaines structuraux (I à IV) très conservés parmi les génotypes. Les domaines II, III et IV complétés par quelques codons de la séquence codant la protéine Core forment un site d'entrée interne des ribosomes (IRES) de type III, assurant la traduction du précurseur polyprotéique par un mécanisme indépendant de la coiffe et par recrutement direct de la sous-unité ribosomale 40S. La région 3' NC d'une longueur de 200 à 250 nucléotides contient 3 domaines distincts : une courte région variable non essentielle (VR), une séquence enrichie en polyuracile/polycytosine (polyU/C) d'une longueur moyenne de 80 nucléotides et un domaine quasi-invariable comprenant les 98 nucléotides terminaux, appelé 3'X fortement enrichi en structures secondaires. Le rôle de ces éléments structuraux au cours des différentes étapes du cycle viral sera explicité en détail dans la \autoref{section:cycle}. Le génome contient cinq sites de liaison au micro-ARN hépato-spécifique miR-122. Ces interactions, qui contribuent au tropisme hépatique du VHC, stabilise la structure de l’IRES, amplifie la traduction et la réplication génomique et protège l’ARN viral de la dégradation par les nucléases cellulaires et de la reconnaissance par les senseurs de l’immunité innée \citep{RN401,RN289,RN290}. 

		\subsection{Tropisme et restriction d’hôte : entrée du virus de l'hépatite C dans les hépatocytes humains}
					\label{section:tropisme}

Après transmission par la voie parentérale, le VHC circule dans le sang du patient et pénètre dans le foie par l’artère hépatique. Les particules lipovirales traversent l’endothélium des sinusoïdes hépatiques et se retrouvent dans l’espace de Disse où elles sont exposées au pôle basolatéral des hépatocytes. Cet accès peut être facilité par les cellules endothéliales sinusoïdales et par les cellules de Kupffer qui sont réfractaires à l’infection mais qui expriment des lectines de type C, L-SIGN et DC-SIGN, des molécules d’adhésion capables de capturer les particules lipovirales en se liant à la glycoprotéine d’enveloppe E2 \citep{RN404,RN405,RN406}. Après un premier cycle d’infection des hépatocytes par la circulation sanguine, les particules lipovirales néoformées peuvent être transmises par contacts directs entre hépatocytes, via les jonctions serrées. Ce processus permet de protéger les particules lipovirales des anticorps neutralisants présents dans le sérum, ce qui favoriserait l’établissement de la persistance virale \citep{RN407}. \\ \\
\indent
L’entrée du VHC au sein des hépatocytes est un processus hautement coordonné qui s’effectue en 5 étapes distinctes : (i) liaison aux co-récepteurs de surface présents sur la face basolatérale ; (ii) translocation vers les jonctions serrées ; (iii) internalisation des particules virales par endocytose; (iv) fusion de l’enveloppe avec les membranes de l’endosome et (v) désassemblage de la capside et libération du génome dans le cytoplasme (\autoref{fig:fig20}). De nombreux facteurs de l’hôte sont engagés transitoirement au cours de l’attachement et de l’internalisation des particules du VHC, dont certains contribuent au tropisme hépatique et à la restriction d’hôte du VHC pour l’Homme. Ceci a des implications pour le développement de modèles animaux susceptibles à l’infection pour l’étude du VHC \textit{in vivo} (voir \autoref{section:assemblage}). La plupart des facteurs d’hôtes et des mécanismes décrits dans ce chapitre ont été identifiés dans le contexte de lignées d’hépatomes cellulaires, qui n’imitent pas l’architecture des hépatocytes polarisés. Toutefois, le recours à des systèmes qui reproduisent plus fidèlement l’environnement hépatique \textit{in vivo} a permis de commencer récemment à valider le modèle d’entrée du VHC proposé à partir des découvertes menées sur cellules en lignées \citep[pour revues,][]{RN408,RN409}.



			\subsubsection{Attachement de la particule lipovirale à la surface des hépatocytes}

L’attachement initial des particules du VHC aux hépatocytes imite la stratégie d’entrée des lipoprotéines du sérum. Les apolipoprotéines associées aux particules lipovirales en coopération avec la glycoprotéine d’enveloppe E2, sont responsables de la reconnaissance et de la liaison aux chaînes glycosaminoglycanes de l’héparane sulfate (HS) \citep{RN410}, au récepteur des lipoprotéines de basse densité (LDL-R) \citep{RN411,RN412} et au récepteur scavenger de classe B, type I (SR-BI) \citep{RN413,RN414} présents sur la face basolatérale des hépatocytes. Récemment, le récepteur de la phosphatidylsérine TIM-1, servant de facteur d’entrée pour divers virus de la famille des \textit{Flaviviridae}, a été identifié comme un nouveau facteur d’hôte contribuant à l'attachement du VHC \citep{RN415,RN416}. L’interaction de la particule avec SR-BI active sa fonction de transfert lipidique et entraîner consécutivement la dissociation des lipoprotéines associées aux particules virales. Ce processus induit un changement de conformation de la glycoprotéine E2 et notamment, une exposition des sites de liaison au récepteur CD81, précédemment protégé des senseurs de l’immunité par la région hyper variable 1 (HVR1) \citep{RN417}. L’engagement de E2 avec CD81 active séquentiellement les voies de signalisation du récepteur du facteur de croissance épidermique (EGFR) \citep{RN418} et des GTPases appartenant à la superfamille RAS \citep{RN419,RN420}, ce qui déclenche la diffusion latérale des complexes CD81 associés aux particules virales vers les jonctions serrées, par réarrangement du réseau d’actine. Cette relocalisation permet à ces complexes d’interagir avec les protéines constitutives des jonctions, la claudine 1 (CLDN1) \citep{RN421} et l’occludine (OLCN) \citep{RN422}. La liaison de CD81 avec la CLDN1 est une condition préalable à l’internalisation du virion, et serait responsable de l’endocytose des particules virales \citep{RN423,RN424,RN425}. Le rôle précis de l’OLCN n’est pas connu, mais serait essentiel pour l’internalisation du virion dans les hépatocytes et contribuerait au tropisme d’espèce du VHC \citep{RN426}. En effet, l’expression des protéines homologues humaines CD81 et OCLN constitue le critère minimal pour rendre les cellules de souris ou de hamster permissives à l’entrée virale \citep{RN380}. Les facteurs cellulaires SR-BI et CLDN1 sont fortement exprimés dans les hépatocytes et contribueraient à définir le tropisme tissulaire du VHC au niveau de l’entrée \citep{RN428}.

			\subsubsection{Internalisation de la particule lipovirale et fusion}

Les particules du VHC sont internalisées par endocytose médiée par la clathrine \citep{RN429} et transportées au sein de vésicules endosomales précoces le long du réseau microtubulaire. Parallèlement, l’acidification progressive de la lumière des endosomes va détacher la particule virale de ses récepteurs et déclencher la fusion de l’enveloppe virale avec la membrane endosomale limitante \citep{RN430}. À ce jour, la résolution de la structure de l’ectodomaine de E2 est quasi-complète, tandis que la structure tridimensionnelle de E1 reste partielle, où seuls les résidus 1 à 79 du domaine N-terminal ont pu être déterminés par cristallographie \citep{RN431}. L’ectodomaine de E2 adopte un schéma compact semblable à celui d’une immunoglobuline, excluant la possibilité que E2 assure l’activité fusogène du VHC. À l’inverse, de récentes études démontrent qu’une séquence hydrophobe conservée dans E1 serait impliquée dans le processus de fusion membranaire et a ainsi été proposée comme peptide de fusion (pFP) \citep{RN432}. La résolution partielle de la structure de E1 représente toutefois un modèle incohérent avec les repliements très structurés des peptides de fusion de classe II et III. D’autre part, la taille de E1 est également estimée trop petite pour relier les membranes cellulaires et virales après l’insertion du peptide de fusion. Par conséquent, la protéine assurant l’activité fusogène du VHC est encore indéterminée aujourd’hui, et le VHC pourrait posséder une nouvelle classe de peptide de fusion membranaire, différente des protéines de fusion de classe II répandues chez les Flavivirus \citep{RN433}. Certains facteurs d’hôtes essentiels à l’étape de fusion seraient sélectivement internalisés au sein des endosomes contenant les particules virales, comme la protéine de transport intracellulaire du cholestérol (NPC1L1) \citep{RN434} et la protéine 1 de liaison du facteur de réponse au sérum (SRFBP1) \citep{RN435}. Bien que NPC1L1 soit localisée au pôle apical des hépatocytes, son internalisation au sein des compartiments endosomaux aiderait à fournir une composition lipidique optimale pour promouvoir la fusion. SRFBP1 aurait, quant à elle, un rôle important dans le transport rétrograde des endosomes contenant les particules virales. Après la fusion, la nucléocapside est rapidement désassemblée ce qui libère le génome viral dans le cytoplasme initie les différentes étapes du cycle viral, décrites dans la \autoref{section:cycle}.

	\begin{figureth}
	\centering
			\includegraphics[width=\linewidth]{Figure_20.png}
		\caption[Modèle d’infection des hépatocytes par le VHC à partir de la circulation sanguine]{\textbf{Modèle d’infection des hépatocytes par le VHC à partir de la circulation sanguine.} Cette illustration résume les facteurs d’hôtes et la séquence des événements qui conduisent de la fixation initiale des particules lipovirales sur la face basolatérale des hépatocytes, à l’internalisation et à la libération du génome viral dans le cytosol. La première étape de fixation implique principalement la composante lipoprotéique de la particule lipovirale au niveau des récepteurs LDLR et SR-BI. L’exposition successive de l’enveloppe virale permet à la glycoprotéine E2 d’interagir spécifiquement avec SR-BI, CD81 puis avec CLDN1 après diffusion latérale vers les jonctions serrées. La particule virale est internalisée avec ces co-récepteurs par endocytose médiée par la clathrine (CAPN5, CBLB) et dépendante du réseau microtubulaire (Dynamin II). Les vésicules endosomales s’acidifient progressivement et la baisse du pH favorise la fusion de l’enveloppe virale avec la membrane de l’endosome et le désassemblage de la capside, libérant le génome viral dans le cytosol (Adapté de \citealt{RN409}).}
				\label{fig:fig20}
	\end{figureth}
		\FloatBarrier

\clearpage

%%%%%%%%%%%%%%%%%%%%%%%%%%%%%%%%%%%%%%%%%%%%%%%%%%%%%%%%%%%%%%%%%%%%%%%%%%%%%%%%%%%%%%%%%%%%%%%%%%%%%%%%%%%%%%%%%%%%%%%%%%%%%%%%%%%%%%%%%%%%%%%%%%%%%%%%%%%%%%%%%%%%

% 6 - Cycle infectieux du virus de l'hépatite C

			\section{Cycle infectieux du virus de l'hépatite C}
			\label{section:cycle}			

Comme tous les virus à ARN positif, le cycle infectieux du VHC est entièrement cytoplasmique. Le cycle viral est coordonné en différentes étapes par l’action séquentielle de protéines virales et par le détournement des machineries cellulaires, en particulier le métabolisme lipidique. Au cours de ce chapitre, l’ensemble des étapes post-entrée du cycle viral seront détaillées sous la forme de quatres sections : (i) la traduction du génome viral infectant, (ii) la biogénèse des usines de réplication virale, (ii) la réplication du génome viral et (iv) l’assemblage et la sécrétion des particules virales dans le milieu extra-cellulaire.

			\subsection{Traduction du génome viral et clivage du précurseur polyprotéique}
			\label{section:traduction}
			
Après désassemblage de la nucléocapside, le génome viral libéré dans le cytoplasme sert directement d’ARN messager (ARNm) pour produire une polyprotéine d’environ 3000 acides aminés (\autoref{fig:fig21}). L’IRES du VHC ne nécessite que les facteurs d’initiation eIF2, eIF3 et l’ARN de transfert (ARNt) pour assembler le ribosome \citep{RN436}. L’initiation de la traduction commence par un changement de conformation de l’IRES, différente de celle en solution, permettant d’interagir avec la sous-unité ribosomale 40S \citep{RN437}. L’IRES recrute successivement le complexe de pré-initiation, formé par eIF3 et par le complexe ternaire eIF2-ARNt associé à une molécule de guanosine triphosphate (GTP) au niveau du codon AUG initiateur. Le facteur miR-122 promeut le recrutement des facteurs d’initiation eIF2 et eIF3. Lors de l’hydrolyse de la GTP, les facteurs d’initiation eIF2 et eIF3 se dissocient de l’IRES et sont remplacés par la sous-unité ribosomale 60S pour assembler le ribosome fonctionnel \citep{RN438}. L’ARNt initiateur est ensuite placé au site P du ribosome, ce qui marque la transition entre l’initiation de la traduction et l’élongation. Des études par cryo-microscopie à l’échelle quasi-atomique prédisent que la partie apicale du domaine II et le domaine III représenteraient les structures les plus mobiles de l’IRES tandis que la partie basale du domaine et le pseudo-noeud III/IV forment une structure beaucoup plus rigide \citep{RN439}. \\ \\
\indent
La polyprotéine virale est adressée à la membrane du réticulum endoplasmique (RE), par quatre peptides signaux présents au sein des séquences codant les protéines structurales. La polyprotéine virale est clivée par l’action séquentielle de protéases cellulaires et virales pendant ou immédiatement après la traduction : (i) le domaine C-terminal de NS2 a une activité protéasique à cystéine qui catalyse en premier le clivage à la jonction NS2/NS3; (ii) la libération de la protéine NS3, associée en position C-terminale à son co-facteur NS4A, amorce son activité protéasique à sérine, ce qui catalyse en deuxième le clivage à la jonction NS5A/NS5B (iii), puis à la jonction NS4B/NS5A et (iv) une peptidase signal (SP) cellulaire catalyse en quatrième les clivages complets aux jonctions Core/E1, E1/E2 et partiels aux jonctions E2/p7 et p7/NS2 \citep{RN440,RN441}. D’après des modèles prédictifs, des déterminants structuraux situés en aval des sites de clivage de p7 et NS2 imposeraient des contraintes structurelles et diminueraient l’efficacité du clivage par la SP, conduisant fréquemment à la production de précurseurs E2-p7-NS2 \citep{RN442}. La protéine Core, qui est alors retenue par un peptide signal, sera clivée par la peptidase du peptide signal (SPP) cellulaire, permettant la migration de sa forme mature vers les GL cytosoliques \citep{RN303}. Les autres protéines virales matures restent associées à la membrane du RE par le biais de domaines transmembranaires (p7, NS2, NS4B), d’une unique hélice transmembranaire (E1, E2, NS4A, NS5B) ou d’une hélice alpha monotopique (NS5A) et sont essentiellement localisées dans la lumière du RE (pour les glycoprotéines d’enveloppe E1, E2) ou orientées vers la face cytosolique (NS2, NS3-4A, NS5A et NS5B).

	\begin{figureth}
	\centering
			\includegraphics[width=\linewidth]{Figure_21.png}
		\caption[Traduction de l’ARN viral, clivage du précurseur polyprotéique et topologie membranaire des protéines virales]{\textbf{Traduction de l’ARN viral, clivage du précurseur polyprotéique et topologie membranaire des protéines virales.} L’ORF codant la polyprotéine du VHC et les structures secondaires théoriques des régions 5’NC et 3’NC sont illustrées en haut. Le clivage co- et post-traductionnel de la polyprotéine virale est indiqué comme suit : peptidase signal (flèches verticales en pointillés), peptidase du peptide signal (étoile), protéase NS2 (flèche courbe en pointillés), protéase NS3-4A (flèches courbes solides). La topologie membranaire des protéines virales matures et leurs fonctions connues sont indiquées en bas. La protéine NS5A est représentée sous la forme d’un dimère reconnu, mais les autres protéines virales peuvent également former des homo- et hétérodimères ou des complexes oligomériques. (Adapté de \citealt{RN484}).}
				\label{fig:fig21}
	\end{figureth}
		\FloatBarrier
		
Le VHC code 10 protéines qui assurent l’ensemble des processus nécessaires au cycle infectieux complet du virus, dénotant un rôle complexe et multi-fonctionnel des protéines virales. La diversité des fonctions de chaque protéine peut être favorisée par des changements fréquents de conformation ou de modifications du microenvironnement cellulaire impliquant un réseau spécifique d’interactions avec des partenaires viraux et cellulaires. Les protéines structurales du VHC qui composent la particule lipovirale sont la protéine de capside Core et les glycoprotéines d’enveloppe E1 et E2. Les protéines p7 et NS2 participent à l’assemblage mais ne sont pas incorporées au sein les particules virales. Les protéines NS3, NS4A, NS4B, NS5A et NS5B sont suffisantes pour assurer la réplication génomique \citep{RN294}. Dans ce contexte, il est fréquemment établi que le génome viral code deux modules protéiques fonctionnels : un module d’assemblage constitué des protéines core à NS2 et un module réplicatif constitué des protéines NS3 à NS5B. Toutefois, il est maintenant admis que l’ensemble des protéines non structurales contribue à la formation des virions, bien que les mécanismes précis ne soient pas encore résolus \citep[pour revue,][]{RN293}. L’implication de ces protéines virales au cours du cycle infectieux sera détaillée ci-après. Les caractéristiques structurales et physiopathologiques de la protéine Core seront approfondies dans la \autoref{section:core}.



			\subsection{Biogénèse des usines de réplication virale}
			
Comme tous les virus à ARN de polarité positive, le VHC remodèle considérablement les membranes intracellulaires afin de générer un compartiment spécialisé dans la réplication génomique et l’assemblage des particules virales \citep[pour revue,][]{RN450}. Ce réseau membranaire ou membranous web (MW) permet (i) d’établir un micro-environnement propice à la réplication génomique du VHC en augmentant la concentration locale des facteurs nécessaires à une réplication efficace de l’ARN viral et en protégeant les protéines virales et l’ARN des senseurs de l’immunité innée et (ii) de coordonner spatialement les différentes étapes du cycle viral. \\ \\
\indent
Ce réseau est formé par une accumulation de vésicules de taille et de morphologie hétérogène dans le cytoplasme, naissant par exvaginations des membranes du RE rugueux (Fig. 11), dont les structures principales sont des vésicules à double membrane (DMV) d’environ 150nm \citep{RN451}. Les protéines virales NS3 à NS5B constituant le module réplicatif, ainsi que l’ARN double brin issu d’une activité réplicase \textit{in vitro} ont été retrouvés associés aux DMV, suggérant que ces structures constituent les sites de réplication génomique du VHC \citep{RN452}. Le site de la synthèse \textit{de novo} de l’ARN viral n’a pas encore été précisément localisé, mais des études biochimiques soulignent que le complexe de réplication réside dans un environnement protégé des nucléases et des protéases endogènes, ce qui appuie l’hypothèse selon laquelle la réplication génomique se produit dans la « lumière » de la vésicule, et non sur la membrane externe \citep{RN453,RN454}. La plupart des DMV sont des structures fermées et seule une proportion d’environ 10\% possède une ouverture vers le cytosol, sous la forme de pores \citep{RN451}. Ces ouvertures pourraient être impliquées dans l’échange de métabolites et de facteurs nécessaires à la réplication, ainsi que la sortie des génomes nouvellement synthétisés. Ces observations impliqueraient que seule une sous-population minoritaire de DMV supporte une réplication active tant qu’elles sont connectées au cytosol et que la réplication pourrait cesser au moment de la fermeture des pores de la membrane. Toutefois, il a été récemment montré que des facteurs impliqués dans le transport membranaire comme les protéines du complexe du pore nucléaire (NUP) sont délocalisés dans les régions du MW et pourraient permettre le traffic des molécules à travers un compartiment membranaire fermé \citep{RN455,RN456}. Au sein du MW, on retrouve plus rarement des vésicules à membrane unique (SMV) définies comme des structures préalables aux DMV \citep{RN457} ou des vésicules multi-membranaires (MMV) qui apparaissent principalement lors des stades tardifs de l’infection et résulteraient de la réponse cellulaire au stress induit par une réplication virale élevée \citep{RN458}. Les structures formant « l’organelle » de réplication génomique du VHC sont morphologiquement similaires à celles induites par les \textit{Coronaviridae} \citep{RN459} et les \textit{Picornaviridae} \citep{RN460}, mais relativement distinctes de celles induites par les autres membres de la famille des \textit{Flaviviridae} tels que le virus de la dengue (DENV), le virus du nil occidental (WNV) ou le virus de l’encéphalite à tique (TBEV) \citep[pour revue,][]{RN461}, qui sont des invaginations à membrane unique dans la lumière du RE. Récemment, une étude a mis en évidence que le MW détecté dans le tissu hépatique serait essentiellement constitué de SMV, contrairement aux observations réalisées \textit{in vitro} \citep{RN462}. Les auteurs proposent que les DMV et les MMV seraient une caractéristique propre aux souches du VHC adaptées en laboratoire ou aux mécanismes de la réponse d’immunité innée des lignées cellulaires d’hépatome, soulignant l’importance d’étudier le cycle viral dans des modèles plus physiologiques. \\ \\
\indent
La formation d’un MW intègre et fonctionnel résulte de l’action concertée des protéines virales NS4B et NS5A et de facteurs cellulaires. Premièrement, cela implique de remodeler les membranes intracellulaires existantes afin de compartimenter le complexe de réplication. La protéine NS4B est une protéine hautement hydrophobe avec une topologie membranaire complexe, comprenant quatre hélices transmembranaires et quatre hélices amphipatiques dans les domaines N et C-terminaux (\autoref{fig:fig22}) \citep{RN463}. La formation des vésicules serait facilitée par une courbure positive de la membrane externe du RE, induite par l’insertion asymétrique des hélices amphipatiques de NS4B dans le feuillet lipidique \citep{RN464}. De plus, la multimérisation de NS4B serait importante pour amplifier le processus de remodelage membranaire \citep{RN465}. Deuxièmement, la biogenèse des « organelles » de réplication nécessite une synthèse \textit{de novo} importante de lipides membranaires, en particulier de cholestérol et de sphingolipides, qui stimulent l’activité de la réplicase en formant des radeaux lipidiques \citep{RN466,RN467}. L’infection par le VHC déclenche le clivage protéolytique des protéines de liaison à l’élément de régulation des stérols (SREBP), une famille de facteurs de transcription qui activent l’expression de gènes impliqués dans la lipogénèse, tels que l’acide gras synthase (FAS) et l’hydroxyméthylglutaryl-CoA réductase (HMG-CoA), l’enzyme catalysant la biosynthèse du cholestérol \citep{RN468}. De plus, la protéine NS5A recrute la kinase PI4KIIIa pour produire une accumulation locale de phosphatidylinositol 4-phosphate (PI4P) au niveau des DMV \citep{RN469,RN470}. Le transport d’autres espèces lipidiques, comme le cholestérol et les sphingolipides, pourrait être médié par des protéines de transfert lipidique ciblant les membranes enrichies en PI4P. Par exemple, NS5A recrute la protéine de liaison à l’oxystérol 1 (OSBP), qui assure le transport du cholestérol en échange du PI4P \citep{RN471}. En parallèle de son rôle central dans le recrutement des facteurs d’hôte, la protéine NS5A modifie les propriétés membranaires, par le biais de son hélice α amphipatique contenue à son extrémité N-terminale, ce qui est essentiel dans la biogenèse du compartiment de réplication \citep{RN472}.

	\begin{figureth}
	\centering
			\includegraphics[width=\linewidth]{Figure_22.png}
		\caption[Organisation et structure tridimensionnelle des usines de réplication génomique du virus de l’hépatite C]{\textbf{Organisation et structure tridimensionnelle des usines de réplication génomique du virus de l’hépatite C.} Représentation schématique du réarrangement membranaire du réticulum endoplasmique pour former les vésicules à double membranes (DMV) ou les vésicules à multiples membranes (MMV) induite par le VHC. À droite, reconstruction tridimensionnelle du réarrangement des membranes observé par tomographie électronique dans des cellules infectées  par le VHC. Les membranes externe et interne des DMV sont respectivement colorisées en brun clair et en orange. Le RE, les filaments du cytosquelette, l’appareil de Golgi et les vésicules à membrane unique sont respectivement coloriées en brun foncé, en bleu, en vert et en violet. En bas à droite, représentation schématique du compartiment de réplication du VHC au sein d’une DMV, qui contient les protéines non structurales NS3-4A, NS4B, NS5A et NS5B responsables de la réplication génomique  (Adapté de \citealt{RN450,RN346}).}
				\label{fig:fig22}
	\end{figureth}
		\FloatBarrier
	

			\subsection{Réplication du génome viral}
			\label{section:replication}

Au sein des « organelles » de réplication, le génome viral est recruté et copié en un intermédiaire de polarité négative, qui servira de matrice pour la production \textit{de novo} de nouvelles molécules d’ARN de polarité positive. La réplication du génome viral est médiée la protéine virale NS5B qui héberge une activité ARN polymérase ARN-dépendante (RdRp) \citep{RN473,RN474}. Comme toutes les RdRp virales, NS5B a une forme de « main droite » englobant les domaines de la paume, du pouce et des doigts et un domaine catalytique comprenant un motif GDD \citep{RN475}. La synthèse d’ARN \textit{de novo} est initiée à partir de l’extrémité 3’ du génome viral par un mécanisme indépendant d’amorce \citep{RN476,RN477}. Outre la polymérase NS5B, la protéine NS3 a une activité d’hélicase essentielle pour assurer une réplication correcte de l’ARN, mais son rôle n’est pas complètement défini. Cette fonction pourrait être requise pour dissocier les éléments d’ARN monocaténaire hautement structurés ou pour dissocier les ARN double brins \citep{RN478}. En parallèle, la protéine NS5A est essentielle à la réplication génomique en stimulant l’activité de la polymérase, par liaison à l’ARN viral ou par interaction directe avec NS5B \citep{RN525} et par le recrutement de facteurs d’hôte : la cyclophiline A (CypA) qui catalyse l’isomérisation de certains domaines de NS5A et NS5B nécessaire au fonctionnement du complexe de réplication \citep{RN522,RN523} et la protéine A associée à une vésicule (VAPA) qui facilite le positionnement correct de la polymérase au sein des rafts lipidiques \citep{RN524}.\\ \\
\indent
Les structures en tiges-boucles présentes tant dans le brin positif que dans le brin négatif sont impliqués dans l’initiation ou la régulation de la réplication (\autoref{fig:fig23}). Dans le génome viral, les domaines I et II contiennent des signaux essentiels pour la réplication génomique bien que l'intégralité de la région soit requise pour assurer une synthèse correcte de l'ARN \citep{RN286,RN287}. Le domaine terminal 3'X ainsi qu'un élément minimal de 25 nucléotides situé dans la région polyU/C sont essentiels pour la réplication génomique en culture cellulaire et \textit{in vivo} \citep{RN249,RN292}. En plus des éléments structuraux présents dans les régions NC, les interactions de longue portée entre la tige-boucle 5BSL3.2 présente dans la région codante de NS5B et les extrémités 5’NC ou 3’X permettraient de circulariser le génome, une conformation plus propice à la réplication \citep{RN480,RN481,RN482}. L'ARN intermédiaire de polarité négative forme des structures secondaires différentes de celles trouvées dans le génome viral (\autoref{fig:fig23}). Son extrémité 3’, incluant les élément minimaux I’ et II’z, servent d’initiateur très efficace de la synthèse d’ARN \citep{RN483}. En revanche, l’extrémité 3’ de l’ARN de polarité positive est dissimulée au sein d’une structure complexe, réduisant l’efficacité de l’initiation \textit{de novo}. Cette différence joue probablement un rôle dans la régulation des processus de réplication et pourrait contribuer à l’excès d’ARN positifs par rapport aux ARN négatifs \citep[pour revue,][]{RN484}.

	\begin{figureth}
	\centering
			\includegraphics[width=\linewidth]{Figure_23.png}
		\caption[Structures secondaires au sein du génome du virus de l’hépatite C (ARN de polarité positive) et de l’intermédiaire de réplication (ARN de polarité négative)]{\textbf{Structures secondaires au sein du génome du virus de l’hépatite C (ARN de polarité positive) et de l’intermédiaire de réplication (ARN de polarité négative).} L’organisation du génome viral, avec l’ORF codant la polyprotéine encadré des structures secondaires théoriques des régions 5’NC et 3’NC, est illustré en haut de la figure. En dessous sont représentés les prédictions des structures secondaires présentes dans trois régions au sein de l’ARN de polarité positive. Les interactions ARN/ARN de longue portée sont indiquées avec des flèches en pointillés et les sites de liaison au miR-122 sont indiqués par des rectangles gris. Les structures secondaires prédites dans l’ARN intermédiaire de polarité négative sont illustrées en bas (Adapté de \citealt{RN484}).}
				\label{fig:fig23}
	\end{figureth}
		\FloatBarrier
		


			\subsection{Assemblage et sécrétion des particules virales}
			\label{section:assemblage}
			
La morphogénèse et la sécrétion des particules virales constituent les étapes les moins comprises du cycle viral encore aujourd’hui. Elles englobent un ensemble de processus, à commencer par l’assemblage de la nucléocapside, l’acquisition de l’enveloppe virale par bourgeonnement dans la lumière du RE, l’acquisition des lipoprotéines intracellulaires pour former une particule « lipovirale » mature et la sortie définitive des virions dans le milieu extracellulaire. Ces étapes complexes reposent sur une action coordonnée entre les protéines virales structurales, non structurales et un nombre important de facteurs d’hôte, dont les détails mécanistiques sont encore flous.

\subsubsection{Formation de la nucléocapside virale}

Une première condition obligatoire à la réalisation de ces processus implique un rapprochement intracellulaire des éléments formant la structure des particules virales : la protéine Core, le génome viral et les glycoprotéines E1 et E2 (\autoref{fig:fig24}\textcolor{blue}{.A}). Après la synthèse et la maturation des protéines virales par clivage post-traductionnel, la protéine Core est recrutée à la surface des GL cytosoliques \citep{RN485} tandis que les protéines E1 et E2 sont retenues au sein des membranes du RE sous la forme d’un hétérodimère non-covalent \citep{RN486}. L’association de Core aux GL est cruciale pour la morphogénèse virale puisque des mutations visant à empêcher le trafic de Core vers ces organites abolissent l’assemblage des particules virales \citep{RN487,RN488,RN489}. La séparation de la protéine Core et du complexe de réplication dans des compartiment distincts serait importante pour le contrôle spatio-temporel de la réplication et de l’assemblage du virus, afin d’éviter toute compétition au niveau de la liaison à l’ARN \citep{RN346}. La protéine Core doit être toutefois re-mobilisée à proximité des molécules d’ARN virale néosynthétisées pour initier l’assemblage de la nucléocapside. Une étude récente d’imagerie à haute résolution a mis en évidence l’existence d’une sous-population de GL enveloppée par le MW et directement connectée à des DMV, au sein des cellules infectées par la souche Jc1 du VHC (\autoref{fig:fig24}\textcolor{blue}{.B}) \citep{RN490}. Ce groupe a constaté que les DMV peuvent être générés directement à partir des membranes du RE entourant les GL, qui contiennent à la fois les protéines du complexe de réplication et les glycoprotéines d’enveloppe. \\ \\
\indent
Ainsi, deux hypothèses principales émergent de ces observations concernant le point de départ putatif de l’assemblage viral : il aurait lieu soit (i) à la surface des GL soit (ii) au niveau des membranes du RE juxtaposées aux GL. L’interaction de la protéine NS5A avec Core est une étape préalable cruciale à l’assemblage des particules virales \citep{RN491}. Sa mobilité entre les membranes cellulaires et sa capacité de fixation à l’ARN laissent supposer que NS5A est responsable du contact entre la protéine Core et le génome viral \citep{RN494}. Dans le premier modèle, le génome viral serait acheminé par NS5A directement au contact de Core, à la surface des GL. En effet, il a été montré que NS5A co-localise transitoirement avec Core à la surface des GL et que l’inhibition de cette interaction abolit l’assemblage viral \citep{RN295,RN493}. Dans le deuxième modèle, la protéine Core serait recrutée dans la membrane du RE avoisinante par une action coordonnée entre les protéines virales NS5A, NS2 et p7 \citep{RN495}. Il a également été montré que NS2 interagit avec l’hétérodimère E1/E2, et serait responsable de la migration des glycoprotéines d’enveloppe vers les sites d’assemblage putatifs \citep{RN496,RN497}. De plus, la protéine p7 est indispensable aux étapes finales de la formation de la nucléocapside, selon un mécanisme encore inconnu \citep{RN498}. Dans ce contexte, l’enveloppement serait une étape directement couplée à l’assemblage de la nucléocapside. Enfin, au sein du complexe de réplication, les protéines NS3-4A, NS4B et NS5B sont également mobilisées lors de la morphogénèse du VHC \citep{RN296,RN297,RN298}. Cependant, il reste à déterminer si ces protéines non structurales jouent un rôle direct ou indirect dans cette étape du cycle viral. Dans ce deuxième modèle, les GL cytosoliques serviraient alors de simples organites de transport, transférant la protéine Core des sites de traduction aux sites d’assemblage putatifs contenus dans les membranes du RE juxtaposées. Toutefois, il a récemment été montré que la mobilisation des lipides à partir des GL cytosoliques favorise la morphogénèse du VHC, traduisant un rôle supplémentaire de ces organites dans la lipidation des particules virales \citep[pour revue,][]{RN500}. En effet, certaines molécules lipidiques stockées au sein des GL cytosoliques, comme le cholestérol et les sphingolipides, sont essentielles pour l’infectiosité des particules virales \citep{RN501}. Le transfert des lipides aux particules virales s’effectuerait par hydrolyse des GL cytosoliques par l'adipose triglycéride lipase (ATGL) sous le contrôle de la protéine 5 contenant un domaine α/β hydrolase (ABHD5) \citep{RN499}. Jusqu’à présent, il n’a pas été possible de privilégier l’un de ces deux modèles et de visualiser l’emplacement précis des sites d’assemblage du VHC, en raison de la nature transitoire de ces évènements. De plus, la discrimination entre la synthèse des particules lipovirales et des lipoprotéines est difficile compte tenu des similarités entre ces deux nano-structures.

	\begin{figureth}
	\centering
			\includegraphics[width=\linewidth]{Figure_24.png}
		\caption[Morphogénèse et sites putatifs d’assemblage du virus de l’hépatite C]{\textbf{Morphogénèse et sites putatifs d’assemblage du virus de l’hépatite C.} (A) Représentation schématique de l’assemblage des particules du VHC. La protéine Core est recrutée par l’action coordonnée des protéines virales p7, NS2 et NS5A aux membranes du RE. L’ARN viral néosynthétisé est déplacé hors des complexes de réplication vers les sites putatifs d’assemblage. Les nucléocapsides s'assemblent et bourgeonnent dans la lumière du RE où elles s’associent aux glycoprotéines d’enveloppe E1-E2 et aux apolipoprotéines, dont apoE à partir des GL luminales (Adapté de \citealt{RN345}). (B) Micrographies électroniques (en haut) et modélisations 3D (en bas) montrant les sous-populations de GL entourées par des membranes du RE et connectées à des DMV (colorisées en jaunes) au sein de cellules infectées. Les barres d'échelle représentent 100 nm (Adapté de \citealt{RN490}).}
				\label{fig:fig24}
	\end{figureth}
		\FloatBarrier

\subsubsection{Maturation et sécrétion de la particule virale}

La plupart des chercheurs favorisent l’hypothèse selon laquelle les étapes finales du cycle du VHC, c'est-à-dire la maturation et la libération des particules virales, seraient étroitement liées à la voie de biosynthèse des VLDL et à la voie de sécrétion par le réseau trans-golgien \citep{RN502,RN503}. Après le bourgeonnement des nucléocapsides dans la lumière du RE, les particules nouvellement enveloppées s’associeraient à des GL luminales, les précurseurs des lipoprotéines, pour acquérir les différentes apolipoprotéines \citep[pour revue,][]{RN504}. De nombreuses apolipoprotéines peuvent être trouvées à la surface des particules circulantes du VHC, telles que apo-B, l'apo-A-I, l'apoC-I, l'apoC-II et l'apoC-III, mais seule l’incorporation d’apoE serait suffisante pour rendre la particule infectieuse, en accord avec son rôle dans l’entrée virale \citep{RN505,RN506}. Par ailleurs, une étude a récemment montré que l’association à certaines lipoprotéines dépendrait de la souche virale et du type cellulaire \citep{RN507}. L’association des GL luminales avec les particules virales serait directement médiée par l’interaction de la protéine d’enveloppe E2 avec apoE, sous le contrôle de l’annexine 3 (ANXA3), une protéine cellulaire régulant le trafic des voies d’endo et d’exocytose \citep{RN508}. Les particules virales matures transiteraient ensuite vers l’appareil de Golgi. À l’appui de cette hypothèse, les glycoprotéines E1 et E2 subissent des modifications post-traductionnelles de type N-glycosylation avec l’acquisition de sucres complexes \citep{RN509}. Par ailleurs, plusieurs témoignages mettent en évidence la présence de particules virales dans des vésicules COP-II, qui assurent le transport antérograde des protéines cargo du RE vers le Golgi \citep{RN510}. De plus, l'inhibition de la GTPase Rab1b, responsable de la régulation du trafic entre le RE et le Golgi, réduit le taux de libération des particules du VHC \citep{RN511}. Les particules virales transiteraient ensuite au sein du réseau trans-golgien pour rejoindre le compartiment endosomal (\autoref{fig:fig25}\textcolor{blue}{.A}) \citep{RN512}, où elles seraient sécrétées via des vésicules de clathrine \citep{RN513} ou des exosomes bourgeonnant à partir des corps multi-vésiculaires (MVB) \citep{RN514,RN515}. Cette voie sécrétoire dite « canonique » est une voie de sortie employée par de nombreux virus enveloppés, tels que le VHB, le DENV et le WNV \citep[pour revue,][]{RN516}.  \\ \\
\indent
Cependant, des découvertes récentes ont mis en évidence des voies sécrétoires non con-ventionnelles impliquées dans la sortie des particules virales, soulignant la possibilité que le VHC exploite plusieurs voies de sécrétion \citep[pour revues,][]{RN517,RN518}. En effet, les facteurs connus pour jouer un rôle dans la sécrétion du VHC, comme les protéines de la famille Rab, ne sont pas seulement impliquées dans le trafic intracellulaire golgien. Par exemple, les vésicules COP-II peuvent directement fusionner avec la membrane plasmique ou avec les endosomes tardifs qui vont successivement libérer le cargo dans le milieu extracellulaire, coutournant ainsi l’appareil de Golgi (\autoref{fig:fig25}\textcolor{blue}{.B}). Cette voie de sécrétion atypique pourrait être stimulée par le stress membranaire induit au cours de la réplication du VHC. Bien qu'il soit hautement spéculatif, ce modèle est compatible avec la modification incomplète des glycoprotéines virales par les enzymes résidant dans le Golgi et l'absence de co-localisation des particules virales avec les marqueurs du Golgi \citep{RN519}. Pour finir, des études récentes mettent en évidence que des apolipoprotéines échangeables pourraient être acquises par les particules virales après leur libération dans le milieu extracellulaire, à partir des lipoprotéines circulant dans le sérum \citep{RN520,RN521}.

	\begin{figureth}
	\centering
			\includegraphics[width=0.70\linewidth]{Figure_25.png}
		\caption[Voies de sécrétion hypothétiques des particules du virus de l’hépatite C]{\textbf{Voies de sécrétion hypothétiques des particules du virus de l’hépatite C.} (A) Dans la voie d’exportation conventionnelle ou canonique, les particules virales matures sont transportées du RE vers l’appareil de Golgi au sein de vésicules COP-II puis transitent au sein du réseau trans-golgien jusqu’aux compartiments endosomaux. À ce stade, les particules virales peuvent être soit transportées vers un corps multivésiculaire (MVB) pour être ensuite libérées par exocytose, soit directement exportées via une vésicule sécrétoire. (B) Dans la voie d’exportation non conventionnelle, les vésicules COP-II vont directement libérer les particules virales dans le milieu extracellulaire ou les transporter dans le compartiment endosomal, sans passer par l’appareil de Golgi. Les facteurs d’hôte connus pour être impliqués dans le trafic intracellulaire des particules virales  sont encadrés de vert. (Adapté de \citealt{RN518}).}
				\label{fig:fig25}
	\end{figureth}
		\FloatBarrier

\clearpage

%%%%%%%%%%%%%%%%%%%%%%%%%%%%%%%%%%%%%%%%%%%%%%%%%%%%%%%%%%%%%%%%%%%%%%%%%%%%%%%%%%%%%%%%%%%%%%%%%%%%%%%%%%%%%%%%%%%%%%%%%%%%%%%%%%%%%%%%%%%%%%%%%%%%%%%%%%%%%%%%%%%%

% 7 - Les modèles d'étude du virus de l'hépatite C

\section{Les modèles d'étude du virus de l'hépatite C}	
		\label{section:modele}

Depuis la découverte du VHC en 1989, l’absence de système en culture cellulaire a longtemps été un obstacle majeur à l’étude de ce nouveau pathogène et à la conception de stratégies prophylactiques et thérapeutiques. Historiquement, les premières tentatives pour établir des modèles d’étude \textit{in vitro} reposaient sur l’infection expérimentale de divers types cellulaires en culture, dont des cellules hépatocytaires, par des souches cliniques du VHC issues de sérum de patients infectés (VHCser), qui représentaient à l’époque la seule source de virus. Bien que plusieurs travaux aient fait état d’une réplication du VHC dans des lignées cellulaires ou des cultures de cellules primaires, l’efficacité de réplication était variable voire douteuse, et aucun modèle n’a abouti à une infection productive \citep[pour revue,][]{RN307}. Les essais suivants se sont alors basés sur les premiers clones moléculaires disponibles en suivant les stratégies établies pour d’autres virus à ARN positif \citep{RN308}, c’est-à-dire introduire une copie d’ADNc du génome viral dans un vecteur d’expression sous le contrôle transcriptionnel d’un promoteur d’ARN polymérase de phage. Ce système par synthèse \textit{in vitro}, permettait alors de disposer d’une source illimitée de génome viral et d’insérer des gènes rapporteurs ou des marqueurs de sélection pour faciliter la détection. La fonctionnalité de cette approche pour le VHC a a été établie pour la première fois suite à l’injection intra-hépatique du génome transcrit \textit{in vitro} de l’isolat H77 de génotype 1a chez des chimpanzés, qui ont alors développé une virémie \citep{RN309,RN310}. Par la suite, d’autres génomes infectieux \textit{in vivo} ont été générés, comme l’isolat J6 de génotype 2a \citep{RN311}. Néanmoins, aucune réplication en culture cellulaire n’a pu être observée avec ces clones, signant le premier échec d'un long et sinueux parcours vers le développement de modèles expérimentaux pour étudier le VHC. \\ \\
\indent
Au cours de ce chapitre, nous détaillerons (i) les différents systèmes d’étude qui ont été développés \textit{in vitro} et leurs implications dans les découvertes fondamentales sur la biologie du virus et dans le développement des thérapies antivirales ; (ii) les perspectives futures pour élaborer des modèles cellulaires plus physiologiques à partir des nouvelles technologies de pointe et (iii) les modèles animaux disponibles pour étudier l'évolution naturelle de l'infection.

\subsection{Réplicons sous-génomiques et pseudo-particules virales}	
			\subsubsection{Les réplicons sous-génomiques}

Le premier modèle d’étude \textit{in vitro} du VHC a été rendu possible par la conception d’ARN sous-génomiques novateurs. Ce modèle expérimental est un « mini-génome » bicistronique, constitué d’un gène codant pour un marqueur de sélection (par exemple, conférant une résistance à un antibiotique) et du module réplicatif du VHC, \textit{i.e.} les séquences des protéines NS3 à NS5B sous le contrôle traductionnel de l’IRES provenant d’un autre virus (par exemple, le virus de l’encéphalomyocardite ou EMCV), encadrés des extrémités 5’ et 3’NC (\autoref{fig:fig26}\textcolor{blue}{.A}). Un réplicon basé sur les séquences de l’isolat Con1 de génotype 1b a été le premier à montrer une réplication stable, bien que de faible niveau, en lignée d’hépatome humaine Huh-7 \citep{RN294}. Le maintien de clones cellulaires hébergeant le réplicon Con1 sous pression de sélection a permis d'obtenir des variants dotés d’une meilleure aptitude réplicative et d’améliorer considérablement ce système. Plusieurs groupes ont découvert que ces variants sous-génomiques contenaient une ou plusieurs mutations génétiques dispersées dans l’ensemble de la région codante de NS3 à NS5B, essentielles pour amplifier la réplication de l’ARN \citep{RN322,RN314}. Les principales mutations adaptatrices identifiées dans les variants du réplicon Con1 ont pu être transférées à des réplicons sous-génomiques dérivant d’autres isolats de génotype 1, comme l’isolat H77 \citep{RN315} en raison des mécanismes d’action conservés. Il était cependant difficile de couvrir l’ensemble des génotypes par cette approche car la majorité des réplicons nécessitait des combinaisons variées de plusieurs mutations adaptatrices pour aboutir à une réplication génomique stable. En raison de ces difficultés, l’établissement de réplicons sous-génomiques couvrant la majorité des génotypes circulants n’a été publié que beaucoup plus tard : les premiers réplicons de génotypes 3 et 4 en 2013 \citep{RN316,RN318}, de génotypes 5 et 6 en 2014 \citep{RN319,RN320} et d’autres sous-types encore plus récemment \citep{RN321}. En plus des mutations compensatrices favorisant la réplication génomique, la permissivité de la cellule hôte constitue également un facteur déterminant pour établir un modèle d’étude du VHC efficace \textit{in vitro}. Plusieurs clones cellulaires hautement permissifs à la réplication du VHC ont été établis, tels que les lignées d’hépatomes humaines Huh-7.5 \citep{RN322} ou Huh-7-Lunet \citep{RN323}. Ces systèmes cellulaires offrent un environnement beaucoup plus favorable à la réplication du VHC, soit en augmentant le niveau d'expression des facteurs cellulaires indispensables à l’activité du module réplicatif, soit en ayant des défauts dans les voies de signalisation de l’immunité innée qui limitent habituellement la multiplication du virus. Dans le cas des cellules Huh-7.5, la permissivité élevée est liée à une mutation  faux-sens qui inactive la fonction du gène inductible par l’acide rétinoïque I (RIG-I), ayant pour conséquence une absence de réponse antivirale \citep{RN324}. \\ \\
\indent
Le développement des réplicons sous-génomiques a constitué une véritable révolution et a permis pour la première fois d’établir les détails moléculaires de la réplication génomique et de caractériser le réseau membranaire qui sous-tend cette étape du cycle viral \citep[pour revues,][]{RN326,RN328}. En effet, le groupe de C. Rice a identifié un site dans la protéine NS5A qui tolère l'insertion de la protéine fluorescente verte (GFP), ce qui a permis d’effectuer les premières études d'imagerie des sites putatifs de la réplication virale en cellules vivantes \citep{RN985,RN329}. Les réplicons sous-génomiques ont également été déterminants dans le développement des thérapies antivirales afin de caractériser l’effet des AAD sur la réplication génomique, à l’aide de réplicons codant des gènes rapporteurs comme la luciférase \citep[pour revue,][]{RN332}. De plus, la récente disponibilité des réplicons représentatifs des génotypes 2, 3, 4, 5 et 6 a permis d’évaluer l’efficacité des molécules antivirales contre la majorité des génotypes circulants et d’aboutir à l’élaboration des combinaisons d’AAD guérissant aujourd’hui >90\% des cas d’infection dans le monde. Bien que le modèle d’étude \textit{in vitro} basé sur les réplicons sous-génomiques ait été crucial pour le développement de médicaments, l'inconvénient de ce système est qu'il ne récapitule pas l’ensemble des étapes du cycle viral, notamment l’entrée ou l’assemblage et la sécrétion des particules infectieuses, une lacune importante pour la compréhension de la biologie du virus.

			\subsubsection{Les pseudo-particules virales}

Le deuxième modèle d’étude \textit{in vitro} du VHC exploite la technologie des pseudo-particules et représente une percée majeure dans l’étude du processus d’entrée virale. Cette méthode se base sur des particules rétro-virales défectives exprimant les glycoprotéines d’enveloppe E1 et E2 du VHC à leur surface \citep{RN333,RN334}. Pour produire ces pseudo-particules du VHC (VHCpp), des cellules 293T sont co-transfectées avec trois vecteurs d’expression codant : (i) les glycoprotéines E1 et E2 du VHC, (ii) les protéines Gag-Pol du VIH-1 ou du virus de la leucémie murine (VLM) et (iii) un génome rétroviral contenant un gène rapporteur tel que la luciférase ou la GFP pour détecter et quantifier facilement l’entrée du VHC dans les cellules hôtes (\autoref{fig:fig26}\textcolor{blue}{.B}). Les VHCpp miment le processus d’infection, ce qui a permis d’étudier le rôle des glycoprotéines E1 et E2 dans l’entrée du VHC, d’identifier et de valider les facteurs d’attachement et les récepteurs candidats et de mettre en lumière les mécanismes d’internalisation du virus \citep[pour revues,][]{RN335,RN336}. De plus, les VHCpp ont permis d’étudier les propriétés neutralisantes des anticorps spécifiques de E1 et E2 et d’identifier des épitopes de neutralisation à large spectre \citep{RN337,RN338}. Un inconvénient majeur à ce système, en plus de ne récapituler que les premières étapes du cycle viral, est qu’il repose sur un processus d’assemblage similaire à celui des rétrovirus. De ce fait, les pseudo-particules ne reproduisent pas l’association du VHC avec les apolipoprotéines et ne constituent donc pas un bon modèle pour étudier la morphologie et la structure des virions naturels.

	\begin{figureth}
	\centering
			\includegraphics[width=\linewidth]{Figure_26.png}
		\caption[Premiers modèles \textit{in vitro} pour étudier la réplication génomique et l’entrée du cycle viral]{\textbf{Premiers modèles \textit{in vitro} pour étudier la réplication génomique et l’entrée du cycle viral.} (A) Le réplicon sous-génomique initialement développé par Lohmann et al. code le gène de résistance à la néomycine (Neo) sous le contrôle traductionnel de l’IRES du VHC dans le premier cistron et les protéines du module réplicatif (NS3 à NS5B) sous le contrôle traductionnel de l’IRES hétérologue de l’EMCV dans le deuxième cistron. L’expression concomitante du gène de résistance à la néomycine permet de sélectionner les cellules hébergeant les réplicons par un traitement avec la généticine (G418). (B) Les pseudo-particules du VHC contiennent l’enveloppe authentique du virus avec les glycoprotéines E1 et E2 assemblées sur une capside rétrovirale et sont produites par la co-transfection de cellules 293T avec trois plasmides codant (i) les protéines E1 et E2 du VHC, (ii) les protéines Gag-Pol du VIH-1 ou du VLM et (iii) un génome rétroviral exprimant un gène rapporteur comme la luciférase (luc) ou la GFP (Adapté de \citealt{RN351,RN332}).}
				\label{fig:fig26}
	\end{figureth}
		\FloatBarrier

\subsection{Premiers systèmes infectieux en lignées d’hépatome humaines}	

Avec le succès sans précédent des réplicons sous-génomiques adaptés à la culture cellulaire, la communauté scientifique a naturellement imaginé un design analogue avec des ADNc couvrant la séquence complète du VHC, dans l’objectif de produire un modèle d’étude \textit{in vitro} du cycle intégral du virus. Des ARN de longueur génomique de génotype 1 ont donc été générés avec les mutations compensatrices favorisant la réplication, mais aucun de ces génomes n’a abouti à une production de particules virales malgré une réplication stable en cellules Huh-7 \citep{RN330}. Les hypothèses étaient que soit la lignée cellulaire utilisée n’était pas permissive à l’assemblage viral, de par l’absence de facteurs d’hôtes pro-viraux, soit les mutations qui favorisent la réplication interféraient avec la production de particules virales. À l’appui de cette seconde hypothèse, un groupe a montré que l’inoculation intra-hépatique de chimpanzés avec l’ARN génomique Con1 contenant les mutations adaptatrices n’établissait pas d’infection, tandis que l’inoculation avec le génome Con1 authentique provoquait une infection persistante \citep{RN331}. Une étude ultérieure a confirmé ces observations, indiquant que les mutations favorisant la réplication \textit{in vitro} sont bel et bien contre-productives \textit{in vivo} \citep{RN339} et qu’un système de culture cellulaire permissif à la réplication et à l’assemblage ne serait \textit{in fine} possible qu’avec une souche virale sauvage. \\ \\
\indent
Quelques années plus tard, un isolat particulier du VHC appartenant au génotype 2a (JFH-1) a été identifié chez un patient japonais atteint d’hépatite fulminante. En plus de montrer une  réplication génomique naturellement élevée \citep{RN340}, cette souche clinique s’est avérée capable de produire spontanément des particules infectieuses en cellules Huh-7 \citep{RN124}. Cet isolat constitue à ce jour la seule souche clinique capable de reproduire naturellement un cycle infectieux complet \textit{in vitro}, établissant une nouvelle source clonale de virus (VHCcc). Bien que la réplication génomique de la souche JFH-1 était particulièrement efficace et ne requérait pas de mutation d’adaptation, les rendements globaux d’infectiosité étaient relativement faibles. Afin d’améliorer ce nouveau modèle viral, des tentatives d’adaptation en culture de la souche JFH-1 ont été menées en vue d’acquérir des variants plus robustes sur le plan infectieux. Plusieurs groupes ont fait état de mutations qui résident non seulement dans séquences des protéines structurales, mais aussi dans celles de p7, NS2 et NS5A, ce qui est cohérent vis-à-vis du rôle de ces protéines dans la coordination de la morphogénèse virale \citep{RN361,RN362}. Le variant JFH-1 adapt (Jad), adapté en culture de cellules Huh-7.5, constituait le mutant le plus robuste avec une combinaison de trois mutations affectant les protéines NS5A et NS5B et conférant un titre infectieux 100 fois plus élevé par rapport au clone original \citep{RN342}. Ces mutations semblent améliorer la transition entre les processus de réplication génomique et d’assemblage des particules virales. Une autre stratégie menée en parallèle pour améliorer le modèle JFH-1 reposait sur la production de dérivés chimériques, par la substitution de séquences hétérologues provenant d’autres souches cliniques.  Une des premières constructions hybrides fonctionnelles s’est basée sur la combinaison des régions NS3 à NS5B du génome JFH-1 avec les régions des protéines Core à NS2 de l’isolat J6 de même génotype, produisant un génome hybride J6/JFH-1 \citep{RN343}. Le dogme établi était alors de séparer les protéines impliquées dans la morphogénèse et les protéines du complexe de réplication. Pour déterminer les meilleures jonctions pour l’hybridation, une étude a cartographié l’ensemble de la séquence NS2 et a mis en évidence que le site de fusion le plus optimal se situait immédiatement en aval du premier segment transmembranaire \citep{RN344}. Le génome chimérique J6/JFH-1 a alors été amélioré en déplaçant le point de jonction au sein de NS2, à l’origine de la souche Jc1 qui constitue aujourd’hui un des modèles d’infection \textit{in vitro} les plus répandus dans le domaine. \\ \\
\indent
La découverte exceptionnelle de l’isolat JFH-1 après 25 années d’essais à cultiver des souches cliniques \textit{in vitro} a permis d’ouvrir une nouvelle voie pour l’étude de la biologie du virus, et de finalement décortiquer les dernières étapes du cycle viral, à savoir les mécanismes d’assemblage, de sécrétion et de propagation des particules du VHC \citep[pour revues,][]{RN345,RN346}. L’étude de la structure des virions JFH-1 et Jc1 par des méthodes de capture de haute affinité a également permis d’obtenir les premiers détails concernant la morphologie et la composition atypique des particules infectieuses du VHC \citep{RN282,RN284,RN285}. Les modèles disponibles se limitaient alors à des souches de génotype 2a, qui ne pouvaient être à elles seules pertinentes sur le plan clinique. La construction de chimères intergénotypiques a alors été entreprise mais a généralement été moins immédiate car elle nécessitait une adaptation complémentaire en culture, en raison de problèmes de compatibilité génétique entre les protéines virales des différents génotypes. Néanmoins, le groupe de J. Bukh a produit avec succès des génomes hybrides fonctionnels exprimant les séquences Core à NS2 des génotypes 1 à 7 \citep{RN350,RN352}. Plus tard, d’autres modèles intergénotypiques ont été développés à partir du génome J6/JFH-1, en substituant les régions codant NS3-4A, NS5A, NS5B ou encore  toutes les régions à l’exception de NS3-4A et NS5B par les séquences équivalentes d’autres génotypes \citep[pour revue,][]{RN351}. En parallèle, d’autres groupes ont cherché à établir des modèles d’infection \textit{in vitro} pour tous les génotypes du VHC. Ce travail s’est avéré particulièrement épineux, car les isolats cliniques nécessitaient des schémas complexes de mutations d’adaptation, à la fois pour améliorer la réplication génomique et pour établir une production de particules virales. Le premier succès a été obtenu avec un génome H77 hautement adapté (H77-S), qui contient un total de 5 mutations compensatrices dispersées dans la séquence virale \citep{RN353}. Plus tard, des variants hyper-adaptés de génotypes 2a (à partir de l’isolat J6) \citep{RN354}, 2b \citep{RN355}, 3a \citep{RN356,RN357} et 6a \citep{RN358} ont été développés. Ces virus, dotés de l’ordre d’une vingtaine de mutations d’adaptation, présentent toutefois un titre infectieux 10 à 100 fois inférieur à celui de la souche sauvage JFH-1 et une proportion plus importante de particules défectives. Néanmoins, l’émergence de ces nouveaux modèles \textit{in vitro} du VHC constitue un atout futur pour caractériser les propriétés pathogéniques spécifiques à certains génotypes, en particulier au génotype 3, associés au développement de la stéatose hépatique et à une progression avancée de la maladie vers les stades de la cirrhose \citep{RN359,RN360}.

\subsection{Cultures d’hépatocytes primaires, cellules souches et organoïdes}	

Jusqu’à présent, les sous-clones permissifs dérivant de la lignée d’hépatome humaine Huh-7 constituent le modèle cellulaire de référence pour étudier le VHC. Néanmoins, s’agissant d’une lignée immortalisée issue d’un carcinome hépatique, ces cellules ne sont pas polarisées ce qui ne permet pas de reproduire la compartimentation des co-récepteurs du VHC ni l’orientation des systèmes sécrétoires des hépatocytes matures. Elle ne constitue pas non plus le modèle le plus pertinent pour étudier certains aspects de l’interaction virus-hôte, tels que l’activation de la réponse immunitaire innée ou la carcinogenèse. L'objectif futur est donc de développer des modèles cellulaires plus physiologiques pour l’étude \textit{in vitro} du VHC. \\ \\
\indent
Les cellules primaires d’hépatocytes humains (PHH) sont les cellules hôtes du VHC lors de l’infection naturelle et sont cultivables \textit{ex vivo}. Toutefois, ces cellules sont difficiles à obtenir car elles proviennent de parties non tumorales de résections hépatiques et sont très variables en fonction du sexe, de l’âge, de l’exposition à diverses substances et des polymorphismes génétiques du donneur. Une fois mises en culture, les PHH ne se divisent pas et perdent rapidement les caractéristiques biologiques d’hépatocytes matures par dédifférenciation, ce qui limite leur durée de vie en monocouche à environ deux semaines \citep{RN363}. Pour prolonger leur durée de vie, certains groupes ont eu recours à des méthodes d’immortalisation par l’introduction d’oncogènes \citep{RN364}. Des progrès considérables ont été réalisés dans le but de générer des hépatocytes matures par différenciation de cellules souches pluripotentes somatiques (IPSC) ou embryonnaires (EPSC) permettant un approvisionnement illimité en cellules et un modèle plus reproductible par rapport aux PHH de donneurs adultes. En général, ces modèles ne sont pas hautement permissifs à l’infection par le VHC, notamment en raison des voies immunitaires intactes de l’hôte. En effet, plusieurs groupes ont documenté un faible taux d’infection de PHH obtenues par résections \citep{RN365} ou dérivant des EPSC/IPSC \citep{RN366,RN367,RN368} à partir d’inoculum de VHCcc et une clairance virale au bout d’une dizaine de jours. \\ \\
\indent
Depuis que la technologie des cellules souches a été établie, une perspective innovante a été le développement de foies humains vascularisés et fonctionnels par des protocoles d’organogénèse \textit{in vitro}. Les organoïdes hépatiques peuvent être dérivés à partir de cellules souches pluripotentes (telles que les EPSC/IPSC) ou de cellules souches bipotentes progénitrices du foie \citep{RN369,RN370}. Les cellules souches sont cultivées dans des matrices extracellulaires et mises en contact avec des facteurs de croissance qui favorisent l’auto-organisation spontanée du tissu (\autoref{fig:fig27}). Cette approche permet de structurer naturellement un réseau constitué des trois principales cellules non parenchymateuses du foie (cellules stellaires, cellules de Kupffer et cellules endothéliales) et des hépatocytes, reproduisant \textit{in vitro} l’architecture complexe de l’organe et l’hétérogénéité cellulaire contrairement aux cultures 2D d’hépatocytes  \citep{RN371,RN372}. Ces structures organoïdes peuvent être maintenues à long terme en culture \textit{in vitro} et continuer à refléter, même après de nombreuses générations, les marqueurs de différenciation du tissu d'origine  \citep{RN373}. Les organoïdes matures peuvent être également transplantés dans des souris immunodéficientes pour générer des souris avec un foie humanisé (voir \autoref{section:animaux}). Récemment, la technologie des organoïdes s'est imposée comme l'outil de culture cellulaire de pointe pour l'étude de la biologie humaine dans le domaine de la santé et des pathologies  \citep[pour revues,][]{RN374,RN375}. Le principal défi de ces modèles pour étendre leur utilisation à l’étude des pathologies infectieuses humaines est de parvenir à établir l’infection naturelle avec l’agent étiologique. Dans le cadre de l’hépatite C, l’induction des réponses immunitaires limite fortement la mise en place d’une infection productive et persistante par le virus. À l’heure actuelle, seuls les organoïdes dérivant de cellules Huh-7.5 sont permissifs à l’infection par le VHC \citep{RN376}, et même si ils rétablissent certains marqueurs caractéristiques des hépatocytes matures lors de la mise en culture 3D (comme la polarisation et les jonctions serrées), ils restent globalement moins physiologiques que les modèles basés sur les hépatocytes humains en culture primaire. \\

	\begin{figureth}
	\centering
			\includegraphics[width=\linewidth]{Figure_27.png}
		\caption[Protocole d’organogénèse \textit{in vitro}pour générer des organoïdes hépatiques et biliaires]{\textbf{Protocole d’organogénèse \textit{in vitro} pour générer des organoïdes hépatiques et biliaires.} Les organoïdes du foie peuvent dériver de différentes sources telles que les cellules bipotentes progénitrices du foie ou les cellules souches pluripotentes. Les cellules progénitrices peuvent être obtenues à partir du tissu hépatique adulte par résection chirurgical, prélèvement de biopsie ou de bile ou à partir des stades embryonnaires de l’organogenèse et stimulées pour former des organoïdes hépatiques (hépatocytes) ou des organoïdes biliaires (cholongiocytes) après incubation dans une matrice extracellulaire avec une combinaison définie de facteurs de croissance. Les cellules souches pluripotentes d’origine embryonnaire (EPS) ou somatique (IPS) nécessitent d’abord un protocole de différenciation en 3 étapes pour générer des hépatoblastes qui sont ensuite incorporés dans une matrice extracellulaire pour promouvoir la croissance et la formation d’organoïdes. BMP4 : protéine morphogénétique osseuse 4 ; EGF : facteur de croissance épidermique ; FGF : facteur de croissance fibroblastique ; GSK3-i : inhibiteur de la glycogène synthase kinase 3 ; HGF : facteur de croissance hépatocytaire ; PI3K-i, inhibiteur de la phosphoinositide 3-kinase ; RA, acide rétinoïque ; TGFa : facteur de croissance transformant α ; TGFBR1-i, inhibiteur du récepteur 1 du facteur de croissance transformant ß.  (Adapté de \citealt{RN375}).}
				\label{fig:fig27}
	\end{figureth}
		\FloatBarrier

\clearpage

\subsection{Modèles animaux sauvages et humanisés}	
		\label{section:animaux}
		
Historiquement, la découverte et l’étude des agents responsables d’hépatites virales ont grandement bénéficié de l’exploitation de primates non humains comme modèle permissif à l’infection aiguë et chronique \citep{RN377,RN378}. Les chimpanzés sont la seule espèce animale autre que l’homme à être naturellement susceptible à l’infection par le VHC et représentaient donc le seul modèle acceptable pour l’évaluation des stratégies thérapeutiques et vaccinales. Depuis 2012, les nouvelles directives éthiques restreignent fortement l’emploi des chimpanzés en tant que modèle d’expérimentation animale pour le VHC et limitent leur utilisation aux États-Unis \citep{RN379}. \\ \\
\indent
Les souris transgéniques humaines ou les souris chimériques à foie humain sont alors devenus les modèles d’expérimentation animale de référence pour l’étude du VHC \textit{in vivo}. Les souris sont naturellement résistantes à l’infection par le VHC, en raison des divergences entre les facteurs  hépatocytaires murins et humains essentiels pour l’entrée virale \citep[pour revue,][]{RN381} et la présence de facteurs de restriction limitant la multiplication virale dans les hépatocytes murins \citep{RN382}. Une méthode qui a permis de surmonter la barrière d’espèce consiste à réaliser une humanisation directe du foie murin par xénotransplantation avec des PHH ou des organoïdes hépatiques humains cultivés \textit{in vitro} (\autoref{fig:fig28}\textcolor{blue}{.A}) \citep{RN385}. Pour faciliter le maintien de la greffe humaine, ce genre d’approche doit être effectuée dans un contexte très immuno-suppresseur, donc à partir de souris déficientes pour les réponses lymphocytaires B, T et NK. Ces lignées de souris expriment également des transgènes hépatotoxiques qui vont induire simultanément des lésions dans la région hépatique murine, déclenchant la production d’hormones régénératrices qui vont stimuler la prolifération et la colonisation des PHH dans tout l’organe. Une autre stratégie pour développer des souris susceptibles à l’infection avec moins de difficultés techniques consiste à introduire transgénétiquement les facteurs homologues humains requis pour l’entrée du virus \citep{RN380}. Dans ce contexte, une lignée de souris transgéniques exprimant les gènes humains codant pour les co-récepteurs CD81, SRB1, CLDN1 et OLCN a été développée et s’est montrée permissive pour l’entrée du VHC \citep{RN383}. Afin de récapituler une infection persistante du virus, il était néanmoins nécessaire de combiner l'expression des facteurs humains d'entrée du VHC avec une atténuation des réponses immunitaires innées de la souris, en utilisant des souris immuno-déficientes (\autoref{fig:fig28}\textcolor{blue}{.B}) \citep{RN384}. Les souris transgéniques et transplantées deviennent alors permissives à l’infection par le VHC mais aucune pathologie n’a été observée malgré la persistance du virus pendant plusieurs mois, probablement en raison de l’absence d’inflammation qui s'établit dans le foie au cours de l’infection chronique chez l’homme. Néanmoins, ces modèles ont été très utiles pour valider de nombreux aspects moléculaires du cycle viral \textit{in vivo} ou pour évaluer l’efficacité des anticorps neutralisants  \citep{RN386,RN387,RN388} ou des combinaisons thérapeutiques en phase pré-clinique \citep{RN389,RN390,RN391}. Plus tard, un groupe a reproduit cette approche sur des souris à fond génétique immuno-compétent (ICR) qui ont développé une virémie soutenue pendant plus de 12 mois parallèlement à des lésions hépatiques signant une progression vers la fibrose et la cirrhose \citep{RN392}. Toutefois, ces données doivent être reproduites indépendamment afin de confirmer l’établissement du premier modèle viable \textit{in vivo} pour étudier la pathogénèse du VHC \citep[pour revue,][]{RN393}. D’autres modèles de souris doublement humanisées par transplantation de PHH et de cellules hématopoïétiques humaines sont en cours de développement, afin d’établir un système pour valider les stratégies vaccinales en phase pré-clinique, mais elles ne s’avèrent pas encore permissives à l’infection par le VHC \citep{RN394,RN395}. \\
	\begin{figureth}
	\centering
			\includegraphics[width=\linewidth]{Figure_28.png}
		\caption[Modèles de souris transgéniques et chimériques pour étudier l’infection naturelle du VHC \textit{in vivo}]{\textbf{Modèles de souris transgéniques et chimériques pour étudier l’infection naturelle du VHC \textit{in vivo}.} (A) Modèle de souris chimériques par transplantation intrasplénique d’hépatocytes humains primaires qui vont coloniser le foie après induction de lésions dans le foie murin. Le fond génétique immuno-déficient permet aux PHH de coloniser jusqu’à 90\% du foie de l’animal, rendant ce modèle permissif à l’infection persistante du VHC. (B) Lignée de souris génétiquement humanisées sur fond génétique STAT1 -/- exprimant les récepteurs humains du VHC (EFT : entry factor transgenic) et récapitulant toutes les étapes du cycle viral. Une persistance virale est établie par la détection d’une virémie jusqu’à 90 jours post-infection (p.i). (Adapté de \citealt{RN351}).}
				\label{fig:fig28}
	\end{figureth}
		\FloatBarrier

\clearpage

%%%%%%%%%%%%%%%%%%%%%%%%%%%%%%%%%%%%%%%%%%%%%%%%%%%%%%%%%%%%%%%%%%%%%%%%%%%%%%%%%%%%%%%%%%%%%%%%%%%%%%%%%%%%%%%%%%%%%%%%%%%%%%%%%%%%%%%%%%%%%%%%%%%%%%%%%%%%%%%%%%%%

% 8 - La protéine de capside Core

\section{La protéine de capside Core}	
		\label{section:core}

	\subsection{Structure et topologie de la protéine Core}
	
La protéine Core constitue l’élément principal de la capside qui entoure et protège le génome viral. La forme initiale immature de Core est une protéine de 191 acides aminés (23kDa), constituée de deux domaines principaux D1 et D2 et d’un domaine provisoire D3 (\autoref{fig:fig29}) \citep[pour revue,][]{RN399}. Le domaine N-terminal D1 (résidus 1-117) est une structure hydrophile très flexible, subdivisée en trois sous-domaines basiques BD1, BD2 et BD3 riches en résidus arginine, lysine, glycine et proline. Il possède des propriétés de fixation à l’ARN viral, assurant l’incorporation du génome dans la nucléocapside \citep{RN1078}. La formation de la capside nécessite l’oligomérisation de Core qui est également assurée par le domaine D1 \citep{RN300}. De plus, la flexibilité de cette structure permet au domaine D1 d’interagir avec de nombreux partenaires cellulaires, ce qui pourrait contribuer à la pathogénèse \citep{RN1112}. Le domaine central D2 (résidus 118-177) contient deux hélices amphipatiques séparées par une boucle hydrophobe fortement enrichie en résidus leucine et alanine. Cette région, qui a une affinité pour les monocouches lipidiques, confère à Core la capacité d’ancrage aux gouttelettes lipidiques (GL). La région d’interaction avec les GL se situe au niveau des résidus 138 et 169 \citep{RN304}. Les 20 résidus C-terminaux qui constituent le domaine D3, forment un segment transmembranaire basique responsable de la rétention de Core au RE après traduction. Il contient une séquence signal qui sera clivée par la SPP, libérant la protéine Core mature de 177 acides aminés (21kDa) \citep{RN303,RN955}. À ce jour, la structure tridimensionnelle de Core n’a pas encore été résolue à l’exception d’une région partielle du peptide signal synthétique du précurseur Core-E1 par résonance magnétique nucléaire (RMN), qui a permis de fournir la base structurelle du mécanisme du clivage par la SPP \citep{RN1085}. Des produits de traduction provenant d'un cadre de lecture alternatif chevauchant la région codante principale, notamment la protéine core +1, ont été identifiés \citep[pour revue,][]{RN1084}, mais l’implication de ces protéines dans le cycle de vie et la pathogenèse du VHC reste à élucider.

	\begin{figureth}
	\centering
			\includegraphics[width=\linewidth]{Figure_29.png}
		\caption[Organisation de la protéine de capside Core du virus de l'hépatite C]{\textbf{Organisation de la protéine de capside Core du virus de l'hépatite C.} La protéine Core est divisée en trois domaines D1 (résidus 1-117), D2 (118-176) et D3 (177-191). Les positions des sous-domaines basiques (BD1, BD2 et BD3), des hélices alpha (H1 et H2) et de la boucle hydrophobe (HL) sont indiquées. Le clivage par la peptidase du peptide signal (SPP) est indiqué par une flèche. (D'après \textit{Hepatitis C Online} \copyright\ 2021, illustration par J. Travnicek et D. Ehlert).}
				\label{fig:fig29}
	\end{figureth}
		\FloatBarrier

	\subsection{Implication de la protéine Core dans le développement des pathologies hépatiques}
		\label{section:corepatho}

La fonction principale de la protéine Core dans le cycle viral est de former la nucléocapside, qui constitue la première étape de l’assemblage des particules virales infectieuses (décrite dans la \autoref{section:assemblage}). Toutefois, plusieurs études mettent en évidence que le rôle de la protéine Core du VHC ne se limite pas à l’assemblage de la capside, mais que celle-ci a également un rôle central dans la pathogénèse \citep[pour revues,][]{RN1123,RN1120}. Ces évidences expérimentales étaient disponibles avant le développement des réplicons sous-génomiques et du premier système infectieux basé sur la souche JFH-1 et reposaient donc principalement sur l’expression isolée de protéines Core recombinantes \textit{in vitro} ou dans des modèles de souris transgéniques. Les premiers témoignages expérimentaux à ce sujet datent de la fin des années 1990, où Core a été directement associée au développement de la stéatose hépatique à l’âge de 3 mois et à des nodules hépatiques caractéristiques des altérations malignes à l’âge de 16 mois dans des lignées de souris transgéniques \citep{RN1117,RN1118,RN1121,RN1122}. En effet, l’expression de Core semble affecter les voies de signalisation cellulaire, les réponses pro-apoptotiques et pro-inflammatoires, le métabolisme des lipides et des lipoprotéines et la prolifération et la transformation cellulaire, ce qui représente des facteurs de risque important pour la stéatose et le CHC. L’un des moteurs clés de la pathogénèse de Core est sa capacité à interagir avec un large éventail de protéines, enzymes et de facteurs de transcription cellulaires \citep{RN1112}. Ces interactions peuvent perturber la fonction des protéines ciblées et l’expression des gènes cellulaires, pouvant altérer \textit{in fine} de nombreux processus biologiques. Bien que les autres protéines virales puissent contribuer à la pathogénèse virale, en particulier NS5A qui interfère également avec de nombreux facteurs d’hôte \citep{RN1141}, l’expression isolée de Core en culture cellulaire semble refléter un grand nombre des dérégulations observées dans les biopsies de foie de patients infectés, qui seront détaillés dans la \autoref{section:pathogenese}.

	\subsection{Mécanisme de pathogénèse de la protéine Core}
	\label{section:pathogenese}

La protéine Core est supposée moduler l'homéostasie lipidique en augmentant la lipogenèse via l'activation de SREBP et en réduisant l'oxydation et l'exportation des lipides \citep{RN468}. Core diminue également l'expression du récepteur $\alpha$ activé par les proliférateurs du peroxysome (PPAR$\alpha$), un récepteur nucléaire régulant plusieurs gènes responsables de la dégradation des acides gras \citep{RN1126}. De plus, Core inhibe l’activité de la protéine de transfert des triglycérides microsomaux (MTP) en se délocalisant transitoirement à la membrane des mitochondries, ce qui réduirait l’assemblage et la sécrétion des VLDL \citep{RN1127}. Cependant, la protéine Core n’étant pas détectée au niveau des mitochondries dans un système d’infection par la souche JFH-1, la pertinence de cette localisation et de cette fonction particulière reste débattue \citep{RN486}. En revanche, l’ensemble de ces données concordent avec le fait que l’expression transitoire de Core serait liée à une régulation positive de la synthèse des triglycérides qui se traduit par une accumulation et un un élargissement des GL cytosoliques, ce qui pourrait contribuer à l’apparition des signes histologiques de la stéatose \citep{RN1139}. Il a également été suggéré que la protéine Core favorise la prolifération cellulaire, l'apoptose, l'angiogenèse et la tumorigenèse en dérégulant un large éventail de gènes tels que le facteur de croissance transformant ß (TGF-ß), le facteur de croissance endothélial vasculaire (VEGF), la voie Wnt/ß-caténine et la cyclo-oxygénase-2 (COX-2), ce qui peut favoriser le développement et le caractère invasif du CHC. Core semble également pouvoir moduler l’expression de proto-oncogènes cellulaires, comme la protéine p53, et l’abondance des micro-ARN qui participent aux réponses anti-tumorales, tel que le miR-122 hépatique, qui possède la propriété intrigante de réguler les gènes suppresseurs de métastases. L’apoptose est un processus cellulaire essentiel dans le contrôle des infections virales, en éliminant le contenu de la cellule contaminée par phagocytose, sans exposer les molécules intracellulaires qui déclencheraient des réponses inflammatoires. Compte tenu de son rôle, Il est fréquent que certains virus aient des stratégies pour éviter l'élicitation de l'apoptose. L’infection par le VHC a une interaction très complexe et mal comprise avec l’apoptose. En effet, L'expression des facteurs cellulaires régulant l'apoptose varie selon la phase aigue ou chronique de l’hépatite C chez les patients \citep{RN1124}. Core présente des propriétés pro-apoptotiques et anti-apoptotiques, telles que sa capacité à stimuler ou à inhiber les voies intrinsèques ou extrinsèques médiées par p53 ou par le facteur de nécrose tumorale $\alpha$ (TNF$\alpha$) \citep{RN1131}.  \\ \\
\indent
Core est la protéine la plus conservée de toutes les protéines du VHC entre les 8 génotypes. Toutefois, des polymorphismes naturels retrouvés dans sa séquence ont été corrélés à des degrés de dérégulations cellulaires plus importants et à l’apparition plus fréquente des complications hépatiques. En effet, des résidus spécifiques aux protéines Core de génotype 1b et 3a ont été liées à une progression plus fréquente vers la stéatose et le CHC \citep{RN989,RN1134,RN1129}. À titre d’exemple, le résidu Phe à la position 164 et la combinaison des résidus Phe/Ile aux positions respectives 182 et 186, retrouvées dans les séquences Core de génotype 3a, ont été associées à un dérèglement accru du métabolisme des lipides et à une régulation positive de la biogénèse des GL \citep{RN990,RN991}. Le résidu Gln à la position 70 fréquent au sein des séquences Core de génotype 1b s’avère être un prédicteur important de l’évolution vers le CHC \citep{RN988} et le résidu Thr à la position 71 a été récemment associée à une dérégulation accrue de la voie Wnt/ß-caténine \citep{RN808}. Enfin, une étude récente met en évidence que l’expression des gènes codant PPAR$\alpha$ et TGF-ß est davantage stimulée par les souches virales hyper-adaptées de génotype 3a par rapport à celles de génotype 1a \citep{RN1128}. L’ensemble de ces évidences expérimentales appuient les études cliniques évoquées dans la \autoref{section:classification}. En revanche, il est important de noter que la pertinence de ces expériences \textit{in vitro} ou \textit{in vivo}, impliquant majoritairement une surexpression de la protéine Core, reste incertaine dans le cadre de l’infection naturelle par le VHC.