%% Copyright (C) 2017-2021 Emeline Simon
%%
%% The current owner of this work is Emeline Simon
%% <contact at emeline.simon@gmail.com>.
%%
%% This is 03_discussion.tex the third chapter for my PhD Thesis.
%%
%%%%%%%%%%%%%%%%%%%%%%%%%%%%%%%%%%%%%%%%%%

\chapter{Discussion et perspectives}
	\minitoc
	\newpage

%%%%%%%%%%%%%%%%%%%%%%%%%%%%%%%%%%%%%%%%%%%%%%%%%%%%%%%%%%%%%%%%%%%%%%%%%%%%%%%%%%%%%%%%%%%%%%%%%%%%%%%%%%%%%%%%%%%%%%%%%%%%%%%%%%%%%%%%%%%%%%%%%%%%%%%%%%%%%%%%%%%%%%%%

\section{Discussion}

Globalement, nos recherches approfondies combinant analyse quantitative par imagerie confocale et utilisation de variants de Core obtenus par mutagènese dirigée démontrent le rôle instrumental jusque là présumé de la protéine Core dans la clusterisation, la redistribution et l’élargissement des GL induits lors de l'infection par le VHC. En outre, il a été démontré que les fonctions de la protéine Core ne se limitent pas à l’assemblage de la capside, mais que celle-ci a également un rôle central dans la pathogénèse. Le développement de modèles pionniers dans le domaine d’étude du VHC, \textit{i.e} les virus intergénotypiques codant des protéines Core hétérologues, ont permis d’explorer le lien direct potentiel entre l’origine génotypique de Core et la stéatose hépatique par imagerie confocale et par crible transcriptomique à haut débit.

\subsection{Analyse combinée par microscopie confocale quantitative et mutagénèse dirigée pour étudier le rôle instrumental de la protéine Core dans la dérégulation des GL au cours de l’infection par le VHC}
\label{section:discussion1}

Les premières observations microscopiques de la localisation de la protéine Core du VHC ont été réalisées avec différentes lignées cellulaires (CHO, HepG2, BHK-21, Huh-7, U-2 OS) dans lesquelles les protéines Core de génotype 1a (souche \textit{Glasgow}) ou 1b (souche japonaise) étaient exprimées de façon stable, transitoire ou inductible \citep{RN485,RN1045,RN1055}. Dès lors, ces études mettent en évidence pour la première fois une interaction singulière de la protéine Core du VHC avec les GL, unique à l’époque chez les \textit{Flaviviridae} et les virus humains, jusqu’à la description d’une interaction de la protéine de capside du DENV avec les GL \citep{RN1037}. Cette localisation, ne serait-ce que transitoire, a été confirmée plus tard avec des systèmes infectieux pour la souche clinique JFH-1 de génotype 2a se répliquant spontanément en culture cellulaire [\citep{RN487}, la souche hybride Jc1 exprimant Core de la souche J6 de génotype 2a \citep{RN495} et plus récemment pour des virus hyper-adaptés de génotypes 1a, 2a, 2b, 2c et 3a \citep{RN1060}, suggérant un mécanisme universel à tous les génotypes. Le recrutement de la protéine Core à la surface des GL est une étape indispensable à la morphogenèse du VHC \citep{RN487,RN488,RN489}. \\ \\
\indent
Au démarrage de cette étude, la protéine Core a déjà été incriminée dans plusieurs processus de dérégulations des GL à des degrés variables selon les modèles. Toutefois, les différents travaux menant à ces découvertes n’apportaient ni de dimension quantitative précise, ni de mécanisme viral concret pouvant être à l’origine de ces dérégulations. Grâce aux récentes avancées dans les approches bio-informatiques applicables en imagerie, et en développant un algorithme novateur sur Python, nous sommes parvenus à quantifier pour la première fois l'évolution au cours de l'infection des évènements d'association de Core avec les GL et de leurs conséquences sur la dynamique des GL. Ainsi, nous avons mis en évidence que l’infection par la souche Jad augmente de 20\% la proportion de GL formant des clusters et double quasiment le nombre de contacts entre elles, au temps le plus tardif étudié. En parallèle, la taille moyenne des GL double, voire triple sans changement de leur contenu global, tandis que leur nombre est proportionnellement réduit. Ces effets apparaissent progressivement au cours de l’infection et sont particulièrement marqués à partir de 3 et 4j p.i. et sont cohérents avec l’étude de Boulant et al., montrant que la souche JFH-1 induit un regroupement, mais pas d’accumulation visible des GL au bout de 3j p.i. \citep{RN930}. Dans les modèles d’expression ectopique de Core de génotype 1a ou 1b, la redistribution et la clusterisation des GL s’effectue en moins de 16h et s’accompagne d’une augmentation du contenu en GL \citep{RN1040,RN947}. Notre étude de corrélation entre les paramètres relatifs à Core et aux GL souligne une variation proportionnelle entre le taux de clusterisation des GL et la quantité de Core recouvrant leur surface, ce qui explique probablement les divergences relevées dans l’ampleur du regroupement et de la redistribution des GL entre les systèmes de forte expression ectopique de Core et d’infection. Par ailleurs, contrairement à ce qui a été noté dans ces systèmes d’expression transitoire \citep{RN1041,RN947}, nous ne détectons pas la protéine Core dans le noyau. Nous observons parfois de rares GL qui transitent au sein du noyau dans la lignée Huh-7.5, mais elles apparaissent totalement dénuées de Core, soulignant qu’une surexpression isolée de la protéine peut mener à des effets biologiques différents de l’infection. \\ \\
\indent
Nos analyses fines par imagerie ont mis en évidence que la protéine Core est localisée préférentiellement dans les zones de contact entre les GL appariées, et que seule la population de GL mobilisée par Core est soumise à cette clusterisation et à cet élargissement progressif, pointant une action potentiellement directe de la protéine Core dans ces dérégulations cellulaires. La réduction ou l’absence respective de ces effets avec des mutants viraux déficients pour l’assemblage, pour lesquels Core est partiellement ou totalement délocalisée de la surface des GL, confirme le rôle instrumental de cette protéine virale dans la dérégulation des GL au cours de l’infection par le VHC. À l’heure actuelle, le mécanisme et le rôle biologique du recrutement de la protéine Core à la surface des GL, ainsi que l’implication de ces organites dans l’assemblage des particules virales ne sont pas encore précisément établis. Notre étude a permis d’établir certaines hypothèses en rapport avec ces différents points, qui seront discutées lors des paragraphes ci-dessous.
\clearpage

	\subsubsection{Recrutement de la protéine Core à la surface des gouttelettes lipidiques cytosoliques : un mécanisme identifiable à celui des protéines de classe I ?}

La protéine Core contient des éléments fréquemment retrouvés dans les protéines de classe I associées aux GL comme les oléosines végétales, tels que des hélices amphipatiques, un noeud de résidus proline (formé par les résidus P138 et P143) et un motif Y/FATG \citep{RN304,RN957}. D’après nos données de mutagénèse dirigée, empêcher le recrutement de la protéine Core mature à la surface des GL par l’insertion de mutations dans le noeud de résidus proline et dans le motif YATG du domaine D2 (Core DP_SATG) déstabilise fortement la protéine. Ce résultat est en accord avec des stratégies visant à empêcher le recrutement de la protéine Core mature aux GL en mutant les hélices amphipatiques, qui aboutissaient à une dégradation protéolytique quasi-complète de la protéine \citep{RN304}. Une première hypothèse est que la protéine Core mature serait libérée transitoirement dans le cytoplasme, car les hélices amphipatiques présentes dans le D2 de Core (de l’ordre de 20aa) ne seraient pas suffisamment longues pour maintenir la protéine mature dans la bicouche du RE, contrairement aux hélices amphipatiques des oléosines végétales (de l’ordre de 72aa) \citep{RN609}, qui restent stablement ancrées dans le RE après la substitution des résidus proline \citep{RN1069}. Une autre hypothèse est que les mutations combinées dans le noeud de résidus proline et du motif Y/FATG perturberaient la topologie correcte de la protéine dans la bicouche du RE, ce qui induirait un stress membranaire et le déclenchement de l’UPR. Quelle qu'en soit la raison, la protéine C_DP_SATG est ciblée et dégradée par la machinerie protéasomale. Seules les mutations du noeud de proline dans la protéine Core (C_DP) ont permis de produire une protéine mutante relativement stable dans le contexte de la souche Jad, pour laquelle nous détectons une co-localisation modeste de l’ordre de 20\% de C_DP avec les GL. Il est difficile de conclure s’il s’agit d’un ancrage fonctionnel à la surface des GL pour cette faible proportion de protéine C_DP, ou d’une rétention à des sites de transfert entre RE et GL, comme évoqué par \citet{RN930}. Cependant, nous avons montré une prédisposition de C_DP à induire une clusterisation et une redistribution partielle des GL à proximité des usines de réplication virale, contrairement à la protéine Core_DP_SATG, ce qui favorise la première hypothèse. La localisation partielle de la protéine C_DP à la surface des GL ou sa rétention dans les sites de transfert du RE protègerait la protéine de la protéolyse. À l’inverse, la protéine C_DP_SATG, dépourvue de distribution ponctuée, serait incapable d’atteindre les sites de transfert entre le RE et les GL et serait rapidement engagée dans la dégradation protéasomale. Les quelques agrégats visibles de C_DP_SATG dispersés dans le cytoplasme se comportent peut-être comme des corps d'inclusion à court terme. L’ensemble de ces résultats et des données disponibles dans la littérature soutiennent que la protéine Core se comporte comme une protéine de classe I, diffusant à travers la bicouche du RE vers les GL par un recrutement actif, après le clivage protéolytique du peptide signal par la SPP. Le recrutement actif de Core vers les GL en cours de biosynthèse serait médié par la protéine cellulaire DGAT1, une enzyme constitutive du RE impliquée dans la biosynthèse \textit{de novo} des lipides neutres \citep{RN944,RN945}. \\ \\
\indent
À ce jour, la question du « timing » du recrutement de Core dans le cycle de vie des GL reste en suspens : s’effectuerait-il uniquement au moment de la biosynthèse \textit{de novo} des GL ou les GL pré-existantes pourraient-elles acquérir Core lors d’échanges ultérieurs avec des zones enrichies en protéine Core sur la membrane externe du RE ? Premièrement, notre étude du profil transcriptomique montre une réduction globale de l’expression des enzymes impliquées dans la synthèse \textit{de novo} de lipides neutres dans les hépatocytes infectées, la première étape qui déclenche la formation de nouvelles GL, ce qui semblerait contre-productif pour distribuer les protéines Core synthétisées en continu à leur organite cible. Deuxièmement, une étude pionnière dont l’objectif était de quantifier l’interactome entre les organites intracellulaires par imagerie spectrale, a révélé que 85\% des GL sont en contact avec le RE à un temps donné, en raison de l’importance des échanges entre ces deux organites pour réguler l’homéostasie lipidique et la lipotoxicité cellulaire \citep{RN580}. Cette caractéristique augmenterait les probabilités de transfert de Core des sites de traduction vers les GL. Dans notre étude, nous avons évalué que 75\% de la population de GL est mobilisée par Core dès 24h p.i. Les GL alternent des cycles de formation et de dégradation plus ou moins dynamiques selon l’état métabolique des cellules et le type cellulaire, rendant le calcul précis de la demi-vie moyenne d’une GL difficile. Toutefois, étant donné que leur fonction principale est le stockage des lipides à long terme, il paraît peu probable que 75\% des GL se soient complètement renouvelées en 24h p.i. Au vu de ces résultats et des informations complémentaires de la littérature, nous pouvons émettre l’hypothèse que la distribution de Core à la surface des GL peut survenir à la fois lors de la formation d’une GL naissante à partir des membranes du RE enrichies en Core et lors des contacts entre les GL pré-existantes et les membranes du RE.

	\subsubsection{Rétention de la protéine Core à la surface des gouttelettes lipidiques cytosoliques : un mécanisme contre-productif pour l’assemblage des particules virales ?}

Bien que l’association de la protéine Core aux GL cytosoliques soit une étape essentielle pour la morphogénèse virale, le passage de Core à la surface des GL peut être transitoire, comme c’est le cas pour la souche hybride Jc1, dont la protéine Core (J6) est retrouvée majoritairement sous forme de puncta intenses localisés à proximité de la membrane du RE \citep{RN489}. Dans le cas de la souche JFH-1 sauvage, Core est majoritairement localisée à la surface des GL, induisant un parfait enveloppement de ces organites \citep{RN486}. En raison du titre infectieux 50 à 100 fois supérieur de la souche Jc1, il a été évoqué que la protéine Core de Jc1 serait plus rapidement mobilisée vers les sites putatifs d’assemblage, contrairement à celle de JFH-1 et que, par conséquent, l’association durable de Core aux GL serait contre-productive pour la morphogénèse virale \citep{RN495}. 

Dans notre étude avec la souche Jad adaptée, ainsi que celle de Kaul et al. qui l’a décrite pour la première fois \citep{RN342}, les titres infectieux sont de l’ordre de 1 à 5x10\up{5} similaires à ceux du virus Jc1, car les mutations d’adaptation situées dans les séquences des protéines NS5A et NS5B améliorent la transition entre la réplication génomique et l’assemblage \citep{RN1051}. Nos résultats montrent que la protéine Core de Jad se comporte comme celle de JFH-1, en s’accumulant progressivement à la surface des GL au cours l’infection, jusqu’à coloniser 85\% de la population de ces organites et envelopper en moyenne 70\% de leur surface au temps le plus tardif de notre étude. Ainsi, nos données indiquent que l’association stable de Core aux GL ne constitue pas une dimension contre-productive à l’assemblage et à la synthèse de particules infectieuses, contrairement au dogme établi. Le virus JFH-1, étant la seule souche clinique aboutissant d’emblée à un cycle viral complet dans la lignée Huh-7.5, pourrait simplement être sous-optimal pour la production de particules virales. De plus, dans notre étude, le virus chimérique Jad/C2a-J6, exprimant la protéine Core de J6 dans le contexte du Jad, montre un profil d’association aux GL similaire à celui du virus Jad, confirmant que les protéines p7 et NS2 de la souche Jc1 seraient nécessaires à la re-mobilisation précoce de Core aux sites putatifs d’assemblage dans les membranes du RE adjacentes observée précédemment \citep{RN495}. Par ailleurs, les protéines Core produites par l’ensemble des virus intergénotypiques sont moins stables que la protéine Core native de la souche Jad et couvrent parfois moins extensivement la surface des GL, sans que cela affecte la capacité de production de particules infectieuses. Ces résultats suggèrent que Jad conduit à une accumulation en « excès » de Core, qui n’est pas entièrement consommée pour produire des particules virales, et que les titres infectieux seraient plafonnés dans les cellules Huh7.5 en raison d'un autre facteur viral ou d’hôte limitant pour l’assemblage de toutes les souches de VHC.

	\subsubsection{Redistribution et clusterisation des gouttelettes lipidiques à proximité des usines de réplication virale : un mécanisme contribuant à l’assemblage des particules virales ?}

Nos observations par microscopie électronique mettent en évidence une redistribution des GL à proximité des remodelages membranaires induits par l’infection, et notamment des DMV, structures qui ont été décrites comme le siège d'une réplication génomique active \citep{RN452}. Ces remodelages du RE et re-distribution des GL prennent place de façon localisée dans le cytoplasme, mais ne sont pas limités à la zone péri-nucléaire, contrairement à ce qui a été précédemment décrit dans des cellules hébergeant des réplicons sous-génomiques sans assemblage de virions \citep{RN1044,RN329}, ou dans les systèmes d'expression transitoire de Core sans MV \citep{RN1040,RN947}. En revanche, nos résultats sont en accord avec ceux visualisant récemment une co-localisation des protéines non structurales (notamment NS5A) responsables du recrutement des génomes viraux néosynthétisés et de la glycoprotéine d’enveloppe E2 au niveau du MV entourant les GL associées à Core, par une approche de microscopie corrélative (CLEM) combinant l’observation de la localisation des protéines virales par fluorescence et des structures intracellulaires par microscopie électronique \citep{RN490}. Selon ces auteurs, les sites putatifs d’assemblage du VHC constitueraient alors les sites de transfert formés par les contacts membranaires entre le RE et les GL, et les GL serviraient de plateforme de stockage essentielles pour délivrer Core à proximité des autres composants de la particule virale. Ce rapprochement permettrait \textit{in fine} d’augmenter la probabilité d’interaction entre la protéine Core du VHC et l’ARN nouvellement synthétisé, favorisant ainsi l’encapsidation du génome qui constitue la première étape de l’assemblage du virus. Nos tentatives d’identifier la sous-population de GL concernée par cette redistribution, \textit{i.e.} déterminer si elle est majoritairement associée à Core, par triple co-marquage des GL, de Core et des ARN viraux par hybridation \textit{in situ}, ce qui n'a jusqu’à présent jamais été exploré dans la littérature, n'ont malheureusement pas abouti à ce jour, en raison de problèmes techniques (sensibilité de la membrane phospholipidique aux détergents inclus dans les réactifs d’hybridation, résultant en une désinsertion des protéines de surface des GL.

	\subsubsection{Élargissement des gouttelettes lipidiques par un mécanisme présumé de fusion homotypique : un effet secondaire de l’assemblage des particules virales ?}

En l’absence de biosynthèse \textit{de novo} accrue de GL et compte tenu de la réduction de l’expression d’enzymes impliquées dans la synthèse locale de lipides neutres dans les cellules infectées, nous postulons que l’élargissement des GL est médié par un mécanisme de fusion homotypique, et serait une conséquence directe de la redistribution et de la clusterisation des GL. En effet, Core pourrait établir un pont physique reliant les GL limitrophes, ce qui pourrait déséquilibrer leur tension de surface et favoriser \textit{a posteriori} leur fusion. Core présente une forte plasticité conformationnelle permettant de s’oligomériser, une étape essentielle pour la formation de la nucléocapside \citep{RN1043,RN1056}. Par conséquent, un mécanisme possible qui sous-tend la liaison des GL adjacentes pourrait être attribué à la capacité de Core à former des homodimères \textit{en trans}, en pontant les protéines Core d'une gouttelette à une autre, selon un mécanisme similaire à celui impliquant les complexes protéiques cellulaires CIDEB et AUP1 \citep{RN596,RN558}. Cette hypothèse assez ambitieuse nécessiterait que le domaine D1 responsable de la dimérisation \citep{RN1062} soit accessible à ce stade de la conformation de Core. Des informations sur la structure tridimensionnelle de Core aideraient à légitimer cette hypothèse, mais seule une structure très partielle de la région du peptide signal résolue à ce jour par RMN est disponible \citep{RN1085}. Un autre mécanisme indirect concevable serait que Core mobilise les complexes cellulaires tels que AUP1 ou CIDEB. Alternativement, la fusion des GL appariées pourrait être indirectement facilitée par la dé-mobilisation de Core vers les sites putatifs d’assemblage situés sur le RE. Celle-ci aurait pour conséquence d'augmenter la tension superficielle des GL (puisque les protéines associées à la surface des GL contribuent à préserver « l’émulsion »), et de déclencher la fusion entre les GL appariées \citep{RN608}. Cette hypothèse serait cohérente avec le fait que la protéine Core_DP, déficiente pour l’assemblage, induit une clusterisation et une redistribution partielle sans élargissement des GL, en raison d’une topologie altérée qui ne permet pas de mobiliser cette protéine mutée pour former la nucléocapside. \\ \\
\indent
Il est intéressant de noter que nos résultats préliminaires avec le virus Jc1 ne documentent pas d’élargissement des GL au cours de l’infection (non montrés), ce qui est cohérent avec des données récentes de la littérature \citep{RN1057}. En ligne avec la discussion ci-dessus, nous pouvons émettre l'hypothèse que le passage de Core de la souche J6 à la surface des GL dans le contexte du virus Jc1 n’est pas suffisamment long pour mettre en place les mécanismes de dérégulations marqués, tels que quantifiés aux temps tardifs de l’infection par la souche Jad. Dans ce contexte, la re-mobilisation rapide de la protéine Core mature par les protéines p7 et NS2 de la souche J6 au niveau des sites putatifs d’assemblage du RE représenteraient un mécanisme alternatif pour délivrer Core auprès des autres éléments viraux pour former les particules virales. Pour confirmer cette hypothèse, il conviendrait de vérifier si Jc1 induit une clusterisation et une redistribution des GL à proximité des usines de réplication virale. \\ \\
\indent
Globalement, l’ensemble de nos résultats a permis de compléter ceux de la littérature et d'apporter des éléments nouveaux avec une dimension quantitative, qui supportent le modèle illustré dans la \Autoref{fig:fig77} : (1) la protéine Core du VHC a une topologie et un mécanisme de recrutement similaire aux protéines de classe I : elle diffuse préférentiellement à la surface des GL selon un recrutement actif médié par la protéine DGAT1 juste après le clivage du peptide signal par la SPP. La monocouche lipidique des GL accommoderait plus stablement les courtes hélices amphipatiques de Core que la bicouche du RE. Les GL peuvent acquérir les protéines Core à leur surface au cours de leur synthèse ou lorsqu’elles établissent des contacts ultérieurs avec des zones riches en protéines Core dans la membrane externe du RE ; (2) la protéine Core induit une redistribution des GL à proximité des DMV qui hébergent les usines de réplication génomique virale, concentrant localement la protéine de capside au plus proche des génomes viraux néosynthétisés pour optimiser la formation des nucléocapsides ; (3) Cette redistribution des GL induit secondairement des évènements d’appariement, soit par un mécanisme directement médié par la capacité de la protéine Core à former des oligomères \textit{en trans}, soit par le recrutement d’un autre facteur d’hôte capable de dimérisation, comme CIDEB ou AUP1 ; (4) Cette clusterisation « forcée », en parallèle d’une re-mobilisation des protéines Core recouvrant la surface des GL vers les sites d’assemblage pour former les nucléocapsides peut déstabiliser la tension superficielle des GL et déclencher leur fusion homotypique.

		\begin{figureth}
	\centering
			\includegraphics[width=\linewidth]{Figure_77.png}
		\caption[Modèle de la dérégulation et de l’implication des gouttelettes lipidiques dans l’étape de morphogénèse du VHC privilégié par notre étude]{\textbf{Modèle de la dérégulation et de l’implication des gouttelettes lipidiques dans l’étape de morphogénèse du VHC privilégié par notre étude.}}
				\label{fig:fig77}
	\end{figureth}
	\FloatBarrier

\subsection{Production de virus chimériques exprimant des protéines Core hétérologues comme nouveaux outils pour évaluer le lien direct potentiel entre l’origine génotypique de Core et la stéatose hépatique}

La recherche visant à décrire et à comprendre les mécanismes moléculaires et physiopathologiques liés à l'infection par le VHC a été freinée par le manque de système de culture approprié et de modèle animal permissif à l’infection. En particulier, les mécanismes liés aux perturbations induites spécifiquement par l’infection par le VHC de génotype 3 demeurent mal compris. En effet, les méta-données cliniques et épidémiologiques obtenues à partir de cohortes de patients révèlent une prévalence de 80\% des cas de stéatose sévère dans les infections chroniques par les souches de génotype 3 \citep{RN359,RN360}, qui a été interprétée comme le résultat d’un effet « stéatogène » direct des virus de ce génotype \citep{RN1225}. À la suite de ces enquêtes, plusieurs évidences expérimentales reposant principalement sur des systèmes d’expression transitoire de protéines virales suggéraient que les désordres du métabolisme lipidique seraient induits de façon génotype-spécifique \citep{RN1206,RN1209,RN1207}. En ce sens, un rôle central et génotype-spécifique dans les dysfonctionnements cellulaires associés au virus, incluant la lipogénèse, la stéatose et la ß-oxydation a été attribué à la protéine Core \citep[pour revue,][]{RN1123}. À l’inverse, d’autres études basées sur des analyses histologiques de biopsies de foie de patients divergent largement à ce sujet, et remettent en cause l’association préférentielle du génotype 3 du VHC avec la stéatose hépatique \citep[pour revue,][]{RN1201}. Ces témoignages contradictoires renforçaient la nécessité d’utiliser des systèmes d'infection pertinents pour de telles études de corrélats physio-pathologiques. \\ \\
\indent
La souche JFH-1 du VHC de génotype 2a est le seul isolat naturel qui récapitule spontanément l'ensemble du cycle viral menant à la production de particules du VHC en lignée cellulaire d'hépatome humain Huh-7.5 \citep{RN124}. Plusieurs équipes sont parvenues à adapter avec succès des souches de divers génotypes du VHC en lignées hépatocytaires : les souches H77-S, TN et HCV1 de génotype 1a \citep{RN353,RN1193,RN1202}, les souches J6 et T9 de génotype 2a \citep{RN354}, les souches DH8 et J6 de génotype 2b \citep{RN355}, la souche clinique S83 de génotype 2c \citep{RN1186}, et les souches cliniques DBN3a, S52, S310 de génotype 3a \citep{RN357,RN356,RN1060}. Néanmoins, non seulement ces souches hyper-adaptées ne reflètent pas la séquence native des souches cliniques puisqu’elles contiennent de multiples mutations d’adaptation, mais leur multiplication reste très variable, ce qui rend leur utilisation difficile. Dans ce contexte, pour aborder la question du lien entre le génotype 3 du VHC et la stéatose hépatique, notre étude s'est initialement concentrée sur la production de virus chimériques codant une série de protéines Core hétérologues.

	\subsubsection{L’interaction entre les protéines Core hétérologues de génotype 3 et les autres éléments viraux est permissive pour l'assemblage des particules virales}

La série de virus chimériques intergénotypiques que nous avons produite avec succès dans le contexte de la souche Jad \citep{RN342,RN811} exprime en lieu et place de la protéine Core native, la protéine équivalente issue de six souches cliniques de sous-type 3a, ainsi que d’une souche clinique de sous-type 1a, qui ont l’avantage d’être associées à un tableau clinique connu, établi à partir des biopsies hépatiques prélevées chez les patients d’origine. Ces patients ont développé différents degrés de stéatose macro- et micro-vésiculaire, avec une accumulation variable de vésicules graisseuses dans les hépatocytes. \\ \\
\indent
À ce jour, aucun VHC recombinant naturel dont le génome résulterait de la substitution exacte de la séquence codante d’une seule protéine virale n’a été détecté sur le terrain. Les « points chauds » des recombinaisons intergénotypiques naturelles identifiées se situent au sein de la séquence de la protéine NS2 ou dans la région N-terminale de NS3, générant ainsi des protéines hybrides, ou encore à la jonction NS2/NS3, séparant le supposé « module d’assemblage » de Core à NS2 du « module de réplication » de NS3 à NS5B \citep[pour revue,][]{RN94}. Cette jonction fréquente pouvait suggérer l’importance de préserver des interactions homologues entre les éléments propres à chaque module, afin de ne pas nuire aux étapes de réplication génomique et d’assemblage des particules virales. C'est pourquoi la plupart des génomes intergénotypiques fonctionnels générés par génie génétique afin d’étudier les propriétés physiopathologiques des autres génotypes viraux et de disposer de modèles pour tester les réponses humorales pan-neutralisantes, portent la séquence codante des protéines Core-E1-E2-p7-NS2 hétérologues, représentant un « module d’assemblage » minimal \citep{RN352,RN351}. Ainsi, aucune des données disponibles dans la littérature, qu’elles soient relatives aux recombinants naturels ou synthétiques, ne pouvait donc prédire si la seule substitution de la séquence complète de Core perturberait la fonctionnalité du génome intergénotypique résultant. Comme preuve de concept, une progénie virale robuste a été obtenue par notre groupe pour la première fois à partir d’un panel de génomes chimériques construits dans le contexte du Jad et codant Core provenant de souches prototypiques ou cliniques de sous-types 1a, 1b, 4a, 4f et 4p du VHC \citep[][et résultats non publiés.]{RN808}. Les génomes chimériques nouvellement produits pour cette étude, exprimant différents variants de la protéine Core de sous-type 3a se sont avérés également aboutir à une synthèse robuste de particules virales infectieuses, bien que ces séquences de Core soient toutes phylogénétiquement plus distantes de la séquence initiale de sous-type 2a que les séquences des autres génotypes testés. Seul un virus intergénotypique codant la protéine Core 376 de sous-type 3a s’est avéré sous-optimal. Il est notable que la séquence consensus de la protéine Core de cet isolat soit la plus divergente parmi les séquences 3a considérées, avec des polymorphismes uniques au niveau des résidus en positions 20, 66, 67, 110, 144, 151, 158. Une adaptation suite à cinq passages successifs des cellules Huh-7.5 transfectées par l’ARN viral correspondant transcrit \textit{in vitro} a permis d’engendrer et de stabiliser à quatres reprises une variante de ce virus aussi robuste que le virus parental. L’examen de la séquence nucléotidique majoritaire de ces variants n’a pas révélé de mutation potentiellement compensatrice au sein de la séquence codante de la polyprotéine, ni dans la partie variable de la région 3'NC ou dans les ~150 nucléotides 3'-terminaux de la région 5'NC. Il serait intéressant de déterminer les séquences des régions 5’ et 3' NC manquantes de ces variants, afin d'identifier si une mutation dans les régions non codantes optimiserait l’étape de morphogénèse virale, ce qui, le cas échéant renforcerait l'hypothèse qu'une interaction directe entre une région NC de l'ARN et la protéine Core serait critique pour l'encapsidation de l'ARN. \\ \\
\indent
Globalement, ces études conjointes de notre groupe démontrent pour la première fois l’existence d’interactions largement permissives entre les protéines Core hétérologues de sous-types 1a, 1b, 3a, 4a, 4f et 4p et les éléments natifs de sous-type 2a, qui conduisent efficacement à la morphogénèse des particules virales. L'intérêt de ce nouveau type de virus intergénotypiques réside dans le fait qu'ils expriment des protéines E1, E2, p7, NS2, NS3, NS4A, NS4B, NS5A et NS5B identiques et que seule la séquence de Core est déclinée. De plus, contrairement aux souches hyper-adaptées, ces virus intergénotypiques présentent des cinétiques de réplication et des taux d’expression protéique comparables. Ainsi, il a été possible d'inférer à la seule protéine Core hétérologue les phénotypes et les mécanismes physiopathologiques différentiels induits par l’infection, comme ceux observés vis à vis des GL et des voies métaboliques impliquées dans la stéatose hépatique décrits ci-dessous.

	\subsubsection{Absence d’effet « stéatogène » spécifique aux protéines Core hétérologues de génotype 3 ou au tableau clinique des patients à l’échelle des gouttelettes lipidiques}

Notre étude reposait sur l’hypothèse que l’observation et la comparaison du contenu et de la taille des GL dans des cellules infectées par les virus intergénotypiques pouvait être un indicateur initial de l’effet « stéatogène ». Toutes les protéines Core hétérologues de génotype 1, 2, 3 ou 4 exprimées par les virus intergénotypiques considérés sont retrouvées en majorité à la surface des GL, ce qui étend les études basées sur des systèmes d’expression transitoire des génotypes 1 et 3 de Core \citep{RN990,RN991,RN1041}. Toutefois, le degré d’association de Core à la surface des GL peut varier de 30 à 45\% selon la séquence de Core, indépendamment de son origine génotypique. Une telle variabilité a été décrite dans l’étude très récemment publiée recourant aux virus de différents génotypes hyper-adaptés à la culture cellulaire \citep{RN1060}. Ces auteurs ont montré que le degré de co-localisation directe de Core aux GL s’étend entre 40 et 60\% selon la souche virale, une fourchette relativement similaire à celle que nous avons observée avec nos modèles de virus chimériques de Core. Ces données combinées confirment que la localisation des protéines Core hétérologues ne semble pas être perturbée par les éléments viraux de la souche Jad. \\ \\
\indent
Notre étude n’a pas révélé de différence significative du volume total des GL par cellule dans les cultures infectées par les différents virus intergénotypiques, indiquant que les protéines Core de génotype 3 ne sont pas particulièrement promptes à induire une biosynthèse des GL, par rapport à celles de génotypes 1, 2 ou 4. Une augmentation modeste est toutefois obtenue pour le virus chimérique exprimant la protéine 389 de sous-type 3a. Pour confirmer qu’elle n’est pas issue d’un dépassement du seuil de significativité lié au hasard de la sélection des cellules dans ces expériences, il serait utile d’élargir l’étude à un nombre plus important de cellules. L'absence générale de régulation du contenu global de GL par cellules se démarque des résultats obtenus dans des systèmes d'expression transitoire \citep{RN990,RN991,RN1223} ou avec les souches hyper-adaptées \citep{RN356,RN1060} qui révélaient une accumulation plus importante de GL cytosoliques pour certaines protéines Core ou pour la souche S310 de sous-type 3a. Les raisons que l'on peut invoquer pour expliquer ces résultats divergents relèvent probablement des différences dans les modèles d'étude ou dans les temps post-infection retenu, les comparaisons entre souches ayant parfois été réalisées après plusieurs dizaines de passages des cellules infectées \citep{RN356}. En revanche, notre étude met en évidence un élargissement différentiel des GL dans les cellules infectées par les virus intergénotypiques. Cet élargissement, déjà noté pour la souche Jad, est plus ou moins marqué selon les séquences de Core, mais ne semble pas particulièrement plus élevé pour les séquences Core de génotype 3 contrairement aux données obtenues dans les systèmes d’expression transitoire \citep{RN1041,RN1208}. De plus, les virus intergénotypiques qui portent le résidu Phe en position 164 ou la combinaison Phe/Ile des résidus aux positions 182 et 186 du D3 dans la séquence Core hétérologue ne conduisent pas systématiquement à un élargissement des GL (Jad/C3a-390, Jad/C3a-395, Jad/C3a-401), ce qui est en désaccord avec les études par expression transitoire de \citet{RN990} et de \citet{RN991}. L’ampleur de l’élargissement semble plutôt corréler avec le taux de recrutement de Core à la surface des GL, confirmant nos conclusions précédentes sur le rôle essentiel de Core dans ce mécanisme physiopathologique (voir \autoref{section:discussion1}). \\ \\
\indent
La biosynthèse des GL pourrait être un effet relatif à la stéatose micro-vésiculaire, caractérisée par une accumulation de nombreuses petites vésicules graisseuses, tandis que l’élargissement des GL pourrait être attribuable à la stéatose macro-vésiculaire, qui se définit par la présence de peu voire d’une unique GL volumineuse dans le cytoplasme des hépatocytes. Cette hypothèse reste toutefois à moduler, étant donné l’absence d’élargissement des GL pour les virus intergénotypiques exprimant les protéines Core des souches cliniques 390 et 395, qui proviennent pourtant de souches cliniques associées à une stéatose sévère. En effet, les biopsies hépatiques prélevées à partir des patients infectés par ces deux souches contiennent respectivement 60\% d’hépatocytes avec des micro ou macro-vésicules graisseuses ou 40\% de macro-vésicules graisseuses. Une étude réalisée sur biopsies de foie de patients infectés par des virus de génotype 1 ou 3 a révélé l’absence de corrélation entre génotype infectant et proportion d’hépatocytes présentant des vésicules graisseuses \citep{RN1188}. Les auteurs soulignent toutefois l’existence d’une corrélation entre les virus de génotype 3 et la taille des vésicules. De plus, ces auteurs mettent en évidence l’existence d’une part significative d’hépatocytes sains qui contiennent de larges GL, suggérant qu’il existerait, en parallèle des effets pathogéniques direct du virus, un mécanisme indirect de l’induction de la stéatose. Le développement de la stéatose micro et macro-vésiculaire hépatique des patients infectés par les souches virales 390 et 395 pourrait être davantage attribuable à des facteurs confondants, comme le surpoids et la consommation excessive d’alcool. L’importance des facteurs d’hôte dans le développement de la stéatose hépatique sévère avait déjà été mis en lumière par les travaux de \citet{RN1228}, qui supportent l’absence d’une relation entre la stéatose hépatique et l’origine génotypique du virus. En effet, les études épidémiologiques en Europe de l’Ouest ont mis en évidence que l’excès d’alcool est fréquemment observé chez les utilisateurs de drogues injectables, et qu’il s’agit d’une population à risque pour la transmission du VHC de génotype 3 \citep{RN1226}. \\ \\
\indent
Globalement, notre étude n’a donc pas permis de confirmer le lien établi par les études cliniques et épidémiologiques entre les protéines Core de génotype 3 et les dérégulations sévères des GL qui pourraient sous-tendre l’apparition des signes histologiques de la stéatose, malgré le recours à des modèles d’infection. La force mais également la limite de notre modèle est qu’il se restreint uniquement à la nature génotypique de Core. Or, il a été montré que d’autres protéines virales, telles que NS5A, pourraient être conjointement impliquées dans le développement de la stéatose hépatique \citep{RN1210,RN1204,RN1187}. Cela impliquerait de développer des virus doublement chimériques de Core et de NS5A pour inclure les éventuels effets causés par les protéines NS5A de génotype 3. Toutefois, cette même conclusion a été atteinte dans le cadre des souches virales adaptées de différents génotypes dont le génotype 3 \citep{RN1060}, confirmant l’absence de lien entre les souches de génotype 3 et les dérégulations sévères des GL dans les systèmes d’infection hépatocytaires.

	\subsubsection{Le transcriptome hépatique est modulé différentiellement selon l’origine génotypique de Core}

Notre étude comparative des transcriptomes d’hépatomes infectés par les différents virus intergénotypiques révèle pour la première fois dans un système infection une dérégulation différentielle du profil d’expression des gènes hépatiques selon l’origine génotypique de Core. Une telle modulation différentielle a été décrite précédemment dans des systèmes d’expression de la protéine Core isolée de génotype 1b, 2a, 3a ou 4d \citep{RN1222,RN1221,RN1211} puis dans des biopsies hépatiques de patients infectés par une souche de génotype 1 ou 3 \citep{RN1219,RN1208,RN1196,RN1192}. Les conclusions de ces auteurs relatives à un effet avéré du génotype 3 sur le métabolisme lipidique ou sur les voies pro-inflammatoires divergent selon les modèles. Une précédente étude de notre groupe s’est focalisée sur le lien entre les polymorphismes naturels de Core présents dans les souches de génotype 4 émergentes en Afrique centrale et la voie de signalisation de la Wnt/ß-caténine qui contribue au développement du CHC \citep{RN808}. La conclusion principale de cette étude révèle que le virus Jad/C4fC induisait une activation significativement plus marquée de la voie Wnt/ß-caténine que la protéine Jad/C4aR, attribuable au résidu Thr en position 71 présent uniquement dans la protéine 4fC. À l’heure actuelle, les données du crible transcriptomique effectuées avec 11 virus intergénotypiques sont en cours d’analyse afin d’identifier si des polymorphismes naturels de Core spécifiques aux souches de génotype 3 sont impliqués dans la dérégulation de certaines voies biologiques représentant des facteurs de risque pour la stéatose. Ces approches constituent un travail pionnier important, qui permettra d'obtenir des informations sur les déterminants viraux impliqués dans la progression de la maladie hépatique.

\clearpage

%%%%%%%%%%%%%%%%%%%%%%%%%%%%%%%%%%%%%%%%%%%%%%%%%%%%%%%%%%%%%%%%%%%%%%%%%%%%%%%%%%%%%%%%%%%%%%%%%%%%%%%%%%%%%%%%%%%%%%%%%%%%%%%%%%%%%%%%%%%%%%%%%%%%%%%%%%%%%%%%%%%%

\section{Perspectives}

Depuis le début du 21ème siècle, le domaine de la recherche biomédicale est en pleine révolution technologique : d’une part, avec l’évolution considérable du domaine de l’imagerie depuis l’invention des premiers prototypes de microscope optique au début du 17ème siècle, qui ont révolutionné l’étude du vivant, jusqu’à la conception de la cryo-microscopie électronique qui révèle maintenant des informations structurelles détaillées presque au niveau atomique ou fournit une résolution de l’ordre de quelques nanomètres dans la cellule vivante \citep{RN1026,RN1028} ; d’autre part, avec l’essor des méthodes d’analyses moléculaires à haut débit dites « omiques » couplées au développement des méthodes d’analyses bio-informatiques qui facilitent le stockage et l’interprétation d’importants volumes de données ou « big data ». Aujourd’hui, les avancées dans le domaine de l’imagerie et dans les technologies « omiques » permettent aux biologistes de regarder dans les deux directions : à l’échelle micro- ou nanoscopique, afin d’observer toujours plus en détail les structures cellulaires et moléculaires, ou à  l’échelle de la cellule unique ou de l’organisme afin d’avoir une image globale plus large et fonctionnelle d’un système biologique.

\subsubsection{Applications de l’imagerie en cellules vivantes et de la cryo-microscopie électronique pour valider les mécanismes viraux présumés dans la clusterisation et dans la fusion des gouttelettes lipidiques induites par l’infection}

Plusieurs lacunes liées à la limitation des systèmes exploités dans cette étude bénéficieront des progrès rapides dans le domaine de la microscopie. Une première direction pour compléter cette étude serait de suivre par microscopie en cellules infectées vivantes la clusterisation et la fusion des GL, à l’aide d’une souche Jad exprimant une molécule fluorescente, telle que la GFP. La séquence codant NS5A s’avère flexible pour l’insertion d’une protéine rapportrice de grande taille, sans affecter de manière excessive la capacité de production virale \citep{RN985,RN983}. La protéine NS5A fusionnée à la GFP dans le contexte de la souche Jad pourrait servir de marqueur subsidiaire des complexes de réplication génomique, ce qui permettrait également de suivre l’évolution de la redistribution des GL à proximité de ces structures. La quantification du phénomène de fusion est difficile en raison de la rareté et de l’imprévisibilité de ce processus. Cependant, un groupe a publié une méthode de quantification de la fusion des GL par récupération de la fluorescence après photoblanchiment (FRAP) \citep{RN1023}, mesurant le taux d’échanges de lipides neutres entre des GL adjacentes, marquées métaboliquement. Il serait intéressant de voir si cette approche est applicable dans notre modèle pour confirmer que l’élargissement local des GL s’effectue par un mécanisme de fusion homotypique. La vitesse d’acquisition des images doit correspondre à la dynamique du processus étudié, par conséquent, la microscopie confocale à balayage est inappropriée pour l’étude en cellules vivantes. En effet, elle recueille le signal d’un seul point focal en balayant le faisceau laser sur l’échantillon, générant lentement une image pixel par pixel. Une alternative serait d’utiliser un microscope confocal à disque rotatif qui éclaire l'échantillon à l'aide d'un réseau d’ouvertures disposées sur un disque, créant des centaines de faisceaux focalisés simultanément pour obtenir une image de l'ensemble du champ de vision. Cependant, le marquage fluorescent des GL par le BODIPY est particulièrement sensible à l’illumination prolongée par les lasers, et la phototoxicité réduit la viabilité des échantillons. La microscopie par diffraction optique tridimensionnelle combinée avec des lampes à épifluorescence, qui permet de visualiser les structures intracellulaires et de s’abstraire de l’utilisation de lasers, serait donc un outil idéal pour cette perspective d’étude \citep{RN1027,RN1030}. Enfin, pour vérifier notre première hypothèse vis-à-vis d’un mécanisme direct de Core par dimérisation en trans dans la formation des clusters de GL, il serait intéressant d’étudier à l’échelle quasi-atomique la structure de la protéine Core associée aux GL (microscope TITAN Krios, Institut Pasteur, Paris). Ce travail permettrait d’obtenir d’une part des données sur la topologie de la protéine Core mature et d’autre part, de vérifier si Core est capable de se multimériser à la surface des GL, impliquant la possibilité que la première étape de l’assemblage des particules virales puisse s’effectuer à la surface des GL et non après la remobilisation de Core vers les sites de transfert localisés sur la membrane externe du RE.

\subsubsection{Applications des technologies « omiques » pour étudier les propriétés pathogènes spécifiques au génotype et aux polymorphismes de Core}

Une deuxième direction pour compléter cette étude serait d’analyser et de comparer le protéome et le lipidome des GL dans les cellules infectées par les différents virus recombinants intergénotypiques, afin de réaliser une étude intégrative combinant plusieurs technologies « omiques » à haut débit. Un protocole de purification des GL a été récemment développé par le groupe de E. Herker et sera mis à disposition pour notre groupe dans le cadre d’un projet collaboratif \citep{RN1024}. En parallèle, il serait intéressant d’évaluer les éventuelles différences de compositions protéique et lipidique des GL associées ou non à Core dans les cellules infectées, afin d’identifier les potentielles fonctions biologiques retirées ou attribuées à cette sous-population de GL détournée pour l’assemblage des particules virales. Enfin, pour vérifier notre deuxième hypothèse vis-à-vis d’un mécanisme indirect par le recrutement de facteurs d’hôte dans la formation des clusters de GL, il serait intéressant d’identifier dans un premier temps les partenaires cellulaires de la protéine Core par une analyse interactomique à haut débit. Ce travail impliquerait préalablement d’aboutir à la production d’une protéine Core portant une étiquette pour la purification par affinité en tandem applicable à une analyse par spectrométrie de masse, sans déstabiliser excessivement la protéine, une approche actuellement en cours de développement dans notre groupe. Dans un second temps, le « silencing » par le biais d’ARN interférent court (siARN) ou la construction de lignées Huh-7.5 KO par CRISPR/Cas9 pour les partenaires cellulaires mis en évidence avec un rôle potentiel dans le but la redistribution, la clusterisation ou la fusion des GL permettrait d’apporter une dimension fonctionnelle à notre étude.

\subsubsection{Évolution vers des systèmes plus physiologiques}

Dans le cadre de l’étude d’agents pathogènes, il est important de prendre en compte les limitations des modèles. Un grand nombre de pathogènes humains ne sont toujours pas cultivables \textit{in vitro}, que ce soit dans le règne procaryote, qui nécessitent des milieux très stricts en terme de pH ou de température, ou dans le cas des parasites ou des virus, qui dépendent obligatoirement d’une cellule hôte. Avec le développement des lignées cellulaires immortelles faciles à entretenir depuis l’avènement des cellules HeLa en 1953, il a été rendu possible de cultiver de nombreux virus humains \textit{in vitro}. Néanmoins, ces modèles cellulaires ne sont pas toujours physiologiquement proches de l’infection naturelle. Des essais préliminaires d’infection de PHH par les virus recombinants produits ont été réalisés en collaboration avec C. Gondeau (Institut de Recherche en Biothérapie, Montpellier, France) mais nous ne sommes pas parvenus à montrer une infection robuste et reproductible d’un donneur à l’autre. Nos analyses complémentaires dans ce modèle ont suggéré qu’il faudrait utiliser des virus intergénotypiques construits dans le contexte natif JFH-1, car les mutations d’adaptation de la souche Jad confèrent un phénotype d’atténuation de la réplication dans ces cultures. Un modèle de cellules Huh-7.5 partiellement différenciées et maintenues sous hypoxie a récemment été développé \citep{RN1031}, reproduisant plus physiologiquement l’environnement hépatique, les caractéristiques des hépatocytes primaires ainsi que la structure native des particules du VHC. Ce nouveau modèle cellulaire plus pertinent pourrait donc représenter un modèle de choix pour les futures directions de nos projets.