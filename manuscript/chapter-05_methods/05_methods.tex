%%%%%%%%%%%%% Copyright (C) 2017-2021 Emeline Simon
%%
%% The current owner of this work is Emeline Simon
%% <contact at emeline.simon@gmail.com>.
%%
%% This is 05_methods.tex the fifth chapter for my PhD Thesis.
%%
%%%%%%%%%%%%%%%%%%%%%%%%%%%%%%%%%%%%%%%%%%

\chapter{Méthodes Expérimentales}
	\minitoc
	\newpage

%%%%%%%%%%%%%%%%%%%%%%%%%%%%%%%%%%%%%%%%%%%%%%%%%%%%%%%%%%%%%%%%%%%%%%%%%%%%%%%%%%%%%%%%%%%%%%%%%%%%%%%%%%%%%%%%%%%%%%%%%%%%%%%%%%%%%%%%%%%%%%%%%%%%%%%%%%%%%%%%%%%%%%%%

% 1 - SOUCHES CLINIQUES
			
	\section{Identification et clonage des séquences de Core des souches cliniques}
	\label{sec:souches}
	
	Les séquences codant Core provenant des souches cliniques S411, S311, S376, S389, S390, S395, S401 ont été obtenues après transcription inverse (RT) des ARN viraux à l'aide d'une amorce non spécifique (pdN(6), Thermo Fisher Scientific) et de \textit{reverse transcriptase SuperScript II} (Thermo Fisher Scientific), suivie d'une amplification par PCR semi-nichée ou nichée à l'aide de \textit{One Taq 2X Master Mix with Standard Buffer} (Thermo Fisher Scientific) et de paires d'amorces conçues pour s’hybrider avec des séquences consensus pour chaque sous-type au niveau de la région 5' NC et de la partie codante N-terminale de E1 (voir \autoref{tab:tabM5}). Les fragments d'ADN amplifiés et purifiés ont été, d'une part, séquencés directement par technique Sanger et électrophorèse capillaire (Eurofins). Les séquences consensus prédominantes codant Core ont été établies à partir de ces analyses. D'autre part, les fragments d'ADN ont été clonés dans un vecteur navette à l'aide du kit \textit{TOPO TA cloning for sequencing} ou \textit{Zero Blunt TOPO PCR cloning for sequencing} (Thermo Fisher Scientific), afin de disposer d'une séquence clonale à substituer à celle de Core du JFH-1 et d'apprécier la présence et la fréquence d'éventuelles quasi-espèces. Pour cette seconde étape, les ADN plasmidiques de 10 à 28 clones issus des produits d'une ou de deux réactions de RT-PCR indépendantes ont été séquencés (Eurofins). \\
	
\begin{tabular}{ |m{6.5cm}|m{6.5cm}|  }
\hline
Nom de l'amorce & Séquence (5' -- 3')\\
\hline
HCV3a_243FW1 & GATCACTAGCCGAGTAGTGTTGG \\
HCV3a_276FW2 (semi-nichée) & GCCTTGTGGTACTGCCTGATAG \\
HCV3a_945RV1 & GCTATTGGAACAGTCGTTGGTAAGG \\
PCR-5'-gt1_4-FW & CTTGTGGTACTGCCTGATAGG \\
Nested-5'-gt1_4-FW (semi-nichée) & GATAGGGTGCTTGCGAGTGC \\
PCR-E1-gt1-RV & GACCAGTTCATCATCATATCCC \\
\hline
\end{tabular}

	\begin{tableth}
\caption[Liste des amorces conçues pour amplifier et séquencer la séquence codante de Core des souches cliniques.]{\textbf{Liste des amorces conçues pour amplifier et séquencer la séquence codante de Core des souches cliniques.} La spécificité génotypique (HCV/gt 3a, 1, 1a) et la polarité (FW : génomique, RV : anti-génomique) des amorces, ainsi que l'utilisation pour la PCR semi-nichée sont indiquées dans la colonne de gauche.}.
			\label{tab:tabM5}
\end{tableth}

Globalement, pour tous les virus, la plupart des mutations silencieuses ou des mutations codantes (donnant lieu à une modification d'acide aminé dans la séquence de Core) a été retrouvée dans une proportion faible de clones (< 5\% si $\geq$24 clones analysés ou < 10\% si 10-12 clones analysés) et peut représenter soit des quasi-espèces mineures circulant chez les patients, soit des erreurs liées aux enzymes durant les étapes de RT ou PCR. Il est à noter qu'un nombre plus important de positions de Core cibles de substitutions minoritaires a été retrouvé pour l'isolat S401 de sous-type 3a, suggérant la présence de quasi-espèces virales variées chez ce patient. Par ailleurs, pour cet isolat, une mutation codante fréquente a été identifiée au niveau du codon en position 15 par rapport à la séquence considérée comme consensus (Ile -->Thr pour 10/24 clones) dans les séries de clones issues des deux réactions RT-PCR indépendantes, représentant ainsi une quasi-espèce non minoritaire. La situation s'est avérée du même ordre pour l'isolat S395 de sous-type 3a, avec 4/28 clones présentant un résidu Leu au lieu de Pro en position 4 et 4/28 clones présentant un résidu Ile au lieu de Val en position 162, ainsi que d'autres quasi-espèces plus minoritaires. Il en est de même pour l'isolat S411 de sous-type 1a, avec un nombre élevé de positions cibles de quasi-espèces silencieuses équilibrées (25-50\%), et le codon en position 75 (Ser) cible de mutations codantes (Thr ou Ala, pour 4 clones sur 24). Ces résultats dénotent une co-évolution plus importante du virus infectant chez ces trois patients par rapport aux autres. 

% 2 - CONSTRUCTION DES PLASMIDES
			
	\section{Construction des plasmides}
	
		\subsubsection{Construction des plasmides pJad et pJad-2EIL3 chimériques}
		
La construction des plasmides contenant les ADNc chimériques repose sur une stratégie de PCR de recouvrement conçue pour échanger uniquement la séquence codant Core. En bref, trois segments d'ADNc viral sont amplifiés dans une première étape en utilisant des amorces dotées d'extensions complémentaires des segments à fusionner, qui permettent le recouvrement des produits et la création des points de jonction exacts. Le produit de fusion est amplifié dans une seconde étape à partir des 3 produits PCR issus de la première étape, puis digéré par les enzymes dont les sites sont présents à ses extrémités, et cloné dans le vecteur pJad digéré par ces mêmes enzymes. Les plasmides pJad/C1aH77, /C4aR et /C4fC sont décrits dans un travail précédent du laboratoire \citep{RN808}. Le plasmide pJad/C3a-311, incluant une séquence codant Core mature identique à celle de l’isolat S311, a été précédemment produit dans le laboratoire. Les plasmides pJad/C1a-411, /C2a-J6, /C3a-376, /C3a-389, /C3a-390, /C3a-395, et /C3a-401 ont été construits pour cette étude. Un premier fragment d'ADN a été amplifié par PCR en utilisant pJad comme matrice avec une amorce de polarité positive s'hybridant en amont du site de restriction unique \textit{Age}I dans la région 5’NC et une amorce de polarité négative complémentaire de l'extrémité 3' de la région 5’NC et dotée en 5' d'une extension correspondant à la partie 5' de la séquence codant Core hétérologue à introduire. Un second fragment a été produit en utilisant le plasmide pCR4-TOPO hébergeant la séquence consensus codant Core d'une souche clinique  ou le plasmide pJc1-2EI3 (Core 2a-J6) comme matrice, avec des amorces s'hybridant respectivement sur les séquences 5’- et 3'- terminales de Core hétérologue et étendues en 5', de l'extrémité 3' de la région 5’NC et en 3', de l'extrémité 5' de la séquence de E1. Un troisième produit de PCR a été généré en utilisant pJad comme matrice, couvrant la séquence codante de E1 depuis son extrémité 5' jusqu'au site de restriction unique \textit{Bsi}WI, chevauchant ainsi l'amorce utilisée pour la deuxième PCR. Les PCR ont été réalisées avec 2,5U de polymérase \textit{Pwo Super Yield DNA Polymerase} (Roche) selon les instructions du fabricant. Une PCR de recouvrement dans laquelle les trois produits de PCR on été mélangés dans des rapports équimolaires a été réalisée en utilisant les deux amorces externes. Le produit PCR résultant couvre un segment de la région 5’NC, les séquences hétérologues codant Core et un segment de la séquence codant E1, encadré des sites de restriction \textit{Age}I et \textit{Bsi}WI. Après purification (\textit{QIAquick PCR Purification Kit}, Qiagen), ce produit PCR a été digéré avec les enzymes de restriction \textit{Age}IHF et \textit{Bsi}WI (New England Biolabs) et cloné dans le plasmide pJad préalablement digéré par ces mêmes enzymes de restriction et déphosphorylé à l’aide de 1U de phosphatase alcaline de crevette recombinante (\textit{Shrimp Alkaline Phosphatase rSAP}, NewEngland Biolabs). La ligation est réalisée avec le kit \textit{Rapid DNA Ligation} (Roche). Des cellules compétentes XL1-Blue ont été transformées avec ce mélange de ligation. Les colonies ont été sélectionnées dans des boîtes de Pétri LB contenant de l’ampicilline, à partir desquelles l’ADN plasmidique a été isolé (\textit{Plasmid Mini Kit}, Qiagen). Les clones ont été criblés avec les enzymes de restriction \textit{Kpn}I, \textit{Blp}I, \textit{Eco}RV et/ou \textit{Cla}I, présents dans les séquences Core hérétologues mais pas dans la séquence Core native du Jad. Les clones positifs ont été amplifiés pour la préparation à large échelle d’ADN (\textit{NucleoBond Xtra Midi Plus}, Macherey-Nagel). Toutes les séquences Core hétérologues de souches cliniques et prototypiques ont été insérées dans le plasmide pJad-2EIL3 en remplaçant le fragment de restriction \textit{Age}I-\textit{Not}I (situé dans la séquence codante de NS2) ou \textit{Age}I-\textit{Bsi}WI par les fragments correspondants isolés des plasmides chimériques dans le contexte pJad, générant ainsi les plasmides chimériques pJad-2EIL3/C1aH77, /C1a-411, /C2a-J6, /C3a-311, /C3a-376, /C3a-389, /C3a-390, /C3a-395, /C3a-401, /C4aR et /C4fC.

		\subsubsection{Construction des plasmides pJad/C_DP, pJad/C_DP_SATG, pJad/C_3aa_SPP et pJad/C_4aa_SPP par mutagenèse dirigée}

Un plasmide dérivé de pJad portant les mutations ponctuelles des codons GCA et CGC induisant les modifications d'acides aminés Y138A et Y143A dans la séquence de Core a été construit selon le même type de stratégie par PCR de recouvrement, mais reposant sur deux fragments qui se chevauchent au niveau de la région à muter et s'étendent en amont du site \textit{Age}I et en aval du site \textit{Bsi}WI. Deux oligonucléotides complémentaires portant les substitutions nucléotidiques d'intérêt ont été utilisés. Les clonages ont été réalisés entre les sites \textit{Age}I et \textit{Bsi}WI de pJad, générant ainsi le plasmide pJad/C_DP. La même stratégie a été employée pour produire (i) à partir de la matrice pJad/C_DP, le plasmide pJad/C_DP_SATG portant la substitution nucléotidique additionnelle TCA altérant le résidu Y164 en S164, (ii) à partir de la matrice pJad, plasmide pJad/C_3aa_SPP portant les substitutions nucléotidiques GTC, CTC et GTC induisant les modifications d’acides aminés A180V, S183L et C184V et (iv) à partir de la matrice pJad/C_3aa_SPP, le plasmide pJad/C_4aa_SPP portant la substitution nucléotidique additionnelle CCT altérant le résidu T186 en L186. Les amorces conçues pour la mutagenèse dirigée par PCR sont listées dans le \autoref{tab:tabM6}. Deux clones indépendants de chaque construction ont été retenus pour chaque construction mutante afin de confirmer par la suite les phénotypes des ARN viraux correspondants. L'intégrité et la séquence de tous les clones a été confirmée par une analyse de la séquence nucléotidique (voir \autoref{sec:sequencing}).

\begin{flushleft}
	\begin{tabular}{ |m{3.25cm}|m{7.75cm}|m{3cm}|  }
 \hline
Nom de l'amorce & Séquence (5' -- 3') & Substitution \\
\hline
DP_CoreJFH- & CCACTAAGC\textbf{G\textcolor{red}{C}G}GCGCCTACGAC\textbf{T\textcolor{red}{G}C}GATGTACCC  & Position 138 : Y -> A   Position 143 : Y -> A \\
CoreJFH_DP+ & GGGTACATC\textbf{G\textcolor{red}{C}A}GTCGTAGGCGC\textbf{C\textcolor{red}{G}C}GCTTAGTGG  & Position 138 : Y -> A   Position 143 : Y -> A \\
Y164S_CoreJFH- & GGTTCCCTGTTGC\textbf{T\textcolor{red}{G}A}ATTAACCCCGTCC & Position 164 : Y -> S \\
CoreJFH_Y164S+ & GGACGGGGTTAAT\textbf{T\textcolor{red}{C}A}GCAACAGGGAACC & Position 164 : Y -> S \\
Mut_SPP_CoreJFH- & GGTGAT\textbf{G\textcolor{red}{AC}G\textcolor{red}{AG}}CAACAG\textbf{G\textcolor{red}{A}C}CAGCAAGAAGATAG & Position 180 : A -> V   Position 183 : S -> L  Position 184 : C -> V \\
CoreJFH_Mut_SPP+ & TGCTG\textbf{G\textcolor{red}{T}C}CCTGTTG\textbf{\textcolor{red}{CT}C\textcolor{red}{GT}C}ATCACCGTTCCGGTC  & Position 180 : A -> V   Position 183 : S -> L   Position 184 : C -> V \\
Mut_T186L_CoreSPP- & GAGACCGGAACG\textbf{\textcolor{red}{AG}G}ATGACGAGCAAC & Position 186 : T -> L \\
CoreSPP_Mut_T86L+ & GTTGCTCGTCAT\textbf{C\textcolor{red}{CT}}CCGTTCCGGTCTC & Position 186 : T -> L \\
\hline
\end{tabular}
\end{flushleft}

	\begin{tableth}
\caption[Liste des amorces conçues pour générer les inserts d’ADN à l’origine des plasmides pJad/C_DP, pJad/C_DP_SATG, pJad/C_3aa_SPP et pJad/C_4aa_SPP.]{\textbf{Liste des amorces conçues pour générer les inserts d’ADN à l’origine des plasmides pJad/C_DP, pJad/C_DP_SATG, pJad/C_3aa_SPP et pJad/C_4aa_SPP.} Les résidus nucléotidiques qui diffèrent de la séquence génomique complémentaire sont indiqués en rouge et les codons sont mis en valeur en gras.}.
			\label{tab:tabM6}
\end{tableth}

% 3 - TRANSCRIPTION
			
	\section{Transcription de l’ARN \textit{in vitro}}
		\label{sec:transcription}
	
Pour préparer les ARN viraux de longueur génomique \textit{in vitro}, le plasmide pJad et les plasmides dérivés codant les ADNc recombinants pJad/C(X) ont été linéarisés par l'enzyme de restriction \textit{Xba}I qui va générer une extrémité 3' protubérante, éliminée par la suite à l’aide d’un traitement à la nucléase \textit{Mung Bean} (New England Biolabs). Le plasmide pJad-2EIL3 et les plasmides dérivés codant les ADNc recombinants pJad-2EIL3/C(X) ou les ADNc contrôles ont été linéarisés par l’enzyme de restriction \textit{Mlu}I et l’extrémité 3’ exacte des ARN synthétiques est obtenue par l’action du ribozyme du VHD dont la séquence est insérée immédiatement en aval de la région 3'NC du VHC et en amont du site \textit{Mlu}I. Les ADN linéarisés ont été purifiés par extraction au phénol/chloroforme, et quantifiés sur des gels d'agarose. Un microgramme d’ADN linéarisé sert ensuite de matrices pour la transcription \textit{in vitro} par l’ARN polymérase du phage T7 (Promega). La matrice d’ADN est ensuite éliminée après un traitement avec 1U de DNase (Promega), puis les ARN sont purifiés par extraction au phénol/chloroforme et précipités avec un volume d’isopropanol complémenté avec de l’acétate de sodium 0,3M. La qualité et la quantité des ARN resuspendus en H\textsubscript{2}O sont évaluées par électrophorèse sur gel d’agarose et par mesure de l’absorbance à 260nm.

% 4 - TRANSFECTION
			
	\section{Transfection}
	\label{sec:transfection}	
	
	Après détachement des monocouches par trypsination, les cellules Huh-7.5 sont lavées en tampon phosphate salin sans Ca\up{2+} ni Mg\up{2+} ou DPBS (\textit{Dulbecco’s Phosphate Buffered Saline}, Gibco) et resuspendues dans du milieu Opti-MEM (Gibco) à une concentration de 5x10\up{6} cellules/mL. Deux millions de cellules sont électroporées en présence de 5µg d'ARN transcrits \textit{in vitro} ou de PBS à l'aide d'un système \textit{Gene Pulse} (Biorad, Munich, \textit{Germany}) à 240V et 900µF, dans une cuvette de 0,4cm de large (Eurogentec). Immédiatement après le choc électrique, les cellules sont transférées dans du milieu complet puis réparties dans (i) des flacons de 75cm\up{2} (1,6x10\up{6} cellules) pour la production de stocks viraux, ou (2x10\up{6} cellules) pour les observations par microscopie électronique, (ii) des flacons de 25cm\up{2} (8x10\up{5} cellules) pour les expériences d’adaptation, (iii) des plaques 6 puits (3x10\up{5} cellules par puits) pour le suivi de la réplication virale par dosage de l’activité FLuc, ou (4x10\up{5} cellules) pour la préparation d’extraits protéiques et sur (iv) des lames en verre (1x10\up{4} cellules par canal) pour l’imagerie (Ibidi). Les cellules transfectées sont placées pendant 4h à 37°C sous 5\% de CO\textsubscript{2}, puis le milieu de culture est remplacé par du milieu complet frais afin d'éliminer des débris issu du choc électrique. Les cellules sont maintenues à 37°C sous 5\% de CO\textsubscript{2} pour une durée dépendant des expériences. À des fins d’inhibition du protéasome, les cellules transfectées ont été traitées avec 0,1\% de DMSO contenant 10µM de MG132 (Sigma) durant 14h avant la récolte. Le surnageant de culture et les extraits protéiques sont collectés à 3j p.tf. pour la production de stocks viraux (voir \autoref{sec:stocks}) et pour le suivi de la réplication virale par dosage de l’activité FLuc (voir \autoref{sec:fluc}). Les extraits protéiques sont récoltés 2 ou 3j p.tf. pour l’analyse de l’abondance des protéines virales par immunoblot (voir \autoref{sec:immunoblot}). Les cellules transfectées sont fixées à 2 ou 3j p.tf. pour les observations par microscopie confocale (voir \autoref{sec:confocal}) et à 3j p.tf. en vue des observations par microscopie électronique (voir \autoref{sec:electronique}).

% 5 - INFECTION
			
	\section{Infection}
	\label{sec:infection}
	
La veille de l’infection, des cellules Huh-7.5 sont ensemencées (i) en plaque 24 puits (3x10\up{4} cellules/puits) pour la quantification des ARN et titres infectieux viraux et pour la préparation des extraits protéiques, ou (1,5x10\up{4} cellules/puits) pour la préparation d’extraits d’ARN total, (ii) en plaques 12 puits (4,5x10\up{4} cellules/puits) pour le suivi de l’infectiosité par dosage de l’activité FLuc, (iii) sur des lames en verre (5x10\up{3} cellules/canal) pour les observations par microscopie confocale, et (iv) dans des flacons de 25cm\up{2} (1,25x10\up{5} cellules) en vue des analyses par microscopie électronique. Les cellules sont infectées avec une MOI de 1, 3 ou 10 TCID50/cellule. Pour le suivi de l'infectiosité des ARN viraux par dosage de l'activité FLuc, les cellules naïves sont mises en contact avec 0,5mL du surnageant récolté à 72h p.tf. Après incubation en présence de l'inoculum viral pendant 4h à 37°C sous 5\% de CO\textsubscript{2}, l'inoculum est remplacé par du milieu complet frais et les cellules maintenues à 37°C pour une durée dépendant des expériences. Des stocks viraux indépendants ont été utilisés pour les réplicats biologiques. Le surnageant de culture et les ARN totaux sont collectés à 1, 2, 3, 4 ou 5j p.i. pour la quantification des titres infectieux (voir \autoref{sec:titrage}) et des ARN viraux intra- et extra-cellulaires (voir \autoref{sec:genome}). Les extraits protéiques sont préparés à 4j p.i. pour les analyses par immunoblot (voir \autoref{sec:immunoblot}). Les ARN totaux sont préparés à 4 ou 5j p.i. pour l’analyse à haut débit par RNA-Seq (voir \autoref{sec:transcriptomique}). Les cellules infectées sont fixées 1, 2, 3 ou 4j p.i. pour les observations par microscopie confocale (voir \autoref{sec:confocal}) et à 5j p.i. pour les analyses par microscopie électronique (voir \autoref{sec:electronique}).

% 6 - STOCKS

	\section{Production des stocks viraux}
	\label{sec:stocks}

La progénie virale des génomes parentaux et recombinants a été obtenue par génétique inverse suite à l’électroporation d'ARN viraux transcrits \textit{in vitro }(décrit dans la \autoref{sec:transfection}) en collectant les surnageants de culture des cellules Huh-7.5 transfectées à 3j p.tf., ou à 25j p.tf. dans le cadre des expériences d’adaptation, puis en les clarifiant par centrifugation à 3000g pendant 10min. Les surnageants clarifiés ont été utilisés pour infecter des cellules Huh-7.5 naïves à une MOI de 0,01 TCID50/cellule. Les cellules infectées sont amplifiées lorsqu'elles atteignent une confluence de 80-100\%, généralement tous les 2 à 3j, par transfert dans des flacons de plus grande surface en présence d'une fraction du surnageant de culture collecté (~20\%) ajouté à du milieu complet frais. Les effets cytopathiques associés à l’infection ont été surveillés quotidiennement pour ajuster le protocole de passage des cellules et la collecte finale d'un large volume de surnageant de culture constituant le stock viral, conservé en mono-doses à -80°C. Avant utilisation des stocks, le titre infectieux est établi suite à un minimum de trois titrages indépendants (voir  \autoref{sec:titrage}).

% 7 - TITRAGE

	\section{Détermination du titre viral}
	\label{sec:titrage}
	
Le titre infectieux des stocks viraux a été déterminé par un test en dilutions limites. Des dilutions sériées des surnageants ont été utilisées pour infecter 2,5x10\up{3} cellules Huh-7.5 par puits de plaques à 96 puits à raison de 8 puits par dilution. Cinq jours après l’infection, les cellules sont fixées par ajout de 200µL de méthanol et incubées avec une solution de DPBS supplémenté de 0,3\% de peroxyde d'hydrogène (Sigma) pendant 5min à température ambiante afin de bloquer l'activité peroxydase endogène. Les sites antigéniques non spécifiques sont saturés à l'aide d'une solution de blocage contenant 1,25\% de sérum de cheval, puis les cellules sont incubées avec 0,4µg/mL d’un anticorps monoclonal de souris anti-NS3 du VHC (BioFront) durant la nuit à 4°C. Le lendemain, les cellules sont incubées avec un conjugué anti-IgG souris couplé à la peroxydase de raifort (\textit{ImmPRESS HRP}, Vector Laboratories) pendant 30min à température ambiante. L'activité peroxydase est révélée par contact avec le substrat chromogénique 3,3'-diaminobenzidine (\textit{DAB Substrate Kit, Vector Laboratories}) pendant 20 minutes à température ambiante, ce qui génère une coloration brune sur le site de l’antigène cible. Les foyers infectieux ont été dénombrés manuellement au microscope optique. Le titre viral est exprimé en doses infectieuses 50\% en culture tissulaire par mL (TCID50/mL) et est calculé par la méthode de Reed et Muench.

% 8 - QUANTIFICATION DES GENOMES VIRAUX

	\section{Quantification des génomes viraux}
	\label{sec:genome}
	
Pour extraire les ARN intracellulaires, le tapis cellulaire des puits de plaques 6 ou 24 puits est mis en contact avec respectivement 200 ou 100µL de RNAzol (\textit{RNAzol RT}, Sigma) et les lysats sont transférés en tubes et congelés à -20°C. Après décongélation, l'ARN total est extrait selon les indications du fournisseur. Le culot d'ARN est ensuite resuspendu dans 15 à 30µL d'H\textsubscript{2}O certifiée sans ribonucléases et conservé à -80°C. La concentration d'ARN intracellulaire total est déterminée par mesure de la densité optique à l'aide d'un nano-spectrophotomètre (\textit{MySpec}, Ozyme). L’ARN viral est extrait de 140µL de surnageant de culture à l’aide d’un tampon de lyse virale supplémenté avec 10ng/µL d’ARN porteur (\textit{QiaAmp Viral RNA Mini}, Qiagen), purifié selon les indications du fournisseur, élué dans 55µL d’H\textsubscript{2}O certifiée sans ribonucléases et conservé à -80°C. Les génomes viraux sont quantifiés à partir de 20ng d’ARN intracellulaire total ou de 5µL d’ARN viral particulaire par RT-PCR quantitative en une étape (\textit{TaqMan Fast-virus 1 Step Master Mix}, Applied Biosystems) à l’aide d’une paire d’amorces qui ciblent la région 5’NC du VHC et de sondes fluorescentes FAM (6-carboxyfluorescéine) et TAMRA (6-carboxytétraméthylrho- damine). Les quantités d'ARN intracellulaires introduites dans le test RT-qPCR sont normalisées par rapport à la quantité d’ARN ribosomal mesurée à l’aide d’amorces et d'une sonde fluorescente VIC (2$'$-chloro-7$'$phenyl-1,4-dichloro-6-carboxy-fluorescéine) spécifiques de l’ARN 18S (\textit{Eukaryotic 18S rRNA Endogenous Control}, Applied Biosystems). Les échantillons d’ARN ont été analysés sur un appareil de PCR à temps réel (\textit{7500 Fast Real-Time PCR System}, Applied Biosystems). L’ARN a été soumis à une RT à 50°C pendant 5min, puis après dénaturation de l’enzyme à 95°C pendant 20s, les étapes de la qPCR sont réalisées sur 40 cycles de 15s  à 95°C et de 1min à 60°C. La quantification absolue des génomes viraux est obtenue à l’aide d’une courbe standard établie à partir d’ARN du VHC transcrit \textit{in vitro} dont la concentration en équivalents génomes par µL (copies/µL) a été déterminée par mesure de l’absorbante à 260nm. Les valeurs finales sont exprimées en copies/mL et en copies/µg d’ARN total.

% 9 - SEQUENCAGE

	\section{Séquençage du génome viral}
	\label{sec:sequencing}
	
Pour séquencer le génome viral des expériences d’adaptation (voir section \autoref{sec:transfection}) ou des stocks (voir \autoref{sec:infection}), les ARN sont extraits des cellules transfectées avec 1mL de RNAzol (\textit{RNAzol RT}, Sigma) ou de 140µL de surnageant à l’aide d’un tampon de lyse virale supplémenté avec 10ng/µL d’ARN porteur (\textit{QiaAmp Viral RNA Mini}, Qiagen) selon les indications des fournisseurs. Les ARN sont resuspendus respectivement dans 30µL ou 55µL d'H\textsubscript{2}O certifiée sans ribonucléases et conservés à -80°C. La concentration d'ARN total est déterminée par mesure de la densité optique à l'aide d'un nano-spectrophotomètre (\textit{MySpec}, Ozyme). Un microgramme d’ARN total ou 10µL d’ARN viral particulaire est additionné de 1µM d’amorces de séquence aléatoire (\textit{hexamer pdN(6)}, Roche) et d’un mélange de 1mM de chaque dNTP (\textit{dNTP set}, Eurobio). Afin d'éliminer les structures secondaires de l'ARN, le mélange est incubé 5min à 65°C dans un thermocycleur (\textit{Eppendorf Mastercycler 5333 Thermal Cycler}, Eppendorf), puis immédiatement transféré dans de la glace. Après refroidissement, le mélange réactionnel est supplémenté de 10mM de dithiothréitol (DTT), 500U d'inhibiteur de ribonucléases (\textit{RNasin}, Promega) et 200U de transcriptase inverse (\textit{SuperScript II Reverse Transcriptase}, Thermo Fisher Scientific) dans un volume final de 20µL, puis incubé dans un thermocycleur pendant 10min à 25°C, 50min à 42°C pour la RT, et 15min à 70°C. Deux microlitres de l’ADNc issu de la réaction de RT est amplifié par PCR en présence du réactif \textit{One Taq 2X Master Mix Polymerase} (New England Biolabs) et de 0,2µM de couples d'amorces amplifiant spécifiquement des segments de 1000 à 1500 pb couvrant les régions génomiques au terme de 30 cycles à 94°C pendant 30s, 55°C pendant 30s et 68°C pendant 1min. Les produits de PCR résultants sont vérifiés par migration sur gel d’agarose et purifiés par l'intermédiaire du Kit \textit{QIAquick PCR Purification} (Qiagen). Les réactions de séquençage sont assemblées selon la méthode de Sanger avec 15ng du produit PCR purifié, 0,32µM d’amorce s'hybridant spécifiquement sur le brin sens ou anti-sens de l'ADN et un mélange constitué de quatres sondes fluorescentes marquant les ddNTPs (\textit{Big Dye Terminator v1.1 or V3.1 Cycle Sequencing}, Applied Biosystems)  et incubées au thermocycleur pour 25 cycles à 95°C pendant 45s, à 50°C pendant 30s et à 60°C pendant 4min. Les réactions de séquençage sont analysées par électrophorèse capillaire (Eurofins).

% 10 - MESURE DE L'ACTIVITE FLUC

	\section{Mesure de l’activité luciférase}
	\label{sec:fluc}
	
Les cellules Huh-7.5 transfectées (voir \autoref{sec:transfection}) ou infectées (voir \autoref{sec:infection}) destinées au suivi de l’activité rapportrice FLuc sont lysées dans 200 ou 100µL de tampon de lyse (\textit{Reporter Lysis Buffer}, Promega) selon la taille de la plaque de culture et congelées à -20°C. Après décongélation, les lysats sont centrifugés à 3000g pendant 5min afin d'éliminer les débris cellulaires. L'activité FLuc de 10µL de lysat clarifié est mesurée en plaques 96 puits blanches à l'aide d'un luminomètre (\textit{TriStar LB 942 Multimode Microplate Reader}, Berthold Technologies) après injection automatique de 50µL de substrat dilué au 1:2 dans de l’H\textsubscript{2}O (\textit{Luciferase Assay Reagent}, Promega).

% 11 - TRANSCRIPTOMIQUE

	\section{Analyse du transcriptome hépatique par RNA-Seq}
	\label{sec:transcriptomique}

L’ARN total des cellules Huh-7.5 infectées en vue de l’analyse du transcriptome hépatique par RNA-Seq (voir \autoref{sec:infection}) est extrait avec 350µL d’un tampon de lyse (\textit{RNeasy Plus Mini}, Qiagen) supplémenté par 0,1\% de ß-mercaptoéthanol (Sigma) selon les instructions du fabricant. Pour garantir l’élimination complète de l’ADN génomique, une digestion à l'aide de DNase a été effectuée sur colonne pendant 15min (\textit{RNase-free DNase Set}, Qiagen). La qualité et l’intégrité de l’ARN ont été évaluées par mesure de la densité optique à l'aide d'un nano-spectrophotomètre (\textit{MySpec}, Ozyme) et d’un bioanalyseur (\textit{2100 Bioanalyzer}, Agilent). Tous les échantillons présentaient une excellente pureté (A\textsubscript{260/230} > 1,8 et A\textsubscript{260/280} > 2,0) et ne présentaient aucun signe visible de dégradation (Score RIN > 9). Les ARNm poly(A) ont été enrichis à partir de 200ng d’ARN totaux avec le kit \textit{Dynabeads mRNA Purification Kit} (Thermo Fisher Scientific) selon le protocole \textit{RNA Direction Library prep set user Manual} de MGI (MGI Tech, Shenzen Shi, Chine). Les ARNm sont clivés en fragments de 150pb, rétro-transcrits et le brin complémentaire synthétisé selon le même protocole. Les adaptateurs MGI sont ensuite ajoutés par ligation avant une étape d'amplification des fragments d'ADN de 14 cycles. La qualité des banques d'ADN est contrôlée par électrophorèse capillaire (\textit{Fragment Analyzer}, Agilent) en utilisant un kit pour fragments d’ADN de 100 à 6000 pb (\textit{HS NGS Fragment Kit}, Agilent). Les banques d'ADN sont ensuite assemblées et préparées selon les protocoles \textit{RNA Directional Library prep set user Manual} et \textit{High-throughput (Rapid) Sequencing Set User Manual} de MGI (MGI Tech, Shenzen Shi, Chine), puis séquencées sur une cellule à haut débit en simple lecture de 100pb (DNBSEQ-G400, MGI Tech, Shenzen Shi, Chine).

% 12 - IMMUNOBLOT

	\section{Analyse quantitative des protéines par immunoblot}
	\label{sec:immunoblot}
	
Les extraits protéiques sont préparés par lyse des cellules Huh-7.5 transfectées (voir \autoref{sec:transfection}) et infectées (voir \autoref{sec:infection}) dans respectivement 200 ou 100µL de tampon dénaturant (\textit{Lithium Dodecyl Sulfate Sample Buffer}, Thermo Fisher Scientific) contenant 5\% de ß-mercaptoéthanol (Sigma). Les lysats protéiques sont systématiquement dénaturés par incubation à 95°C pendant 15min et conservés à -20°C. Les protéines sont séparées par électrophorèse sur gel de polyacrylamide à 4-12\% ou 12\% d’acrylamide (\textit{NuPAGE Bis-Tris Gels}, Thermo Fisher Scientific) en tampon de migration (\textit{NuPAGE MES SDS Running Buffer} ou \textit{NuPAGE MOPS SDS Running Buffer}, Thermo Fisher Scientific) et transférées sur une membrane de nitrocellulose (\textit{Nitrocellulose Premium, 0.45μm}, Amersham) en système liquide (\textit{NuPAGE Transfer Buffer}, Thermo Fisher Scientific). Les membranes sont saturées pendant 1h dans une solution de DPBS contenant 0,1\% de Tween-20 (DPBS-T) et 5\% de lait écrémé déshydraté et incubées durant la nuit à 4°C en présence d’anticorps primaires dilués en DPBS-T contenant 1\% de lait. Le lendemain, les membranes sont extensivement rincées au DPBS puis incubées pendant 1h avec des anticorps secondaires anti-IgG de souris ou de lapin couplés à un fluorochrome DyLight 680 ou 800 (Thermo Fisher Scientific) dilués en DPBS-T contenant 1\% de lait. L’ensemble des anticorps primaires et secondaires utilisés pour cette étude sont listés dans les Tableaux \ref{tab:tabM1} et \ref{tab:tabM2}. Après élimination des anticorps par un traitement de 15min avec un tampon de dissociation (\textit{Stripping Buffer}, Euromedex), les protéines totales sont révélées par une incubation de 5min dans un colorant protéique (\textit{Revert 700 Total Protein Stain}, Li-Cor Biosciences), suivie d'une fixation dans une solution aqueuse contenant 30\% d’éthanol et 6,7\% d’acide acétique glacial. Les signaux de fluorescence aux deux longueurs d'onde dans le proche infrarouge sont détectés à l'aide du système d'imagerie laser Odyssey CLx (Li-Cor Biosciences) et quantifiés à l’aide du logiciel ImageStudioLite (Li-Cor Biosciences).

% 13 - IMMUNOMARQUAGE

	\section{Marquages fluorescents des lipides, des protéines et de l’ARN viral}
	\label{sec:immunomarquage}
	
Pour réaliser le marquage des GL, les cellules Huh-7.5 vivantes sont mises en contact avec une sonde fluorescente dérivée de l’acide dodécanoïque (BODIPY® 558/568 C12, Thermo Fisher Scientific) diluée dans du milieu sans sérum pendant 15h à 37°C préalablement à la fixation. Les cellules Huh-7.5 transfectées (voir \autoref{sec:transfection}) et infectées (voir \autoref{sec:infection}) destinées aux visualisations par imagerie confocale sont fixées pendant 20min avec une solution de paraformaldéhyde à 4\%, puis perméabilisées pendant 30min avec une solution de digitonine à 40µg/mL pour préserver la structure des GL. Comme méthode alternative de marquage des GL, les cellules fixées ont été incubées pendant 30min avec un colorant lipidique (\textit{LipidTox Red Neutral Lipid Stain}, Thermo Scientific). La révélation des ARN viraux de polarité positive et négative est réalisée par une approche d’hybridation \textit{in situ} (\textit{ViewRNA Cell Plus Assay}, Thermo Fisher Scientific) à l’aide de sondes qui s’hybrident respectivement aux bases 3733-4870 du brin positif (VF1-10121) ou 4904-5911 du brin négatif (VF6-11102) de la souche JFH-1, conçues par Shulla et al. \citep{RN818}. Ces régions sont situées dans les séquences codantes de NS5A et NS5B mais ne couvrent pas les mutations d’adaptation de la souche Jad, par conséquent, ces sondes s’hybrident parfaitement à l’ARN de Jad. En parallèle, des sondes contrôles ciblant les ARNm cellulaires du gène de l'\textit{ACTB} conçues par le fabricant ont été employées. Les étapes d’incubation avec les sondes spécifiques des brins (+) et (-) du VHC ou des transcrits de l'\textit{ACTB}, d’amplification et de révélation du signal ont été effectuées selon les instructions du fabricant. Pour les lames dédiées à révéler l’ARN viral, l’ensemble des solutions précédant l’hybridation ont été complémentés avec des inhibiteurs de ribonucléases. Afin de limiter l’action des détergents sur l’intégrité des GL, la solution de lavage incluse dans le kit a été remplacée par du DPBS. Pour le marquage fluorescent des protéines virales ou cellulaires, une saturation des sites antigéniques non spécifiques a été réalisée pendant 30min avec une solution de DPBS contenant du sérum de chèvre ou d’âne à 5\% selon l’espèce de l’anticorps secondaire employé. Les cellules ont ensuite été incubées pendant 1h avec l’anticorps primaire spécifique des protéines virales ou cellulaires dilué dans du DPBS contenant 1\% de sérum de chèvre ou d’âne. Les complexes immuns ont ensuite été révélés par incubation pendant 1h à l’obscurité en présence d’anticorps secondaires conjugués à un fluorochrome Alexa Fluor 488, 555 ou 647 (Thermo Fisher Scientific). L’ensemble des anticorps primaires et secondaires utilisés pour cette étude sont listés dans les Tableaux \ref{tab:tabM1} et \ref{tab:tabM2}. Les noyaux ont été révélés avec une solution à 1µg/mL de 4-,6-diamidino-2-phénylindole (DAPI) pendant 10min (Thermo Fisher Scientific), puis les lames sont conservées dans un milieu de montage non polymérisant (\textit{Ibidi mounting medium}, Ibidi).

% 14 - MICROSCOPIE CONFOCALE

	\section{Acquisition, traitement et analyse d’images par microscopie confocale}
	\label{sec:confocal}
	
Les lames sont observées avec un microscope confocal inversé (LSM 700, Zeiss) contenant quatre diodes laser de longueur d’émission de 405, 488, 555 et 639nm. Les images ont été acquises avec des objectifs 40X ou 63X en immersion à huile NA 1.4 à l’aide du logiciel de l’instrument (ZEN, Zeiss). Les filtres d'émission sont configurés de façon à détecter les signaux de manière séquentielle avec une restriction $\leq$460nm pour les noyaux colorés au DAPI, 490 à 555nm pour le fluorochrome Alexa Fluor 488, 550 à 620nm pour le fluorochrome Alexa Fluor 555, et $\geq$650nm pour le fluorochrome Alexa Fluor 647 ou les sondes d’hybridation \textit{in situ}. Le diaphragme confocal est ajusté à 1 unité Airy (UA) de manière à empêcher la lumière hors du plan focal d'atteindre le détecteur. La résolution des images lors de l’acquisition a été adaptée afin d’obtenir une taille de pixels de 0,05µm, pour se placer dans les paramètres optimaux de déconvolution (voir \autoref{section:preambule1}). Les images tridimensionnelles ont été automatiquement empilées en utilisant un intervalle vertical de 0,15µm sur le plan z entre chaque section acquise. Après l'acquisition, les images sont restaurées par l'outil de déconvolution du logiciel Huygens Professional (Scientific Volume Imaging), une procédure itérative qui permet de résoudre les biais induits par la limite de résolution de l'objectif. L'analyse volumétrique a été réalisée à l'aide de l'outil \textit{Huygens Object Analyzer}. Sur la base d’un seuil défini automatiquement par le logiciel, le signal de chaque canal a été segmenté en « objets » volumétriques permettant de calculer le nombre, le volume individuel de chaque objet et l’intersection spatial entre les objets de de chaque canal en voxels. Pour réaliser l’étude sur l’interaction des GL dites « appariées » et les zones de contact, un programme innovant sur Python a été développé spécialement pour ce projet par D. Ershov (\textit{Image Analysis Hub}, Institut Pasteur, Paris). En bref, les GL sont détectées à l’aide d’un détecteur multi-échelle et converties en un masque sphérique. La région représentant la couche protéique en surface a été calculée autour de chaque masque sphérique. L’épaisseur moyenne de l’enveloppe protéique a été fixée à 200nm sur la base du signal de Core mesuré dans les cellules infectées. Étant une estimation limitée par la diffraction, l’enveloppe est calculée de telle sorte à être immergée dans le masque sphérique des GL sur 100nm et à s’étendre à l’extérieur de celui-ci sur 100nm. Pour vérifier la précision de la segmentation, nous avons contrôlé que l’extraction des masques correspondant aux GL et à l’enveloppe protéique chevauchaient parfaitement les signaux sous-jacents respectifs. Les masques sphériques situés à une distance surface-à-surface inférieure à 200nm définissent les GL « appariées ». Les sites de contact représentent la zone de chevauchement des deux masques de l’enveloppe protéique appartenant aux GL « appariées ». L’intensité du signal extrait hors et à l’intérieur des zones de contact a été corrigé à partir du signal cytoplasmique et normalisé par rapport à une valeur moyenne du bruit de fond calculé sur les cellules naïves de la même expérience. Les analyses ont été effectuées sur 40 à 60 images 3D individuelles provenant de deux à trois expériences indépendantes. Les reconstructions tridimensionnelles utilisées comme illustrations ont été générées à l'aide d'Imaris (ImarisViewer 9.5.1, Oxford Instruments).

% 15 - MICROSCOPIE ELECTRONIQUE

	\section{Acquisition d’images par microscopie électronique}
	\label{sec:electronique}
	
Les cellules Huh-7.5 transfectées (voir \autoref{sec:transfection}) ou infectées (voir \autoref{sec:infection}) destinées aux études de microscopie électronique ont été fixées à froid dans un tampon à pH neutre contenant 4\% de paraformaldéhyde, 1\% de glutaraldéhyde et 0,1M de phosphate monosodique. Après 24h d’incubation à 4°C, les cellules ont été brièvement resuspendues en DPBS puis placées dans une solution à 2\% de tétrode d’osmium (AgarScientific, Stansted, \textit{United Kingdom}). Les échantillons ont été entièrement déshydratés par incubation successive avec des solutions graduelles d’éthanol et d'oxyde de propylène, puis laissés pendant une nuit dans de la résine Eson pure (Sigma). La polymérisation de la résine a été faite par incubation à 60°C pendant 48h. Des sections ultrafines (90nm) de ces blocs de résine ont été obtenues avec un ultramicrotome Leica EM UC7 (Wetzlar, \textit{Germany}) puis colorées avec 2\% d'acétate d'uranyle (Agar Scientific) et 5\% de citrate de plomb (Sigma). Les observations ont été effectuées avec un microscope électronique à transmission (JEOL 1011, Tokyo, \textit{Japan}).

% 13 - BIOINFORMATIQUES ET STATISTIQUES

	\section{Approches bioinformatiques et statistiques}
	\label{sec:bioinfo}

Les représentations graphiques des résultats ont été générées à l'aide des logiciel Prism 9 (GraphPad, San Diego, \textit{United States}), RStudio (Boston, \textit{United States}), Python (Fredericksburg, \textit{United States}) et Cytoscape (Seattle, \textit{United States}) \citep{RN822,RN823} puis assemblées dans Illustrator (Adobe, \textit{Denmark}) pour la préparation des figures  \citep{RN824}. L'analyse statistique des études volumétriques des GL et de Core (voir \autoref{section:stabilisation}, \ref{section:cluster}, \ref{section:coregl}, \ref{section:mutants} et \ref{section:gouttelettes}) a été réalisée sous R par H. Varet (Plateforme de Bioinformatique et Biostatistique, Institut Pasteur, Paris) à l'aide d'un modèle linéaire à effet mixte issu du paquet R lme4 version 1.1-23. Les variables liées à l’infection ont été considérées comme fixes, tandis que l’effet lié aux dates des expériences a été inclus comme aléatoire. Les cellules infectées et non infectées ont été comparées à chaque temps à l'aide du package R emmeans version 1.4.7 et les \textit{p-values} ont été ajustées à l'aide de la méthode de Tukey. L’analyse statistique des cinétiques de réplication (voir \ref{section:virus}) a été réalisée sous R par E. Jacquemet (Plateforme de Bioinformatique et Biostatistique, Institut Pasteur, Paris) à l'aide d'un modèle linéaire à effet mixte. Les trois paramètres mesurés ont été ajustés à l’aide d'un modèle similaire, en considérant les variables liées à l’infection comme la souche virale et le temps post-infection comme effets fixes. Les réplicats et la répétition des expériences qui font office de source de variabilité expérimentale ont été inclus comme effet aléatoire. L’analyse statistique des autres résultats quantitatifs a été effectuée sur Prism 9 à l’aide d’un t-test paramétrique ou d’une ANOVA univariée à comparaisons multiples selon la forme des données. Pour l’analyse de corrélation entre les variables relatives aux GL et à Core (voir \autoref{section:correlation}), les coefficients de corrélation r compris entre 1 et -1 ont été calculées par la méthode de Spearman à partir de l’algorithme standard inclus dans RStudio. La significance statistique des coefficients de corrélation a été évaluée à l’aide du paquet R corrplot version 0.84.

	\cleardoublepage