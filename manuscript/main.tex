%% Copyright (C) 2017-2021 Emeline Simon
%%
%% The current owner of this work is Emeline Simon
%% <contact at emeline.simon@gmail.com>.
%%
%% This is main.tex for my PhD Thesis.
%%
%%%%%%%%%%%%%%%%%%%%%%%%%%%%%%%%%%%%%%%%%%

\documentclass[a4paper, 12pt, twoside, openright]{report}
%%%% Font et typos %%%%%%%%%%%
\usepackage[utf8]{inputenc}
\usepackage[T1]{fontenc}

\usepackage[square,sort&compress,sectionbib]{natbib}	
%\bibpunct{\textcolor{black}{[}}{\textcolor{black}{]}}{\textcolor{black}{,}}{a}{}{\textcolor{black}{;}}
\usepackage{chapterbib}
		\renewcommand{\bibsection}{\chapter*{Références}}	

\usepackage[french]{babel}
\usepackage{amsmath,amsfonts,amssymb}
\usepackage{lmodern}
\usepackage{ae,aecompl}	 % Utilisation des fontes vectorielles modernes
\usepackage[upright]{fourier}

\usepackage{textgreek}
\usepackage{underscore}
\usepackage{blindtext}
%%%% Allure g�n�rale du document %%%%%%%%%%%
\usepackage{enumerate}
\usepackage{enumitem}
\usepackage[section]{placeins}	% Place un FloatBarrier � chaque nouvelle section
\usepackage{epigraph}
\usepackage[font=footnotesize]{caption}
\usepackage[francais,nohints]{minitoc}		% Mini table des mati�res
	\setcounter{minitocdepth}{2}	% Mini-toc d�taill�s (section/sous-section)
\usepackage[notbib]{tocbibind}		% Ajoute les Tables des Mati�res/Figures/Tableaux � la table des mati�res
\usepackage{setspace}
\onehalfspacing
\usepackage[raggedright]{titlesec}

\addtolength{\skip\footins}{2pc plus 5pt}

\newcommand\blfootnote[1]{%
  \begingroup
  \renewcommand\thefootnote{}\footnote{#1}%
  \addtocounter{footnote}{-1}%
  \endgroup}
  
%%%% Tableaux %%%%%%%%%%%
\usepackage{multirow}
\usepackage{booktabs}
\usepackage{colortbl}
\usepackage{tabularx}
\usepackage{multirow}
\usepackage{threeparttable}
\usepackage[table]{xcolor}
\definecolor{Gray}{gray}{0.9}
{\rowcolors{3}{white}{Gray}
\usepackage{stackengine}
\newcommand\xrowht[2][0]{\addstackgap[.5\dimexpr#2\relax]{\vphantom{#1}}}
\usepackage{array}
\newcolumntype{M}[1]{>{\centering\arraybackslash}m{#1}}
\usepackage{etoolbox}
	\appto\TPTnoteSettings{\footnotesize}
	\AtBeginEnvironment{tabular}{\footnotesize}
\addto\captionsfrench{\def\tablename{{\textsc{Tableau}}}}	% Renomme 'table' en 'tableau'

%%%% El�ments graphiques %%%%%%%%%%%                 
\usepackage{graphicx}			% Permet l'inclusion d'images
\usepackage{subcaption}
\usepackage{floatrow}
\usepackage{pdfpages}
\usepackage{rotating}
\usepackage{pgfplots}
	\usepgfplotslibrary{groupplots}
\usepackage{tikz}
	\usetikzlibrary{backgrounds,automata}
	\pgfplotsset{width=7cm,compat=1.3}
	\tikzset{every picture/.style={execute at begin picture={
   		\shorthandoff{:;!?};}
	}}
	\pgfplotsset{every linear axis/.append style={
		/pgf/number format/.cd,
		use comma,
		1000 sep={\,},
	}}
\usepackage{eso-pic}
\usepackage{import}
\usepackage{chngcntr}
\counterwithout{figure}{chapter}
\counterwithout{table}{chapter}

\usepackage[normalem]{ulem}

%%%% Navigation dans le document %%%%%%%%%%%      
\usepackage[pdftex,pdfborder={0 0 0},
			colorlinks=true,
			linkcolor=blue,
			citecolor=black,
			pagebackref=true,
			]{hyperref}	% Cr��ra automatiquement les liens internes au PDF

% \Autoref is for the beginning of the sentence
\let\orgautoref\autoref
\providecommand{\Autoref}{%
\def\figureautorefname{Figure}%
\def\tableautorefname{Tableau}%
\def\subfigureautorefname{Figure}%
\orgautoref}
% \autoref is used inside the sentence to produce Fig., and Eq. for figures, subfigures, and equations
\renewcommand{\autoref}{%
\def\figureautorefname{Fig.}%
\def\tableautorefname{Tableau}%
\def\subfigureautorefname{Fig.}%
\orgautoref}

\usepackage{nameref}

\newcommand{\fullref}[1]{\autoref{#1}. \nameref{#1}}

\usepackage{tocloft}
\newlength{\mylen}
\renewcommand*\cftfigpresnum{Figure~}
\settowidth{\mylen}{\cftfigpresnum\cftfigaftersnum}
\addtolength{\cftfignumwidth}{\mylen}

\renewcommand*\cfttabpresnum{Tableau~}
\settowidth{\mylen}{\cfttabpresnum\cfttabaftersnum}
\addtolength{\cfttabnumwidth}{\mylen}

%% PACKAGES CHARGES EN DERNIER  %%%%%%%%%%%      
	             
\usepackage[top=2.5cm, bottom=2cm, left=2.5cm, right=2.5cm,
			headheight=15pt]{geometry}

\usepackage{fancyhdr}			% Ent�te et pieds de page (apr�s geometry)
	\pagestyle{fancy}		% Indique que le style de la page sera justement fancy
	\lfoot[\thepage]{} 		% gauche du pied de page
	\cfoot{} 			% milieu du pied de page
	\rfoot[]{\thepage} 		% droite du pied de page
	\fancyhead[RO, LE] {}	
	
\usepackage[acronym, nonumberlist, nomain, nogroupskip,nopostdot]{glossaries}
\usepackage{glossary-mcols}
\renewcommand{\glossarypreamble}{\small}
\renewcommand{\glsnamefont}[1]{\textnormal{#1}}

	\makeglossaries
	\loadglsentries{glossary.tex}			% Utilisation d'un fichier externe pour la définition des entr�es (glossary.tex)		

%% Figures centr�es, et en position 'here, top, bottom or page'
\newenvironment{figureth}{%
		\begin{figure}[htb]
			\centering
	}{
		\end{figure}
		}
		
		
%% Tableaux centr�es, et en position 'here, top, bottom or page'
\newenvironment{tableth}{%
		\begin{table}[htb]
			\centering
	}{
		\end{table}
		}
%% Copyright (C) 2017-2021 Emeline Simon
%%
%% The current owner of this work is Emeline Simon
%% <contact at emeline.simon@gmail.com>.
%%
%% This is titlepage.tex for my PhD Thesis.
%%
%%%%%%%%%%%%%%%%%%%%%%%%%%%%%%%%%%%%%%%%%%

\makeatletter

\newcommand{\titlepagenew}{
\newgeometry{top=2.5cm, bottom=2cm, left=2cm, right=1cm}
  \begin{titlepage}
\begin{picture}(0,0)
\put(170,20){\makebox(0,0)[rt]{\includegraphics[width=6.5cm]{ressources/logo-ip.png}}}
\put(465,15){\makebox(0,0)[rt]{\includegraphics[width=7.5cm]{ressources/logo-virip.png}}}
\end{picture}
      \vspace{3.5cm}
      
\centering
      {\huge Université de Paris}\\
    \vspace{1cm}
      	{\Large École doctorale Bio Sorbonne Paris Cité - ED 562}\\
    \vspace{0.5cm}
   		{\large Institut Pasteur, Université de Paris, CNRS, UMR 3569}\\
   	    {\large Génétique Moléculaire des virus à ARN}\\
    \vspace{1.5cm}
    	{\LARGE \textbf{Liens entre l'origine génotypique de la protéine Core}}\\
    	{\LARGE \textbf{du virus de l'hépatite C et la stéatose hépatique}}\\
    \vspace{1.5cm}
     {\Large Par Emeline Simon}\\
    \vspace{1cm}
    	{\Large Thèse de doctorat d'Infectiologie}\\
    \vspace{1cm}
    	{\large Dirigée par le Dr. Annette Martin}\\
    \vspace{1cm}
    	{\large Présentée et soutenue publiquement le 7 décembre 2021}\\
      \vspace{1cm}
	\hspace{-6cm}
	{Devant un jury composé de :}\\
    \vfill
    \begin{tabbing}
	Mirjam Zeisel \hspace{9em} \= Université de Lyon \hspace{5em} \= CR-HDR \hspace{2em} \= Rapportrice \\
	Joachim Lupberger \> Université de Strasbourg \> CR-HDR \> Rapporteur \\
	 Yves Rouillé \> Université de Lille \> DR-HDR \> Examinateur \\
	Pierre-Emmanuel Ceccaldi \> Université de Paris \> PU-HDR \> Président du jury \\
	Annette Martin \> Université de Paris \> DR-HDR \> Directrice de thèse
	\end{tabbing}
	\vfill
  \end{titlepage}

\restoregeometry
}
\makeatother
%% Copyright (C) 2017-2021 Emeline Simon
%%
%% The current owner of this work is Emeline Simon
%% <contact at emeline.simon@gmail.com>.
%%
%% This is coverpage.tex for my PhD Thesis.
%%
%%%%%%%%%%%%%%%%%%%%%%%%%%%%%%%%%%%%%%%%%%

\makeatletter

\newcommand{\coverpage}{
\newgeometry{top=4cm, bottom=2cm, left=2.75cm, right=2.75cm}
  \begin{titlepage}
    {\LARGE \textbf{Résumé}} \\
    \vspace{0.5cm}
\begin{singlespace}
L'infection chronique par le virus de l'hépatite C (VHC) est une maladie du foie progressive qui peut mener à une stéatose, une accumulation anormale de gouttelettes lipidiques (GL) dans les hépatocytes. Des études cliniques ont mis en évidence une corrélation entre l'infection chronique par les VHC de génotype 3 et une prévalence élevée de stéatose, ainsi qu'un taux de progression accru vers la cirrhose et le carcinome hépatocellulaire. Parmi les facteurs viraux, la protéine de capside (Core) présente une association singulière et critique à la surface des GL et joue un rôle clé dans la dérégulation des voies de signalisation hépatiques, représentant des facteurs de risque potentiels pour la stéatose. Dans ce contexte, les objectifs de mon projet de thèse étaient (i) d'évaluer l'impact de l'infection par le VHC sur la biogénèse et la dynamique des GL, et (ii) d'identifier si les génotypes ou les polymorphismes de Core sont impliqués dans la dérégulation de l'homéostasie des GL et des voies métaboliques hépatocytaires clés qui pourraient sous-tendre les aggravations cliniques. \\ \\
\indent
En lien avec le premier objectif, des analyses cinétiques par imagerie quantitative d'hépatocytes infectés par une souche prototypique du VHC ont révélé que le contenu global des GL est inchangé, ce qui suggère que l'infection n'a pas d'impact sur la biogenèse globale des GL. Cependant, au fur et à mesure que l’infection progresse, les GL montrent un élargissement et une redistribution marquée sous la forme d’agrégats compacts, portant Core à leurs sites de contact et à leur surface. En utilisant des ARN du VHC codant une protéine Core mutée incapable de s'associer aux GL, aucun élargissement ou regroupement de GL n'a été observé. Ces résultats indiquent que Core est le principal moteur de l'agrégation des GL, potentiellement en reliant physiquement les GL limitrophes, en déstabilisant leur tension de surface et en favorisant leur fusion. Le rôle fonctionnel de l'agrégation et de la redistribution des GL pourrait être de délivrer la protéine de capside aux sites d'assemblage du VHC. \\ \\
\indent
En lien avec le second objectif, de nouveaux virus intergénotypiques ont été produits, exprimant des protéines Core hétérologues dérivées d'isolats cliniques de sous-types 1a, 2a, 3a, 4a et 4f et associés à divers degrés de stéatose. L'ampleur de l'élargissement des GL dans les cellules infectées par ces différents virus varie selon les séquences de Core, mais indépendamment des génotypes ou des degrés de stéatose. En parallèle, des analyses transcriptomiques comparatives à haut débit d'hépatocytes infectés par les virus intergénotypiques ont mis en évidence des dérégulations hépatiques différentielles marquées. En outre, plusieurs facteurs prédictifs de la stéatose se sont avérés spécifiquement modulés par les protéines Core de génotype 1a, 3a ou 4. Ces résultats révèlent une modulation différentielle de facteurs pro-stéatogènes clés en lien avec l'origine génotypique de la protéine Core du VHC, ce qui pourrait \textit{in fine} fournir des marqueurs de progression de cette pathologie hépatique.
\end{singlespace}
  \end{titlepage}

\restoregeometry
}

% Méta-données du PDF
\hypersetup{
    pdfauthor={Emeline Simon},
    pdfsubject={Manuscrit de thèse de doctorat},
    pdftitle={Liens entre l'origine génotypique de la protéine Core du virus de l'hépatite C et la stéatose hépatique},
    pdfkeywords={virus de l'hépatite C, origine génotypique, protéine Core, stéatose, gouttelettes lipidiques, clusterisation, fusion}
}

\begin{document}
	\pagenumbering{roman}
	\coverpage
	
	\cleardoublepage
		
	% Avant-propos
\chapter*{Avant-propos}
\phantomsection\addcontentsline{toc}{chapter}{Avant-propos}
\section*{Préface}

\noindent
Le manuscrit et le projet de la thèse de doctorat ont été réalisés par Emeline Simon sous la direction du Dr. Annette Martin.\\ \\
\noindent
Ce projet de thèse a été financé pour trois ans suite à l'attribution d'une bourse nominative du concours de l'école doctorale BioSorbonne Paris Cité (ED 562). Une quatrième année a été financée par l'obtention d'une allocation de recherche par l'ANRS Maladies Infectieuses Émergentes. \\ \ \\
\textbf{Institut Pasteur, Département de Virologie} \\
Groupe Interactions Hepacivirus/Hôte (HEPA) \\
Unité de Génétique Moléculaire des virus à ARN (GMVR) \\ 
25 rue du Docteur Roux, Bâtiment Borrel, 75724 PARIS CEDEX 15  \\
Chef de groupe : Dr. Annette Martin \\
Directrice de l'unité : Pr. Sylvie van der Werf \\ \\
\textbf{Université de Paris}  \\
45 rue des Saints-Pères, 75298 PARIS CEDEX 06  \\
Directrice de l'université : Pr. Christine Clerici  \\  \\
\textbf{École Doctorale BioSorbonne Paris Cité - ED 562}   \\ 
35 rue Hélène Brion, 75205 PARIS CEDEX 13  \\
Directrice de l'école doctorale : Pr. Caroline Le Van Kim \\
Co-directrice de l'école doctorale : Dr. Chantal Desdouets \\
Gestionnaire : Joanna Levy \\ \\
\textbf{ANRS - Maladies Infectieuses Émergentes}   \\
101 rue de Tolbiac, 75205 PARIS CEDEX 13  \\
Directeur de l'agence : Pr. Yazdan Yazdanpanah \\
Responsable du département scientifique : Dr. Cécile Peltekian

	\cleardoublepage
\section*{Remerciements}

\noindent
Je tiens tout d'abord à adresser mes plus grands remerciements au \textbf{Dr. Annette Martin} pour m'avoir encadrée pendant ces 4 années de thèse. Merci pour la confiance et l'indépendance que tu m'as accordée tout au long du projet, pour tes nombreux conseils avisés et toutes ces discussions dont je suis sortie enrichie tant sur le plan scientifique que sur le plan humain, pour tes encouragements face aux résultats négatifs et à ma frustration et pour avoir toujours été une oreille disponible, critique et solidaire. Merci pour toutes les fois où tu as su te rendre disponible lors de mes besoins. Merci de m'avoir formée avec tant de patience et de pédagogie, de m'avoir fait grandir et de m'avoir fait découvrir le magnifique métier de chercheur avec tant de dévouement. Je suis ravie d'avoir partagé cette aventure de la thèse dans ton équipe qui restera à jamais dans ma mémoire. \\ \\
\noindent
Je remercie le \textbf{Pr. Sylvie van der Werf}, directrice de l'unité de Génétique Moléculaire des virus à ARN à l'Institut Pasteur, pour son soutien et pour m'avoir accueillie dans son unité durant ces travaux de thèse. \\ \\
\noindent
Je tiens particulièrement à exprimer ma gratitude aux \textbf{Drs. Mirjam Zeisel} et \textbf{Joachim Lupberger} pour avoir accepté volontiers d'évaluer mon travail de thèse et pour leurs suggestions qui ont contribué à améliorer ce manuscrit. Je remercie tout autant le \textbf{Dr. Yves Rouillé} et le \textbf{Pr. Pierre-Emmanuel Ceccaldi} pour leur disponibilité et pour avoir gentiment consenti à faire partie de mon jury de thèse. \\ \\
\noindent
Tous mes remerciements à nos collaborateurs, en particulier au \textbf{Pr. Philippe Roingeard} pour sa bienveillance, sa disponibilité, ses nombreux conseils et pour avoir partagé de bon cœur son expérience en microscopie électronique au cours de ces quatres années. Merci au \textbf{Dr. Claire Gondeau} pour sa contribution dans les premières étapes du projet, pour avoir fourni et caractérisé les souches cliniques et pour avoir pris en main les expériences préliminaires sur les cultures primaires d'hépatocytes. \\ \\
\noindent
I would like to sincerely thank \textbf{Dr. Dmitry Ershov} for agreeing to participate in this work, which was originally intended to be a short-term collaboration. He turned out to be an outstanding collaborator whom I warmly thank for his kindness, for the interest he naturally and quickly showed for the project, for his availability during the long brain storming sessions and for the rigorous work he provided which allowed me to obtain such advanced results. \\ \\
\clearpage
\noindent
Je tiens également à remercier les équipes des \textbf{Drs. Laleh Majlessi} et \textbf{Nicolas Escriou} pour m'avoir permis de participer à cette belle aventure sur les projets vaccinaux du SARS-CoV-2 menés à l'Institut Pasteur. Un grand merci tout particulier à Nicolas pour s'être fréquemment rendu disponible et pour ses nombreux conseils lors des répétitions de mes présentations orales. \\ \\
\noindent
Je tiens à remercier le \textbf{Dr. Matthieu Fritz}, ancien doctorant au sein de l'équipe HEPA, pour m'avoir encadrée dès mon arrivée en Master et et pour le savoir technique et théorique qu'il m'a prodiguée jusqu'à son départ. \\ \\
\noindent
I am also grateful to \textbf{Dr. Stephanie Aicher}, former doctoral student in the HEPA team, who initiated the basis of the work I have pursued during this project and for providing valuable viral tools. \\ \\
\noindent
Je tiens à remercier chaleureusement \textbf{Brigitte Blumen} et \textbf{Damien Batalie}, techniciens au sein de l'équipe HEPA, pour leur soutien et leur contribution considérable à ce projet sur le plan technique. Merci à toi, Brigitte, pour ton aide précieuse sur la biologie moléculaire, pour toutes les constructions et les mutants que tu as rigoureusement préparés. Merci à toi, Damien, pour m'avoir aidée à préparer tous ces stocks viraux et pour avoir finalisé de nombreuses expériences. Je vous souhaite à tous les deux une très bonne continuation dans l'équipe et plein de bonnes choses tant sur le plan professionnel que personnel. Merci à \textbf{Houda Tabbal}, post-doctorante au sein de l'équipe HEPA, pour avoir pris le temps d'écouter et de me conseiller sur les répétitions de mes présentations orales. \\ \\
\noindent
A big thank to \textbf{Angeliki-Anna Beka}, for having participated in my project during her Erasmus internship, for her friendship, for her enthusiasm and for her constant encouragements. I hope I was a good tutor even though I was much less present in the last year. I wish you a wonderful thesis, which, I have no doubt, will be full of beautiful discoveries ! \\ \\
\noindent
Un grand merci à \textbf{Martine Vandaele}, \textbf{Edmond Bellance}, \textbf{Jennifer Chevalier} et \textbf{Mohamed Hicaine} pour leur aide au quotidien dans l'organisation et la préparation du matériel du laboratoire et pour leur travail dans « l'ombre » essentiel au succès des projets de recherche. \\ \\
\noindent
Merci à \textbf{François Pénin}, anciennement responsable d'entité à l'Institut de Biologie et Chimie des Protéines à Lyon, pour avoir gentiment partagé son expertise et ses prédictions structurales sur les protéines mutantes. \\ \\
\clearpage
\noindent
I would particularly like to express my respect and gratitude to \textbf{Dr. Eva Herker}, group leader in the University of Marburg, for the interest she showed in my modest work, for her availability and for her scientific advices which helped me to progress in this project. I am deeply sorry that the COVID crisis has impacted our project plans, I would have loved to discover your laboratory and carry out this collaborative work with your team. \\ \\
\noindent
Je souhaite remercier \textbf{Julien Fernandes}, \textbf{Lesly Raulin} et \textbf{Jean Yves Tinevez}, ingénieurs de recherche aux plateformes de Bioimagerie Photonique et d'Analyse d'Images de l'Institut Pasteur, pour m'avoir formée et conseillée sur l'utilisation du microscope confocal et des logiciels d'imagerie. \\ \\
\noindent
Je tiens également à remercier les membres de la plateforme Biomics de l'Institut Pasteur et plus particulièrement \textbf{Thomas Cokelaer}, \textbf{Juliana Pipoli Da Fonseca}, \textbf{Rachel Legendre}, \textbf{Saliha Azebi} et \textbf{Marc Monot} pour l'obtention et l'analyse des résultats du crible transcriptomique. \\ \\
\noindent
Je remercie également \textbf{Hugo Varet} et \textbf{Elise Jacquemet}, ingénieurs de recherche aux plateformes de Métabolomique et de Bioinformatique et Biostatistique de l'Institut Pasteur, pour leur contribution très significative à ce projet (\textit{p-value} < 0.0001) et pour avoir réalisé la majeure partie des analyses statistiques de mes résultats. \\ \\
\noindent
Je remercie les \textbf{Drs. Takaji Wakita}, \textbf{Charles Rice}, \textbf{Ralf Bartenschlager}, \textbf{John McLauchlan}, \textbf{Agata Budkowska}, \textbf{Athanasios Kakkanas} et \textbf{Penelope Mavromara} pour le don de matériels biologiques. \\ \\
\noindent
Merci aux \textbf{Drs. Serge Gangloff} et \textbf{Valérie Choumet} pour avoir gentiment accepté de participer à mon comité de suivi de thèse et pour le temps qu'ils ont bien voulu accorder à l'évaluation annuelle de mes travaux. \\ \\
\noindent
Je tiens à remercier \textbf{Elise Caliot}, conseillère carrière au sein de la MAASC, pour m'avoir encouragée et soutenue tout au long de la thèse et pour m'avoir donné de nombreux conseils et contacts afin de préparer ma future carrière. Je m'excuse d'avoir négligé cet aspect important de la vie professionnelle par manque de temps sur la dernière année, mais je vous promets de me concentrer à nouveau très bientôt sur mon avenir. \\ \\
\noindent
Un grand merci à \textbf{Coraline Méheux} pour son aide précieuse dans l'organisation de la soutenance de thèse. \\ \\
\clearpage
\noindent
Je tiens à témoigner toute ma reconnaissance aux personnes suivantes, pour l'expérience inoubliable qu'elles m'ont fait vivre durant ces quatres années au sein de l'Institut Pasteur : \\ \\
\noindent
Au nouveau \textbf{Dr. Mathieu de Jode}, sans conteste un des meilleurs amis que je me suis faite à Paris, pour son soutien, pour avoir toujours cru en mes capacités (pourtant, c'est un vrai génie par rapport à moi :D), les longues discussions pendant les pauses café, nos anniversaires en commun et tous les moments géniaux que nous avons passé ensemble. J'espère de tout mon cœur que ton nouveau poste à Bruxelles sera enrichissant mais ne nous éloignera pas trop et je te souhaite de réussir dans ta carrière et dans la vie. \\ \\
\noindent
À tous les futurs docteurs de la même promotion avec qui j'ai partagé ce chemin, en particulier \textbf{Nicolas Minier} pour son amitié, sa gentillesse et cette campagne géniale de D\&D qui m'a permis de souffler tout au long de la thèse et \textbf{Marta Mastrogiovanni} pour son grand cœur, sa positivité, son humour et pour avoir partagé avec moi des séances de microscopie de dernière minute ;). Je suis très heureuse d'avoir partagé cette grande aventure qu'est la thèse, y compris toutes ses difficultés, à vos côtés. Je vous souhaite à tous beaucoup de bonheur pour l'avenir, tant sur le plan professionnel que personnel.  \\ \\
\noindent
À tous mes autres compagnons et ex-compagnons pasteuriens, en particulier \textbf{Sumit Kumar}, \textbf{Ségolène Gracias}, \textbf{Yves-Joël Diegues}, \textbf{Thibault Snollaerts}, \textbf{Florian Bakoa}, \textbf{Rémi Denise}, \textbf{Warren Bakkay} et \textbf{Giulia Gallo} pour leur énorme soutien, pour avoir fait naître cette belle amitié et pour tous les moments géniaux passés ensemble, que ce soit à l'intérieur ou à l'extérieur du campus. \\ \\
\noindent
To all the former and current members of \textbf{StaPa} with whom I shared this sometimes difficult but rewarding task of making this association growth, which is so dear to my heart and has allowed me to meet incredible people and most of my friends here on the campus. \\ \\
\noindent
Tous mes remerciements à mes anciens camarades du Master 2 de Virologie, en particulier \textbf{Manon Lemasson}, \textbf{Alexandre Nicolas}, \textbf{Sarah Lesage}, \textbf{Salomé Bourgeau} et \textbf{Kevin Villette} qui ont fortement animé ma vie d'étudiante parisienne, pour toutes ces bonnes soirées passées ensemble et pour avoir partagé un bout de chemin avec moi avant et pendant le doctorat. Sans vous, l'année du Master aurait été bien plus fade ;) \\ \\
\noindent
J'adresse également des énormes remerciements à mes amis de la licence de Biologie à Rennes, \textbf{Maëla Semery}, \textbf{Laëtitia Hervot}, \textbf{Aëla Vancassel}, \textbf{Laure Olivier}, \textbf{Maxence Lejars} et \textbf{Maxime Royez} avec qui j'ai découvert un amour commun pour la science. Merci pour tous ces souvenirs et soirées mémorables passées ensemble et pour avoir tant fait grandir la petite « Emy » en jeune adulte avec votre générosité, votre maturité, votre ouverture et bien sûr, votre longue amitié. \\ \\
\noindent
Je tiens à remercier chaleureusement mes amis d'enfance \textbf{Mossa Diallo}, \textbf{Liza Robert}, \textbf{Donia Benhlal} et \textbf{Youen Chapron} pour avoir été pendant si longtemps le cœur de ma vie bretonne, pour cette longue et durable amitié et pour avoir été un véritable soutien malgré la distance. Merci de m'avoir toujours accueillie avec enthousiasme et d'avoir souvent soulagé ma stressante vie parisienne avec votre fraîcheur et votre naturel. \\ \\
\noindent
J'adresse mes plus chaleureuses pensées et ma plus grande gratitude à ma famille, en particulier à mes parents \textbf{Didier} et \textbf{Anne} et à mon frère \textbf{Nicolas} qui ont toujours cru en moi, m'ont soutenu jusqu'au bout et m'ont donné l'énergie et les ressources pour réussir. Merci pour votre immense soutien émotionnel, pour votre confiance et pour m'avoir acceptée telle que je suis, je me sens vraiment chanceuse de vous avoir. Grâce à vous, je crois que j'ai enfin « passé mon bac ». Merci à ma grand-mère \textbf{Marie}, pour son incroyable tendresse, pour avoir pris soin de moi pendant des années étant enfant (et même adulte), pour sa patience et sa compréhension malgré la distance et pour ses délicieuses galettes le vendredi, sans conteste les meilleures qui existent en Bretagne. Merci à mon oncle \textbf{Daniel}, ma tante \textbf{Christiane} et mes cousins \textbf{François} et \textbf{Laura} pour leur soutien, leur présence et pour cette superbe atmosphère qui règne dans la famille. Pour finir, j'adresse une pensée affectueuse à mon grand-père \textbf{Joseph}, un homme tendre et admirable décédé au cours de l'année, avec qui je regrette énormément de ne pas pouvoir partager cette importante étape de ma vie. \\ \\
\noindent
Un grande ringraziamento alla famiglia \textbf{Pinto}, in particolare a \textbf{Erika} per la sua gentilezza e il suo amore per i francesi :p, ad \textbf{Antonio} per la sua calda accoglienza e per avermi fatto sentire a mio agio fin dal primo minuto e a \textbf{Fabia} per la sua generosità, la sua tenerezza e per questa meravigliosa zuppa di verdure. \\ \\
\noindent
I would like to express my heartfelt thanks to \textbf{Giulia Pinto}, my most beautiful discovery of the last two years. Thank you for bringing so much happiness into my life and for transforming my daily routine. Thank you for being my main ally and true cheerleader, for having supported and relieved my doubts, my angers, my fears and my little too tenacious imposter syndrome, for believing in me so much and for pushing me until the day I finally wrote these lines. I am proud to have made you proud and I have no doubt that you will soon find the keys that will lead you to a wonderful future. \\ \\
\noindent
Enfin, ce travail n'aurait pas été possible sans le soutien de l'\textbf{Université de Paris}, de la commission de l'\textbf{ANRS} et du \textbf{Département de Virologie} de l'Institut Pasteur, qui m'ont permis, grâce à une bourse doctorale, une allocation de recherche et diverses aides financières de réaliser et de terminer cette thèse avec sérénité.

	\cleardoublepage

\chapter*{Résumé}
\phantomsection\addcontentsline{toc}{chapter}{Résumé}
\begin{singlespace}
L'infection chronique par le virus de l'hépatite C (VHC) est une maladie du foie progressive qui peut mener à une stéatose, une accumulation anormale de gouttelettes lipidiques (GL) dans les hépatocytes. Des études cliniques ont mis en évidence une corrélation entre l'infection chronique par les VHC de génotype 3 et une prévalence élevée de stéatose, ainsi qu'un taux de progression accru vers la cirrhose et le carcinome hépatocellulaire. Parmi les facteurs viraux, la protéine de capside (Core) présente une association singulière et critique à la surface des GL et joue un rôle clé dans la dérégulation des voies de signalisation hépatiques, représentant des facteurs de risque potentiels pour la stéatose. Dans ce contexte, les objectifs de mon projet de thèse étaient (i) d'évaluer l'impact de l'infection par le VHC sur la biogénèse et la dynamique des GL, et (ii) d'identifier si les génotypes ou les polymorphismes de Core sont impliqués dans la dérégulation de l'homéostasie des GL et des voies métaboliques hépatocytaires clés qui pourraient sous-tendre les aggravations cliniques. \\ \\
\indent
En lien avec le premier objectif, des analyses cinétiques par imagerie quantitative d'hépatocytes infectés par une souche prototypique du VHC ont révélé que le contenu global des GL est inchangé, ce qui suggère que l'infection n'a pas d'impact sur la biogenèse globale des GL. Cependant, au fur et à mesure que l'infection progresse, les GL montrent un élargissement et une redistribution marquée sous la forme d'agrégats compacts, portant Core à leurs sites de contact et à leur surface. En utilisant des ARN du VHC codant une protéine Core mutée incapable de s'associer aux GL, aucun élargissement ou regroupement de GL n'a été observé. Ces résultats indiquent que Core est le principal moteur de l'agrégation des GL, potentiellement en reliant physiquement les GL limitrophes, en déstabilisant leur tension de surface et en favorisant leur fusion. Le rôle fonctionnel de l'agrégation et de la redistribution des GL pourrait être de délivrer la protéine de capside aux sites d'assemblage du VHC. \\ \\
\indent
En lien avec le second objectif, de nouveaux virus intergénotypiques ont été produits, exprimant des protéines Core hétérologues dérivées d'isolats cliniques de sous-types 1a, 2a, 3a, 4a et 4f et associés à divers degrés de stéatose. L'ampleur de l'élargissement des GL dans les cellules infectées par ces différents virus varie selon les séquences de Core, mais indépendamment des génotypes ou des degrés de stéatose. En parallèle, des analyses transcriptomiques comparatives à haut débit d'hépatocytes infectés par les virus intergénotypiques ont mis en évidence des dérégulations hépatiques différentielles marquées. En outre, plusieurs facteurs prédictifs de la stéatose se sont avérés spécifiquement modulés par les protéines Core de génotype 1a, 3a ou 4. Ces résultats révèlent une modulation différentielle de facteurs pro-stéatogènes clés en lien avec l'origine génotypique de la protéine Core du VHC, ce qui pourrait \textit{in fine} fournir des marqueurs de progression de cette pathologie hépatique. \\ \\
Mots clés : virus de l'hépatite C - origine génotypique - protéine Core - stéatose - \\ gouttelettes lipidiques - clusterisation - fusion
\end{singlespace}

	\cleardoublepage
	
\chapter*{Summary}
\begin{singlespace}
Chronic hepatitis C virus (HCV) infection is a slow, progressive liver disease leading to steatosis, an abnormal accumulation of lipid droplets (LDs) in hepatocytes. Clinical studies have reported a correlation between HCV genotype 3 chronic infection and a higher prevalence of steatosis, as well as increased progression rate toward cirrhosis and hepatocellular carcinoma. Among viral factors, the capsid (Core) protein displays a peculiar and critical association at the surface of cytosolic LDs and has a key role in the dysregulation of hepatocyte signaling and metabolic pathways, representing potentially important risk factors for steatosis.  In this context, the aims of my PhD project were to (i) assess how HCV infection impacts LD biogenesis and dynamics, and (ii) identify whether genotype-specific Core polymorphisms are involved in the dysregulation of hepatocyte LD homeostasis and key metabolic pathways that may account for clinical manifestations. \\ \\
\indent
In addressing the first aim, quantitative imaging time-course analyses of hepatocytes infected with an HCV prototypic strain revealed that global LD content remained unaltered, suggesting that infection did not impact the overall LD biogenesis. However, as infection progressed, LDs displayed a marked redistribution in large and compact clusters, bearing Core at their contact sites and surface. Using HCV RNAs encoding mutated Core unable to associate to LDs, no LD enlargement or clustering was observed. These results indicate that Core is the main driver of LD clustering, possibly by physically bridging neighboring LDs, destabilizing their surface tension and favoring their fusion. The functional role of LD aggregation and redistribution may be the delivery of capsid protein at the HCV assembly sites.\\ \\
\indent
To address the second aim, a set of novel intergenotypic viruses was produced that expressed heterologous Core proteins derived from clinical isolates of subtypes 1a, 2a, 3a, 4a and 4f and associated with various steatosis grades. The extent of LD local enlargement in cells infected with these intergenotypic viruses was found to vary according to Core sequences, yet apparently irrespective of HCV genotypes or steatosis grades. In parallel, comparative high-throughput transcriptomic analyses of hepatocytes infected with the recombinant viruses showed marked differential impact in hepatic gene dysregulations. In addition, several steatosis-predictive factors were found to be specifically modulated by genotype 1a, 3a or 4 Core. These findings reveal a differential modulation of key pro-steatogenic factors in link with the genotypic origin of HCV Core that may ultimately provide markers of liver disease progression. \\ \\
Key words : hepatitis C virus - genotypic origin - core protein - steatosis - lipid droplets - clustering - fusion
\end{singlespace}
\cleardoublepage
	

	% Table des matières
\setcounter{tocdepth}{1}
\dominitoc
\tableofcontents
						
	% Liste des acronymes
			\printglossary[type=\acronymtype, style=indexgroup, title={Liste des acronymes}, toctitle={Liste des acronymes}]
			\glsaddall
			
\cleardoublepage

	% Liste des figures
			\renewcommand*\listfigurename{Liste des figures}

\listoffigures%

			\captionsetup[figure]{labelfont={bf}}

\cleardoublepage

	% Liste des tableaux
\listoftables%

			\captionsetup[table]{labelfont={bf}}
			
\cleardoublepage

%%%%%%%%%%%%%%%%%%%%%%%%%%%%%%%%%%%%%		
%        Contenu du document        %
%%%%%%%%%%%%%%%%%%%%%%%%%%%%%%%%%%%%%
	\setcounter{mtc}{5}	% "Corrige" les minitocs décalés cause des chapitres *
	\pagenumbering{arabic}
	%% Copyright (C) 2017-2021 Emeline Simon
%%
%% The current owner of this work is Emeline Simon
%% <contact at emeline.simon@gmail.com>.
%%
%% This is 01_introduction.tex the first chapter of my PhD Thesis.
%%
%%%%%%%%%%%%%%%%%%%%%%%%%%%%%%%%%%%%%%%%%%

\chapter{\'{E}tude bibliographique}
	\minitoc
	\newpage

%%%%%%%%%%%%%%%%%%%%%%%%%%%%%%%%%%%%%%%%%%%%%%%%%%%%%%%%%%%%%%%%%%%%%%%%%%%%%%%%%%%%%%%%%%%%%%%%%%%%%%%%%%%%%%%%%%%%%%%%%%%%%%%%%%%%%%%%%%%%%%%%%%%%%%%%%%%%%%%%%%%%%%%%
	
% 1 - LE FOIE HUMAIN

\section{Le foie humain}

Le foie est l’organe le plus volumineux du corps humain : une masse spongieuse brune rougeâtre d’environ 1,5kg et 30 cm de large, qui se situe au niveau de la cavité abdominale, sous le diaphragme. Il est le centre du métabolisme des nutriments et de l’élimination des déchets métaboliques, faisant de lui un des organes les plus vitaux. Son principal rôle est de contrôler le parcours et l’effet notoire des substances absorbés par le système digestif avant de les redistribuer au système circulatoire. \\ \\
\indent
Au cours de ce chapitre nous allons brièvement traiter de la structure et des fonctions principales du foie pour mieux comprendre les aspects physiologiques qui peuvent mener au développement de la stéatose, principale pathologie hépatique au coeur de cette étude, détaillée dans la \autoref{section:steatose}.

	\subsection{Anatomie, histologie et structure du foie}

\subsubsection{L'organe}
Le foie est divisé en deux lobes cunéiformes (droit et gauche) de taille et de forme inégale séparés par un ligament falciforme \citep{RN1237}. On peut également individualiser deux lobes mineurs situés sur la face viscérale du foie : le lobe caudé et le lobe carré. La vésicule biliaire est attachée au foie à la limite du lobe carré et du lobe hépatique droit (\autoref{fig:fig1}). Ils sont séparés par un sillon appelé le hile du foie : c’est à ce niveau que les vaisseaux sanguins pénètrent dans le foie et que passent les canaux biliaires majeurs. \\ \\
\indent
Le foie est relié à deux grands vaisseaux sanguins : l’artère hépatique et la veine porte. L'artère hépatique transporte le sang riche en oxygène de l'aorte, tandis que la veine porte transporte le sang riche en nutriments digérés par le tube digestif. En pénétrant dans le foie, ces vaisseaux sanguins se subdivisent en petits capillaires appelés sinusoïdes hépatiques, qui permettent l’apport des nutriments et de l’oxygène aux différentes cellules du foie. Parallèlement aux vaisseaux sanguins, le foie est parcouru par un grand nombre de canaux biliaires. Ils collectent la bile stockée dans la vésicule biliaire et la transportent vers le duodénum où elle sera utilisée pour la digestion.

	\begin{figureth}
	\centering
			\includegraphics[width=0.65\linewidth]{Figure_1.png}
		\caption[Le foie et les organes connexes]{\textbf{Le foie et les organes connexes.} Les deux lobes majeurs du foie (en brun) et les structures du système digestif connectées au système hépatique sont indiqués (Adapté de \citealt{RN1230}).}
				\label{fig:fig1}
	\end{figureth}
		\FloatBarrier
	

	\subsubsection{Structure du tissu hépatique}
Le tissu hépatique est composé de différents types cellulaires : les hépatocytes, les cellules épithéliales biliaires, les cellulaires stellaires, les cellules de Kupffer et les cellules endothéliales sinusoïdales. Chaque cellule hépatique a des fonctions uniques bien définies qui peuvent être liées à la structuration de l’organe, au métabolisme ou à la protection de l’organisme contre les pathogènes ou les substances xénobiotiques \citep[pour revue,][]{RN251}. \\ \\
\indent
Les hépatocytes sont des cellules cubiques hautement spécialisées, d’une taille allant de 20 à 30µm. Ils représentent la majorité de la masse hépatique (60\% des cellules constitutives du foie) et assurent la plupart des fonctions hépatiques liées au métabolisme. Les hépatocytes sont des cellules polarisées : c’est-à-dire qu’elles remplissent des fonctions distinctes selon l’orientation de leur membrane plasmique, basale ou apicale. Les hépatocytes peuvent contenir jusqu’à 8 jeux de chromosomes, ce qui faciliterait la régénération du foie. Les cellules stellaires jouent un rôle fondamental dans l’absorption et le stockage des vitamines liposolubles A et D. Des atteintes au foie peuvent activer et transformer les cellules stellaires en myofibroblastes, qui vont produire des composants de la matrice extracellulaire comme du collagène I, de la fibronectine et des protéoglycanes. Bien que ces éléments contribuent à la cicatrisation, une activation chronique des cellulaires stellaires représente un facteur important dans la progression des stades fibrotiques. Les cellules de Kupffer constituent la population de macrophages résidents du foie. Ils assurent la tolérance et l’élimination des corps étrangers constamment introduits par le système digestif. \\ \\
\indent
A l’échelle microscopique, les cellules hépatiques sont organisées sous forme de lobules, la plus petite unité fonctionnelle du foie \citep{RN1236}. Ce sont des structures hexagonales formées autour d’une veine centrale, dans lesquelles s’agencent des rangées d’hépatocytes, des sinusoïdes et des canalicules biliaires (\autoref{fig:fig2}\textcolor{blue}{.A}). Ces petites structures secondaires prennent naissance à partir du canal biliaire, de la veine portale et de l’artère hépatique qui forment la triade portale, à chaque angle du lobule. La structure des sinusoïdes est maintenue par les cellules endothéliales sinusoïdales qui, contrairement à l’endothélium des capillaires classiques, ont une membrane basale discontinue et des parois minces afin d’augmenter la perméabilité du tissu et les échanges entre le plasma sanguin et les hépatocytes. Une fine matrice de collagène appelée l’espace de Disse sépare les sinusoïdes des hépatocytes, dans lequel circulent les cellules stellaires et les cellules de Kupffer. Le sang riche en oxygène issu de l’artère hépatique est mélangé avec les nutriments dans la zone péri-portale, puis circule dans les sinusoïdes au contact des différentes cellules hépatiques avant d’être drainé par la veine centrale. Lors du passage du sang, les nutriments et les toxines sont extraits et métabolisés par les hépatocytes tandis que les cellules sanguines endommagées et les corps étrangers sont éliminés par les macrophages résidents. Les hépatocytes collectent les nutriments par les transporteurs présents sur la membrane basolatérale au contact des sinusoïdes (\autoref{fig:fig2}\textcolor{blue}{.B}). Ils produisent également des acides biliaires qui sont sécrétés par les transporteurs de la membrane apicale, au contact des canalicules biliaires. Des jonctions serrées imperméables permettent une séparation stricte entre la bile et le sang, qui peut être rompue en cas de dommages au foie.

	\begin{figureth}
	\centering
			\includegraphics[width=\linewidth]{Figure_2.png}
		\caption[Structure du tissu hépatique à l’échelle microscopique]{\textbf{Structure du tissu hépatique à l’échelle microscopique.} (A) Vues transversale et verticale de l’organisation d’un lobule hépatique (Adapté de \citealt{RN1230}). (B) Zoom sur une section du lobule représentant la structure des hépatocytes et leur implication dans les fonctions hépatiques (Adapté de \citealt{RN826}).}
				\label{fig:fig2}
	\end{figureth}
	\FloatBarrier

	\subsection{Fonctions hépatiques}

	\subsubsection{Action centrale dans le métabolisme énergétique}

L’organisme dépend de l’alimentation pour son approvisionnement en source d’énergie. Après le repas, les sucres et les graisses alimentaires sont absorbées par les entérocytes de l’intestin grêle et transférées vers le foie sous la forme de chylomicrons. Les nutriments qui arrivent de l'intestin dans le foie sont des molécules complexes qui nécessitent d’être transformées en formes utilisables par les cellules de l'organisme \citep[pour revue,][]{RN251}. Les sucres sont transformés en glucose qui est libéré dans la circulation sanguine vers les tissus extra-hépatiques. Les graisses sont transformées en acide gras et transportées dans le sang par les lipoprotéines. Les protéines sont converties en acides aminés et transférées aux muscles squelettiques. Le foie est responsable de 85-90\% de l’apport en protéines dans l’organisme. Comme la prise de nourriture est discontinue et entrecoupée de périodes plus ou moins longues de jeûne, l’organisme a besoin de stocker l’apport en nutriments excédentaire. Cette réserve d’énergie est conservée sous la forme de glycogène pour le glucose, de protéines pour les acides aminés et de triglycérides pour les acides gras. Le foie adapte son action selon le statut nutritionnel (stockage ou production d’énergie) qui dépend de l’équilibre entre la concentration sanguine de l’insuline et du glucagon \citep[pour revues,][]{RN827, RN1249}. Lors d’une forte demande en énergie par l’organisme, le foie va libérer de nouveau du glucose par la gluconéogénèse. En cas de carence en glycogène, le foie est capable de dégrader les triglycérides et les protéines et de métaboliser les acides gras qui peuvent fournir de l’énergie par la ß-oxydation ou les acides aminés qui peuvent servir de substrats pour les cycles de néoglucogénèse. Le foie est également responsable de la synthèse de nombreuses protéines du sérum, comme l’albumine, des facteurs de coagulation et des hormones stéroïdiennes.
	
	\subsubsection{Sécrétion de la bile}

Le foie assiste la digestion intestinale par la sécrétion de bile, un fluide jaune-verdâtre alcalin qui contient des sels biliaires, de la bilirubine (responsable de sa pigmentation) et des électrolytes. Les sels biliaires ont des propriétés émulsifiantes, aidant à l’absorption des acides gras par la muqueuse intestinale. Chaque jour, le foie sécrète environ 800 à 1000 mL de bile, qui sera directement déversée dans le duodénum ou stockée et concentrée dans la vésicule biliaire. La bile est également le moyen d'excrétion de certains principes actifs endogènes et exogènes.

	\subsubsection{Protection contre les substances xénobiotiques}

Le foie est souvent considéré comme un organe épurateur. Il constitue le premier site de passage du sang qui provient du tractus digestif et qui est donc porteur de nombreuses substances exogènes issues de l’alimentation. Celles-ci peuvent comporter différents pathogènes, qui seront reconnus et éliminés par les cellules du système immunitaire circulant dans le foie, ou des toxines qui seront inactivées par les cellules hépatiques. Le foie est également responsable de l’élimination de substances endogènes, tels que les déchets azotés issus du catabolisme. Les hépatocytes sont responsables de la dénaturation des molécules chimiques endogènes et exogènes, pour réduire leur toxicité et faciliter leur excrétion par les voies rénales ou intestinales. Alors que les substances hydrosolubles peuvent être directement éliminées par les reins, les substances lipophiles doivent être au préalable transformées par le foie. 

	\subsubsection{Système immunitaire}

Le foie est riche en glandes lymphatiques, ce qui assure un apport constant en cellules du système immunitaire. Après les macrophages de l’intestin, les cellules de Kupffer constituent la deuxième ligne de défense contre les bactéries alimentaires. Elles éliminent directement les bactéries par phagocytose ou production d’oxyde nitrique et sécrètent des cytokines inflammatoires qui recrutent les autres acteurs du système immunitaire. Le foie permet également de mettre en place un seuil de tolérance contre les antigènes issus de l’alimentation.

	\subsubsection{Stockage de minéraux et de vitamines}
	
Le foie est responsable du stockage de certaines vitamines et de minéraux telles que le fer sous la forme de ferritine, qui sera secrété lors des besoins de production d’hématies, le cuivre, la vitamine B12, la vitamine D, la vitamine A, la vitamine E et la vitamine K.

\clearpage
%%%%%%%%%%%%%%%%%%%%%%%%%%%%%%%%%%%%%%%%%%%%%%%%%%%%%%%%%%%%%%%%%%%%%%%%%%%%%%%%%%%%%%%%%%%%%%%%%%%%%%%%%%%%%%%%%%%%%%%%%%%%%%%%%%%%%%%%%%%%%%%%%%%%%%%%%%%%%%%%%%%%

% 2 - La stéatose hépatique

\section{La stéatose hépatique}
		\label{section:steatose}
		
La stéatose hépatique ou le syndrome du foie « gras » se caractérise par une accumulation de lipides dans les hépatocytes. Auparavant, la plupart des cas de stéatose étaient attribués à une consommation excessive d'alcool, dans le cadre de la maladie du foie gras alcoolique (AALD) (voir \autoref{section:etiologie}). En 1980, Ludwig et ses collègues ont décrit des patients d’âge moyen, présentant des résultats anormaux aux tests biochimiques du foie et des signatures histologiques identiques à l’hépatite alcoolique, c’est-à-dire, une stéatose modérée à sévère sans consommation apparente d’alcool \citep{RN847}. Cette cause aujourd’hui fréquente de stéatose chez les adultes a été baptisée maladie du foie gras non alcoolique (NAFLD), et a une prévalence mondiale estimée à 25\% \citep{RN846}. Les hépatites virale, en particulier l’hépatite C chronique, sont également des conditions médicales à risque dans le développement de la stéatose hépatique. Cette pathologie était traditionnellement considérée comme bénigne et réversible avec seule une petite proportion des patients évoluant vers une cirrhose avec un risque de développer un carcinome hépatocellulaire (CHC) (voir \autoref{section:histopathologie}). Toutefois, en raison de sa prévalence élevée, tant chez les sujets en surpoids, de poids normal ou chez les sujets maigres, elle est responsable de près de 2 millions de décès par an dans le monde et représente aujourd’hui la deuxième indication principale pour la transplantation hépatique aux États-Unis \citep{RN829}. Approximativement 2 milliards d’humains sont en surpoids ou obèses, 400 millions souffrent de diabète et 325 millions sont atteints d’hépatites virales. Avec l’accroissement de la population obèse et diabétique (qui s’est multipliée par 6 sur les quatres dernières décennies), l’augmentation préoccupante du taux d’enfants en surpoids et la stabilité du taux d’hépatites virales, les scientifiques prédisent que le fardeau mondial des pathologies hépatiques aiguës et chroniques risque d’augmenter de façon critique à l’avenir et les transplantations hépatiques deviendront une ressource inestimable.

\subsection{Étiologies}
		\label{section:etiologie}

Dans les pays occidentaux, la stéatose hépatique est majoritairement attribuée soit à une consommation excessive d'alcool, \textit{i.e.} l’AALD, soit au surpoids ou à l’obésité, \textit{i.e.} la NAFLD. L’excès de nutrition peut mener à une désensibilisation du foie à la sécrétion d’insuline. La résistance à l’insuline favorise la libération d’acides gras libres du tissu adipeux dans le sang et leur ré-absorption par le foie \citep{RN837}. L’excès d’alcool, quant à lui, inhibe la mobilisation des graisses par le simple changement dans l’équilibre d’oxydo-réduction par la détoxification récurrente de l’éthanol en acétate \citep{RN831}. Il n’existe pas de consensus sur ce qui représente une consommation « excessive » d’alcool d’après les différentes études scientifiques sur le sujet, mais le seuil de 210g/semaine pour les hommes et 140g/semaine pour les femmes sur une période minimale de 2 ans précédant le diagnostic histologique est généralement utilisé pour différencier l’AALD de la NAFLD \citep{RN866}. La progression vers la cirrhose est plus importante chez les patients atteints d’AALD que de NAFLD, en raison de l’accumulation d’éthanal, un produit toxique issu de l’oxydation partielle de l’éthanol, qui peut occasionner des lésions hépatiques. Globalement, 50\% de la mortalité liée aux cirrhoses est attribuable à l’AALD. La progression vers un stade avancé de fibrose ou de cirrhose est relativement lente pour les patients atteints de NAFLD et la plupart d’entre deux meurent de causes indépendantes comme des accidents cardiovasculaires. \\ \\
\indent
Dans les pays en développement, l’hépatite C chronique restent la cause majeure de développement de la stéatose hépatique. La primo-infection par le virus de l’hépatite C (VHC) est majoritairement asymptomatique mais le virus persiste dans 80\% des cas d’infection \citep{RN834}. La détection du VHC dans le sang au-delà de 6 mois signe l’évolution de l’hépatite C vers une forme chronique. Au fil des années, le virus provoque des atteintes hépatiques subtils mais cumulatifs et une stéatose hépatique apparaît dans 40 à 80\% des cas d’infection selon le génotype viral infectant (voir \autoref{section:classification}). Les mécanismes virologiques liés au développement de la stéatose ne sont pas encore établis, mais le VHC est connu pour interférer avec les processus d’homéostasie du métabolisme des lipides. Le stockage excessif des triglycérides dans les GL pourraient résulter de différents processus : (i) une augmentation de la synthèse des triglycérides, combinée à la biogénèse des GL, (ii) une diminution du catabolisme des lipides et de la lipolyse des GL (voir \autoref{section:formationgl}), et (iii) une altération de la sécrétion des lipoprotéines \citep[pour revue,][]{RN627}. L’hépatite C chronique est également une prédisposition majeure au développement de complications hépatiques, comme la cirrhose et le CHC, résultant en moyenne en 500.000 décès par an, tuant plus de personnes que le paludisme ou la tuberculose \citep{RN833}. En effet, les protéines virales interfèrent avec les voies de signalisation impliquées dans la survie cellulaire, la prolifération et la transformation \citep{RN838}. Après 25 à 30 ans d’infection, 15 à 35\% des patients développent une cirrhose et le risque d’atteindre un état de décompensation ou un cancer est plus important dans le cas de l’hépatite C que d'autres étiologies. Un quart des cancers du foie, la 4ème cause de décès par cancers dans le monde, est attribuable à une infection par le VHC, faisant de ce virus l’un des sept virus oncogènes chez l’Homme \citep{RN832}. L’évolution globale de la pathologie de l’hépatite C est résumée sur la \Autoref{fig:fig3}.

	\begin{figureth}
	\centering
			\includegraphics[width=\linewidth]{Figure_3.png}
		\caption[Évolution et stades pathologiques de l’hépatite C chronique]{\textbf{Évolution et stades pathologiques de l’hépatite C chronique.} Les mécanismes biologiques principaux impliqués dans la progression des différentes stades pathologiques sont encadrés en rouges. ROS : dérivés réactifs de l’oxygène (Adapté de \citealt{RN855}).}
				\label{fig:fig3}
	\end{figureth}
	\FloatBarrier

Par ailleurs, la stéatose hépatique peut également survenir dans une variété de conditions médicales ou être déclenchée par des médicaments ou des régimes alimentaires riches en omégas 6 et en fructose, qui stimulent la lipogénèse \textit{de novo}. Les populations hispaniques ou asiatiques seraient plus sensibles au syndrome métabolique du foie « gras » que leurs homologues occidentaux. Enfin, des polymorphismes génétiques dans les gènes \textit{PNPLA3} \citep{RN851} et \textit{TM6SF2} \citep{RN850}, impliqués dans un défaut de lipidation des lipoprotéines, sont également des facteurs de risque du développement de la stéatose.


\subsection{Histopathologie, diagnostic et traitements}
		\label{section:histopathologie}

Traditionnellement, une teneur en graisse hépatique dépassant 5\% du poids du foie est considérée comme la définition d’une stéatose hépatique. Cette définition est difficilement applicable dans le contexte clinique, de ce fait, le diagnostic de la stéatose est établi si > 5\% des hépatocytes contiennent des vacuoles graisseuses, plus communément appelées gouttelettes lipidiques (GL) \citep{RN1242}. La stéatose hépatique est classée en deux étiologies : la stéatose macro-vésiculaire et la stéatose micro-vésiculaire. La plupart des stéatoses sont de type macro-vésiculaire, dans laquelle une unique vacuole volumineuse occupe le cytoplasme de la cellule, déformant l’architecture cellulaire (\autoref{fig:fig4}\textcolor{blue}{.A}). La stéatose micro-vésiculaire est moins fréquente mais généralement plus grave et correspond à un fractionnement de la graisse en nombreuses petites GL (\autoref{fig:fig4}\textcolor{blue}{.B}). La sévérité de la stéatose est déterminée histologiquement par le pourcentage d'hépatocytes comportant une ou plusieurs vésicules graisseuses dans le cytoplasme \citep[pour revue,][]{RN1245}. On parle de stéatose légère en dessous de 20\%, stéatose modérée entre 20 et 50\% et stéatose sévère au-delà de 50\%. La stéatose comprend un spectre de lésions hépatiques allant de la simple stéatose, c’est-à-dire, uniquement des vacuoles graisseuses qui ne perturbent pas la fonction hépatique, à la stéatohépatite (SH), un état dans lequel la stéatose est associée à une inflammation lobulaire avec une infiltration mixte de neutrophiles, de lymphocytes et de macrophages (\autoref{fig:fig4}\textcolor{blue}{.C}). Le phénomène qui mène d’une simple accumulation de lipides considérée comme « inoffensive » à l’inflammation n’est pas entièrement comprise. Une théorie proposée en 1975 par l'hépatologiste Heribert Thaler établit que « la cause de la stéatose et non l’accumulation de graisses serait responsable d’une évolution vers la fibrose et la cirrhose », expliquant les différents degrés d’évolution selon l’origine de la pathologie \citep{RN1244}. Une autre hypothèse souligne que la capacité du foie à stocker les acides gras serait dépassée ce qui conduit à l’accumulation d’espèces lipidiques toxiques \citep{RN1246}. Ces métabolites induisent un stress hépatocellulaire et des lésions qui entraînent la mort cellulaire, responsable de l’inflammation. Dans ce contexte, la stéatose est souvent considérée comme une étape favorisant le développement de complications hépatiques graves. En effet, la destruction substantielle du tissu hépatique par nécrose résultant de l’inflammation peut déclencher une synthèse de matrice extracellulaire issue du processus de cicatrisation : la fibrose. Quand la fibrose s’étend et éventuellement remplace toute l’architecture du foie jusqu’à entraîner une congestion des nodules hépatiques, elle peut mener à une dégénérescence hépatique grave, la cirrhose avec un risque de développement de CHC de 2 à 5\% par an. Lorsque les défaillances hépatiques s’aggravent, la progression d’une cirrhose vers un état de décompensation implique des hémorragies dues à la formation de varices hépatiques et à l’hypertension du système veineux du foie.

	\begin{figureth}
	\centering
			\includegraphics[width=\linewidth]{Figure_4.png}
		\caption[Coupes histologiques du tissu hépatique présentant les différents types de stéatose]{\textbf{Coupes histologiques du tissu hépatique présentant les différents types de stéatose.} (A) La stéatose macro-vésiculaire est caractérisée par des hépatocytes présentant des GL volumineuses, arrondies et bien définies (souvent uniques) dans le cytoplasme et les noyaux sont déplacés à la périphérie. (B) La stéatose micro-vésiculaire est caractérisée par des hépatocytes comportant de nombreuses gouttelettes lipidiques et leurs noyaux restent centrés. (C) Stéatohépatite avec au centre des infiltrats inflammatoires (D'après \citealt{RN1239} et \citealt{RN1245}).}
				\label{fig:fig4}
	\end{figureth}
	\FloatBarrier

Les individus atteints de stéatose hépatique ne présentent souvent aucun symptôme perceptible et la pathologie n’est souvent détectée qu’au cours d’analyses sanguines de routine, par l’augmentation anormale des enzymes du foie de type transaminases (ASAT, ALAT) ou des biomarqueurs comme l’albumine et la bilirubine. L’analyse histologique des tissus par biopsie est le seul test largement accepté pour diagnostiquer et distinguer définitivement la stéatose des autres formes de pathologies hépatiques et pour évaluer la gravité de l’inflammation et de la fibrose qui en résulte. L’aspect invasif de cette méthode peut néanmoins entraîner des complications, en plus de ne pas nécessairement être représentative de la situation. En effet, l’échantillon de foie obtenu par biopsie hépatique ne représente que 1/50.000 de la masse totale du foie selon l’individu \citep{RN841}. Les limites inhérentes à la biopsie ont encouragé le développement de techniques non invasives. Initialement, l'échographie était utilisée car elle présentait une excellente sensibilité et spécificité pour les stéatoses modérées et sévères. La mesure de la teneur en triglycérides hépatiques et l'utilisation de la résonance magnétique et de l'élastographie ont également évolué au cours de la dernière décennie et sont devenues des méthodes prometteuses pour diagnostiquer et quantifier la stéatose et les stades fibrotiques dans le domaine de la recherche clinique. \\ \\
\indent
À l’heure actuelle, il n’existe pas de traitements approuvés pour soigner la stéatose, bien que certains soient en cours de développement pré-clinique ou en phase d’essais cliniques. Toutefois, la stéatose est une pathologie réversible, qui peut disparaître en éliminant la cause : une perte de poids et une alimentation plus adaptée, par exemple en instaurant un régime méditerranéen ou végétarien pauvre en calories dans le cas de la NAFLD \citep{RN836,RN844} ; un arrêt de la consommation d’alcool dans le cas de l’AALD ; une guérison dans le cas de l’hépatite C chronique (voir \autoref{section:elimination}). Cependant, si les stades plus avancés de la pathologie sont atteints, comme la cirrhose, l’élimination de la cause n’est généralement pas suffisante et la progression vers un état de décompensation ou un cancer reste un risque.  En effet, il est désormais admis que le risque de développer un CHC peut persister après guérison par thérapie antivirale chez les patients qui étaient atteints de formes graves d’hépatite C \citep{RN1076,RN1077}. À ce stade, les seules options thérapeutiques sont la transplantation hépatique ou la résection de la tumeur. L’espérance de vie sans transplantation hépatique est d’au maximum 3 ans dans le cas d’une cirrhose décompensée sous surveillance étroite et traitement médical lourd et de 18\% après 5 ans pour le CHC. La transplantation hépatique est la seconde transplantation d’organe la plus commune après celle du rein. Cependant, moins de 10\% de la demande en transplantations est honorée au vu du nombre actuel de donneurs \citep{RN839}. Malgré les mesures prises pour encourager les dons et développer les procédures de greffes d’urgence suite à des arrêts cardiaques, le nombre de transplants hépatiques reste très insuffisant dans les pays occidentaux et dans les pays en développement.

\clearpage
%%%%%%%%%%%%%%%%%%%%%%%%%%%%%%%%%%%%%%%%%%%%%%%%%%%%%%%%%%%%%%%%%%%%%%%%%%%%%%%%%%%%%%%%%%%%%%%%%%%%%%%%%%%%%%%%%%%%%%%%%%%%%%%%%%%%%%%%%%%%%%%%%%%%%%%%%%%%%%%%%%%%

% 3 - Les gouttelettes lipidiques

\section{Les gouttelettes lipidiques}	
		\label{section:gl}

Les être vivants déploient un flux constant d'énergie essentiel au maintien de leurs fonctions biologiques vitales. Les nouvelles sources d’énergie n’étant pas toujours disponibles, la capacité à stocker les carbohydrates, les lipides et les protéines dans les tissus est un mécanisme évolutif crucial à la survie. Les tissus sont capables de conditionner et de stocker les lipides excédentaires sous une forme inerte dans des organites intracellulaires spécialisés. Leur première description date du 19ème siècle lorsque des chercheurs identifient la présence de « gouttes d’huile » dans le cytoplasme d’embryons d’oursin en développement. Peu après, ces organites ont été reconnus comme un composant inerte de la plupart des cellules eucaryotes et appelées « liposomes ». Depuis l’invention des liposomes artificiels à la fin des années 1960, elles ont été renommées selon différentes appellations, en passant par « corps lipidiques », « adiposomes », « vacuoles lipidiques », « vésicules lipidiques » jusqu’à l’officialisation récente de GL comme nomenclature officielle \citep{RN871}. Longtemps perçues comme de simples inclusions cytoplasmiques stockant les lipides neutres, les GL sont apparues ces dernières années comme des organites dynamiques assurant des fonctions clés dans l’homéostasie lipidique et énergétique. Elles sont particulièrement importantes dans les tissus spécialisés dans le stockage de l'énergie ou le renouvellement des lipides, comme le tissu adipeux, le foie et l'intestin, mais elles sont également retrouvées dans les muscles squelettiques, le cortex surrénalien et les glandes mammaires. Bien que les GL soient des organites universels retrouvés dans la plupart des cellules eucaryotes, y compris chez les organismes végétaux, leur contenu, leur morphologie, leur composition, leur organisation spatiale et leurs fonctions physiologiques varient considérablement entre les types cellulaires et les tissus. Au cours de ce chapitre, nous détaillerons l’ensemble de ces points en nous centrant principalement sur les caractéristiques et les fonctions des GL propres aux hépatocytes.

	\subsection{Caractéristiques principales des gouttelettes lipidiques}
	
	\subsubsection{Structure, composition et distribution des gouttelettes lipidiques}
	
Les GL possèdent une architecture unique sous la forme d’un noyau hydrophobe composé de lipides neutres et entouré d’une monocouche phospholipidique décorée par un ensemble de protéines résidentes ou simplement localisées de façon transitoires (\autoref{fig:fig5}). Toutes les gouttelettes lipidiques présentent une organisation structurale similaire qui les distingue des autres organites. Le noyau central est principalement constitué de triacylglycérols (TAG) et d’esters de stérol et la monocouche superficielle est riche en phosphatidylcholine (PC), phosphatidylethanolamine (PE), phosphatidylinositol (PI), phosphatidylsérine (PS) et en lysophospholipides, ce qui diffère des bicouches phospholipidiques typiques des autres membranes intracellulaires \citep{RN872,RN873}.

	\begin{figureth}
	\centering
			\includegraphics[width=0.5\linewidth]{Figure_5.png}
		\caption[Structure et composition de la gouttelette lipidique]{\textbf{Structure et composition de la gouttelette lipidique.} Le noyau central des GL est composé principalement de lipides neutres, comme le triacylglycérol et les esters de stérol. La membrane des GL est une monocouche phospholipidique associée à des protéines adhérentes ou intégrées (Adapté de \citealt{RN645}).}
				\label{fig:fig5}
	\end{figureth}
	\FloatBarrier
	
Les premières protéines constitutives des GL découvertes appartiennent à la famille des périlipines (PLIN) \citep{RN875}. Cette famille de protéines de mammifères est constituée de 5 membres, PLIN1 à PLIN5, qui présentent un profil d’expression différent selon le type cellulaire et qui jouent de multiples rôles dans la biologie des GL \citep{RN874}. Depuis, le développement récent des technologies à haut débit a permis de révéler au cours d’études protéomiques une grande variété de protéines co-purifiées avec les GL, qui varient selon les types cellulaires, tissus et organismes \citep[pour revue,][]{RN878}. Au sein d’une cellule de mammifère prototypique, le protéome des GL est généralement composé de 100 à 150 protéines. Ces protéines sont adhérentes à la périphérie des GL ou intégrées au sein de la monocouche de phospholipides par le biais de domaines hydrophobes ou d’hélices amphipatiques. Le mécanisme de recrutement des protéines sera détaillé dans la \autoref{section:recrutement}. La composition lipidique et protéique des gouttelettes lipidiques est influencée par le rôle physiologique de la cellule d’origine, ce qui résulte en une panoplie de fonctions possibles assurées par les GL, décrites dans la \autoref{section:fonctions}. À titre d’exemple, les cellules stellaires contiennent des GL riches en esters de rétinol peu courants, en raison de leur rôle dans le stockage de la vitamine A. De plus, au sein d’une même cellule, la composition lipidique et protéique des GL évolue en fonction de leur état métabolique et de leur devenir (\autoref{fig:fig6}\textcolor{blue}{.A}), par exemple s’il s’agit d’une GL en cours d’expansion ou de dégradation (voir \autoref{section:formationgl}). En effet, certaines protéines, dont les périlipines, s’associent à différentes sous-populations de GL, soit pour promouvoir leur biogénèse et expansion pour PLIN3, soit pour les stabiliser, dans le cas de PLIN1 et PLIN2. \\ \\
\indent
En plus des variations dans la composition, le contenu et la morphologie des GL est très hétérogène entre les types cellulaires (\autoref{fig:fig6}\textcolor{blue}{.B}) \citep[pour revue,][]{RN876}. Le nombre de GL dans les cellules peut varier de 4 à 8 dans les levures à des centaines dans les cellules de mammifères en culture. Toutes les cellules eucaryotes sont capables de former des petites GL de l’ordre de 300 à 800nm lors de la biosynthèse, mais certaines cellules spécialisées dans le stockage des lipides peuvent présenter des GL supérieures à 1µm, comme les adipocytes ou les hépatocytes. Les adipocytes blancs présentent parfois même une unique GL dite « uniloculaire » allant jusqu’à 100µm de diamètre, remplissant le cytoplasme et repoussant mécaniquement le noyau à la périphérie. Une GL large présente un rapport surface/volume plus faible qu’un ensemble de petites GL ayant la même teneur totale en lipides neutres. Les grandes GL seraient donc plus adaptées au stockage à long terme des lipides neutres, en minimisant le besoin en phospholipides. Les petites GL en revanche, sont plus adaptées aux situations dynamiques nécessitant une libération et un stockage rapide de lipides lors des interactions avec d'autres organites \citep{RN877}. De plus, les GL peuvent présenter une distribution intracellulaire dispersée, groupée ou polarisée selon l’état métabolique de la cellule (\autoref{fig:fig6}\textcolor{blue}{.C}). Par exemple, les GL ont tendance à se regrouper et à s'étendre pendant la lipogenèse et à se redisperser en cas de besoins énergétiques. En effet, les changements de statut énergétique influencent hautement le devenir, la fonction et le positionnement des GL, démontrant la versatilité de cette organite et son rôle crucial dans l’homéostasie et la survie cellulaire.

	\begin{figureth}
	\centering
			\includegraphics[width=0.65\linewidth]{Figure_6.png}
		\caption[Diversité dans le contenu, la morphologie et la distribution intracellulaire des gouttelettes lipidiques]{\textbf{Diversité dans le contenu, la morphologie et la distribution intracellulaire des gouttelettes lipidiques.} (A) La composition du noyau lipidique et de l’enveloppe protéique des GL individuelles peut différer selon leur état métabolique et leur fonction, par exemple si les GL grossissent (+) ou rétrécissent (-). (B) Les GL peuvent avoir une taille uniforme, plus ou moins hétérogène ou géante, comme les GL uniloculaires retrouvées dans les adipocytes blancs de mammifères. (C) Les GL peuvent être dispersées, regroupées ou être localisées de manière polarisée (Adapté de \citealt{RN876}).}
				\label{fig:fig6}
	\end{figureth}
	\FloatBarrier
	
\subsubsection{Propriétés biophysiques des gouttelettes lipidiques}

Compte tenu de leur nature, les GL se comportent comme la phase dispersée d’une émulsion « huile dans l’eau », le cytosol étant le milieu aqueux des cellules, et suivent donc les principes biophysiques des émulsions \citep[pour revue,][]{RN608}. La plupart des émulsions sont thermodynamiquement instables, en raison de l’absence d’interactions cohésives entre les molécules de la phase dispersée et de la phase aqueuse. Une tension de surface est alors générée à l’interface entre les deux phases. La clé de la stabilité des émulsions est la présence de tensioactifs, qui minimisent la surface de contact entre les fluides non miscibles, ce qui diminue le coût énergétique. Pour les GL, la monocouche de phospholipides constitue le principal surfactant \citep{RN882}. Elle permet de fournir des interactions favorables avec la phase aqueuse grâce aux têtes polaires orientées vers le cytosol et avec la phase dispersée grâce aux chaînes acyles directement en contact avec le noyau lipidique. Les protéines présentes à la surface de la gouttelette lipidique peuvent également augmenter l'élasticité de l'interface, contribuant ainsi à la stabilité thermodynamique de l’organelle. Cette tension de surface est d’ailleurs responsable de la forme arrondie des gouttelettes lipidiques qui permet de minimiser la surface de lipides neutres potentiellement en contact avec le cytosol aqueux.


	\subsection{Cycle de formation et de dégradation des gouttelettes lipidiques}
	\label{section:formationgl}

	Les GL sont des organites dynamiques qui se forment, s’élargissent, se rétractent ou se dégradent en réponse aux changements d’état métabolique des cellules, par exemple s’il s’agit d’une période d’alimentation ou de jeûne avec une forte demande en besoins énergétiques. Notre compréhension actuelle du cycle de formation et de dégradation des GL est principalement dérivée de travaux sur les adipocytes et les mécanismes ou les acteurs protéiques impliqués dans toutes ces étapes peuvent différer dans d'autres types de cellules.

	\subsubsection{Biosynthèse des gouttelettes lipidiques}

Dans tous les organismes eucaryotes, la formation des GL est initiée par une accumulation de lipides neutres entre le feuillet interne et le feuillet externe du RE \citep[pour revue,][]{RN645}. La synthèse de ces lipides est catalysée par des enzymes impliquées dans la transformation des acides gras en excès ou dans la lipogénèse \textit{de novo} à partir de glycérol. L’étape finale de la synthèse des triacylglycérides (TAG) résulte de l’activité des acyl-CoA diacylglycérol 1 (DGAT1) et 2 (DGAT2)  et les esters de stérol sont le produit des acétyl-CoA acétyltransférase 1 (ACAT1) et 2 (ACAT2) et des stérol O-acyltransférase 1 (SOAT1) et 2 (SOAT2). À faible concentration, les lipides neutres sont solubilisés entre les feuillets de la bicouche lipidique et s’équilibrent à travers le réseau réticulé. Lorsqu’un seuil de concentration locale en lipides est atteint, une « lentille » lipidique se forme au sein de la bicouche (\autoref{fig:fig7}). Ces précurseurs sont généralement très petits (50 nm) et ont une durée de vie de l’ordre de quelques minutes, rendant difficile l’observation des sites de nucléation des GL en cellules vivantes. Dans des conditions de lipogénèse prolongée, la lentille lipidique s’agrandit et déforme la bicouche, jusqu’à ce qu’elle atteigne une taille et une tension de surface critique qui déclenche le bourgeonnement d’une GL « naissante » d’environ 100 à 300 nm. Pour maintenir l’homéostasie de la membrane lors de l’expansion de la lentille, la synthèse de lipides neutres est couplée à une synthèse \textit{de novo} de phospholipides grâce à l’activation de l’enzyme limitante phosphocholine cytidylyltransférase α (CCTα). Plusieurs études suggèrent que les principes physico-chimiques suffisent pour expliquer la formation et le bourgeonnement de la lentille, ces processus étant énergétiquement favorables pour réduire l’interaction des lipides neutres avec les autres composants membranaires. Or, des travaux récents ont identifié plusieurs complexes protéiques qui régulent et facilitent ce processus, au-delà de leur rôle d’enzyme catalysant la synthèse des lipides neutres ou des phospholipides, en induisant par exemple une courbure appropriée du feuillet phospholipidique externe et en contrôlant la direction du bourgeonnement \citep{RN549}. C’est le cas des protéines transmembranaires induisant le stockage des graisses 1 (FIT1) et 2 (FIT2) et des oligomères de seipine, qui sont capables de cibler les sites enrichis en lipides neutres et de réguler le bourgeonnement pour que les GL émergentes soient plus uniformes par rapport aux cellules dépourvues de ces protéines. Au cours de cette étape, les GL naissantes acquièrent un certain nombre de protéines, qui diffusent directement à partir du RE ou qui proviennent du cytosol et se délocalisent préférentiellement à leur surface. Seule la monocouche externe peut accueillir des protéines cytosoliques, comme la périlipine Pln1 chez la levure (orthologue de PLIN3 chez les mammifères), qui s’associe à la surface des GL naissantes lors des étapes très précoces de bourgeonnement. Cette asymétrie dans le recrutement des protéines entre les deux feuillets modifierait l’équilibre de la tension superficielle, ce qui déclencherait le bourgeonnement de la GL exclusivement vers le cytosol (\autoref{fig:fig7}\textcolor{blue}{, encadré}). Ches les eucaryotes supérieurs, une population de GL se détache et diffuse librement dans le cytosol selon un mécanisme de fission encore indéterminé.

	\begin{figureth}
	\centering
			\includegraphics[width=\linewidth]{Figure_7.png}
		\caption[Biosynthèse \textit{de novo} des gouttelettes lipidiques cytosoliques]{\textbf{Biosynthèse \textit{de novo} des gouttelettes lipidiques cytosoliques.} Les réactions successives de la lipogénèse \textit{de novo} sont illustrées dans l’encadré sur fond gris. G3P : glycérol-3-phosphate : GPAT : glycérol-3-phosphate acyltransférase ; LPA : acide lysophosphatidique ; AGPAT :  acylglycérolphosphate acyltransférase ; PA : acide phosphatidique ; PAP : acide phosphatidique phosphatase ; DAG : diacylglycérol ; DGAT : acyl-CoA diacylglycérol ; TAG ; triacylglycérol. (Adapté de \citealt{RN645}).}
				\label{fig:fig7}
	\end{figureth}
	\FloatBarrier
	
\subsubsection{Expansion des gouttelettes lipidiques}

En cas de forte période d’alimentation, une surcharge du processus de biosynthèse \textit{de novo} des GL peut être compensée par un élargissement des GL pré-existantes par synthèse locale en lipides neutres (\autoref{fig:fig8}\textcolor{blue}{.A}). Ce processus est médié par la présence d’enzymes comme les glycérol-3-phosphate acyltransférase (GPAT), les 1-acylglycérol-3-phosphate O-acyltrans- férase (AGPAT) ou DGAT2 qui catalysent les étapes successives de la synthèse des TAG directement à la surface des GL \citep[pour revue,][]{RN607}. De plus, les membres de la famille des acyl-CoA synthétases à longue chaîne (ASCL), qui convertissent les acides gras libres à longue chaîne en esters d'acyl-CoA, seraient également mobilisés pour la synthèse locale de lipides neutres. En l’absence de ces enzymes, les GL peuvent s’approvisionner en lipides neutres en interagissant de nouveau avec les membranes du RE. Le mécanisme de ré-attachement au RE sera explicité en détail dans la \autoref{section:recrutement}. De nouveau au contact avec le RE, les GL seront supplémentées en lipides neutres et en enzymes qui pourront se délocaliser préférentiellement à leur surface. Il est également possible que certaines enzymes impliquées dans la synthèse des lipides neutres soient recrutées directement à la surface des GL à partir du cytosol. \\ \\
\indent
Un mécanisme alternatif d’élargissement des GL repose sur la fusion homotypique de deux GL adjacentes (\autoref{fig:fig8}\textcolor{blue}{.B}). La fusion est initialement déclenchée par un contact entre les membranes adjacentes de deux GL qui, si l’épaisseur du film aqueux les séparant s’approche de quelques nanomètres, peuvent former un pore reliant directement les deux phases lipidiques neutres et favoriser leur mélange lent au fur et à mesure de l'expansion du pore. La fusion est un processus relativement long qui peut prendre jusqu'à 15 minutes et qui est fortement régulé par les protéines de l'hôte et par l'équilibre entre les lipides neutres et les phospholipides composant la monocouche de surface \citep{RN555}. Parmi les régulateurs connus du mécanisme de fusion des GL, on retrouve la famille des effecteurs induisant la mort cellulaire (CIDE), constituée de trois membres : CIDEA, CIDEB et CIDEC. Dans le tissu adipeux, CIDEA et CIDEC régulent conjointement le mécanisme de fusion des GL, tandis que CIDEB est constitutif du tissu hépatique \citep{RN556}. Les protéines CIDE partagent le même mécanisme d’action, à savoir la formation d’oligomères stables qui diffusent sur la surface des GL et qui sont capables d’interagir avec les protéines CIDE homologues localisées sur une autre GL. Cette interaction permet de relier physiquement deux GL adjacentes, ce qui provoque leur fusion \citep{RN557,RN558}. D’autres protéines favorisent le processus de fusion comme l’hydrolyse associé aux GL (LDAH) selon un mécanisme d’action encore inconnu \citep{RN560}. La coalescence spontanée des GL par simple principe physico-chimique peut également se produire mais ce phénomène est peu fréquent dans des conditions normales de croissance. En cas de stress membranaire où les besoins en phospholipides sont importants ou dans des cellules présentant un défaut de synthèse des phospholipides, les GL adjacentes fusionnent, ce qui diminue la quantité de phospholipides nécessaires pour couvrir densément la surface du noyau hydrophobe \citep{RN561}. Ainsi, la fusion des GL peut être un moyen de fournir des phospholipides membranaires pour d’autres processus cellulaires et de limiter le besoin en phospholipides pour la synthèse des GL.

	\begin{figureth}
	\centering
			\includegraphics[width=\linewidth]{Figure_8.png}
		\caption[Mécanismes d’expansion des gouttelettes lipidiques cytosoliques par synthèse locale ou par fusion]{\textbf{Mécanismes d’expansion des gouttelettes lipidiques cytosoliques par synthèse locale ou par fusion.} (A) La synthèse locale de lipides neutres peut être médiée par des enzymes de biosynthèse des TAG ou d’esters de stérol délocalisés à la surface des GL lors l’étape de la biosynthèse ou recrutées dans le cytosol. L’approvisionnement en lipides neutres et en enzymes peut également s’effectuer lors des contacts ultérieurs avec le RE, par la formation d’un pont entre les membranes facilité par le complexe ARF1/COPI. (B) La coalescence ou la fusion peut avoir lieu spontanément en cas de défaut dans la couverture phospholipidique ou peut être activement médiée par l’intermédiaire des complexes protéiques CIDE (Adapté de \citealt{RN627}).}
				\label{fig:fig8}
	\end{figureth}
	\FloatBarrier

\subsubsection{Dégradation des gouttelettes lipidiques}

La dégradation des GL a lieu lorsque des précurseurs lipidiques doivent être mobilisés comme carburant métabolique ou pour la synthèse de composants comme les membranes cellulaires. Comme pour la synthèse locale, ce processus peut s’effectuer par l’action de lipases qui agissent directement à la surface des GL et qui vont catalyser la digestion des lipides neutres : on parle de « lipolyse » (\autoref{fig:fig9}\textcolor{blue}{.A}). La première réaction est catalysée par la patatine contenant un domaine phospholipase 2 (PNPLA2), également connue sous le nom de lipase triglycéride adipeuse (ATGL), qui hydrolyse les TAG pour générer du diacylglycérol (DAG) et une molécule d’acide gras libre. Le DAG est ensuite clivé en monoacylglycérol (MAG) libérant une autre molécule d’acide gras libre par la lipase sensible aux hormones (HSL). Enfin, le MAG est clivé en glycérol et en acide gras libre par la monoglycéride lipase (MGL) \citep[pour revue,][]{RN562}. Contrairement à PNPLA2 et HSL qui hydrolysent des composés strictement hydrophobes, MGL cible les MAG qui sont davantage solubles et qui peuvent diffuser librement dans le cytosol. MGL ne requiert donc pas une localisation stricte à la surface des GL et la dernière étape de la digestion est généralement cytoplasmique. Les enzymes ATGL et MGL sont exprimées dans le foie, bien qu’à des niveaux plus faibles que dans les adipocytes, tandis que l’expression de l’enzyme HSL qui catalyse la deuxième réaction semble négligeable dans le tissu hépatique. Il n’est donc pas certain que ces enzymes constituent les principales enzymes de la lipolyse hépatique. En revanche, un nombre grandissant de témoignages pointent d’autres lipases qui pourraient avoir un rôle majeur dans la lipolyse hépatique, comme la patatine contenant un domaine phospholipase 3 (PNPLA3) et la triacylglycérol lipase hépatique (LIPC). Les phospholipides pourraient être éliminés en parallèle de la lipolyse par voie enzymatique, les rendant solubles dans l’eau. À l’appui de cette hypothèse, plusieurs membres de la famille des phospholipases A2 sont retrouvés à la surface des GL. Lorsque les lipases catalysent la digestion du noyau hydrophobe, la taille des GL diminue, ce qui génère un encombrement stérique des protéines présentes à leur surface \citep{RN564}. En général, les protéines qui ciblent les GL à partir du cytosol via une hélice amphipatique sont faiblement liées et sont susceptibles d’être détachées aisément et remobilisées dans d’autres processus cellulaires. À l’inverse, les protéines ancrées via un domaine plus hydrophobe sont davantage susceptibles d’être dégradées en parallèle du noyau lipidique. Elles sont reconnues et extraites sélectivement par la protéine de choc thermique 70 (HSP70), transférées aux lysosomes pour être dégradée par autophagie \citep{RN567}. \\ \\
\indent
Plus récemment, une forme de macro-autophagie appelée « lipophagie » a été découverte en 2009 et représente une voie catabolique importante dans la mobilisation rapide des lipides \citep{RN568}. La lipophagie ne peut recycler qu’une sous-population de GL de petites tailles produites après lipolyse, qui se prêtent à l’internalisation lipophagique \citep{RN880}. Au cours de ce processus, les petites GL sont recouvertes par une bicouche membranaire pour former un autophagosome, qui sera acheminé vers les lysosomes, avec lesquels ils fusionnent pour former un autolysosome (\autoref{fig:fig9}\textcolor{blue}{.B}). La dégradation lysosomal des lipides neutres est effectuée par la lipase acide (LIPA) en raison de son activité optimale au pH lysosomal de 4,5 - 5. Elle présente une large spécificité de substrat, hydrolysant le TAG, le DAG et les esters de stérol \citep[pour revue,][]{RN574}. En revanche, on ne sait pas si LIPA hydrolyse le MAG ou si celui-ci est libéré pour une digestion successive par la MGL dans le cytoplasme. En parallèle, les phospholipases et les peptidases lysosomales dégradent les phospholipides et les protéines associées au noyau hydrophobe. Une étude récente a confirmé l’importance du mécanisme de lipophagie dans la dégradation des GL dans le tissu hépatique à partir de cultures primaires d’hépatocytes et de cellules en lignées \citep{RN573}. Compte tenu du rôle crucial de la lipolyse et de la lipophagie dans l’homéostasie des lipides, de nombreuses études valorisent le rôle protecteur de ces voies cataboliques dans les pathologies hépatiques. À ce titre, divers traitements connus pour induire les voies d’autophagie conventionnelles sont actuellement en cours d’étude en tant que pistes thérapeutiques pour le traitement de la stéatose.

	\begin{figureth}
	\centering
			\includegraphics[width=0.80\linewidth]{Figure_9.png}
		\caption[Mécanisme de consommation des gouttelettes lipidiques]{\textbf{Mécanisme de consommation des gouttelettes lipidiques.} (A) Représentation schématique du mécanisme de la lipolyse. (B) Représentation schématique du mécanisme de l’autophagie des gouttelettes lipidiques ou « lipophagie ». Adapté de \citealt{RN608}).}
				\label{fig:fig9}
	\end{figureth}
	\FloatBarrier

	\subsection{Interface entre les gouttelettes lipidiques et les autres organites}
	\label{section:interface}

 Au cours de leur cycle de vie, les GL communiquent fréquemment avec les autres organites intracellulaires tels que le RE, l’appareil de Golgi, les mitochondries, les lysosomes, les peroxysomes, ainsi qu’avec les membranes plasmique et nucléaire \autoref{fig:fig10} \citep[pour revues,][]{RN578,RN579}. Une étude récente par imagerie multi-spectrale a montré qu’environ 85\% des GL sont physiquement connectées avec le RE dans les cellules de mammifères, soulignant le rôle important de cet organite dans la biogénèse des GL et dans l’apport de la plupart de leurs molécules constitutives \citep{RN580}. Pour interagir avec les autres organites, les GL se déplacent à travers le réseau microtubulaire selon un mouvement bidirectionnel qui dépend du recrutement de protéines motrices. \\ \\
 \indent
 Les contacts membranaires homo- ou hétéro-typiques sont reconnus comme une voie majeure de trafic intracellulaire de molécules entre les différents compartiments cellulaires \citep[pour revue,][]{RN583,RN584}. Dans le contexte des GL, ces interactions permettent d’approvisionner les différents compartiments cellulaires en lipides et en protéines, stockés et transportés par les GL. Par exemple, le repositionnement spatial des GL vers les mitochondries pendant l’état de lipolyse, permet d’optimiser la canalisation des acides gras libérés dans la ß-oxydation. La base moléculaire de ces interactions reste aujourd’hui mal comprise, mais plusieurs études mettent en lumière le rôle de complexes d’ancrage protéiques dans l'établissement d’une connexion physique entre les GL et les autres organites \citep{RN586}. Premièrement, la monocouche des GL s’associe à la membrane externe de la bicouche des organistes partenaires par un mécanisme d’hémi-fusion. Ce processus serait facilité par un complexe formé par la GTPase facteur 1 d'ADP-ribosylation (ARF1) et le coatomère COPI (ARF1/COPI) \citep{RN588,RN589} qui provoque le détachement des phospholipides de la monocouche sous la forme de très petites gouttelettes lipidiques appelées « nano-GL » (\autoref{fig:fig8}). Cet appauvrissement en phospholipides diminue la couverture surfactante du noyau hydrophobe ce qui entraîne une augmentation de la tension superficielle et favorise conséquemment le pontage entre la monocouche des GL et la membrane externe de l’organite. De plus, de nombreux membres de la famille Rab, les GTPases qui contrôlent le trafic vésiculaire \citep[pour revue,][]{RN590}, sont retrouvées à la surface des GL et favoriseraient la formation et l’expansion du pont entre les GL et l’organite cible, en interagissant avec la machinerie SNARE de fusion membranaire eucaryote \citep{RN592}. Ce pont permet la diffusion mono- ou bi-directionnelle passive de lipides, de métabolites, de protéines et d’ions entre les lumières des deux organites, sans nécessiter de transfert actif médié par des protéines \citep[pour revue,][]{RN593}. Un autre mécanisme d’ancrage pourrait s’effectuer par la dimérisation homotypique de protéines localisées à la surface à la fois des GL et des autres organites. Par exemple, les protéines AUP1 et CIDE seraient responsables des contacts homotypiques entre GL et de leur clusterisation \citep{RN557,RN596}.

	\begin{figureth}
	\centering
			\includegraphics[width=\linewidth]{Figure_10.png}
		\caption[Interface entre les gouttelettes lipidiques et les organites intracellulaires]{\textbf{Interface entre les gouttelettes lipidiques et les organites intracellulaires.} (Adapté de \citealt{RN869}).}
				\label{fig:fig10}
	\end{figureth}
	\FloatBarrier

Si la majorité des GL sont localisées dans le cytosol, certaines GL dites « luminales » ont été observées dans la lumière du RE et à l’intérieur du compartiment nucléaire \citep{RN598}. Les GL nucléaires ont été principalement observées dans des cellules d'origine hépatique et se formeraient par des invaginations de l’enveloppe nucléaire \citep{RN600} ou par le reflux des GL luminales vers le nucléoplasme \citep{RN639}. Les GL nucléaires ont été proposées comme plateforme de régulation de l’expression des gènes mais leur rôle n’est pas complètement établi (voir \autoref{section:fonctions}).

	\subsection{Mécanisme de recrutement des protéines à la surface des gouttelettes lipidiques}
	\label{section:recrutement}
	
	Les protéines associées physiquement aux GL se répartissent en deux catégories selon leur mécanisme d’ancrage : les protéines membranaires qui diffusent à partir du RE vers les GL appartiennent à la classe I ; les protéines solubles directement recrutées à la surface des GL à partir du cytosol définissent la classe II \citep[pour revue,][]{RN602}. Il est probable que des protéines provenant des mitochondries, de l’appareil de Golgi, des peroxysomes ou des lysosomes soient également recrutées à la surface des GL, par le biais des sites de contact établis entre les GL et ces autres organites cellulaires. Les protéines sont intégrées dans la monocouche phospholipidique des GL par le biais de divers mécanismes et motifs de liaison. Jusqu'à présent, les hélices amphipathiques et les domaines hydrophobes ont été identifiés comme les motifs les plus courants d’association stable à la surface des GL. Les protéines peuvent également s’associer indirectement et de façon réversible à la périphérie des GL en se liant à des protéines intégrées dans la membrane.

	\subsubsection{Les protéines de classe I : ciblage depuis le réticulum endoplasmique}
	
Les protéines de classe I sont initialement insérées dans la membrane du RE et se répartissent à travers le feuillet externe vers la monocouche délimitante de la GL par diffusion latérale (\autoref{fig:fig11}). Cette redistribution est susceptible de se produire au moment du bourgeonnement des GL lors de la synthèse \textit{de novo} où à des stades ultérieurs lorsque les GL matures forment de nouveaux contacts physiques avec le RE. Les domaines hydrophobes des protéines de classe I forment généralement des motifs en épingles à cheveux, qui s’insèrent à mi-chemin dans les bicouches lipidiques, les extrémités amino- et carboxy- terminales face au cytosol. Cette conformation monotonique, dépourvue de boucles luminales, permet à ces protéines d’être logées à la fois dans la bicouche lipidique du RE et dans la monocouche des GL \citep{RN604}. Cette topologie a été clairement démontrée pour les oléosines, une protéine constitutive des  GL végétales \citep{RN605}. On trouve des protéines aux fonctions diverses parmi les protéines de classe I associées aux GL, telles que des enzymes de biosynthèse des lipides neutres comme ASCL3, AGPAT3, GPAT4 et DGAT2 et des protéines impliquées dans la protéolyse dépendante de l'ubiquitine comme AUP1. \\ \\
\indent
Les mécanismes qui conduisent à l’accumulation des protéines de classe I au niveau des gouttelettes lipidiques sont encore obscurs, mais plusieurs auteurs supposent que cette redistribution dépendrait de l’état métabolique par (i) un recrutement actif des populations de protéines destinées aux GL ou (ii) d’une structure particulière ne permettant qu'à des sous-populations adéquates de protéines de traverser la monocouche des GL \citep[pour revue,][]{RN603}. Les propriétés biophysiques de la bicouche du RE et de la monocouche des GL créent probablement une première barrière de sélection qui contrôle le type de protéines capables de se répartir entre ces membranes \citep{RN608}. Par exemple, les GL ne peuvent pas accueillir les protéines avec des régions luminales hydrophiles car l'exposition de ces domaines dans le noyau neutre hydrophobe est énergétiquement défavorable. Inversement, les bicouches sont perméables aux molécules d'eau et une protéine constituée d’un domaine strictement hydrophobe sera plus favorablement recrutée sur les GL qui offrent un environnement exclusivement constitué de lipides neutres. De plus, la longueur des régions hydrophobes peut être critique pour diriger la délocalisation des protéines vers la surface des GL. Les oléosines végétales possèdent un domaine hydrophobe d’une longueur de 72 acides aminés \citep{RN609}, plus long que l’épaisseur limitante de la bicouche lipidique du RE d’environ 3 nm. Leur intégration dans la bicouche peut provoquer un stress membranaire à cause des mésappariements hydrophobes et la délocalisation vers la membrane des GL pourrait être un état plus favorable. Pour finir, il existe des éléments de séquence retrouvées dans les protéines de classe I qui facilitent la délocalisation vers les GL, requis initialement pour l’insertion dans la membrane du RE. À titre d’exemple, la topologie monotopique est favorisée par la présence d’un noeud de proline au centre de la région hydrophobe. La mutation de ce noeud de proline dans la séquence hydrophobe de AUP1 ou dans celle des oléosines retient les protéines au sein du feuillet externe du RE et restreint leur partitionnement vers les GL.

	\subsubsection{Les protéines de classe II : Ciblage depuis le cytosol}

Les protéines de classe II sont traduites dans le cytosol et sont recrutées directement à la surface des GL (\autoref{fig:fig11}). Elles ont pour la plupart des hélices amphipathiques dépliées en solution qui se replient en hélice lorsqu'elles interagissent avec une membrane phospholipidique. Elles peuvent également s'insérer dans la monocouche phospholipidique par le biais d'ancres lipidiques qui modifient la nature des acides gras comme la GTPase Rab18. D’autres s’associent en périphérie par liaison avec une protéine déjà intégrée dans la monocouche lipidique, comme la lipase HSL qui interagit avec PLIN1 ou la protéine de capside du virus de la dengue qui est recrutée par PLIN3. \\ \\
\indent
La monocouche des GL peut accueillir une densité ~13\% moins riche en phospholipides que le feuillet externe des bicouches lipidiques et est généralement moins statique. Des défauts dans le positionnement des phospholipides peuvent alors apparaître à la surface des GL, ce qui provoque une exposition transitoire des lipides neutres hydrophobes à la phase aqueuse (\autoref{fig:fig11}\textcolor{blue}{, encadré}). L’insertion des protéines de classe II peut alors être déclenchée au niveau de l’espace disponible entre les phospholipides afin de réduire la tension de surface \citep{RN617}. Si une pénurie en phospholipides constitue la force motrice de la liaison des protéines de classe II, certains mécanismes de régulation existent pour éviter le recrutement non sélectif. Par exemple, l’encombrement stérique de la surface des GL par des protéines « gardiennes », telles que celles de la famille des périlipines constituerait une barrière pour empêcher le recrutement d’autres protéines cytosoliques \citep{RN564}. Elles sont particulièrement importantes pour prévenir d’une dégradation indésirable des GL en bloquant l’accès aux lipases \citep{RN620}. En cas de changement d’état métabolique, les périlipines sont ciblées et dégradées par la machinerie d’autophagie médiée par les chaperones afin de libérer de la surface disponible pour recruter d’autres protéines, comme les enzymes responsables de la lipolyse ou de l’expansion des GL \citep{RN567}. Contrairement aux adipocytes, PLIN1 n'est pas exprimée dans les hépatocytes et PLIN2 s'associe moins fortement avec les GL, rendant ainsi la surface des GL plus permissive dans ce type cellulaire.

	\begin{figureth}
	\centering
			\includegraphics[width=\linewidth]{Figure_11.png}
		\caption[Mécanismes de recrutement des protéines de classes I et II à la surface des gouttelettes lipidiques]{\textbf{Mécanismes de recrutement des protéines de classes I et II à la surface des gouttelettes lipidiques.} Les protéines de classe I sont initialement insérées dans la membrane du RE selon une topologie monotopique en épingle à cheveux et diffusent latéralement ou sont recrutées activement à la surface des GL (vert). Les protéines de classe II sont recrutées à la surface GL à partir du cytosol en s’associant directement avec la monocouche phospholipidique via des hélices amphipathiques ou des ancres lipidiques, ou par des interactions protéine-protéine (bleu). (Adapté de \citealt{RN603}).}
				\label{fig:fig11}
	\end{figureth}
	\FloatBarrier
	
	\subsection{Fonctions des gouttelettes lipidiques}
	\label{section:fonctions}
	
Les GL sont des organites multi-tâches qui remplissent de nombreux rôles selon le type cellulaire et l’état physiologique de la cellule. Leur fonction primaire est de stocker les lipides lors des conditions de surplus de nutriments et de les re-mobiliser pour la production d'énergie ou pour la synthèse des phospholipides membranaires. Au-delà de leur rôle dans l’homéostasie énergétique, les GL assurent également des fonctions liées à la réponse généralisée face au stress lipotoxique ou oxydatif et servent de plateformes de repliement moléculaire \citep[pour revues,][]{RN625,RN645}. Les différentes fonctions communes aux GL seront décrites dans les paragraphes suivants et sont illustrées sur le schéma récapitulatif de la \Autoref{fig:fig12}. Le dysfonctionnement des gouttelettes lipidiques est lié au développement de nombreuses pathologies, telles que le diabète de type 2, l’insulino-résistance, l’athérosclérose et la stéatose hépatique \citep[pour revues,][]{RN626,RN627}.

	\subsubsection{Stockage des lipides et autres molécules lipophiliques}

Les GL sont responsables du stockage des acides gras en excès sous la forme de triacylglycérides ou d’esters de stérol. En cas de privation de nutriments, les acides gras stockés sont mobilisés par lipolyse ou lipophagie afin d’alimenter les processus métaboliques comme la ß-oxydation, la synthèse des lipoprotéines et autres médiateurs lipidiques. Lors de la croissance cellulaire, les acides gras stockés seront fortement mobilisés pour synthétiser les phospholipides nécessaires à l’extension des membranes plasmique ou nucléaire. À l’échelle de l’organisme, la sécrétion des lipoprotéines à partir des tissus adipeux ou hépatiques permet d’alimenter les autres tissus en substrat énergétique en cas d’insuffisance de nutriments. En fonction de l’organisme et du tissu, les gouttelettes lipidiques peuvent stocker d’autres molécules lipophiliques impliquées dans des fonctions diverses qui ne sont pas directement liées à la synthèse d’énergie, telles que des vitamines liposolubles, des précurseurs d’hormones stéroïdiennes ou des médiateurs de l’inflammation.
	
	\subsubsection{Neutralisation des composés lipotoxiques}

Malgré leur rôle essentiel dans les processus biologiques, une concentration élevée en acides gras  libres est toxique en raison de leur solubilité limitée et de leur potentiel de transformation en espèces bioactives hautement cytotoxiques qui peuvent entraîner la mort cellulaire \citep{RN629}. En effet, une augmentation locale de stérols libres perturbe l’intégrité et la rigidité des membranes cellulaires et peut directement agir comme détergent en induisant une lyse de la membrane. La conversion des stérols et des acides gras libres sous la forme relativement inerte et stable d’esters et de triacylglycérides et leur incorporation dans les GL protège les cellules des effets toxiques liés à une quantité excessive en lipides, plus communément définis par « lipotoxicité ». En parallèle des acide gras endogènes, les GL peuvent séquestrer une grande variété de molécules cytotoxiques présentes dans l’environnement extérieur. Les polluants exogènes étant généralement lipophiles (dérivés d’hydrocarbures), ils ont une forte pré-disposition à diffuser au travers des membranes cellulaires, à la manière des acides gras libres. L’accumulation de polluants exogènes dans les GL renforce l’idée que ces organites ont un rôle essentiel dans la préservation des autres organites cellulaires. La liste des espèces lipidiques toxiques séquestrées au sein des GL s’est récemment allongée, avec l’identification des polyprénols tels que les dolichols. Une concentration élevée de polluants exogènes dans les tissus adipeux des mammifères et des poissons est à présent reconnue comme un signe d’exposition chronique à des contaminants écologiques. Toutefois, les propriétés de séquestration des GL ne sont pas toujours bénéfiques : elles peuvent réduire l’efficacité thérapeutique, comme cela a été montré dans le contexte de la chimiothérapie avec le docétaxel, un médicament utilisé pour traiter le cancer du sein \citep{RN632}.

	\subsubsection{Réponse au stress oxydatif et à l’hypoxie}

Lors de l’utilisation des lipides comme substrat pour produire de l’adénosine triphosphate (ATP) par ß-oxydation, des dérivés réactifs de l’oxygène (ROS) sont générés le long de la chaîne de transport d’électrons de la mitochondrie. Les acides gras qui contiennent des chaînes polyinsaturées sont particulièrement sensibles à la peroxydation induite par une production excessive de ROS. La redistribution des acides gras polyinsaturés au sein des GL empêche leur incorporation dans les phospholipides, ce qui protège les parois membranaires des dommages oxydatifs. L’hypoxie favorise la lipogénèse en modulant l’expression des gènes impliqués dans l’absorption et la synthèse des acides gras. Afin de réduire l’effet lipotoxique lié à l’accumulation excessive de lipides sous condition hypoxique, la formation des GL est accrue.

	\subsubsection{Plateforme de stockage et de repliement des protéines}

La réponse protéique non repliée (UPR) a été initialement définie comme une réponse cellulaire adaptative qui s’active en cas de surcharge du RE dans la capacité de repliement des protéines \citep{RN635}. Les GL sont essentielles au maintien de l’homéostasie du RE en participant à la clairance des protéines aggrégées ou mal repliées qui s’accumulent dans la bicouche lipidique. Lors de leur formation, les GL servent alors de dépôt pour les protéines avant leur dégradation par la voie de dégradation des protéines associées au RE (ERAD). En lien avec la fonction de stockage des GL, il a été proposé que ces organites servent également au dépôt temporaire de protéines pour faire face à des situations de besoins aigus.

	\subsubsection{Régulation des gènes et entretien de l’ADN}

Dans les embryons de drosophile, les GL sont associées à des quantités importantes d'histones et peuvent être transférées vers les noyaux, ce qui favoriserait la structuration de la chromatine \citep{RN638}. Il a été également proposé que les GL nucléaires servent de plateforme de régulation de l’expression des gènes en transportant des facteurs de transcription. Par exemple, une étude a récemment mis en évidence une synthèse et une mobilisation de GL nucléaires transportant la protéine CCTα lors d’une augmentation importante en acides gras libres, ce qui active la synthèse de la PC selon un mécanisme rétro-actif \citep{RN639}.

	\begin{figureth}
	\centering
			\includegraphics[width=0.75\linewidth]{Figure_12.png}
		\caption[Diversité des fonctions communes aux gouttelettes lipidiques]{\textbf{Diversité des fonctions communes aux gouttelettes lipidiques.} Les fonctions essentielles des GL être classifiées en trois domaines qui concernent (i) l’homéostasie énergétique, où les GL servent de réservoirs pour diverses espèces lipidiques (ii) le stockage ou la dégradation des protéines et (iii) la réponse au stress cellulaire où les GL ont un rôle de protection contre les ROS et les composés lipotoxiques (Adapté de \citealt{RN579}).}
				\label{fig:fig12}
	\end{figureth}
	\FloatBarrier
	
	\subsubsection{Vers de nouvelles fonctions ?}
	
Récemment, le protéome des GL d’une lignée hépatocytaire a été établi \citep{RN508,RN641}. Il est dominé par des enzymes impliquées dans la régulation du métabolisme énergétique. Toutefois, on retrouve des protéines impliquées dans d’autres fonctions biologiques, comme le traffic vésiculaire, l’organisation du cytosquelette, la signalisation cellulaire, l’autophagie, les régulations transcriptionnelle et post-traductionnelle, renseignant sur de nombreuses fonctions potentiellement assurées par les GL au sein du tissu hépatique (\autoref{fig:fig13}).

	\begin{figureth}
	\centering
			\includegraphics[width=\linewidth]{Figure_13.png}
		\caption[Cartographie du protéome des gouttelettes lipidiques hépatocytaires]{\textbf{Cartographie du protéome des gouttelettes lipidiques hépatocytaires.} Les protéines associées aux GL sont regroupées en modules fonctionnels sur la base de l’analyse GO et des annotations fonctionnelles UNIPROT. Les lignes pleines représentent les interactions physiques au sein des modules fonctionnels et les lignes transparentes représentent les interactions entre les protéines de modules distincts. L’intensité de la couleur bleue au sein d’un noeud indique le degré de confiance avec lequel la protéine fait partie du module. Les facteurs entourés en rouge représentent les protéines qui ont été précédemment validées par études de co-localisation (Adapté de \citealt{RN641}).}
				\label{fig:fig13}
	\end{figureth}
	\FloatBarrier
	
	\subsection{Interface entre les gouttelettes lipidiques et les pathogènes humains}
	
L'implication des GL dans le cycle infectieux des virus a été documenté pour la première fois en 2007 avec le VHC \citep{RN488} où elles sont mobilisées pour le stockage de la protéine de capside et pour l’assemblage des particules virales (voir \autoref{section:cycle}). D'autres espèces virales, principalement des virus à ARN en raison de leur réplication cytoplasmique, dépendent également des GL au cours de leur cycle : le virus de la dengue (DENV), le rotavirus (RV), l’orthoréovirus et le virus de l’hépatite B (VHB) \citep[pour revue,][]{RN643} et plus récemment le coronavirus du syndrome respiratoire aigu sévère (SARS-CoV2) \citep{RN867}.

\clearpage

%%%%%%%%%%%%%%%%%%%%%%%%%%%%%%%%%%%%%%%%%%%%%%%%%%%%%%%%%%%%%%%%%%%%%%%%%%%%%%%%%%%%%%%%%%%%%%%%%%%%%%%%%%%%%%%%%%%%%%%%%%%%%%%%%%%%%%%%%%%%%%%%%%%%%%%%%%%%%%%%%%%%

% 4 - Généralités sur le VHC

\section{Généralités sur le virus de l'hépatite C}	

	\subsection{Identification et histoire naturelle}

L’identification du virus de l'hépatite C (VHC) n’a pas été une mince affaire et les très nombreuses tentatives utilisant les méthodes conventionnelles apparurent extraordinairement laborieuses et décevantes pendant les 15 premières années de recherche à son sujet \citep[pour revue,][]{RN127}. Le domaine d’étude des hépatites virales a débuté dans les années 1960 avec la distinction de deux hépatites virales majeures sur les plans clinique et immunologique, plus tard prouvées comme étant attribuables à l’infection par le virus de l'hépatite A (VHA) \citep{RN96} et par le virus de l'hépatite B (VHB) \citep{RN88}. Avec l’avancée des tests sérologiques, il a été montré que la plupart des cas de transmission d’hépatites par voie parentérale n’était pas due à ces virus, mais à un agent étiologique inconnu. La terminologie « hépatites non A, non B » fut alors introduite pour désigner ces hépatites, dont les agents responsables n’étaient pas identifiés mais qui apparaissaient sérologiquement distinctes des hépatites A et B. Les chercheurs se sont rapidement mobilisés afin d’identifier ce nouvel agent pathogène en utilisant les méthodes connues de l’époque (ultra-filtration, immunodiffusion, dosage immuno-enzymatique (ELISA), dosage radio-immunologique (RIA) mais sans succès malgré des années de travaux intensifs. L’émergence des approches moléculaires dans les années 1980 n’a pas non plus permis de caractériser l’agent pathogène, mais a permis de mettre en évidence sa proximité avec des familles virales connues (\textit{Flavirividae}, \textit{Togaviridae}, \textit{Hepadnaviridae} et \textit{Picornaviridae}) puis avec le virus de l'hépatite Delta (VHD) découvert en 1977 \citep{RN113}. Des essais pour cultiver l’agent pathogène \textit{in vitro} dans différents systèmes cellulaires ont également été menés mais jamais aucun cycle viral productif n’a pu s’établir (cet « exploit » n’a été accompli qu’en 2005 par l’équipe de T. Wakita à partir d’une souche dérivé d’un patient japonais \citep{RN124}). En parallèle, les essais de visualisation par microscopie électronique de l’éventuelle progénie virale circulant dans le sérum des patients n’ont pas été particulièrement instructifs. Les premières particules virales infectieuses du VHC dans le sérum n’ont été détectées que très récemment \citep{RN125} (voir \autoref{section:organisation}). \\ \\
\indent
À la fin des années 1970, le développement de l’immuno-criblage permettait une détection précise de la présence d’antigènes, mais l’efficacité de cette technique dépend grandement de la qualité et de la spécificité des anticorps employés. Dans le cas du VHC, aucun anticorps ciblant les composants viraux n’ont pu être purifiés à partir des sérums de patients malgré les tentatives répétées par de nombreux laboratoires de recherche \citep{RN116}, ce qui limitait alors grandement le recours à des méthodes basées sur des anticorps dans le cadre de l’identification de cet agent pathogène. En 1989, l’équipe du Dr. Houghton a décidé de recourir à une méthode particulièrement hasardeuse d’immuno-criblage en partant directement du sérum d’un patient souffrant d’hépatite chronique en guise de source d’anticorps anti-viraux. En utilisant ce sérum contre une librairie d’acide désoxyribonucléique complémentaire (ADNc) générée de manière aléatoire à partir du plasma de chimpanzés infectés, les scientifiques sont parvenus à mettre en évidence 6 clones positifs, malgré les difficultés techniques rencontrées et la quantité importante de contaminants. Un des clones positifs contenait un acide ribonucléique (ARN) simple brin de polarité positive et d’une longueur d’environ 10.000 nucléotides qui ne correspondait pas à un gène cellulaire et qui était spécifique des animaux infectés. Les tests qui suivirent le clonage et l’étude de cet ARN confirmaient son caractère immunogène et pathogène. Il fut ainsi caractérisé comme l’agent étiologique majeur des hépatites transmises par voie parentérale et nommé le VHC \citep{RN91}. La séquence du VHC demeurait toutefois incomplète jusqu’à la découverte de la région 3’X en 1996, un élément hautement conservé situé à l’extrémité du génome viral et indispensable à sa réplication autonome \citep{RN249}. Le prix Nobel de Médecine 2020 a été attribué aux Drs. Harvey J. Alter, Michael Houghton et Charles M. Rice pour leur contribution décisive à la découverte et à la caractérisation du VHC. Le comité Nobel a déclaré que ces trois scientifiques avaient « rendu possibles des méthodes de diagnostic sanguin et des solutions thérapeutiques qui ont permis de sauver des millions de vies ».

	\subsection{Classification et variabilité génétique}
	\label{section:classification}
	
La séquence du génome du VHC fut rapidement décrite et analysée suite à son identification et le virus fut classé dans la famille des \textit{Flaviviridae} puisqu’il présente une organisation génomique proche de celles des Pestivirus et des Flavivirus, deux genres appartenant à cette famille virale \citep{RN92}. Néanmoins, en raison de la faible homologie de séquences nucléotidique et protéique entre le VHC et les autres espèces virales connues, il devint le représentant prototype d’un nouveau genre, les Hepacivirus \citep{RN114}. Depuis, d’autres espèces virales infectant une grande variété d’hôtes ont été décrites et attribuées au genre Hepacivirus, c’est le cas du GB virus B (GBV-B) infectant les singes du Nouveau Monde \citep{RN119} et plus récemment des espèces infectant les rongeurs, chauve-souris, oiseaux, bovins, requins et chevaux \citep[pour revue,][]{RN100}. Un quatrième genre Pegivirus fut introduit à la famille des \textit{Flaviviridae} suite à la découverte de nouveaux virus hépatotropes humains, le GB virus A (GBV-A), GB virus C (GBV-C) et le GB virus D (GBV-D) relativement éloignés des Hepacivirus sur le plan phylogénétique \citep{RN121}.

	\begin{figureth}
	\centering
			\includegraphics[width=0.80\linewidth]{Figure_14.png}
		\caption[Arbre phylogénétique des membres représentatifs des 4 genres de la famille des \textit{Flaviviridae}]{\textbf{Arbre phylogénétique des membres représentatifs des 4 genres de la famille des \textit{Flaviviridae}.} L'analyse est basée sur la comparaison des séquences codant l'hélicase NS3. La longueur des branches est proportionnelle au nombre de substitutions par site. (D'après \citealt{RN100}).}
				\label{fig:fig14}
	\end{figureth}
		\FloatBarrier

Les \textit{Flaviviridae} sont des virus à ARN simple brin de polarité positive, ils ont par conséquent une dynamique d’évolution très forte et circulent sous la forme d’une distribution complexe de variants génétiques. Au sein d’une même espèce virale, ces variants peuvent être relativement éloignés sur le plan phylogénétique : c’est le fruit d’une longue évolution par la génération et le maintien continu des variants les plus adaptés à un environnement et à une population donnée au fil des années. La dynamique des populations de virus à ARN peut également s’observer à l’échelle de l’individu : la réplication rapide favorise l’apparition de mutations au cours d’une infection chronique ce qui aboutit à la formation d’un ensemble de variants génétiquement proches communément définit par l’appellation « quasi-espèces ». La mutagénèse spontanée des virus à ARN pose un grave problème de santé publique, en premier lieu sur le plan clinique en raison de l’émergence de mutations de résistance sous la pression de sélection exercée par les molécules antivirales, qui constitue la principale raison des échecs thérapeutiques (voir \autoref{section:elimination}), mais également sur le plan épidémiologique puisque les \textit{Flaviviridae} sont responsables d’épidémies émergentes ou ré-émergentes contemporaines.

	\subsubsection{Variabilité inter-individus : Génotypes et sous-types viraux}
	
Peu après la publication de la première séquence complète du génome du VHC, la forte hétérogénéité génétique entre les isolats cliniques étudiés par les laboratoires du monde entier fut rapidement remarquée. Face à l’accumulation de séquences et de terminologies diverses dans la littérature, l’élaboration d’un système universel pour classer les variants du VHC devint une priorité afin de simplifier les futures études épidémiologiques et phylogénétiques. Une première nomenclature uniforme fut proposée en 1994, reposant sur les similarités de séquence en nucléotides : les séquences analogues entre les variants furent regroupées sous le terme de « génotypes » et les groupes de séquences les plus proches au sein de ces génotypes furent désignés sous le terme de « sous-types » \citep{RN117}. Les génotypes et sous-types sont attribués aux séquences par ordre de découverte chronologique, par exemple le premier clone mis en évidence par Choo et ses collègues en 1989 appartient au génotype 1 et sous-type a. L’appartenance à un génotype et à un sous-type est déterminée par l’analyse de petites régions peu variables au sein du génome, souvent dans les extrémités 5' non codante (5'NC) et 3' non codante (3'NC) ou dans les gènes codant la protéine Core ou la protéine non structurale 5B (NS5B). Les séquences nucléotidiques des souches appartenant à des génotypes hétérologues diffèrent de 30 à 35\% et de plus de 15\% entre sous-types \citep{RN120}. En mai 2019, les bases de données publiques comptabilisent plus de 1300 séquences complètes du génome du VHC, classées en 8 génotypes et 90 sous-types reconnus \citep{RN267}. Malgré l’élargissement considérable des données relatives à la diversité du VHC, la classification basée sur les génotypes et sous-types reste très robuste aujourd’hui, bien qu’elle ne référence pas encore les rares recombinants naturels (voir paragraphe \nameref{section:recombinants}). Tous les génotypes sont responsables de maladies hépatiques, toutefois des études cliniques ont pu mettre en évidence des disparités dans la progression et la sévérité de la pathologie, ainsi que dans la sensibilité aux traitements selon l’origine génotypique de la souche infectante. En effet, les infections chroniques par les souches de génotype 3 seraient liées à une prévalence plus élevée de stéatose hépatique et par conséquent, à un risque accru de progression vers des complications hépatiques graves \citep{RN306,RN360} tandis que les infections chroniques par les souches de génotype 1b seraient associées à un risque plus important de CHC \citep{RN1072,RN1073}. Ces découvertes indiquent l’importance de coupler aux études cliniques des analyses plus fondamentales pour identifier des signatures moléculaires propres à certains génotypes, voire directement des résidus « à risque » au sein des séquences virales, et contribuer ainsi à la mise en place d’une cartographie globale permettant de prédire l’évolution de la pathologie hépatique chez les individus chroniquement infectés.

	\begin{figureth}
	\centering
			\includegraphics[width=0.80\linewidth]{Figure_15.png}
		\caption[Classification du virus de l'hépatite C en 8 génotypes majeurs et 90 sous-types]{\textbf{Classification du virus de l'hépatite C en 7 génotypes majeurs et 90 sous-types.} L'arbre phylogénétique repose sur l'analyse comparative des séquences des cadres de lecture ouverts des différentes souches du VHC. L'isolat unique de génotype 8 identifié en 2019 est manquant sur la figure. (D'après \citealt{RN1233}).}
				\label{fig:fig15}
	\end{figureth}
		\FloatBarrier
		
		\subsubsection{Variabilité intra-individus : Quasi-espèces}
		
La notion de « quasi-espèces » a été introduite par Eigen et Schuster en 1977 \citep{RN95} dans le contexte de leurs travaux sur la diversité des acides nucléiques née des erreurs de leur réplication autonome, dans le but d’établir un modèle mathématique théorisant le processus d’évolution des formes de vie primitives. Ce concept a rapidement été adopté en virologie avec la première étude sur le sujet démontrant que la progénie virale du bactériophage Qß n’était pas génétiquement homogène \citep{RN93}. Ce postulat constitue le premier témoignage expérimental du comportement des virus à ARN sous la forme de quasi-espèces. Depuis, cette caractéristique a été démontrée pour les représentants majeurs de groupes de virus à ARN humains, animaux ou végétaux en culture et \textit{in vivo}, parmi lesquels on retrouve le virus de l'immunodéficience humaine 1 (VIH-1), le virus de la grippe et le VHC \citep{RN106}. Les variants qui composent une quasi-espèce ont une séquence génomique qui ne diffère généralement pas plus de 1 à 3\%. L’existence de ces variants est le résultat de l’absence d’activité de relecture de la réplicase de ces virus, induisant une accumulation de mutations au fur et à mesure des cycles réplicatifs. \\ \\
\indent
Chez le VHC, le taux de mutation par cycle réplicatif est estimé à 3,5x10\up{-5} \citep{RN97}. Ces mutations apparaissent dans l’intégralité du génome du VHC bien qu’il existe des régions plus ou moins sujettes à une variabilité intrinsèque. En effet, les séquences du génome qui codent des fonctions virales essentielles (telles que la traduction et la réplication) ou à des domaines structuraux majeurs (les extrémités 3’NC et 5’NC et la capside) sont généralement plus conservées. Cela provient certainement du fait qu’une mutation dans ces domaines plus « sensibles » pourrait plus fréquemment aboutir à la production d’un génome déficient. À l’inverse, des régions considérées comme hypervariables peuvent atteindre des taux de mutation de l’ordre de 1x10\up{-3}, c’est le cas par exemple de certaines séquences codantes de E2 \citep{RN97}. Ainsi, la forte hétérogénéité génétique du VHC à l’échelle de l’individu résulte d’une combinaison de deux facteurs : l’infidélité de la réplicase virale couplée à un processus d’infection très dynamique avec des cycles de production et de clairance des particules virales estimés à 10\up{12} virions par jour et une demi-vie des virions de l’ordre de quelques heures \citep{RN109}. Au cours de l’infection chronique, l’ensemble des variants produits sont théoriquement neutres en terme de valeur sélective et évoluent autour d’une séquence moyenne stable dite « consensus ». Toutefois, sous une forte pression de sélection, la nature de la souche circulante dominante peut changer à tout instant pour une souche mineure pré-existante plus adaptée au nouvel environnement. Une conséquence de ce phénomène d’hétérogénéité permettrait ainsi au VHC d’échapper aux acteurs du système immunitaire, en particulier aux anticorps neutralisants \citep[pour revue,][]{RN87}, mais également de résister aux traitements antiviraux \citep[pour revue,][]{RN218} puisqu’une mutation de résistance pourrait pré-exister au sein de la population virale avant la prise en charge des patients sous thérapie. En plus de représenter un challenge majeur pour la vaccination, ce phénomène a des implications thérapeutiques importantes puisque l’efficacité des traitements antiviraux reposerait à la fois sur une utilisation précoce avant que la population virale ne se diversifie et sur un ciblage plus global de la distribution des variants plutôt que sur une entité génomique individuelle.

		\subsubsection{Les recombinants naturels}
		\label{section:recombinants}
		
La recombinaison désigne le mécanisme par lequel un génome comportant des portions chimériques d'origine phylogénétique distincte est formé. Des évènements de recombinaison ont été identifiés parmi les virus à ARN de polarités négative et positive, notamment chez plusieurs membres de la famille des \textit{Flaviviridae} du genre Flavivirus \citep{RN101}, Pestivirus \citep{RN89} et Hepacivirus \citep{RN123}. Ce processus joue un rôle important dans l’évolution des virus à ARN en générant de la variabilité génétique lors d’évènements d’échanges de séquences nucléotidiques entre des génomes hétérologues. Lors d’une réplication concomitante des deux génomes, la polymérase passe d'une matrice ARN à l’autre, tout en conservant le cadre de lecture, ce qui aboutit à la formation de brins hybrides complémentaires. Le mécanisme exact de l'échange de brin n'est pas connu mais pourrait être facilité par les temps de «  pause » de la réplicase lors de l'élongation du brin et la présence de structures secondaires pouvant déstabiliser la liaison à l’ARN \citep{RN94}. Dans le cadre du VHC, il a été montré que les cellules infectées deviennent réfractaires à une infection consécutive limitant fortement la possibilité d’avoir deux virus dans la même cellule \citep{RN141}. Toutefois, la recombinaison apparaît comme un processus important puisqu’elle serait présente chez 18\% des patients chroniquement infectés \citep{RN115}. Il est d’ailleurs très probable que le taux de recombinaison soit sous-estimé au sein des quasi-espèces virales, étant donné que la détection d’une discordance dans l’alignement phylogénétique du génome résultant repose sur le fait que les deux virus parentaux doivent différer suffisamment dans leurs séquences nucléotidiques. \\ \\
\indent
Ce phénomène serait moins fréquent pour les recombinants inter- et intra-génotypiques car il requiert que l’individu soit co-infecté par au moins deux souches de génotypes ou sous-types distincts. La prévalence des co-infections varie entre 1 et 20\% selon le groupe de populations \citep{RN147,RN99}. Les co-infections seraient favorisées par une activité insuffisante du système immunitaire, qui ne parviendrait pas à empêcher les évènements de réinfection. À ce jour, 14 recombinants naturels du VHC ont été identifiés dans la population mais un seul circule activement, notamment en Irlande \citep{RN148}, en Uzbekistan \citep{RN151}, en France \citep{RN154} et à Chypre \citep{RN155}. Il s’agit d’un virus recombinant intergénotypique codant les gènes structuraux d’une souche de sous-type 2k et les gènes non structuraux d’une souche de sous-type 1b, identifié pour la première fois à St Petersburg en 2002 \citep{RN102}. Des séquences appartenant au génotype 2 sont présentes dans la majorité des recombinants naturels décrits jusqu'à présent ce qui pourrait suggérer un rôle critique d’éléments propres au génotype 2 dans la stabilité et la fonctionnalité du génome recombinant résultant. Pour l’heure, il n’existe pas de méthode pour classifier les formes recombinantes du VHC qui nécessitent un séquençage complet du génome afin de définir les jonctions entre les génotypes et les sous-types \citep{RN126}. Toutefois, certains points « chauds » de recombinaison ont été identifiés dans les gènes codant NS2 et NS3 ainsi qu’à la jonction NS2/NS3 \citep{RN98}. Cet échange d’information génétique est un mécanisme clé pour la production de nouveaux génomes conférant des avantages sélectifs. La recombinaison peut agir comme un catalyseur de la résistance antivirale en combinant plusieurs mutations de résistance issues des virus parentaux. Ainsi, l’étude des mécanismes et des conséquences de la recombinaison est importante en raison de son potentiel à produire de nouvelles souches hybrides qui pourraient avoir de nouvelles propriétés pathogéniques, en particulier dans le cadre de la conception de vaccins vivants atténués multivalents \citep[pour revue,][]{RN268}.

	\subsection{Épidémiologie, prévalence et facteurs de risque}
			\label{section:epidemiologie}
	
	D’après les données épidémiologiques de l’Organisation Mondiale de la Santé (OMS), la séroprévalence globale du VHC est estimée à 1,6\% soit 115 millions d’individus. Il est à noter que ces chiffres prennent en compte les infections résolues spontanément et les succès thérapeutiques. Le nombre actuel d’individus chroniquement infectés est quant à lui estimé à 58 millions en 2021, ce qui correspond à une prévalence inférieure à 1\%. L’absence de données pour un nombre important de pays, en particulier les pays à faible revenus (où seulement 29\% d’entre eux renseignent des données), contraint à extrapoler la prévalence du VHC et la fréquence génotypique au niveau mondial à partir d’études cliniques et épidémiologiques publiées après 2013, restreintes à 100 pays. A l’échelle du globe, la prévalence du VHC est très hétérogène, avec 5 pays comptabilisant à eux seuls la moitié des cas totaux d’infections : la Chine, le Pakistan, l’Inde, l’Egypte et la Russie. La forte prévalence dans ces pays est attribuable à une transmission iatrogène du virus favorisée par l’utilisation de dispositifs médicaux contaminés, avant la découverte de l’agent pathogène et la mise en place des premiers diagnostics dans les années 1990. La source d’infection par le VHC a notamment bien été documentée en Egypte où lors d’une campagne anti-parasitaire massive dans les années 1960-1970 qui consistait en une succession d’injections intraveineuses avec du matériel n’ayant pas été systématiquement stérilisé a déclenché la dissémination du virus au sein de la population \citep{RN86}. Depuis les années 1990, la majorité des infections par le VHC survient chez les usagers de drogues par voie intraveineuse, une pratique qui constitue le facteur de risque principal dans les pays occidentaux (concerne 50 à 60\% des infections aigües) \citep{RN112}. 
	
	\begin{figureth}
	\centering
			\includegraphics[width=\linewidth]{Figure_16.png}
		\caption[Prévalence relative des génotypes 1 à 7 du virus de l’hépatite C à l’échelle mondiale]{\textbf{Prévalence relative des génotypes 1 à 7 du virus de l’hépatite C à l’échelle mondiale.} La répartition des différents génotypes du VHC déterminée dans 21 régions (codées par couleurs) est représentée sous la forme de diagrammes circulaires dont le diamètre est proportionnel au nombre d’individus infectés par région (Adapté de \citealt{RN111}).}
		\label{fig:fig16}
	\end{figureth}
		\FloatBarrier

Les génotypes 1 et 3 sont les plus prévalents au niveau mondial : ils comptabilisent respectivement 46\% et 22\% des cas totaux d’infection par le VHC \citep{RN111}. Leur dissémination massive aurait été engendrée par la distribution à l’échelle mondiale de produits sanguins contaminés par les sous-types 1a et 1b \citep{RN103} et par la forte association du sous-type 3a avec l’utilisation de drogues injectables \citep{RN107}. Les génotypes 2, 4 et 6 sont responsables d’une plus faible proportion d’infections, avec des taux respectifs de 13\%, 13\% et 2\% et le génotype 5 plus rare, a un taux d’infection inférieur à 1\%. Le génotype 7 n’a été retrouvé jusqu’à présent que dans quelques cas d’infection chez des patients originaires de la République Démocratique du Congo \citep{RN108,RN270}. Récemment, une équipe du Canada a isolé un nouveau génotype 8 à partir de patients infectés originaires de l’état de Punjab en Inde \citep{RN272}. La distribution des génotypes du VHC varie considérablement selon les zones géographiques, à l’exception du génotype 1 omniprésent à la surface du globe. Des souches considérés comme « endémiques » circulent depuis plusieurs siècles dans des secteurs plus restreints : le génotype 2 est principalement retrouvé en Afrique de l’Ouest, le génotype 3 en Asie du Sud, le génotype 4 en Afrique Centrale et au Moyen-Orient et le génotype 6 en Asie du Sud-Est (\autoref{fig:fig14}). L’expansion de la diversité génétique du VHC entre les continents aurait été influencée par les migrations de populations humaines au cours de l’histoire et plus particulièrement entre les zones ayant entretenu des fortes relations humaines, favorisant ainsi le transfert de variants. Une des périodes particulièrement propice à ces transferts fut lors des années 1700 à 1850 à l’apogée de l’expansion coloniale, où le commerce d’esclaves trans-atlantiques a été le moteur de la diffusion de souches de génotype 2 originaires d’Afrique de l’Ouest vers des zones plus éloignées, parfois sur un autre continent, comme en Amérique du Nord et en Europe \citep{RN105}. De la même manière, les migrations de populations venant des pays où le génotype 3 est dominant comme l’Inde ou le Pakistan a largement permis sa dissémination à la surface du globe \citep{RN122}. Globalement, le nombre total de personnes chroniquement infectées par le VHC décroît depuis 2007 bien qu’il y ait des variations majeures entre les zones géographiques. Ce phénomène s’explique par un taux de mortalité lié aux maladies du foie plus important que les nouvelles infections dans la plupart des pays, estimées à 1,5 millions de nouveaux cas en 2019. 

	\subsection{Élimination du virus de l'hépatite C en tant que problème de santé publique}
	\label{section:elimination}
	
Le récent développement et l’implémentation en clinique des agents antiviraux à action directe (AAD) constituent une véritable révolution thérapeutique pour l’hépatite C. Dans ce contexte, l’OMS a fixé l’objectif d’éliminer l’hépatite C en tant que problème de santé publique d’ici 2030, c’est-à-dire, réduire l’incidence des nouvelles infections de 90\% et la mortalité liée au VHC de 65\%. Atteindre cet objectif ambitieux repose sur la mise en place de politiques de santé et de plans d’action à l’échelle locale et nationale ainsi que la coopération de tous les acteurs de la santé publique, les chercheurs, les cliniciens et les entreprises pharmaceutiques. De nombreux pays occidentaux ne devraient pas y parvenir avant 2050 et cela paraît irréalisable pour les pays à faible et moyens revenus, souvent les plus touchés par l’hépatite C, qui commencent tout juste à aborder la question \citep[pour revue,][]{RN156}. Dans ce contexte, les défis majeurs sont : (i) l’identification et la prise en charge systématique de tous les porteurs sains du VHC, (ii) l’accès au dépistage et aux thérapies antivirales dans les pays aux ressources limitées, et (iii) le développement de nouvelles options thérapeutiques afin d’assurer un taux de guérison global de 100\% pour tous les génotypes et sous-types viraux \citep[pour revue,][]{RN128}. Combiner l’ensemble de ces efforts devraient aboutir à une réduction drastique de la problématique de l’hépatite C au niveau mondial, mais la question de l’éradication du VHC reste difficilement concevable sans le développement d’un vaccin préventif.

	\subsubsection{Dépistage, diagnostic et outils de surveillance}

Les approches traditionnelles de dépistage et de diagnostic du VHC reposent sur des ELISA visant à documenter l’exposition par détection de la présence d’anticorps dans le sérum ou le plasma des patients. Les tests sérologiques ne conviennent pas pour détecter les infections précoces et reposent sur une séroconversion qui intervient généralement 6 à 12 semaines après l'exposition \citep{RN157}. Ils ne permettent également pas de témoigner d’une infection active puisque les anticorps peuvent persister plusieurs années voir décennies chez les patients ayant résolus l’infection \citep{RN161}. En cas de dépistage sérologique positif, la détection de l’ARN viral par réaction en chaîne de la polymérase (PCR) en temps réel ou des antigènes viraux dans le sang sont donc indispensables pour repérer une réplication active du virus et identifier les individus nécessitant une prise en charge thérapeutique \citep{RN162}. Le suivi du niveau d’ARN ou des antigènes viraux ne permet pas de prédire l’évolution et l’issue de la pathologie mais peut servir d’indicateur de la réponse au traitement. La guérison se traduit par un niveau indétectable de l’ARN viral 12 semaines après la fin du traitement \citep{RN163}. Ces procédures s’effectuent majoritairement dans des laboratoires médicaux et autres infrastructures de santé, généralement chez les patients souffrant déjà d’atteintes hépatiques graves, qui ne constituent pas la majorité des porteurs du VHC. En effet, 80\% des individus infectés ignorent leur condition dans la plupart des pays  puisque l’infection aigüe ne s’accompagne généralement pas de symptômes apparents jusqu’aux stades plus avancés de la pathologie, ce qui facilite la transmission continue du VHC dans les populations à risques. Un dépistage de masse apparaît donc aujourd’hui indispensable pour renseigner tous les porteurs sains du VHC sur leur statut virologique afin de réduire le risque de transmission et de les prendre en charge avant une détérioration de leur état de santé. \\ \\
\indent
Dans la perspective d’améliorer le dépistage et l’accès au soin en particulier auprès des populations marginalisées (milieu carcéral, milieu rural, squats, migrants, sans-abris) ou dans les pays à ressources limitées disposant peu de centres cliniques spécialisés, un test rapide d'orientation diagnostic (TROD) a été conçu en 2010, permettant de doser en moins de 30 minutes les anticorps par simple prélèvement de sang par piqûre de doigt \citep{RN190}. Depuis 2017, ces petits dispositifs ont été perfectionnés afin de mesurer en parallèle les marqueurs de l’infection par un système automatisé d’extraction et amplification d’acide nucléique, fournissant un résultat en 100 minutes. Un diagnostic complet peut être ainsi réalisé en moins de 2 heures \citep{RN192}. Pour finir, le recours à la simple prévention par la sensibilisation des individus aux pratiques à risques, la formation des prestataires de soin sur les procédures de stérilisation et l’approvisionnement suffisant en aiguilles et seringues pour couvrir toutes les injections pourraient réduire l’incidence du VHC de 80\% \citep[pour revue,][]{RN193}.

	\subsubsection{Thérapies à l’interféron α et à la ribavirine (1989-2011)}
	
À partir de 1989, les patients atteints d’hépatites C chroniques étaient naturellement traités à l’interféron α (IFNα), un médiateur clé de la réponse antivirale innée, couramment utilisé pour d’autres types d’infections virales. Toutefois, le succès des monothérapies à l’IFNα pour l’hépatite C s’avéra très limité avec un taux de guérison de 12\% pour un traitement sur 6 mois et de 20\% lors d’une extension sur 12 mois \citep{RN136}. En 1998, l’addition de ribavirine, un agent antiviral à large spectre, en combinaison avec l’IFNα a permis d’augmenter le taux de guérison à 35-40\% \citep{RN199}. Jusqu’en 2011, la bithérapie reposant sur des injections pendant 24 à 48 semaines de PEG-IFNα (un dérivé comportant une liaison covalente à du polyéthylène-glycol (PEG) améliorant la demi-vie et l’absorption de la molécule) et de ribavirine constituait alors la norme thérapeutique pour l’hépatite C \citep{RN203,RN205}. Cette bithérapie permettait d’atteindre des taux de guérison globale de 90\% pour les infection aigües et de 55\% pour les infections chroniques \citep{RN207,RN209} mais la réponse au traitement variait fortement selon les caractéristiques de l’hôte (âge, genre, ethnicité, obésité, stade de la maladie, consommation d’alcool) et les caractéristiques virales telles que le génotype. En effet, 80\% des patients infectés par les souches de génotype 2 ou 3 éliminaient avec succès le virus, en revanche seulement 40\% des patients infectés par le le génotype 1, plus commun, parvenaient une clairance virale \citep{RN211}. L’utilisation de ces molécules pouvaient entraîner des effets secondaires sévères tels que des troubles hématologiques (anémie, neutropénie, thrombocytopénie), neuropsychiatriques (dépression) ou des maladies auto-immunes conduisant à l’arrêt prématuré du traitement chez environ 10\% des patients.


	\subsubsection{Thérapies antivirales à action directe (depuis 2011)}
	
Le développement successif des systèmes d’étude du VHC (décrits dans la \autoref{section:modele}) et la résolution des structures tridimensionnelles des protéines virales dotées d’activité enzymatique a énormément contribué au progrès de nos connaissances sur le cycle infectieux. Ces données ont guidé la conception des AAD ciblant de manière spécifique des processus clés du virus : la maturation de la polyprotéine virale par blocage du clivage par la protéine non structurale 3-4A (NS3-4A) (boceprevir, telapravir en 2011), la réplication virale par interférence avec la polymérase NS5B (sofosbuvir en 2013) et l’activité de la protéine non structurale 5A (NS5A) (daclatasvir, ledipasvir en 2014). En 2019, 17 AAD ciblant ces trois protéines virales sont approuvés dans les pays occidentaux \citep[pour revue,][]{RN212}. \\ \\
\indent
La communauté scientifique s’accorde sur le fait que les AAD doivent être combinés sous la forme de bi- ou tri-thérapie afin de limiter le risque d’émergence de mutations de résistance dans les protéines ciblées, puisqu’il a été montré qu’une unique substitution dans la séquence de la protéine ciblée serait suffisante pour conférer une résistance \textit{in vitro} \citep{RN213}. Dans les pays occidentaux, l’émergence des traitements basés sur une combinaison de trois AAD a permis d’atteindre des taux de guérison >90\% pour la majorité des génotypes. En plus d’être facilement administrables (voie orale), les tri-thérapies ont permis de limiter le recours à l’IFNα et la ribavirine particulièrement mal tolérés, ce qui a considérablement amélioré la qualité de vie des patients lors du traitement. L’action directe des AAD vis-à-vis du virus a également permis de réduire le traitement de 8 à 24 semaines par rapport aux précédentes thérapies. En ce qui concerne les infections aigües, des essais cliniques sont en cours afin d’optimiser les traitements sur des courtes durées, de 4 à 6 semaines, dans le but d’améliorer la rentabilité économique \citep[pour revue,][]{RN137}. En effet, le coût très élevé des thérapies antivirales freine son accès dans les pays aux ressources limitées, souvent les plus touchés par l’hépatite C et l’OMS estime que seulement 62\% de la population dépistée bénéficie de ces traitements. Les thérapies basé sur l’IFNα sont donc encore très largement répandues dans les pays où les AAD sont inabordables, en Europe de l’Est ou en Asie. Dans le cadre des stratégies locales d’élimination du VHC, des médicaments génériques doivent être établis afin d’étendre l'utilisation de ces traitements dans les pays en voie de développement. \\ \\
\indent
En dépit de l’optimisation de ces traitements, les études cliniques déclarent jusqu’à 15\% d’échecs thérapeutiques selon le groupe de patients. Les échecs thérapeutiques sont majoritairement liés à la présence de variants qui peuvent contenir des polymorphismes naturellement résistants dans les régions ciblées par les AAD, réduisant ainsi leur susceptibilité face à ces molécules \citep[pour revues,][]{RN218,RN129}. Les variants résistants les plus stables peuvent persister plusieurs années après l’échec du traitement ce qui pose un grave problème pour les thérapies de 2ème intention. Et ce problème ne s’amenuisera pas puisque l’utilisation généralisée des AAD va contribuer à augmenter la prévalence et la transmission des variants résistants. En France, parmi les patients en échec thérapeutique entre 2015 et 2018, 22.3\% d’entre eux étaient infectés par des souches de génotype 4 et parmi eux, on note une pré-dominance du sous-type 4r qui comporte fréquemment des mutations de résistance dans les régions NS3, NS5A et NS5B \citep{RN221}, rendant ce sous-type viral particulièrement difficile à traiter quelle que soit la combinaison d’AAD utilisée. D’autres études cliniques récentes relatent des taux sous-optimaux de clairance virale chez les patients infectés par des souches de sous-types 1e, 1g, 1h ou 1l \citep{RN225}. Ces sous-types inhabituels en Europe et aux Etats-Unis sont prévalents dans les régions africaines et représentent 15\% des cas de VHC dans le monde, laissant en conséquence plusieurs millions de personnes sans option thérapeutique. Ainsi, il est indispensable de continuer à développer de nouvelles combinaisons d’AAD, ciblant par exemple les autres protéines virales \citep{RN226,RN274,RN275} pour les patients en échec ou infectés par ces sous-types multi-résistants et de rendre systématique la détermination du génotype et du sous-type pour guider les indications thérapeutiques.

	\subsubsection{Stratégies vaccinales}

Aucun vaccin prophylactique n’est disponible contre l’infection par le VHC depuis la découverte du virus dans les années 1990. La conception d’un vaccin a été fortement contrainte par l’absence de souche cultivable jusqu’en 2005, qui rendait impossible le recours aux méthodes traditionnelles les plus prometteuses dans l’histoire de la vaccination. Les différentes stratégies vaccinales pour le VHC reposaient alors principalement sur des peptides ou protéines recombinantes, des vecteurs, des particules pseudo-virales ou des vaccins à ADN à différents stades de développement ou phases d’essais cliniques. Depuis l’émergence des premiers systèmes de culture, les vaccins basés sur des virus vivant atténué ou inactivé sont désormais envisageables pour l’hépatite C et ouvrent de nouvelles possibilités \citep[pour revues,][]{RN228,RN234,RN277}. \\ \\
\indent
Les barrières au développement d’un vaccin préventif pour l’hépatite C sont la grande variabilité génétique du VHC impliquant la mise en place d’une protection croisée entre les génotypes et la méconnaissance des corrélats de protection. L’immunité protectrice résiduelle chez les patients suite à la guérison par les thérapies antivirales est insuffisante et ils restent susceptibles aux réinfections et à la persistence virale \citep{RN276}. Toutefois, des études cliniques montrent qu’il y a une diminution de 80\% du risque de développer une chronicité dans le cadre des réinfections récurrentes, signifiant qu’une mémoire immunitaire partielle subsiste bien que non stérilisante. Ces données renforcent l’idée qu’une vaccination contre l’hépatite C serait réalisable, au minimum pour empêcher la pathologie chronique, en concentrant les efforts de recherche sur l’identification des corrélats de protection. Il a été mis en évidence que la clairance de l’infection virale serait liée à l’action concertée d’une expansion rapide et continue en lymphocytes T CD4+ et CD8+ effecteurs et mémoires intrahépatiques \citep{RN236} et à la production précoce d’anticorps neutralisants à large spectre \citep{RN87}. Dans ce contexte, le défi des futures stratégies vaccinales est de réussir à combiner l’induction d’une réponse humorale neutralisante ciblant les protéines structurales du virus et d’une réponse cellulaire T CD4+ et CD8+ dirigée contre les régions plus conservées du génome viral, comme les protéines non structurales NS3, NS4, NS5A et NS5B. Le choix d’un adjuvant approprié pour maintenir les réponses cellulaires T et B sera également crucial pour la protection. À ce jour, seuls deux candidats vaccins ont été testés lors d’essais cliniques de phase I. Après validation de leur innocuité et pouvoir immunogène, seul le vecteur poxviral atténué codant les protéines NS3 à NS5B du VHC (MVA-NSmut) \citep{RN241} conçu pour générer une immunité cellulaire, a atteint la phase II mais les résultats publiés récemment ont été décevants. Enfin, des progrès ont été faits pour développer un modèle animal permissif à l’infection et immuno-compétent afin de valider les stratégies vaccinales en phases pré-cliniques et limiter l’emploi des chimpanzés, jusqu’à présent utilisés pour tous les tests de vaccins (voir section \nameref{section:animaux}). Les connaissances accumulées sur ce sujet ces 20 dernières années laissent espérer que la mise sur le marché d’un vaccin au moins partiellement protecteur serait réalisable d’ici les 5 à 10 prochaines années. En diminuant le risque de persistance, ce vaccin sera déjà un grand pas pour réduire la charge mondiale des maladies hépatiques et la transmission du VHC \citep[pour revues,][]{RN212,RN244,RN245}.

\medskip
	\begin{figureth}
	\centering
			\includegraphics[width=0.90\linewidth]{Figure_17.png}
		\caption[Historique des thérapies antivirales sur les 30 dernières années depuis la découverte de l'agent étiologique et défis futurs pour éradiquer le virus de l'hépatite C]{\textbf{Historique des thérapies antivirales sur les 30 dernières années depuis la découverte de l'agent étiologique et défis futurs pour éradiquer le virus de l'hépatite C.} Figure composée à partir des données de la littérature.}
				\label{fig:fig17}
	\end{figureth}
		\FloatBarrier

\clearpage
%%%%%%%%%%%%%%%%%%%%%%%%%%%%%%%%%%%%%%%%%%%%%%%%%%%%%%%%%%%%%%%%%%%%%%%%%%%%%%%%%%%%%%%%%%%%%%%%%%%%%%%%%%%%%%%%%%%%%%%%%%%%%%%%%%%%%%%%%%%%%%%%%%%%%%%%%%%%%%%%%%%%

% 5 - Organisation moléculaire et structurale du virus de l'hépatite C

\section{Organisation moléculaire et structurale du virus de l'hépatite C}	
				\label{section:organisation}
				
	\subsection{Structure des particules virales}
						\label{section:particule}

La particule virale est constituée d’une copie du génome viral incorporée au sein d’une capside non-icosaédrique et d’une enveloppe lipoprotéique recouverte de lipides neutres (esters de cholestérol et triglycérides), des glycoprotéines virales et des apolipoprotéines cellulaires B (ApoB), C (ApoC) et E (ApoE) (\autoref{fig:fig18}). Contrairement aux Flavivirus qui codent une unique glycoprotéine E d’environ 50kDa à l’origine d’une enveloppe virale uniforme, le VHC possède deux glycoprotéines distinctes E1 et E2 \citep[pour revue,][]{RN279}. Ces protéines d’une longueur de 192 et 363 acides aminés sont constituées d’un ectodomaine N-terminal et d’un court domaine C-terminal transmembranaire d’environ 30 acides aminés et possèdent de nombreux sites de N-glycosylations. En effet, les glycoprotéines E1 et E2 possèdent respectivement 5 et 11 sites de N-glycosylations qui, en plus d’assurer le repliement correct des protéines, ont un rôle dans la protection contre la neutralisation \citep{RN396}. À la surface des particules virales, E1 et E2 s’associent sous la forme d’hétérodimère par un processus lent de réarrangement de leur domaine transmembranaire respectif. Puis, les hétérodimères E1-E2 adoptent une conformation complexe structurée en trimères d’hétérodimères stabilisées par des ponts disulfures, puis en 12 pentamères de trimères, selon de récentes prédictions bioinformatiques \citep{RN397,RN398}. À la surface des virions, la glycoprotéine E2 exposent des régions hypervariables (HVR1, HVR2 et IgVR) qui diffèrent jusqu’à 80\% entre les génotypes \citep{RN399}. Ces régions contiennent des épitopes très immunostimulateurs qui fonctionnent comme « leurre » immunologique et protègent les régions les plus conservées. L’action concertée des glycoprotéines E1 et E2 assure l’entrée virale dans la cellule-hôte par la reconnaissance et la fixation aux récepteurs cibles et la fusion avec la membrane des endosomes décrite dans la \autoref{section:tropisme} \citep{RN400}. L’association de l’enveloppe à des composés lipidiques confère à la particule dite «  lipovirale » une faible densité, de l’ordre de 1,06 à 1,10g/mL, comparable à celle des lipoprotéines de basse densité (LDL) ou de très basse densité (VLDL) \citep{RN283,RN284,RN285}. Les premières observations de l'ultrastructure des particules purifiées à partir de sérum de patients ou issues de cultures cellulaires infectées, par cryo-microscopie, tomographie et microscopie électronique à transmission (MET), ont mis en évidence le caractère sphérique des virions et l’hétérogénéité de la taille des particules allant de 40 à 100nm selon leur contenu lipidique \citep{RN281,RN282,RN125}. L’aspect pléomorphique des particules lipovirales, qui imitent la morphologie des lipoprotéines cellulaires, contribuerait à réduire l’immunogénicité des virions circulant dans le sang. De plus, les apolipoprotéines présentes à la surface des virions pourraient protéger de la neutralisation en masquant les épitopes viraux, ce qui diminue l’accessibilité des particules virales aux anticorps anti-E2. Contrairement aux Flavivirus apparentés qui ont une forme icosaédrique et une taille constante d’environ 50nm, les particules lipovirales du VHC ne présentent aucune symétrie apparente de leur enveloppe ou de leur capside, faisant d’elles des structures atypiques parmi les virus eucaryotes.

	\begin{figureth}
	\centering
			\includegraphics[width=\linewidth]{Figure_18.png}
		\caption[Organisation structurale de la particule lipovirale du virus de l'hépatite C]{\textbf{Organisation structurale de la particule lipovirale du virus de l'hépatite C.} ((A) Modèle théorique de la particule lipovirale illustrant les similarités de structure et de composition avec les lipoprotéines du sérum. Les glycoprotéines de surface E1 et E2 , les apolipoprotéines apoB, apoC-I, apoC-II et apoE et la nucléocapside composée de Core et de l’ARN viral sont indiquées (Adapté de \citealt{RN345}). (B) Micrographie électronique représentative des particules lipovirales capturées à l’aide d’anticorps anti-E2 partir du sérum de patients infectés par le VHC (D’après \citealt{RN125}).}
				\label{fig:fig18}
	\end{figureth}
		\FloatBarrier

			\subsubsection{Organisation génomique}

Le génome du VHC est un ARN monocaténaire de polarité positive de 9600 nucléotides. Il est constitué d'une unique phase ouverte de lecture (ORF) encadrée par des régions non codantes hautement structurées 5’NC et 3'NC (\autoref{fig:fig19}). Ce génome joue un rôle essentiel dans plusieurs étapes distinctes du cycle viral : (i) en tant qu’ARN messager (ARNm) pour la synthèse des protéines virales (partie détaillée dans le paragraphe \nameref{section:traduction}); (ii) en servant de matrice pour la synthèse du brin de polarité négative lors de la réplication génomique (partie détaillée dans le paragraphe \nameref{section:replication}) et (iii) en tant que copie d’ARN encapsidée dans les particules virales néosynthétisées (partie détaillée dans le paragraphe \nameref{section:assemblage}).

	\begin{figureth}
	\centering
			\includegraphics[width=\linewidth]{Figure_19.png}
		\caption[Organisation du génome du virus de l'hépatite C]{\textbf{Organisation du génome du virus de l'hépatite C.} Le génome du VHC est représenté avec les structures secondaires théoriques des régions 5’NC, 3’NC (en rouge) et de l’ORF (en bleu). Deux copies du micro-ARN miR-122 interagissant avec l’extrémité 5’ sont illustrées. L’emplacement de l’IRES, du codon initiateur AUG, du codon terminateur STOP, de la région variable (VR), de la région riche en polyU/C et du domaine 3’X sont indiqués (Adapté de \citealt{RN1231}).}
				\label{fig:fig19}
	\end{figureth}
		\FloatBarrier

L’ARN viral contient un ensemble de structures secondaires en tiges-boucles et pseudo-noeuds, présents à la fois dans les régions non codantes et dans l’ORF, qui contribuent à la stabilité du génome et aux interactions inter- et intra-moléculaires importantes sur le plan fonctionnel \citep[pour revues,][]{RN402,RN403}. En effet, ces structures sont impliquées dans les interactions ARN/ARN en cis et dans le recrutement des protéines virales et cellulaires lors des processus de traduction et de réplication génomique. La région 5'NC d'une longueur d’environ 340 nucléotides est constituée de 4 domaines structuraux (I à IV) très conservés parmi les génotypes. Les domaines II, III et IV complétés par quelques codons de la séquence codant la protéine Core forment un site d'entrée interne des ribosomes (IRES) de type III, assurant la traduction du précurseur polyprotéique par un mécanisme indépendant de la coiffe et par recrutement direct de la sous-unité ribosomale 40S. La région 3' NC d'une longueur de 200 à 250 nucléotides contient 3 domaines distincts : une courte région variable non essentielle (VR), une séquence enrichie en polyuracile/polycytosine (polyU/C) d'une longueur moyenne de 80 nucléotides et un domaine quasi-invariable comprenant les 98 nucléotides terminaux, appelé 3'X fortement enrichi en structures secondaires. Le rôle de ces éléments structuraux au cours des différentes étapes du cycle viral sera explicité en détail dans la \autoref{section:cycle}. Le génome contient cinq sites de liaison au micro-ARN hépato-spécifique miR-122. Ces interactions, qui contribuent au tropisme hépatique du VHC, stabilise la structure de l’IRES, amplifie la traduction et la réplication génomique et protège l’ARN viral de la dégradation par les nucléases cellulaires et de la reconnaissance par les senseurs de l’immunité innée \citep{RN401,RN289,RN290}. 

		\subsection{Tropisme et restriction d’hôte : entrée du virus de l'hépatite C dans les hépatocytes humains}
					\label{section:tropisme}

Après transmission par la voie parentérale, le VHC circule dans le sang du patient et pénètre dans le foie par l’artère hépatique. Les particules lipovirales traversent l’endothélium des sinusoïdes hépatiques et se retrouvent dans l’espace de Disse où elles sont exposées au pôle basolatéral des hépatocytes. Cet accès peut être facilité par les cellules endothéliales sinusoïdales et par les cellules de Kupffer qui sont réfractaires à l’infection mais qui expriment des lectines de type C, L-SIGN et DC-SIGN, des molécules d’adhésion capables de capturer les particules lipovirales en se liant à la glycoprotéine d’enveloppe E2 \citep{RN404,RN405,RN406}. Après un premier cycle d’infection des hépatocytes par la circulation sanguine, les particules lipovirales néoformées peuvent être transmises par contacts directs entre hépatocytes, via les jonctions serrées. Ce processus permet de protéger les particules lipovirales des anticorps neutralisants présents dans le sérum, ce qui favoriserait l’établissement de la persistance virale \citep{RN407}. \\ \\
\indent
L’entrée du VHC au sein des hépatocytes est un processus hautement coordonné qui s’effectue en 5 étapes distinctes : (i) liaison aux co-récepteurs de surface présents sur la face basolatérale ; (ii) translocation vers les jonctions serrées ; (iii) internalisation des particules virales par endocytose; (iv) fusion de l’enveloppe avec les membranes de l’endosome et (v) désassemblage de la capside et libération du génome dans le cytoplasme (\autoref{fig:fig20}). De nombreux facteurs de l’hôte sont engagés transitoirement au cours de l’attachement et de l’internalisation des particules du VHC, dont certains contribuent au tropisme hépatique et à la restriction d’hôte du VHC pour l’Homme. Ceci a des implications pour le développement de modèles animaux susceptibles à l’infection pour l’étude du VHC \textit{in vivo} (voir \autoref{section:assemblage}). La plupart des facteurs d’hôtes et des mécanismes décrits dans ce chapitre ont été identifiés dans le contexte de lignées d’hépatomes cellulaires, qui n’imitent pas l’architecture des hépatocytes polarisés. Toutefois, le recours à des systèmes qui reproduisent plus fidèlement l’environnement hépatique \textit{in vivo} a permis de commencer récemment à valider le modèle d’entrée du VHC proposé à partir des découvertes menées sur cellules en lignées \citep[pour revues,][]{RN408,RN409}.



			\subsubsection{Attachement de la particule lipovirale à la surface des hépatocytes}

L’attachement initial des particules du VHC aux hépatocytes imite la stratégie d’entrée des lipoprotéines du sérum. Les apolipoprotéines associées aux particules lipovirales en coopération avec la glycoprotéine d’enveloppe E2, sont responsables de la reconnaissance et de la liaison aux chaînes glycosaminoglycanes de l’héparane sulfate (HS) \citep{RN410}, au récepteur des lipoprotéines de basse densité (LDL-R) \citep{RN411,RN412} et au récepteur scavenger de classe B, type I (SR-BI) \citep{RN413,RN414} présents sur la face basolatérale des hépatocytes. Récemment, le récepteur de la phosphatidylsérine TIM-1, servant de facteur d’entrée pour divers virus de la famille des \textit{Flaviviridae}, a été identifié comme un nouveau facteur d’hôte contribuant à l'attachement du VHC \citep{RN415,RN416}. L’interaction de la particule avec SR-BI active sa fonction de transfert lipidique et entraîner consécutivement la dissociation des lipoprotéines associées aux particules virales. Ce processus induit un changement de conformation de la glycoprotéine E2 et notamment, une exposition des sites de liaison au récepteur CD81, précédemment protégé des senseurs de l’immunité par la région hyper variable 1 (HVR1) \citep{RN417}. L’engagement de E2 avec CD81 active séquentiellement les voies de signalisation du récepteur du facteur de croissance épidermique (EGFR) \citep{RN418} et des GTPases appartenant à la superfamille RAS \citep{RN419,RN420}, ce qui déclenche la diffusion latérale des complexes CD81 associés aux particules virales vers les jonctions serrées, par réarrangement du réseau d’actine. Cette relocalisation permet à ces complexes d’interagir avec les protéines constitutives des jonctions, la claudine 1 (CLDN1) \citep{RN421} et l’occludine (OLCN) \citep{RN422}. La liaison de CD81 avec la CLDN1 est une condition préalable à l’internalisation du virion, et serait responsable de l’endocytose des particules virales \citep{RN423,RN424,RN425}. Le rôle précis de l’OLCN n’est pas connu, mais serait essentiel pour l’internalisation du virion dans les hépatocytes et contribuerait au tropisme d’espèce du VHC \citep{RN426}. En effet, l’expression des protéines homologues humaines CD81 et OCLN constitue le critère minimal pour rendre les cellules de souris ou de hamster permissives à l’entrée virale \citep{RN380}. Les facteurs cellulaires SR-BI et CLDN1 sont fortement exprimés dans les hépatocytes et contribueraient à définir le tropisme tissulaire du VHC au niveau de l’entrée \citep{RN428}.

			\subsubsection{Internalisation de la particule lipovirale et fusion}

Les particules du VHC sont internalisées par endocytose médiée par la clathrine \citep{RN429} et transportées au sein de vésicules endosomales précoces le long du réseau microtubulaire. Parallèlement, l’acidification progressive de la lumière des endosomes va détacher la particule virale de ses récepteurs et déclencher la fusion de l’enveloppe virale avec la membrane endosomale limitante \citep{RN430}. À ce jour, la résolution de la structure de l’ectodomaine de E2 est quasi-complète, tandis que la structure tridimensionnelle de E1 reste partielle, où seuls les résidus 1 à 79 du domaine N-terminal ont pu être déterminés par cristallographie \citep{RN431}. L’ectodomaine de E2 adopte un schéma compact semblable à celui d’une immunoglobuline, excluant la possibilité que E2 assure l’activité fusogène du VHC. À l’inverse, de récentes études démontrent qu’une séquence hydrophobe conservée dans E1 serait impliquée dans le processus de fusion membranaire et a ainsi été proposée comme peptide de fusion (pFP) \citep{RN432}. La résolution partielle de la structure de E1 représente toutefois un modèle incohérent avec les repliements très structurés des peptides de fusion de classe II et III. D’autre part, la taille de E1 est également estimée trop petite pour relier les membranes cellulaires et virales après l’insertion du peptide de fusion. Par conséquent, la protéine assurant l’activité fusogène du VHC est encore indéterminée aujourd’hui, et le VHC pourrait posséder une nouvelle classe de peptide de fusion membranaire, différente des protéines de fusion de classe II répandues chez les Flavivirus \citep{RN433}. Certains facteurs d’hôtes essentiels à l’étape de fusion seraient sélectivement internalisés au sein des endosomes contenant les particules virales, comme la protéine de transport intracellulaire du cholestérol (NPC1L1) \citep{RN434} et la protéine 1 de liaison du facteur de réponse au sérum (SRFBP1) \citep{RN435}. Bien que NPC1L1 soit localisée au pôle apical des hépatocytes, son internalisation au sein des compartiments endosomaux aiderait à fournir une composition lipidique optimale pour promouvoir la fusion. SRFBP1 aurait, quant à elle, un rôle important dans le transport rétrograde des endosomes contenant les particules virales. Après la fusion, la nucléocapside est rapidement désassemblée ce qui libère le génome viral dans le cytoplasme initie les différentes étapes du cycle viral, décrites dans la \autoref{section:cycle}.

	\begin{figureth}
	\centering
			\includegraphics[width=\linewidth]{Figure_20.png}
		\caption[Modèle d’infection des hépatocytes par le VHC à partir de la circulation sanguine]{\textbf{Modèle d’infection des hépatocytes par le VHC à partir de la circulation sanguine.} Cette illustration résume les facteurs d’hôtes et la séquence des événements qui conduisent de la fixation initiale des particules lipovirales sur la face basolatérale des hépatocytes, à l’internalisation et à la libération du génome viral dans le cytosol. La première étape de fixation implique principalement la composante lipoprotéique de la particule lipovirale au niveau des récepteurs LDLR et SR-BI. L’exposition successive de l’enveloppe virale permet à la glycoprotéine E2 d’interagir spécifiquement avec SR-BI, CD81 puis avec CLDN1 après diffusion latérale vers les jonctions serrées. La particule virale est internalisée avec ces co-récepteurs par endocytose médiée par la clathrine (CAPN5, CBLB) et dépendante du réseau microtubulaire (Dynamin II). Les vésicules endosomales s’acidifient progressivement et la baisse du pH favorise la fusion de l’enveloppe virale avec la membrane de l’endosome et le désassemblage de la capside, libérant le génome viral dans le cytosol (Adapté de \citealt{RN409}).}
				\label{fig:fig20}
	\end{figureth}
		\FloatBarrier

\clearpage

%%%%%%%%%%%%%%%%%%%%%%%%%%%%%%%%%%%%%%%%%%%%%%%%%%%%%%%%%%%%%%%%%%%%%%%%%%%%%%%%%%%%%%%%%%%%%%%%%%%%%%%%%%%%%%%%%%%%%%%%%%%%%%%%%%%%%%%%%%%%%%%%%%%%%%%%%%%%%%%%%%%%

% 6 - Cycle infectieux du virus de l'hépatite C

			\section{Cycle infectieux du virus de l'hépatite C}
			\label{section:cycle}			

Comme tous les virus à ARN positif, le cycle infectieux du VHC est entièrement cytoplasmique. Le cycle viral est coordonné en différentes étapes par l’action séquentielle de protéines virales et par le détournement des machineries cellulaires, en particulier le métabolisme lipidique. Au cours de ce chapitre, l’ensemble des étapes post-entrée du cycle viral seront détaillées sous la forme de quatres sections : (i) la traduction du génome viral infectant, (ii) la biogénèse des usines de réplication virale, (ii) la réplication du génome viral et (iv) l’assemblage et la sécrétion des particules virales dans le milieu extra-cellulaire.

			\subsection{Traduction du génome viral et clivage du précurseur polyprotéique}
			\label{section:traduction}
			
Après désassemblage de la nucléocapside, le génome viral libéré dans le cytoplasme sert directement d’ARN messager (ARNm) pour produire une polyprotéine d’environ 3000 acides aminés (\autoref{fig:fig21}). L’IRES du VHC ne nécessite que les facteurs d’initiation eIF2, eIF3 et l’ARN de transfert (ARNt) pour assembler le ribosome \citep{RN436}. L’initiation de la traduction commence par un changement de conformation de l’IRES, différente de celle en solution, permettant d’interagir avec la sous-unité ribosomale 40S \citep{RN437}. L’IRES recrute successivement le complexe de pré-initiation, formé par eIF3 et par le complexe ternaire eIF2-ARNt associé à une molécule de guanosine triphosphate (GTP) au niveau du codon AUG initiateur. Le facteur miR-122 promeut le recrutement des facteurs d’initiation eIF2 et eIF3. Lors de l’hydrolyse de la GTP, les facteurs d’initiation eIF2 et eIF3 se dissocient de l’IRES et sont remplacés par la sous-unité ribosomale 60S pour assembler le ribosome fonctionnel \citep{RN438}. L’ARNt initiateur est ensuite placé au site P du ribosome, ce qui marque la transition entre l’initiation de la traduction et l’élongation. Des études par cryo-microscopie à l’échelle quasi-atomique prédisent que la partie apicale du domaine II et le domaine III représenteraient les structures les plus mobiles de l’IRES tandis que la partie basale du domaine et le pseudo-noeud III/IV forment une structure beaucoup plus rigide \citep{RN439}. \\ \\
\indent
La polyprotéine virale est adressée à la membrane du réticulum endoplasmique (RE), par quatre peptides signaux présents au sein des séquences codant les protéines structurales. La polyprotéine virale est clivée par l’action séquentielle de protéases cellulaires et virales pendant ou immédiatement après la traduction : (i) le domaine C-terminal de NS2 a une activité protéasique à cystéine qui catalyse en premier le clivage à la jonction NS2/NS3; (ii) la libération de la protéine NS3, associée en position C-terminale à son co-facteur NS4A, amorce son activité protéasique à sérine, ce qui catalyse en deuxième le clivage à la jonction NS5A/NS5B (iii), puis à la jonction NS4B/NS5A et (iv) une peptidase signal (SP) cellulaire catalyse en quatrième les clivages complets aux jonctions Core/E1, E1/E2 et partiels aux jonctions E2/p7 et p7/NS2 \citep{RN440,RN441}. D’après des modèles prédictifs, des déterminants structuraux situés en aval des sites de clivage de p7 et NS2 imposeraient des contraintes structurelles et diminueraient l’efficacité du clivage par la SP, conduisant fréquemment à la production de précurseurs E2-p7-NS2 \citep{RN442}. La protéine Core, qui est alors retenue par un peptide signal, sera clivée par la peptidase du peptide signal (SPP) cellulaire, permettant la migration de sa forme mature vers les GL cytosoliques \citep{RN303}. Les autres protéines virales matures restent associées à la membrane du RE par le biais de domaines transmembranaires (p7, NS2, NS4B), d’une unique hélice transmembranaire (E1, E2, NS4A, NS5B) ou d’une hélice alpha monotopique (NS5A) et sont essentiellement localisées dans la lumière du RE (pour les glycoprotéines d’enveloppe E1, E2) ou orientées vers la face cytosolique (NS2, NS3-4A, NS5A et NS5B).

	\begin{figureth}
	\centering
			\includegraphics[width=\linewidth]{Figure_21.png}
		\caption[Traduction de l’ARN viral, clivage du précurseur polyprotéique et topologie membranaire des protéines virales]{\textbf{Traduction de l’ARN viral, clivage du précurseur polyprotéique et topologie membranaire des protéines virales.} L’ORF codant la polyprotéine du VHC et les structures secondaires théoriques des régions 5’NC et 3’NC sont illustrées en haut. Le clivage co- et post-traductionnel de la polyprotéine virale est indiqué comme suit : peptidase signal (flèches verticales en pointillés), peptidase du peptide signal (étoile), protéase NS2 (flèche courbe en pointillés), protéase NS3-4A (flèches courbes solides). La topologie membranaire des protéines virales matures et leurs fonctions connues sont indiquées en bas. La protéine NS5A est représentée sous la forme d’un dimère reconnu, mais les autres protéines virales peuvent également former des homo- et hétérodimères ou des complexes oligomériques. (Adapté de \citealt{RN484}).}
				\label{fig:fig21}
	\end{figureth}
		\FloatBarrier
		
Le VHC code 10 protéines qui assurent l’ensemble des processus nécessaires au cycle infectieux complet du virus, dénotant un rôle complexe et multi-fonctionnel des protéines virales. La diversité des fonctions de chaque protéine peut être favorisée par des changements fréquents de conformation ou de modifications du microenvironnement cellulaire impliquant un réseau spécifique d’interactions avec des partenaires viraux et cellulaires. Les protéines structurales du VHC qui composent la particule lipovirale sont la protéine de capside Core et les glycoprotéines d’enveloppe E1 et E2. Les protéines p7 et NS2 participent à l’assemblage mais ne sont pas incorporées au sein les particules virales. Les protéines NS3, NS4A, NS4B, NS5A et NS5B sont suffisantes pour assurer la réplication génomique \citep{RN294}. Dans ce contexte, il est fréquemment établi que le génome viral code deux modules protéiques fonctionnels : un module d’assemblage constitué des protéines core à NS2 et un module réplicatif constitué des protéines NS3 à NS5B. Toutefois, il est maintenant admis que l’ensemble des protéines non structurales contribue à la formation des virions, bien que les mécanismes précis ne soient pas encore résolus \citep[pour revue,][]{RN293}. L’implication de ces protéines virales au cours du cycle infectieux sera détaillée ci-après. Les caractéristiques structurales et physiopathologiques de la protéine Core seront approfondies dans la \autoref{section:core}.



			\subsection{Biogénèse des usines de réplication virale}
			
Comme tous les virus à ARN de polarité positive, le VHC remodèle considérablement les membranes intracellulaires afin de générer un compartiment spécialisé dans la réplication génomique et l’assemblage des particules virales \citep[pour revue,][]{RN450}. Ce réseau membranaire ou membranous web (MW) permet (i) d’établir un micro-environnement propice à la réplication génomique du VHC en augmentant la concentration locale des facteurs nécessaires à une réplication efficace de l’ARN viral et en protégeant les protéines virales et l’ARN des senseurs de l’immunité innée et (ii) de coordonner spatialement les différentes étapes du cycle viral. \\ \\
\indent
Ce réseau est formé par une accumulation de vésicules de taille et de morphologie hétérogène dans le cytoplasme, naissant par exvaginations des membranes du RE rugueux (Fig. 11), dont les structures principales sont des vésicules à double membrane (DMV) d’environ 150nm \citep{RN451}. Les protéines virales NS3 à NS5B constituant le module réplicatif, ainsi que l’ARN double brin issu d’une activité réplicase \textit{in vitro} ont été retrouvés associés aux DMV, suggérant que ces structures constituent les sites de réplication génomique du VHC \citep{RN452}. Le site de la synthèse \textit{de novo} de l’ARN viral n’a pas encore été précisément localisé, mais des études biochimiques soulignent que le complexe de réplication réside dans un environnement protégé des nucléases et des protéases endogènes, ce qui appuie l’hypothèse selon laquelle la réplication génomique se produit dans la « lumière » de la vésicule, et non sur la membrane externe \citep{RN453,RN454}. La plupart des DMV sont des structures fermées et seule une proportion d’environ 10\% possède une ouverture vers le cytosol, sous la forme de pores \citep{RN451}. Ces ouvertures pourraient être impliquées dans l’échange de métabolites et de facteurs nécessaires à la réplication, ainsi que la sortie des génomes nouvellement synthétisés. Ces observations impliqueraient que seule une sous-population minoritaire de DMV supporte une réplication active tant qu’elles sont connectées au cytosol et que la réplication pourrait cesser au moment de la fermeture des pores de la membrane. Toutefois, il a été récemment montré que des facteurs impliqués dans le transport membranaire comme les protéines du complexe du pore nucléaire (NUP) sont délocalisés dans les régions du MW et pourraient permettre le traffic des molécules à travers un compartiment membranaire fermé \citep{RN455,RN456}. Au sein du MW, on retrouve plus rarement des vésicules à membrane unique (SMV) définies comme des structures préalables aux DMV \citep{RN457} ou des vésicules multi-membranaires (MMV) qui apparaissent principalement lors des stades tardifs de l’infection et résulteraient de la réponse cellulaire au stress induit par une réplication virale élevée \citep{RN458}. Les structures formant « l’organelle » de réplication génomique du VHC sont morphologiquement similaires à celles induites par les \textit{Coronaviridae} \citep{RN459} et les \textit{Picornaviridae} \citep{RN460}, mais relativement distinctes de celles induites par les autres membres de la famille des \textit{Flaviviridae} tels que le virus de la dengue (DENV), le virus du nil occidental (WNV) ou le virus de l’encéphalite à tique (TBEV) \citep[pour revue,][]{RN461}, qui sont des invaginations à membrane unique dans la lumière du RE. Récemment, une étude a mis en évidence que le MW détecté dans le tissu hépatique serait essentiellement constitué de SMV, contrairement aux observations réalisées \textit{in vitro} \citep{RN462}. Les auteurs proposent que les DMV et les MMV seraient une caractéristique propre aux souches du VHC adaptées en laboratoire ou aux mécanismes de la réponse d’immunité innée des lignées cellulaires d’hépatome, soulignant l’importance d’étudier le cycle viral dans des modèles plus physiologiques. \\ \\
\indent
La formation d’un MW intègre et fonctionnel résulte de l’action concertée des protéines virales NS4B et NS5A et de facteurs cellulaires. Premièrement, cela implique de remodeler les membranes intracellulaires existantes afin de compartimenter le complexe de réplication. La protéine NS4B est une protéine hautement hydrophobe avec une topologie membranaire complexe, comprenant quatre hélices transmembranaires et quatre hélices amphipatiques dans les domaines N et C-terminaux (\autoref{fig:fig22}) \citep{RN463}. La formation des vésicules serait facilitée par une courbure positive de la membrane externe du RE, induite par l’insertion asymétrique des hélices amphipatiques de NS4B dans le feuillet lipidique \citep{RN464}. De plus, la multimérisation de NS4B serait importante pour amplifier le processus de remodelage membranaire \citep{RN465}. Deuxièmement, la biogenèse des « organelles » de réplication nécessite une synthèse \textit{de novo} importante de lipides membranaires, en particulier de cholestérol et de sphingolipides, qui stimulent l’activité de la réplicase en formant des radeaux lipidiques \citep{RN466,RN467}. L’infection par le VHC déclenche le clivage protéolytique des protéines de liaison à l’élément de régulation des stérols (SREBP), une famille de facteurs de transcription qui activent l’expression de gènes impliqués dans la lipogénèse, tels que l’acide gras synthase (FAS) et l’hydroxyméthylglutaryl-CoA réductase (HMG-CoA), l’enzyme catalysant la biosynthèse du cholestérol \citep{RN468}. De plus, la protéine NS5A recrute la kinase PI4KIIIa pour produire une accumulation locale de phosphatidylinositol 4-phosphate (PI4P) au niveau des DMV \citep{RN469,RN470}. Le transport d’autres espèces lipidiques, comme le cholestérol et les sphingolipides, pourrait être médié par des protéines de transfert lipidique ciblant les membranes enrichies en PI4P. Par exemple, NS5A recrute la protéine de liaison à l’oxystérol 1 (OSBP), qui assure le transport du cholestérol en échange du PI4P \citep{RN471}. En parallèle de son rôle central dans le recrutement des facteurs d’hôte, la protéine NS5A modifie les propriétés membranaires, par le biais de son hélice α amphipatique contenue à son extrémité N-terminale, ce qui est essentiel dans la biogenèse du compartiment de réplication \citep{RN472}.

	\begin{figureth}
	\centering
			\includegraphics[width=\linewidth]{Figure_22.png}
		\caption[Organisation et structure tridimensionnelle des usines de réplication génomique du virus de l’hépatite C]{\textbf{Organisation et structure tridimensionnelle des usines de réplication génomique du virus de l’hépatite C.} Représentation schématique du réarrangement membranaire du réticulum endoplasmique pour former les vésicules à double membranes (DMV) ou les vésicules à multiples membranes (MMV) induite par le VHC. À droite, reconstruction tridimensionnelle du réarrangement des membranes observé par tomographie électronique dans des cellules infectées  par le VHC. Les membranes externe et interne des DMV sont respectivement colorisées en brun clair et en orange. Le RE, les filaments du cytosquelette, l’appareil de Golgi et les vésicules à membrane unique sont respectivement coloriées en brun foncé, en bleu, en vert et en violet. En bas à droite, représentation schématique du compartiment de réplication du VHC au sein d’une DMV, qui contient les protéines non structurales NS3-4A, NS4B, NS5A et NS5B responsables de la réplication génomique  (Adapté de \citealt{RN450,RN346}).}
				\label{fig:fig22}
	\end{figureth}
		\FloatBarrier
	

			\subsection{Réplication du génome viral}
			\label{section:replication}

Au sein des « organelles » de réplication, le génome viral est recruté et copié en un intermédiaire de polarité négative, qui servira de matrice pour la production \textit{de novo} de nouvelles molécules d’ARN de polarité positive. La réplication du génome viral est médiée la protéine virale NS5B qui héberge une activité ARN polymérase ARN-dépendante (RdRp) \citep{RN473,RN474}. Comme toutes les RdRp virales, NS5B a une forme de « main droite » englobant les domaines de la paume, du pouce et des doigts et un domaine catalytique comprenant un motif GDD \citep{RN475}. La synthèse d’ARN \textit{de novo} est initiée à partir de l’extrémité 3’ du génome viral par un mécanisme indépendant d’amorce \citep{RN476,RN477}. Outre la polymérase NS5B, la protéine NS3 a une activité d’hélicase essentielle pour assurer une réplication correcte de l’ARN, mais son rôle n’est pas complètement défini. Cette fonction pourrait être requise pour dissocier les éléments d’ARN monocaténaire hautement structurés ou pour dissocier les ARN double brins \citep{RN478}. En parallèle, la protéine NS5A est essentielle à la réplication génomique en stimulant l’activité de la polymérase, par liaison à l’ARN viral ou par interaction directe avec NS5B \citep{RN525} et par le recrutement de facteurs d’hôte : la cyclophiline A (CypA) qui catalyse l’isomérisation de certains domaines de NS5A et NS5B nécessaire au fonctionnement du complexe de réplication \citep{RN522,RN523} et la protéine A associée à une vésicule (VAPA) qui facilite le positionnement correct de la polymérase au sein des rafts lipidiques \citep{RN524}.\\ \\
\indent
Les structures en tiges-boucles présentes tant dans le brin positif que dans le brin négatif sont impliqués dans l’initiation ou la régulation de la réplication (\autoref{fig:fig23}). Dans le génome viral, les domaines I et II contiennent des signaux essentiels pour la réplication génomique bien que l'intégralité de la région soit requise pour assurer une synthèse correcte de l'ARN \citep{RN286,RN287}. Le domaine terminal 3'X ainsi qu'un élément minimal de 25 nucléotides situé dans la région polyU/C sont essentiels pour la réplication génomique en culture cellulaire et \textit{in vivo} \citep{RN249,RN292}. En plus des éléments structuraux présents dans les régions NC, les interactions de longue portée entre la tige-boucle 5BSL3.2 présente dans la région codante de NS5B et les extrémités 5’NC ou 3’X permettraient de circulariser le génome, une conformation plus propice à la réplication \citep{RN480,RN481,RN482}. L'ARN intermédiaire de polarité négative forme des structures secondaires différentes de celles trouvées dans le génome viral (\autoref{fig:fig23}). Son extrémité 3’, incluant les élément minimaux I’ et II’z, servent d’initiateur très efficace de la synthèse d’ARN \citep{RN483}. En revanche, l’extrémité 3’ de l’ARN de polarité positive est dissimulée au sein d’une structure complexe, réduisant l’efficacité de l’initiation \textit{de novo}. Cette différence joue probablement un rôle dans la régulation des processus de réplication et pourrait contribuer à l’excès d’ARN positifs par rapport aux ARN négatifs \citep[pour revue,][]{RN484}.

	\begin{figureth}
	\centering
			\includegraphics[width=\linewidth]{Figure_23.png}
		\caption[Structures secondaires au sein du génome du virus de l’hépatite C (ARN de polarité positive) et de l’intermédiaire de réplication (ARN de polarité négative)]{\textbf{Structures secondaires au sein du génome du virus de l’hépatite C (ARN de polarité positive) et de l’intermédiaire de réplication (ARN de polarité négative).} L’organisation du génome viral, avec l’ORF codant la polyprotéine encadré des structures secondaires théoriques des régions 5’NC et 3’NC, est illustré en haut de la figure. En dessous sont représentés les prédictions des structures secondaires présentes dans trois régions au sein de l’ARN de polarité positive. Les interactions ARN/ARN de longue portée sont indiquées avec des flèches en pointillés et les sites de liaison au miR-122 sont indiqués par des rectangles gris. Les structures secondaires prédites dans l’ARN intermédiaire de polarité négative sont illustrées en bas (Adapté de \citealt{RN484}).}
				\label{fig:fig23}
	\end{figureth}
		\FloatBarrier
		


			\subsection{Assemblage et sécrétion des particules virales}
			\label{section:assemblage}
			
La morphogénèse et la sécrétion des particules virales constituent les étapes les moins comprises du cycle viral encore aujourd’hui. Elles englobent un ensemble de processus, à commencer par l’assemblage de la nucléocapside, l’acquisition de l’enveloppe virale par bourgeonnement dans la lumière du RE, l’acquisition des lipoprotéines intracellulaires pour former une particule « lipovirale » mature et la sortie définitive des virions dans le milieu extracellulaire. Ces étapes complexes reposent sur une action coordonnée entre les protéines virales structurales, non structurales et un nombre important de facteurs d’hôte, dont les détails mécanistiques sont encore flous.

\subsubsection{Formation de la nucléocapside virale}

Une première condition obligatoire à la réalisation de ces processus implique un rapprochement intracellulaire des éléments formant la structure des particules virales : la protéine Core, le génome viral et les glycoprotéines E1 et E2 (\autoref{fig:fig24}\textcolor{blue}{.A}). Après la synthèse et la maturation des protéines virales par clivage post-traductionnel, la protéine Core est recrutée à la surface des GL cytosoliques \citep{RN485} tandis que les protéines E1 et E2 sont retenues au sein des membranes du RE sous la forme d’un hétérodimère non-covalent \citep{RN486}. L’association de Core aux GL est cruciale pour la morphogénèse virale puisque des mutations visant à empêcher le trafic de Core vers ces organites abolissent l’assemblage des particules virales \citep{RN487,RN488,RN489}. La séparation de la protéine Core et du complexe de réplication dans des compartiment distincts serait importante pour le contrôle spatio-temporel de la réplication et de l’assemblage du virus, afin d’éviter toute compétition au niveau de la liaison à l’ARN \citep{RN346}. La protéine Core doit être toutefois re-mobilisée à proximité des molécules d’ARN virale néosynthétisées pour initier l’assemblage de la nucléocapside. Une étude récente d’imagerie à haute résolution a mis en évidence l’existence d’une sous-population de GL enveloppée par le MW et directement connectée à des DMV, au sein des cellules infectées par la souche Jc1 du VHC (\autoref{fig:fig24}\textcolor{blue}{.B}) \citep{RN490}. Ce groupe a constaté que les DMV peuvent être générés directement à partir des membranes du RE entourant les GL, qui contiennent à la fois les protéines du complexe de réplication et les glycoprotéines d’enveloppe. \\ \\
\indent
Ainsi, deux hypothèses principales émergent de ces observations concernant le point de départ putatif de l’assemblage viral : il aurait lieu soit (i) à la surface des GL soit (ii) au niveau des membranes du RE juxtaposées aux GL. L’interaction de la protéine NS5A avec Core est une étape préalable cruciale à l’assemblage des particules virales \citep{RN491}. Sa mobilité entre les membranes cellulaires et sa capacité de fixation à l’ARN laissent supposer que NS5A est responsable du contact entre la protéine Core et le génome viral \citep{RN494}. Dans le premier modèle, le génome viral serait acheminé par NS5A directement au contact de Core, à la surface des GL. En effet, il a été montré que NS5A co-localise transitoirement avec Core à la surface des GL et que l’inhibition de cette interaction abolit l’assemblage viral \citep{RN295,RN493}. Dans le deuxième modèle, la protéine Core serait recrutée dans la membrane du RE avoisinante par une action coordonnée entre les protéines virales NS5A, NS2 et p7 \citep{RN495}. Il a également été montré que NS2 interagit avec l’hétérodimère E1/E2, et serait responsable de la migration des glycoprotéines d’enveloppe vers les sites d’assemblage putatifs \citep{RN496,RN497}. De plus, la protéine p7 est indispensable aux étapes finales de la formation de la nucléocapside, selon un mécanisme encore inconnu \citep{RN498}. Dans ce contexte, l’enveloppement serait une étape directement couplée à l’assemblage de la nucléocapside. Enfin, au sein du complexe de réplication, les protéines NS3-4A, NS4B et NS5B sont également mobilisées lors de la morphogénèse du VHC \citep{RN296,RN297,RN298}. Cependant, il reste à déterminer si ces protéines non structurales jouent un rôle direct ou indirect dans cette étape du cycle viral. Dans ce deuxième modèle, les GL cytosoliques serviraient alors de simples organites de transport, transférant la protéine Core des sites de traduction aux sites d’assemblage putatifs contenus dans les membranes du RE juxtaposées. Toutefois, il a récemment été montré que la mobilisation des lipides à partir des GL cytosoliques favorise la morphogénèse du VHC, traduisant un rôle supplémentaire de ces organites dans la lipidation des particules virales \citep[pour revue,][]{RN500}. En effet, certaines molécules lipidiques stockées au sein des GL cytosoliques, comme le cholestérol et les sphingolipides, sont essentielles pour l’infectiosité des particules virales \citep{RN501}. Le transfert des lipides aux particules virales s’effectuerait par hydrolyse des GL cytosoliques par l'adipose triglycéride lipase (ATGL) sous le contrôle de la protéine 5 contenant un domaine α/β hydrolase (ABHD5) \citep{RN499}. Jusqu’à présent, il n’a pas été possible de privilégier l’un de ces deux modèles et de visualiser l’emplacement précis des sites d’assemblage du VHC, en raison de la nature transitoire de ces évènements. De plus, la discrimination entre la synthèse des particules lipovirales et des lipoprotéines est difficile compte tenu des similarités entre ces deux nano-structures.

	\begin{figureth}
	\centering
			\includegraphics[width=\linewidth]{Figure_24.png}
		\caption[Morphogénèse et sites putatifs d’assemblage du virus de l’hépatite C]{\textbf{Morphogénèse et sites putatifs d’assemblage du virus de l’hépatite C.} (A) Représentation schématique de l’assemblage des particules du VHC. La protéine Core est recrutée par l’action coordonnée des protéines virales p7, NS2 et NS5A aux membranes du RE. L’ARN viral néosynthétisé est déplacé hors des complexes de réplication vers les sites putatifs d’assemblage. Les nucléocapsides s'assemblent et bourgeonnent dans la lumière du RE où elles s’associent aux glycoprotéines d’enveloppe E1-E2 et aux apolipoprotéines, dont apoE à partir des GL luminales (Adapté de \citealt{RN345}). (B) Micrographies électroniques (en haut) et modélisations 3D (en bas) montrant les sous-populations de GL entourées par des membranes du RE et connectées à des DMV (colorisées en jaunes) au sein de cellules infectées. Les barres d'échelle représentent 100 nm (Adapté de \citealt{RN490}).}
				\label{fig:fig24}
	\end{figureth}
		\FloatBarrier

\subsubsection{Maturation et sécrétion de la particule virale}

La plupart des chercheurs favorisent l’hypothèse selon laquelle les étapes finales du cycle du VHC, c'est-à-dire la maturation et la libération des particules virales, seraient étroitement liées à la voie de biosynthèse des VLDL et à la voie de sécrétion par le réseau trans-golgien \citep{RN502,RN503}. Après le bourgeonnement des nucléocapsides dans la lumière du RE, les particules nouvellement enveloppées s’associeraient à des GL luminales, les précurseurs des lipoprotéines, pour acquérir les différentes apolipoprotéines \citep[pour revue,][]{RN504}. De nombreuses apolipoprotéines peuvent être trouvées à la surface des particules circulantes du VHC, telles que apo-B, l'apo-A-I, l'apoC-I, l'apoC-II et l'apoC-III, mais seule l’incorporation d’apoE serait suffisante pour rendre la particule infectieuse, en accord avec son rôle dans l’entrée virale \citep{RN505,RN506}. Par ailleurs, une étude a récemment montré que l’association à certaines lipoprotéines dépendrait de la souche virale et du type cellulaire \citep{RN507}. L’association des GL luminales avec les particules virales serait directement médiée par l’interaction de la protéine d’enveloppe E2 avec apoE, sous le contrôle de l’annexine 3 (ANXA3), une protéine cellulaire régulant le trafic des voies d’endo et d’exocytose \citep{RN508}. Les particules virales matures transiteraient ensuite vers l’appareil de Golgi. À l’appui de cette hypothèse, les glycoprotéines E1 et E2 subissent des modifications post-traductionnelles de type N-glycosylation avec l’acquisition de sucres complexes \citep{RN509}. Par ailleurs, plusieurs témoignages mettent en évidence la présence de particules virales dans des vésicules COP-II, qui assurent le transport antérograde des protéines cargo du RE vers le Golgi \citep{RN510}. De plus, l'inhibition de la GTPase Rab1b, responsable de la régulation du trafic entre le RE et le Golgi, réduit le taux de libération des particules du VHC \citep{RN511}. Les particules virales transiteraient ensuite au sein du réseau trans-golgien pour rejoindre le compartiment endosomal (\autoref{fig:fig25}\textcolor{blue}{.A}) \citep{RN512}, où elles seraient sécrétées via des vésicules de clathrine \citep{RN513} ou des exosomes bourgeonnant à partir des corps multi-vésiculaires (MVB) \citep{RN514,RN515}. Cette voie sécrétoire dite « canonique » est une voie de sortie employée par de nombreux virus enveloppés, tels que le VHB, le DENV et le WNV \citep[pour revue,][]{RN516}.  \\ \\
\indent
Cependant, des découvertes récentes ont mis en évidence des voies sécrétoires non con-ventionnelles impliquées dans la sortie des particules virales, soulignant la possibilité que le VHC exploite plusieurs voies de sécrétion \citep[pour revues,][]{RN517,RN518}. En effet, les facteurs connus pour jouer un rôle dans la sécrétion du VHC, comme les protéines de la famille Rab, ne sont pas seulement impliquées dans le trafic intracellulaire golgien. Par exemple, les vésicules COP-II peuvent directement fusionner avec la membrane plasmique ou avec les endosomes tardifs qui vont successivement libérer le cargo dans le milieu extracellulaire, coutournant ainsi l’appareil de Golgi (\autoref{fig:fig25}\textcolor{blue}{.B}). Cette voie de sécrétion atypique pourrait être stimulée par le stress membranaire induit au cours de la réplication du VHC. Bien qu'il soit hautement spéculatif, ce modèle est compatible avec la modification incomplète des glycoprotéines virales par les enzymes résidant dans le Golgi et l'absence de co-localisation des particules virales avec les marqueurs du Golgi \citep{RN519}. Pour finir, des études récentes mettent en évidence que des apolipoprotéines échangeables pourraient être acquises par les particules virales après leur libération dans le milieu extracellulaire, à partir des lipoprotéines circulant dans le sérum \citep{RN520,RN521}.

	\begin{figureth}
	\centering
			\includegraphics[width=0.70\linewidth]{Figure_25.png}
		\caption[Voies de sécrétion hypothétiques des particules du virus de l’hépatite C]{\textbf{Voies de sécrétion hypothétiques des particules du virus de l’hépatite C.} (A) Dans la voie d’exportation conventionnelle ou canonique, les particules virales matures sont transportées du RE vers l’appareil de Golgi au sein de vésicules COP-II puis transitent au sein du réseau trans-golgien jusqu’aux compartiments endosomaux. À ce stade, les particules virales peuvent être soit transportées vers un corps multivésiculaire (MVB) pour être ensuite libérées par exocytose, soit directement exportées via une vésicule sécrétoire. (B) Dans la voie d’exportation non conventionnelle, les vésicules COP-II vont directement libérer les particules virales dans le milieu extracellulaire ou les transporter dans le compartiment endosomal, sans passer par l’appareil de Golgi. Les facteurs d’hôte connus pour être impliqués dans le trafic intracellulaire des particules virales  sont encadrés de vert. (Adapté de \citealt{RN518}).}
				\label{fig:fig25}
	\end{figureth}
		\FloatBarrier

\clearpage

%%%%%%%%%%%%%%%%%%%%%%%%%%%%%%%%%%%%%%%%%%%%%%%%%%%%%%%%%%%%%%%%%%%%%%%%%%%%%%%%%%%%%%%%%%%%%%%%%%%%%%%%%%%%%%%%%%%%%%%%%%%%%%%%%%%%%%%%%%%%%%%%%%%%%%%%%%%%%%%%%%%%

% 7 - Les modèles d'étude du virus de l'hépatite C

\section{Les modèles d'étude du virus de l'hépatite C}	
		\label{section:modele}

Depuis la découverte du VHC en 1989, l’absence de système en culture cellulaire a longtemps été un obstacle majeur à l’étude de ce nouveau pathogène et à la conception de stratégies prophylactiques et thérapeutiques. Historiquement, les premières tentatives pour établir des modèles d’étude \textit{in vitro} reposaient sur l’infection expérimentale de divers types cellulaires en culture, dont des cellules hépatocytaires, par des souches cliniques du VHC issues de sérum de patients infectés (VHCser), qui représentaient à l’époque la seule source de virus. Bien que plusieurs travaux aient fait état d’une réplication du VHC dans des lignées cellulaires ou des cultures de cellules primaires, l’efficacité de réplication était variable voire douteuse, et aucun modèle n’a abouti à une infection productive \citep[pour revue,][]{RN307}. Les essais suivants se sont alors basés sur les premiers clones moléculaires disponibles en suivant les stratégies établies pour d’autres virus à ARN positif \citep{RN308}, c’est-à-dire introduire une copie d’ADNc du génome viral dans un vecteur d’expression sous le contrôle transcriptionnel d’un promoteur d’ARN polymérase de phage. Ce système par synthèse \textit{in vitro}, permettait alors de disposer d’une source illimitée de génome viral et d’insérer des gènes rapporteurs ou des marqueurs de sélection pour faciliter la détection. La fonctionnalité de cette approche pour le VHC a a été établie pour la première fois suite à l’injection intra-hépatique du génome transcrit \textit{in vitro} de l’isolat H77 de génotype 1a chez des chimpanzés, qui ont alors développé une virémie \citep{RN309,RN310}. Par la suite, d’autres génomes infectieux \textit{in vivo} ont été générés, comme l’isolat J6 de génotype 2a \citep{RN311}. Néanmoins, aucune réplication en culture cellulaire n’a pu être observée avec ces clones, signant le premier échec d'un long et sinueux parcours vers le développement de modèles expérimentaux pour étudier le VHC. \\ \\
\indent
Au cours de ce chapitre, nous détaillerons (i) les différents systèmes d’étude qui ont été développés \textit{in vitro} et leurs implications dans les découvertes fondamentales sur la biologie du virus et dans le développement des thérapies antivirales ; (ii) les perspectives futures pour élaborer des modèles cellulaires plus physiologiques à partir des nouvelles technologies de pointe et (iii) les modèles animaux disponibles pour étudier l'évolution naturelle de l'infection.

\subsection{Réplicons sous-génomiques et pseudo-particules virales}	
			\subsubsection{Les réplicons sous-génomiques}

Le premier modèle d’étude \textit{in vitro} du VHC a été rendu possible par la conception d’ARN sous-génomiques novateurs. Ce modèle expérimental est un « mini-génome » bicistronique, constitué d’un gène codant pour un marqueur de sélection (par exemple, conférant une résistance à un antibiotique) et du module réplicatif du VHC, \textit{i.e.} les séquences des protéines NS3 à NS5B sous le contrôle traductionnel de l’IRES provenant d’un autre virus (par exemple, le virus de l’encéphalomyocardite ou EMCV), encadrés des extrémités 5’ et 3’NC (\autoref{fig:fig26}\textcolor{blue}{.A}). Un réplicon basé sur les séquences de l’isolat Con1 de génotype 1b a été le premier à montrer une réplication stable, bien que de faible niveau, en lignée d’hépatome humaine Huh-7 \citep{RN294}. Le maintien de clones cellulaires hébergeant le réplicon Con1 sous pression de sélection a permis d'obtenir des variants dotés d’une meilleure aptitude réplicative et d’améliorer considérablement ce système. Plusieurs groupes ont découvert que ces variants sous-génomiques contenaient une ou plusieurs mutations génétiques dispersées dans l’ensemble de la région codante de NS3 à NS5B, essentielles pour amplifier la réplication de l’ARN \citep{RN322,RN314}. Les principales mutations adaptatrices identifiées dans les variants du réplicon Con1 ont pu être transférées à des réplicons sous-génomiques dérivant d’autres isolats de génotype 1, comme l’isolat H77 \citep{RN315} en raison des mécanismes d’action conservés. Il était cependant difficile de couvrir l’ensemble des génotypes par cette approche car la majorité des réplicons nécessitait des combinaisons variées de plusieurs mutations adaptatrices pour aboutir à une réplication génomique stable. En raison de ces difficultés, l’établissement de réplicons sous-génomiques couvrant la majorité des génotypes circulants n’a été publié que beaucoup plus tard : les premiers réplicons de génotypes 3 et 4 en 2013 \citep{RN316,RN318}, de génotypes 5 et 6 en 2014 \citep{RN319,RN320} et d’autres sous-types encore plus récemment \citep{RN321}. En plus des mutations compensatrices favorisant la réplication génomique, la permissivité de la cellule hôte constitue également un facteur déterminant pour établir un modèle d’étude du VHC efficace \textit{in vitro}. Plusieurs clones cellulaires hautement permissifs à la réplication du VHC ont été établis, tels que les lignées d’hépatomes humaines Huh-7.5 \citep{RN322} ou Huh-7-Lunet \citep{RN323}. Ces systèmes cellulaires offrent un environnement beaucoup plus favorable à la réplication du VHC, soit en augmentant le niveau d'expression des facteurs cellulaires indispensables à l’activité du module réplicatif, soit en ayant des défauts dans les voies de signalisation de l’immunité innée qui limitent habituellement la multiplication du virus. Dans le cas des cellules Huh-7.5, la permissivité élevée est liée à une mutation  faux-sens qui inactive la fonction du gène inductible par l’acide rétinoïque I (RIG-I), ayant pour conséquence une absence de réponse antivirale \citep{RN324}. \\ \\
\indent
Le développement des réplicons sous-génomiques a constitué une véritable révolution et a permis pour la première fois d’établir les détails moléculaires de la réplication génomique et de caractériser le réseau membranaire qui sous-tend cette étape du cycle viral \citep[pour revues,][]{RN326,RN328}. En effet, le groupe de C. Rice a identifié un site dans la protéine NS5A qui tolère l'insertion de la protéine fluorescente verte (GFP), ce qui a permis d’effectuer les premières études d'imagerie des sites putatifs de la réplication virale en cellules vivantes \citep{RN985,RN329}. Les réplicons sous-génomiques ont également été déterminants dans le développement des thérapies antivirales afin de caractériser l’effet des AAD sur la réplication génomique, à l’aide de réplicons codant des gènes rapporteurs comme la luciférase \citep[pour revue,][]{RN332}. De plus, la récente disponibilité des réplicons représentatifs des génotypes 2, 3, 4, 5 et 6 a permis d’évaluer l’efficacité des molécules antivirales contre la majorité des génotypes circulants et d’aboutir à l’élaboration des combinaisons d’AAD guérissant aujourd’hui >90\% des cas d’infection dans le monde. Bien que le modèle d’étude \textit{in vitro} basé sur les réplicons sous-génomiques ait été crucial pour le développement de médicaments, l'inconvénient de ce système est qu'il ne récapitule pas l’ensemble des étapes du cycle viral, notamment l’entrée ou l’assemblage et la sécrétion des particules infectieuses, une lacune importante pour la compréhension de la biologie du virus.

			\subsubsection{Les pseudo-particules virales}

Le deuxième modèle d’étude \textit{in vitro} du VHC exploite la technologie des pseudo-particules et représente une percée majeure dans l’étude du processus d’entrée virale. Cette méthode se base sur des particules rétro-virales défectives exprimant les glycoprotéines d’enveloppe E1 et E2 du VHC à leur surface \citep{RN333,RN334}. Pour produire ces pseudo-particules du VHC (VHCpp), des cellules 293T sont co-transfectées avec trois vecteurs d’expression codant : (i) les glycoprotéines E1 et E2 du VHC, (ii) les protéines Gag-Pol du VIH-1 ou du virus de la leucémie murine (VLM) et (iii) un génome rétroviral contenant un gène rapporteur tel que la luciférase ou la GFP pour détecter et quantifier facilement l’entrée du VHC dans les cellules hôtes (\autoref{fig:fig26}\textcolor{blue}{.B}). Les VHCpp miment le processus d’infection, ce qui a permis d’étudier le rôle des glycoprotéines E1 et E2 dans l’entrée du VHC, d’identifier et de valider les facteurs d’attachement et les récepteurs candidats et de mettre en lumière les mécanismes d’internalisation du virus \citep[pour revues,][]{RN335,RN336}. De plus, les VHCpp ont permis d’étudier les propriétés neutralisantes des anticorps spécifiques de E1 et E2 et d’identifier des épitopes de neutralisation à large spectre \citep{RN337,RN338}. Un inconvénient majeur à ce système, en plus de ne récapituler que les premières étapes du cycle viral, est qu’il repose sur un processus d’assemblage similaire à celui des rétrovirus. De ce fait, les pseudo-particules ne reproduisent pas l’association du VHC avec les apolipoprotéines et ne constituent donc pas un bon modèle pour étudier la morphologie et la structure des virions naturels.

	\begin{figureth}
	\centering
			\includegraphics[width=\linewidth]{Figure_26.png}
		\caption[Premiers modèles \textit{in vitro} pour étudier la réplication génomique et l’entrée du cycle viral]{\textbf{Premiers modèles \textit{in vitro} pour étudier la réplication génomique et l’entrée du cycle viral.} (A) Le réplicon sous-génomique initialement développé par Lohmann et al. code le gène de résistance à la néomycine (Neo) sous le contrôle traductionnel de l’IRES du VHC dans le premier cistron et les protéines du module réplicatif (NS3 à NS5B) sous le contrôle traductionnel de l’IRES hétérologue de l’EMCV dans le deuxième cistron. L’expression concomitante du gène de résistance à la néomycine permet de sélectionner les cellules hébergeant les réplicons par un traitement avec la généticine (G418). (B) Les pseudo-particules du VHC contiennent l’enveloppe authentique du virus avec les glycoprotéines E1 et E2 assemblées sur une capside rétrovirale et sont produites par la co-transfection de cellules 293T avec trois plasmides codant (i) les protéines E1 et E2 du VHC, (ii) les protéines Gag-Pol du VIH-1 ou du VLM et (iii) un génome rétroviral exprimant un gène rapporteur comme la luciférase (luc) ou la GFP (Adapté de \citealt{RN351,RN332}).}
				\label{fig:fig26}
	\end{figureth}
		\FloatBarrier

\subsection{Premiers systèmes infectieux en lignées d’hépatome humaines}	

Avec le succès sans précédent des réplicons sous-génomiques adaptés à la culture cellulaire, la communauté scientifique a naturellement imaginé un design analogue avec des ADNc couvrant la séquence complète du VHC, dans l’objectif de produire un modèle d’étude \textit{in vitro} du cycle intégral du virus. Des ARN de longueur génomique de génotype 1 ont donc été générés avec les mutations compensatrices favorisant la réplication, mais aucun de ces génomes n’a abouti à une production de particules virales malgré une réplication stable en cellules Huh-7 \citep{RN330}. Les hypothèses étaient que soit la lignée cellulaire utilisée n’était pas permissive à l’assemblage viral, de par l’absence de facteurs d’hôtes pro-viraux, soit les mutations qui favorisent la réplication interféraient avec la production de particules virales. À l’appui de cette seconde hypothèse, un groupe a montré que l’inoculation intra-hépatique de chimpanzés avec l’ARN génomique Con1 contenant les mutations adaptatrices n’établissait pas d’infection, tandis que l’inoculation avec le génome Con1 authentique provoquait une infection persistante \citep{RN331}. Une étude ultérieure a confirmé ces observations, indiquant que les mutations favorisant la réplication \textit{in vitro} sont bel et bien contre-productives \textit{in vivo} \citep{RN339} et qu’un système de culture cellulaire permissif à la réplication et à l’assemblage ne serait \textit{in fine} possible qu’avec une souche virale sauvage. \\ \\
\indent
Quelques années plus tard, un isolat particulier du VHC appartenant au génotype 2a (JFH-1) a été identifié chez un patient japonais atteint d’hépatite fulminante. En plus de montrer une  réplication génomique naturellement élevée \citep{RN340}, cette souche clinique s’est avérée capable de produire spontanément des particules infectieuses en cellules Huh-7 \citep{RN124}. Cet isolat constitue à ce jour la seule souche clinique capable de reproduire naturellement un cycle infectieux complet \textit{in vitro}, établissant une nouvelle source clonale de virus (VHCcc). Bien que la réplication génomique de la souche JFH-1 était particulièrement efficace et ne requérait pas de mutation d’adaptation, les rendements globaux d’infectiosité étaient relativement faibles. Afin d’améliorer ce nouveau modèle viral, des tentatives d’adaptation en culture de la souche JFH-1 ont été menées en vue d’acquérir des variants plus robustes sur le plan infectieux. Plusieurs groupes ont fait état de mutations qui résident non seulement dans séquences des protéines structurales, mais aussi dans celles de p7, NS2 et NS5A, ce qui est cohérent vis-à-vis du rôle de ces protéines dans la coordination de la morphogénèse virale \citep{RN361,RN362}. Le variant JFH-1 adapt (Jad), adapté en culture de cellules Huh-7.5, constituait le mutant le plus robuste avec une combinaison de trois mutations affectant les protéines NS5A et NS5B et conférant un titre infectieux 100 fois plus élevé par rapport au clone original \citep{RN342}. Ces mutations semblent améliorer la transition entre les processus de réplication génomique et d’assemblage des particules virales. Une autre stratégie menée en parallèle pour améliorer le modèle JFH-1 reposait sur la production de dérivés chimériques, par la substitution de séquences hétérologues provenant d’autres souches cliniques.  Une des premières constructions hybrides fonctionnelles s’est basée sur la combinaison des régions NS3 à NS5B du génome JFH-1 avec les régions des protéines Core à NS2 de l’isolat J6 de même génotype, produisant un génome hybride J6/JFH-1 \citep{RN343}. Le dogme établi était alors de séparer les protéines impliquées dans la morphogénèse et les protéines du complexe de réplication. Pour déterminer les meilleures jonctions pour l’hybridation, une étude a cartographié l’ensemble de la séquence NS2 et a mis en évidence que le site de fusion le plus optimal se situait immédiatement en aval du premier segment transmembranaire \citep{RN344}. Le génome chimérique J6/JFH-1 a alors été amélioré en déplaçant le point de jonction au sein de NS2, à l’origine de la souche Jc1 qui constitue aujourd’hui un des modèles d’infection \textit{in vitro} les plus répandus dans le domaine. \\ \\
\indent
La découverte exceptionnelle de l’isolat JFH-1 après 25 années d’essais à cultiver des souches cliniques \textit{in vitro} a permis d’ouvrir une nouvelle voie pour l’étude de la biologie du virus, et de finalement décortiquer les dernières étapes du cycle viral, à savoir les mécanismes d’assemblage, de sécrétion et de propagation des particules du VHC \citep[pour revues,][]{RN345,RN346}. L’étude de la structure des virions JFH-1 et Jc1 par des méthodes de capture de haute affinité a également permis d’obtenir les premiers détails concernant la morphologie et la composition atypique des particules infectieuses du VHC \citep{RN282,RN284,RN285}. Les modèles disponibles se limitaient alors à des souches de génotype 2a, qui ne pouvaient être à elles seules pertinentes sur le plan clinique. La construction de chimères intergénotypiques a alors été entreprise mais a généralement été moins immédiate car elle nécessitait une adaptation complémentaire en culture, en raison de problèmes de compatibilité génétique entre les protéines virales des différents génotypes. Néanmoins, le groupe de J. Bukh a produit avec succès des génomes hybrides fonctionnels exprimant les séquences Core à NS2 des génotypes 1 à 7 \citep{RN350,RN352}. Plus tard, d’autres modèles intergénotypiques ont été développés à partir du génome J6/JFH-1, en substituant les régions codant NS3-4A, NS5A, NS5B ou encore  toutes les régions à l’exception de NS3-4A et NS5B par les séquences équivalentes d’autres génotypes \citep[pour revue,][]{RN351}. En parallèle, d’autres groupes ont cherché à établir des modèles d’infection \textit{in vitro} pour tous les génotypes du VHC. Ce travail s’est avéré particulièrement épineux, car les isolats cliniques nécessitaient des schémas complexes de mutations d’adaptation, à la fois pour améliorer la réplication génomique et pour établir une production de particules virales. Le premier succès a été obtenu avec un génome H77 hautement adapté (H77-S), qui contient un total de 5 mutations compensatrices dispersées dans la séquence virale \citep{RN353}. Plus tard, des variants hyper-adaptés de génotypes 2a (à partir de l’isolat J6) \citep{RN354}, 2b \citep{RN355}, 3a \citep{RN356,RN357} et 6a \citep{RN358} ont été développés. Ces virus, dotés de l’ordre d’une vingtaine de mutations d’adaptation, présentent toutefois un titre infectieux 10 à 100 fois inférieur à celui de la souche sauvage JFH-1 et une proportion plus importante de particules défectives. Néanmoins, l’émergence de ces nouveaux modèles \textit{in vitro} du VHC constitue un atout futur pour caractériser les propriétés pathogéniques spécifiques à certains génotypes, en particulier au génotype 3, associés au développement de la stéatose hépatique et à une progression avancée de la maladie vers les stades de la cirrhose \citep{RN359,RN360}.

\subsection{Cultures d’hépatocytes primaires, cellules souches et organoïdes}	

Jusqu’à présent, les sous-clones permissifs dérivant de la lignée d’hépatome humaine Huh-7 constituent le modèle cellulaire de référence pour étudier le VHC. Néanmoins, s’agissant d’une lignée immortalisée issue d’un carcinome hépatique, ces cellules ne sont pas polarisées ce qui ne permet pas de reproduire la compartimentation des co-récepteurs du VHC ni l’orientation des systèmes sécrétoires des hépatocytes matures. Elle ne constitue pas non plus le modèle le plus pertinent pour étudier certains aspects de l’interaction virus-hôte, tels que l’activation de la réponse immunitaire innée ou la carcinogenèse. L'objectif futur est donc de développer des modèles cellulaires plus physiologiques pour l’étude \textit{in vitro} du VHC. \\ \\
\indent
Les cellules primaires d’hépatocytes humains (PHH) sont les cellules hôtes du VHC lors de l’infection naturelle et sont cultivables \textit{ex vivo}. Toutefois, ces cellules sont difficiles à obtenir car elles proviennent de parties non tumorales de résections hépatiques et sont très variables en fonction du sexe, de l’âge, de l’exposition à diverses substances et des polymorphismes génétiques du donneur. Une fois mises en culture, les PHH ne se divisent pas et perdent rapidement les caractéristiques biologiques d’hépatocytes matures par dédifférenciation, ce qui limite leur durée de vie en monocouche à environ deux semaines \citep{RN363}. Pour prolonger leur durée de vie, certains groupes ont eu recours à des méthodes d’immortalisation par l’introduction d’oncogènes \citep{RN364}. Des progrès considérables ont été réalisés dans le but de générer des hépatocytes matures par différenciation de cellules souches pluripotentes somatiques (IPSC) ou embryonnaires (EPSC) permettant un approvisionnement illimité en cellules et un modèle plus reproductible par rapport aux PHH de donneurs adultes. En général, ces modèles ne sont pas hautement permissifs à l’infection par le VHC, notamment en raison des voies immunitaires intactes de l’hôte. En effet, plusieurs groupes ont documenté un faible taux d’infection de PHH obtenues par résections \citep{RN365} ou dérivant des EPSC/IPSC \citep{RN366,RN367,RN368} à partir d’inoculum de VHCcc et une clairance virale au bout d’une dizaine de jours. \\ \\
\indent
Depuis que la technologie des cellules souches a été établie, une perspective innovante a été le développement de foies humains vascularisés et fonctionnels par des protocoles d’organogénèse \textit{in vitro}. Les organoïdes hépatiques peuvent être dérivés à partir de cellules souches pluripotentes (telles que les EPSC/IPSC) ou de cellules souches bipotentes progénitrices du foie \citep{RN369,RN370}. Les cellules souches sont cultivées dans des matrices extracellulaires et mises en contact avec des facteurs de croissance qui favorisent l’auto-organisation spontanée du tissu (\autoref{fig:fig27}). Cette approche permet de structurer naturellement un réseau constitué des trois principales cellules non parenchymateuses du foie (cellules stellaires, cellules de Kupffer et cellules endothéliales) et des hépatocytes, reproduisant \textit{in vitro} l’architecture complexe de l’organe et l’hétérogénéité cellulaire contrairement aux cultures 2D d’hépatocytes  \citep{RN371,RN372}. Ces structures organoïdes peuvent être maintenues à long terme en culture \textit{in vitro} et continuer à refléter, même après de nombreuses générations, les marqueurs de différenciation du tissu d'origine  \citep{RN373}. Les organoïdes matures peuvent être également transplantés dans des souris immunodéficientes pour générer des souris avec un foie humanisé (voir \autoref{section:animaux}). Récemment, la technologie des organoïdes s'est imposée comme l'outil de culture cellulaire de pointe pour l'étude de la biologie humaine dans le domaine de la santé et des pathologies  \citep[pour revues,][]{RN374,RN375}. Le principal défi de ces modèles pour étendre leur utilisation à l’étude des pathologies infectieuses humaines est de parvenir à établir l’infection naturelle avec l’agent étiologique. Dans le cadre de l’hépatite C, l’induction des réponses immunitaires limite fortement la mise en place d’une infection productive et persistante par le virus. À l’heure actuelle, seuls les organoïdes dérivant de cellules Huh-7.5 sont permissifs à l’infection par le VHC \citep{RN376}, et même si ils rétablissent certains marqueurs caractéristiques des hépatocytes matures lors de la mise en culture 3D (comme la polarisation et les jonctions serrées), ils restent globalement moins physiologiques que les modèles basés sur les hépatocytes humains en culture primaire. \\

	\begin{figureth}
	\centering
			\includegraphics[width=\linewidth]{Figure_27.png}
		\caption[Protocole d’organogénèse \textit{in vitro}pour générer des organoïdes hépatiques et biliaires]{\textbf{Protocole d’organogénèse \textit{in vitro} pour générer des organoïdes hépatiques et biliaires.} Les organoïdes du foie peuvent dériver de différentes sources telles que les cellules bipotentes progénitrices du foie ou les cellules souches pluripotentes. Les cellules progénitrices peuvent être obtenues à partir du tissu hépatique adulte par résection chirurgical, prélèvement de biopsie ou de bile ou à partir des stades embryonnaires de l’organogenèse et stimulées pour former des organoïdes hépatiques (hépatocytes) ou des organoïdes biliaires (cholongiocytes) après incubation dans une matrice extracellulaire avec une combinaison définie de facteurs de croissance. Les cellules souches pluripotentes d’origine embryonnaire (EPS) ou somatique (IPS) nécessitent d’abord un protocole de différenciation en 3 étapes pour générer des hépatoblastes qui sont ensuite incorporés dans une matrice extracellulaire pour promouvoir la croissance et la formation d’organoïdes. BMP4 : protéine morphogénétique osseuse 4 ; EGF : facteur de croissance épidermique ; FGF : facteur de croissance fibroblastique ; GSK3-i : inhibiteur de la glycogène synthase kinase 3 ; HGF : facteur de croissance hépatocytaire ; PI3K-i, inhibiteur de la phosphoinositide 3-kinase ; RA, acide rétinoïque ; TGFa : facteur de croissance transformant α ; TGFBR1-i, inhibiteur du récepteur 1 du facteur de croissance transformant ß.  (Adapté de \citealt{RN375}).}
				\label{fig:fig27}
	\end{figureth}
		\FloatBarrier

\clearpage

\subsection{Modèles animaux sauvages et humanisés}	
		\label{section:animaux}
		
Historiquement, la découverte et l’étude des agents responsables d’hépatites virales ont grandement bénéficié de l’exploitation de primates non humains comme modèle permissif à l’infection aiguë et chronique \citep{RN377,RN378}. Les chimpanzés sont la seule espèce animale autre que l’homme à être naturellement susceptible à l’infection par le VHC et représentaient donc le seul modèle acceptable pour l’évaluation des stratégies thérapeutiques et vaccinales. Depuis 2012, les nouvelles directives éthiques restreignent fortement l’emploi des chimpanzés en tant que modèle d’expérimentation animale pour le VHC et limitent leur utilisation aux États-Unis \citep{RN379}. \\ \\
\indent
Les souris transgéniques humaines ou les souris chimériques à foie humain sont alors devenus les modèles d’expérimentation animale de référence pour l’étude du VHC \textit{in vivo}. Les souris sont naturellement résistantes à l’infection par le VHC, en raison des divergences entre les facteurs  hépatocytaires murins et humains essentiels pour l’entrée virale \citep[pour revue,][]{RN381} et la présence de facteurs de restriction limitant la multiplication virale dans les hépatocytes murins \citep{RN382}. Une méthode qui a permis de surmonter la barrière d’espèce consiste à réaliser une humanisation directe du foie murin par xénotransplantation avec des PHH ou des organoïdes hépatiques humains cultivés \textit{in vitro} (\autoref{fig:fig28}\textcolor{blue}{.A}) \citep{RN385}. Pour faciliter le maintien de la greffe humaine, ce genre d’approche doit être effectuée dans un contexte très immuno-suppresseur, donc à partir de souris déficientes pour les réponses lymphocytaires B, T et NK. Ces lignées de souris expriment également des transgènes hépatotoxiques qui vont induire simultanément des lésions dans la région hépatique murine, déclenchant la production d’hormones régénératrices qui vont stimuler la prolifération et la colonisation des PHH dans tout l’organe. Une autre stratégie pour développer des souris susceptibles à l’infection avec moins de difficultés techniques consiste à introduire transgénétiquement les facteurs homologues humains requis pour l’entrée du virus \citep{RN380}. Dans ce contexte, une lignée de souris transgéniques exprimant les gènes humains codant pour les co-récepteurs CD81, SRB1, CLDN1 et OLCN a été développée et s’est montrée permissive pour l’entrée du VHC \citep{RN383}. Afin de récapituler une infection persistante du virus, il était néanmoins nécessaire de combiner l'expression des facteurs humains d'entrée du VHC avec une atténuation des réponses immunitaires innées de la souris, en utilisant des souris immuno-déficientes (\autoref{fig:fig28}\textcolor{blue}{.B}) \citep{RN384}. Les souris transgéniques et transplantées deviennent alors permissives à l’infection par le VHC mais aucune pathologie n’a été observée malgré la persistance du virus pendant plusieurs mois, probablement en raison de l’absence d’inflammation qui s'établit dans le foie au cours de l’infection chronique chez l’homme. Néanmoins, ces modèles ont été très utiles pour valider de nombreux aspects moléculaires du cycle viral \textit{in vivo} ou pour évaluer l’efficacité des anticorps neutralisants  \citep{RN386,RN387,RN388} ou des combinaisons thérapeutiques en phase pré-clinique \citep{RN389,RN390,RN391}. Plus tard, un groupe a reproduit cette approche sur des souris à fond génétique immuno-compétent (ICR) qui ont développé une virémie soutenue pendant plus de 12 mois parallèlement à des lésions hépatiques signant une progression vers la fibrose et la cirrhose \citep{RN392}. Toutefois, ces données doivent être reproduites indépendamment afin de confirmer l’établissement du premier modèle viable \textit{in vivo} pour étudier la pathogénèse du VHC \citep[pour revue,][]{RN393}. D’autres modèles de souris doublement humanisées par transplantation de PHH et de cellules hématopoïétiques humaines sont en cours de développement, afin d’établir un système pour valider les stratégies vaccinales en phase pré-clinique, mais elles ne s’avèrent pas encore permissives à l’infection par le VHC \citep{RN394,RN395}. \\
	\begin{figureth}
	\centering
			\includegraphics[width=\linewidth]{Figure_28.png}
		\caption[Modèles de souris transgéniques et chimériques pour étudier l’infection naturelle du VHC \textit{in vivo}]{\textbf{Modèles de souris transgéniques et chimériques pour étudier l’infection naturelle du VHC \textit{in vivo}.} (A) Modèle de souris chimériques par transplantation intrasplénique d’hépatocytes humains primaires qui vont coloniser le foie après induction de lésions dans le foie murin. Le fond génétique immuno-déficient permet aux PHH de coloniser jusqu’à 90\% du foie de l’animal, rendant ce modèle permissif à l’infection persistante du VHC. (B) Lignée de souris génétiquement humanisées sur fond génétique STAT1 -/- exprimant les récepteurs humains du VHC (EFT : entry factor transgenic) et récapitulant toutes les étapes du cycle viral. Une persistance virale est établie par la détection d’une virémie jusqu’à 90 jours post-infection (p.i). (Adapté de \citealt{RN351}).}
				\label{fig:fig28}
	\end{figureth}
		\FloatBarrier

\clearpage

%%%%%%%%%%%%%%%%%%%%%%%%%%%%%%%%%%%%%%%%%%%%%%%%%%%%%%%%%%%%%%%%%%%%%%%%%%%%%%%%%%%%%%%%%%%%%%%%%%%%%%%%%%%%%%%%%%%%%%%%%%%%%%%%%%%%%%%%%%%%%%%%%%%%%%%%%%%%%%%%%%%%

% 8 - La protéine de capside Core

\section{La protéine de capside Core}	
		\label{section:core}

	\subsection{Structure et topologie de la protéine Core}
	
La protéine Core constitue l’élément principal de la capside qui entoure et protège le génome viral. La forme initiale immature de Core est une protéine de 191 acides aminés (23kDa), constituée de deux domaines principaux D1 et D2 et d’un domaine provisoire D3 (\autoref{fig:fig29}) \citep[pour revue,][]{RN399}. Le domaine N-terminal D1 (résidus 1-117) est une structure hydrophile très flexible, subdivisée en trois sous-domaines basiques BD1, BD2 et BD3 riches en résidus arginine, lysine, glycine et proline. Il possède des propriétés de fixation à l’ARN viral, assurant l’incorporation du génome dans la nucléocapside \citep{RN1078}. La formation de la capside nécessite l’oligomérisation de Core qui est également assurée par le domaine D1 \citep{RN300}. De plus, la flexibilité de cette structure permet au domaine D1 d’interagir avec de nombreux partenaires cellulaires, ce qui pourrait contribuer à la pathogénèse \citep{RN1112}. Le domaine central D2 (résidus 118-177) contient deux hélices amphipatiques séparées par une boucle hydrophobe fortement enrichie en résidus leucine et alanine. Cette région, qui a une affinité pour les monocouches lipidiques, confère à Core la capacité d’ancrage aux gouttelettes lipidiques (GL). La région d’interaction avec les GL se situe au niveau des résidus 138 et 169 \citep{RN304}. Les 20 résidus C-terminaux qui constituent le domaine D3, forment un segment transmembranaire basique responsable de la rétention de Core au RE après traduction. Il contient une séquence signal qui sera clivée par la SPP, libérant la protéine Core mature de 177 acides aminés (21kDa) \citep{RN303,RN955}. À ce jour, la structure tridimensionnelle de Core n’a pas encore été résolue à l’exception d’une région partielle du peptide signal synthétique du précurseur Core-E1 par résonance magnétique nucléaire (RMN), qui a permis de fournir la base structurelle du mécanisme du clivage par la SPP \citep{RN1085}. Des produits de traduction provenant d'un cadre de lecture alternatif chevauchant la région codante principale, notamment la protéine core +1, ont été identifiés \citep[pour revue,][]{RN1084}, mais l’implication de ces protéines dans le cycle de vie et la pathogenèse du VHC reste à élucider.

	\begin{figureth}
	\centering
			\includegraphics[width=\linewidth]{Figure_29.png}
		\caption[Organisation de la protéine de capside Core du virus de l'hépatite C]{\textbf{Organisation de la protéine de capside Core du virus de l'hépatite C.} La protéine Core est divisée en trois domaines D1 (résidus 1-117), D2 (118-176) et D3 (177-191). Les positions des sous-domaines basiques (BD1, BD2 et BD3), des hélices alpha (H1 et H2) et de la boucle hydrophobe (HL) sont indiquées. Le clivage par la peptidase du peptide signal (SPP) est indiqué par une flèche. (D'après \textit{Hepatitis C Online} \copyright\ 2021, illustration par J. Travnicek et D. Ehlert).}
				\label{fig:fig29}
	\end{figureth}
		\FloatBarrier

	\subsection{Implication de la protéine Core dans le développement des pathologies hépatiques}
		\label{section:corepatho}

La fonction principale de la protéine Core dans le cycle viral est de former la nucléocapside, qui constitue la première étape de l’assemblage des particules virales infectieuses (décrite dans la \autoref{section:assemblage}). Toutefois, plusieurs études mettent en évidence que le rôle de la protéine Core du VHC ne se limite pas à l’assemblage de la capside, mais que celle-ci a également un rôle central dans la pathogénèse \citep[pour revues,][]{RN1123,RN1120}. Ces évidences expérimentales étaient disponibles avant le développement des réplicons sous-génomiques et du premier système infectieux basé sur la souche JFH-1 et reposaient donc principalement sur l’expression isolée de protéines Core recombinantes \textit{in vitro} ou dans des modèles de souris transgéniques. Les premiers témoignages expérimentaux à ce sujet datent de la fin des années 1990, où Core a été directement associée au développement de la stéatose hépatique à l’âge de 3 mois et à des nodules hépatiques caractéristiques des altérations malignes à l’âge de 16 mois dans des lignées de souris transgéniques \citep{RN1117,RN1118,RN1121,RN1122}. En effet, l’expression de Core semble affecter les voies de signalisation cellulaire, les réponses pro-apoptotiques et pro-inflammatoires, le métabolisme des lipides et des lipoprotéines et la prolifération et la transformation cellulaire, ce qui représente des facteurs de risque important pour la stéatose et le CHC. L’un des moteurs clés de la pathogénèse de Core est sa capacité à interagir avec un large éventail de protéines, enzymes et de facteurs de transcription cellulaires \citep{RN1112}. Ces interactions peuvent perturber la fonction des protéines ciblées et l’expression des gènes cellulaires, pouvant altérer \textit{in fine} de nombreux processus biologiques. Bien que les autres protéines virales puissent contribuer à la pathogénèse virale, en particulier NS5A qui interfère également avec de nombreux facteurs d’hôte \citep{RN1141}, l’expression isolée de Core en culture cellulaire semble refléter un grand nombre des dérégulations observées dans les biopsies de foie de patients infectés, qui seront détaillés dans la \autoref{section:pathogenese}.

	\subsection{Mécanisme de pathogénèse de la protéine Core}
	\label{section:pathogenese}

La protéine Core est supposée moduler l'homéostasie lipidique en augmentant la lipogenèse via l'activation de SREBP et en réduisant l'oxydation et l'exportation des lipides \citep{RN468}. Core diminue également l'expression du récepteur $\alpha$ activé par les proliférateurs du peroxysome (PPAR$\alpha$), un récepteur nucléaire régulant plusieurs gènes responsables de la dégradation des acides gras \citep{RN1126}. De plus, Core inhibe l’activité de la protéine de transfert des triglycérides microsomaux (MTP) en se délocalisant transitoirement à la membrane des mitochondries, ce qui réduirait l’assemblage et la sécrétion des VLDL \citep{RN1127}. Cependant, la protéine Core n’étant pas détectée au niveau des mitochondries dans un système d’infection par la souche JFH-1, la pertinence de cette localisation et de cette fonction particulière reste débattue \citep{RN486}. En revanche, l’ensemble de ces données concordent avec le fait que l’expression transitoire de Core serait liée à une régulation positive de la synthèse des triglycérides qui se traduit par une accumulation et un un élargissement des GL cytosoliques, ce qui pourrait contribuer à l’apparition des signes histologiques de la stéatose \citep{RN1139}. Il a également été suggéré que la protéine Core favorise la prolifération cellulaire, l'apoptose, l'angiogenèse et la tumorigenèse en dérégulant un large éventail de gènes tels que le facteur de croissance transformant ß (TGF-ß), le facteur de croissance endothélial vasculaire (VEGF), la voie Wnt/ß-caténine et la cyclo-oxygénase-2 (COX-2), ce qui peut favoriser le développement et le caractère invasif du CHC. Core semble également pouvoir moduler l’expression de proto-oncogènes cellulaires, comme la protéine p53, et l’abondance des micro-ARN qui participent aux réponses anti-tumorales, tel que le miR-122 hépatique, qui possède la propriété intrigante de réguler les gènes suppresseurs de métastases. L’apoptose est un processus cellulaire essentiel dans le contrôle des infections virales, en éliminant le contenu de la cellule contaminée par phagocytose, sans exposer les molécules intracellulaires qui déclencheraient des réponses inflammatoires. Compte tenu de son rôle, Il est fréquent que certains virus aient des stratégies pour éviter l'élicitation de l'apoptose. L’infection par le VHC a une interaction très complexe et mal comprise avec l’apoptose. En effet, L'expression des facteurs cellulaires régulant l'apoptose varie selon la phase aigue ou chronique de l’hépatite C chez les patients \citep{RN1124}. Core présente des propriétés pro-apoptotiques et anti-apoptotiques, telles que sa capacité à stimuler ou à inhiber les voies intrinsèques ou extrinsèques médiées par p53 ou par le facteur de nécrose tumorale $\alpha$ (TNF$\alpha$) \citep{RN1131}.  \\ \\
\indent
Core est la protéine la plus conservée de toutes les protéines du VHC entre les 8 génotypes. Toutefois, des polymorphismes naturels retrouvés dans sa séquence ont été corrélés à des degrés de dérégulations cellulaires plus importants et à l’apparition plus fréquente des complications hépatiques. En effet, des résidus spécifiques aux protéines Core de génotype 1b et 3a ont été liées à une progression plus fréquente vers la stéatose et le CHC \citep{RN989,RN1134,RN1129}. À titre d’exemple, le résidu Phe à la position 164 et la combinaison des résidus Phe/Ile aux positions respectives 182 et 186, retrouvées dans les séquences Core de génotype 3a, ont été associées à un dérèglement accru du métabolisme des lipides et à une régulation positive de la biogénèse des GL \citep{RN990,RN991}. Le résidu Gln à la position 70 fréquent au sein des séquences Core de génotype 1b s’avère être un prédicteur important de l’évolution vers le CHC \citep{RN988} et le résidu Thr à la position 71 a été récemment associée à une dérégulation accrue de la voie Wnt/ß-caténine \citep{RN808}. Enfin, une étude récente met en évidence que l’expression des gènes codant PPAR$\alpha$ et TGF-ß est davantage stimulée par les souches virales hyper-adaptées de génotype 3a par rapport à celles de génotype 1a \citep{RN1128}. L’ensemble de ces évidences expérimentales appuient les études cliniques évoquées dans la \autoref{section:classification}. En revanche, il est important de noter que la pertinence de ces expériences \textit{in vitro} ou \textit{in vivo}, impliquant majoritairement une surexpression de la protéine Core, reste incertaine dans le cadre de l’infection naturelle par le VHC.
	%% Copyright (C) 2017-2021 Emeline Simon
%%
%% The current owner of this work is Emeline Simon
%% <contact at emeline.simon@gmail.com>.
%%
%% This is 02_project.tex the second chapter for my PhD Thesis.
%%
%%%%%%%%%%%%%%%%%%%%%%%%%%%%%%%%%%%%%%%%%%

\chapter{Le Projet}
	\minitoc
	\newpage

%%%%%%%%%%%%%%%%%%%%%%%%%%%%%%%%%%%%%%%%%%%%%%%%%%%%%%%%%%%%%%%%%%%%%%%%%%%%%%%%%%%%%%%%%%%%%%%%%%%%%%%%%%%%%%%%%%%%%%%%%%%%%%%%%%%%%%%%%%%%%%%%%%%%%%%%%%%%%%%%%%%%%%%%

% 1 - CONTEXTE ET OBJECTIFS
			
	\section{Contexte et objectifs du projet}

L’infection chronique par le VHC est une maladie progressive et longtemps asymptomatique qui peut mener au développement de la stéatose, une accumulation anormale de lipides neutres sous la forme de GL dans les hépatocytes. En lien avec la grande diversité génétique du VHC, reflétée par l'existence de 8 génotypes et >90 sous-types (voir \autoref{section:epidemiologie}), des études cliniques ont rapporté une corrélation entre les infections chroniques par les souches du VHC de génotype 3, fréquentes chez les usagers de drogues en Europe, et une prévalence élevée de stéatose, de l’ordre de 80\%. Cette pathologie hépatique est également une évolution fréquente chez les patients souffrant de syndromes métaboliques « modernes » comme la NAFLD et AALD (voir \autoref{section:steatose}). Antérieurement considérée comme bénigne, la stéatose s’est avérée être un indicateur pronostique de la progression vers des complications hépatiques graves telles que la cirrhose et le CHC, responsables de 2 millions de décès annuels. Dans un contexte de recrudescence des maladies métaboliques tant dans les pays riches que dans les pays en développement, l’OMS a déclaré en 2010 que celles-ci constituaient un problème majeur de santé publique à l'échelle mondiale. Pour l’hépatite C chronique, le développement d’AAD permettant de guérir aujourd'hui $\geq$95\% des patients infectés est une révolution thérapeutique sans précédent (voir \autoref{section:elimination}). Cependant, en raison de leur coût, l'utilisation des AAD est encore majoritairement limitée aux pays occidentaux. De plus, la guérison n’élimine pas toujours le risque de développer des complications hépatiques. \\ \\ 
\indent
Les manifestations cliniques qui surviennent au cours de l’infection chronique par le VHC résultent d'une combinaison complexe de facteurs indirects, tels que l'inflammation chronique, la susceptibilité génétique et les habitudes individuelles, et de facteurs directs viro-induits. Comme il a été décrit au cours de l’introduction, de nombreux travaux mettent en évidence l’implication de la protéine Core du VHC dans le développement de la stéatose hépatique (voir \autoref{section:corepatho}). Premièrement, Core présente une association singulière à la surface des GL cytosoliques, dont la fonction essentielle à l’homéostasie lipidique est altérée au profit de la morphogénèse des virions. Le rôle exact de cette interaction cruciale dans l’assemblage des particules virales n’a toujours pas été élucidé et seule une vision incomplète de la façon dont le VHC détourne les GL est disponible à ce jour. De plus, la contribution directe de la dérégulation des GL par l’infection dans l'apparition des signes histologiques, à savoir, l’accumulation anormale des vacuoles lipidiques dans les hépatocytes, est mal comprise. Deuxièmement, Core semble jouer un rôle clé dans la dérégulation des voies de signalisation pro-inflammatoires et métaboliques des hépatocytes, ce qui représente un facteur de risque important pour la stéatose. Certains polymorphismes naturels des acides aminés de Core spécifiques aux souches de génotype 3 comme le résidu Phe en position 164 ou la combinaison des résidus Phe et Ile en positions respectives 182 et 186, ont été associés à un fort élargissement des GL et à une lipogénèse \textit{de novo} accrue, ce qui renforce les conclusions des études cliniques. Cependant, en l’absence de modèles animaux immunocompétents permissifs à l’infection pour étudier la pathogenèse du VHC et de systèmes de culture cellulaire permettant la multiplication de souches cliniques du VHC (voir \autoref{section:modele}), les résultats relatifs à une association entre les déterminants spécifiques de génotype et les troubles du métabolisme lipidique reposent principalement sur des systèmes d'expression transitoire de Core \textit{in vitro}. Par conséquent, on ne dispose à ce jour que d'une vue incomplète ou potentiellement biaisée de l'implication directe de Core dans la stéatose hépatique, et les connaissances mécanistiques dans des systèmes d'infection hépatique appropriés font défaut. \\ \\
\indent
Notre équipe s'intéresse d’une part, à approfondir les connaissances fondamentales du mécanisme d’assemblage du VHC et d’autre part, à déchiffrer le rôle de Core dans la pathobiologie de l'hépatite C et à identifier les mécanismes génotype-spécifiques et les déterminants viraux impliqués dans les réponses de l'hôte à l'infection, en utilisant des systèmes d'infection physiologiquement pertinents. Dans ce contexte, les objectifs de mon projet de thèse sont de : \\

\begin{itemize}
  \item[$\bullet$] Étudier l’interface spatio-temporelle entre la protéine Core et les GL afin d’identifier les mécanismes de dérégulation de ces organites au cours de l’infection et de déterminer leur rôle précis dans le mécanisme d’assemblage des particules virales.

  \item[$\bullet$] Développer de nouveaux modèles virologiques capables de se répliquer en cellules d’hépatome humain et codant les protéines Core de souches cliniques de différents génotypes isolées à partir de patients présentant divers degrés de stéatose, afin de répondre aux questions suivantes.

  \item[$\bullet$] Déterminer si l'origine génotypique de Core a une incidence différentielle sur la biogénèse et l’élargissement des GL, en lien avec les manifestations cliniques de la stéatose hépatique.

  \item[$\bullet$] Étudier si et dans quelle mesure l'origine génotypique de Core module de façon différentielle le transcriptome hépatique afin de mettre en évidence des propriétés pathogènes propres à certaines souches ou génotypes viraux. \\
\end{itemize}
\indent Ce projet s'inscrit dans le cadre d'un travail collaboratif soutenu par l'ANRS et permettra, à terme, une meilleure compréhension des mécanismes biologiques impliqués dans la pathogenèse du VHC et de l’importance des facteurs viraux directs et des polymorphismes génotypiques du VHC dans le développement de la stéatose hépatique.

\clearpage

%%%%%%%%%%%%%%%%%%%%%%%%%%%%%%%%%%%%%%%%%%%%%%%%%%%%%%%%%%%%%%%%%%%%%%%%%%%%%%%%%%%%%%%%%%%%%%%%%%%%%%%%%%%%%%%%%%%%%%%%%%%%%%%%%%%%%%%%%%%%%%%%%%%%%%%%%%%%%%%%%%%%%%%%

% 2 - PROJET LIPID DROPLET
			
	\section{Étude de la régulation de la biogénèse et de la dynamique des GL au cours de l’infection par le virus de l’hépatite C}

		\subsection{Préambule}
		\label{section:preambule1}

L’étude des GL par imagerie quantitative étant un nouveau domaine dans notre équipe, j’ai commencé par mettre au point les conditions expérimentales en approfondissant mes connaissances sur ce sujet à partir des données disponibles dans la littérature. L’imagerie quantitative par microscopie implique de sélectionner un système d’acquisition approprié pour répondre à la question biologique, de préparer rigoureusement des échantillons sur un support compatible avec le système d’acquisition choisi, de configurer précisément les paramètres d’acquisition et de prendre en compte toutes les précautions liées aux limitations optiques afin de se placer dans les conditions expérimentales les plus reproductibles et optimales.

			\subsubsection{Notes sur la mise au point des conditions expérimentales pour la préparation des échantillons}

Sur le plan strictement expérimental, les étapes les plus critiques de la préparation des échantillons sont (i) la fixation, (ii) la perméabilisation, (iii) le blocage et la révélation et (iv) le montage. (i) Les méthodes de fixation chimique à l’aide de méthanol, d’éthanol ou d’acétone à froid sont utilisées traditionnellement pour visualiser les éléments du cytosquelette. Ces solutions précipitantes ont toutefois été jugées incompatibles avec l’étude des GL car elles extraient la majorité des phospholipides cellulaires et provoquent une déstructuration de l’enveloppe des GL. La fixation des cellules ou des tissus par utilisation de paraformaldéhyde constitue la méthode de choix, car les cellules conservent leur contenu lipidique et la structure des GL est préservée \citep{RN927}. (ii) Les anticorps et les fluorophores sont généralement incapables de traverser la membrane plasmique et d’atteindre le cytoplasme, à moins que des détergents soient utilisées pour perméabiliser la membrane plasmique. Le Triton-X est un détergent non ionique qui interagit sélectivement avec le cholestérol, produisant de petits pores dans la membrane plasmique sans affecter les membranes pauvres en cholestérol telles que celles des mitochondries et de l’enveloppe nucléaire. La monocouche phospholipidique des GL est susceptible d'être solubilisée par des détergents tels que le Triton-X, libérant les protéines associées à leur surface \citep{RN924}. Les détergents plus doux comme la saponine ou la digitonine ne dissolvent pas les membranes pauvres en cholestérol, comme la mono-couche des GL et constituent donc les solutions perméabilisantes de choix pour cette étude. (iii) Des étapes de blocage doivent être introduites afin de minimiser la liaison non spécifique des anticorps primaires ou secondaires qui peut entraîner un bruit de fond, au mieux avec une solution de sérum dérivé de la même espèce qui a servi à produire le conjugué. (iv) Le montage à base de solution polymérisante a tendance à déformer les structures cellulaires en « aplatissant » l’échantillon \citep{RN925}, il est donc préférable de choisir un milieu de montage liquide, bien qu’il ne permette pas une conservation à long terme de l’échantillon. De plus, l’indice de réfraction (IR) du milieu de montage doit être proche de celui de l’objectif du microscope et du milieu d’immersion afin d’éviter la diffraction du faisceaux lumineux et les aberrations sphériques. Notre choix final s’est donc porté sur un milieu de montage liquide à base de glycérol avec un IR de 1.42 - 1.44 identique à l’ouverture numérique (NA) de l’objectif fixée à 1.4. Enfin, les objectifs modernes sont conçus pour être utilisés avec des lamelles d’épaisseur de 160 à 190 µm. Notre support est composé d’une lamelle en verre de haute précision (\#1.5H) avec une épaisseur optimale de 170µm et une variance ne dépassant pas ±5 μm. Le verre a l’avantage d’avoir un IR (1.515) très proche du milieu d’immersion à huile (1.518).

		\subsubsection{Notes sur l’optimisation des paramètres d’acquisition et les limitations de la microscopie confocale}

La microscopie confocale a l’avantage de générer des images tridimensionnelles par sectionnement optique, permettant de reconstruire le volume de l’échantillon. Elle offre la possibilité d’étudier la disposition spatiale et le volume des molécules fluorescentes avec une grande précision, contrairement aux microscopes conventionnels à épifluorescence ou à large champ (\textit{widefield}), dont les images résultent d’une superposition d’éléments nets à l’intérieur du plan focal et d’élements flous à l’extérieur. Cette technologie, toujours basée sur l’optique, reste donc limitée en terme de résolution et régie par un ensemble de lois physiques qui ne peuvent être facilement surmontées par la conception de l’objectif. Imposée par la diffraction de la lumière lorsqu’elle traverse le diaphragme du plan focal, la limite de résolution que l’on appelle communément « barrière de diffraction » restreint la capacité des instruments optiques à distinguer deux objets séparés par une distance latérale inférieure à ~200nm et une distance axiale inférieure à ~400-600nm. Pour surmonter les limites de résolution des microscopes optiques, un algorithme de déconvolution peut être appliqué sur les images digitales après l’acquisition \citep{RN926}. La déconvolution désigne le processus d’inversion de la distorsion optique, ce qui élimine la barrière de diffraction et rétablit le signal original de l’échantillon. Cet algorithme repose sur une fonction mathématique appelée la « fonction d’étalement du point » (PSF), qui calcule théoriquement la distorsion à partir du chemin emprunté par la lumière à travers l’instrument, d’où l’importance de respecter des IR proches. Afin de se placer dans les conditions optimales de déconvolution et de discriminer \textit{in fine} les petites structures, les images doivent être acquises selon le critère d’échantillonnage de Nyquist. Ce critère stipule que la taille des pixels d’une image doit être au minimum 2,3 fois plus petite que la résolution limite de l’objectif, c’est-à-dire la barrière de diffraction. Sur les plans axiaux et latéraux, cela se traduit par des tailles maximales de pixel respectivement d'environ 90nm et de 170-260nm. Afin de minimiser le chevauchement spectral lors de l’acquisition de plusieurs fluorophores, chaque canal a été imagé séquentiellement et des filtres d’émission ont été appliqués à chaque canal (voir \autoref{sec:confocal}).

		\subsection{L’infection par Jad induit une stabilisation et un élargissement local des gouttelettes lipidiques cytosoliques}
		\label{section:stabilisation}

Afin d’évaluer l’effet de l’infection par le VHC sur la biogenèse des GL, j’ai réalisé une analyse quantitative du volume total, du volume moyen et du nombre des GL dans des hépatocytes naïfs et dans des hépatocytes infectés par une souche dérivée de la souche prototypique JFH-1 (Jad) \citep{RN342} par imagerie quantitative à différents temps post-infection (p.i.) (\autoref{fig:fig31}\textcolor{blue}{.A, B}). Les méthodes conventionnelles de révélation des GL par fluorescence reposent sur l’utilisation d’anticorps ciblant les protéines constitutives de la couche superficielle des GL ou sur des dérivés fluorescents d’acides gras abondants dans le corps lipidique de ces organites \citep{RN929}. Parmi les anticorps disponibles dans le commerce, la protéine résidente aux GL la plus fréquemment ciblée est la périlipine, ou PLIN2. Toutefois, une équipe a montré que la protéine Core du VHC déplace PLIN2 de la surface des GL, réduisant son signal en-dessous d’un niveau détectable à la surface des GL recouvertes par Core, rendant ainsi cette méthode incompatible pour notre étude \citep{RN930}. De plus, notre étude se base sur des analyses volumétriques et le marquage indirect des protéines résidentes de la surface des GL, bien qu’il soit efficace pour localiser ces organites, ne constitue pas une méthode rationnelle pour renseigner sur l’éventuelle dérégulation de la biosynthèse des GL, qui ne peut être évaluée qu’avec un marquage constitutif du noyau lipidique. Dans ce contexte, le colorant utilisé dans cette étude (BODIPY 558/568) est un analogue fluorescent de l'acide dodécanoïque, qui se comporte comme un précurseur synthétique des phospholipides et s’incorpore dans les GL au cours de leur formation. Le traitement du BODIPY dure 15h pour s'assurer que les GL nouvellement synthétisées ainsi que celles déjà existantes incorporent efficacement le colorant. La révélation parallèle de la protéine Core sert de marqueur de l’infection.

	\begin{figureth}
	\centering
			\includegraphics[width=\linewidth]{Figure_31.png}
		\caption[Analyse du volume total, du volume moyen et du nombre des gouttelettes lipidiques au cours de l’infection (Partie 1)]{\textbf{Analyse du volume total, du volume moyen et du nombre des gouttelettes lipidiques au cours de l’infection (Partie 1).} (A) Schéma simplifié de l’approche expérimentale pour analyser la biosynthèse des GL cytosoliques au cours de l’infection. Des cellules Huh-7.5 ont été infectées par la souche parentale Jad à une MOI de 1 TCID50/cellule. Les GL ont été révélées par incorporation du BODIPY en cellules vivantes pendant 15h et la protéine Core a été révélée par immunomarquage à l’aide de l’anticorps monoclonal ACAP27. (B) Images 3D représentatives d’une cellule non infectée (panneaux du haut) et d’une cellule infectée par la souche prototypique Jad (panneaux du bas) acquises à 72h p.i. Dans les colonnes de gauche, du milieu et de droite figurent respectivement la révélation de la protéine Core (en cyan), du corps lipidique des GL (en magenta) et la combinaison des marquages de Core et des GL. Les barres d’échelle indiquent 5µm. Sur la droite sont représentés les agrandissements des zones indiquées par les cadres en pointillés blancs. Trois paramètres relatifs aux GL ont été mesurés à l’aide du logiciel Huygens Professional après déconvolution : leur volume total (1), leur volume moyen (2) et leur nombre (3).}
				\label{fig:fig31}
	\end{figureth}
	\FloatBarrier

Premièrement, il n’existe pas de différence significative consistante entre le volume total des GL des cellules infectées par rapport aux cellules non infectées au cours de deux expériences indépendantes, ce qui suggère que l'infection par la souche Jad n'induit pas globalement d’augmentation ou de diminution de la biogenèse des GL (\autoref{fig:fig32}\textcolor{blue}{.A}). Il est intéressant de noter que les cellules acquises lors de la deuxième expérience (représentées par les points triangulaires) ont un contenu globalement plus important en GL. Ceci pourrait être lié à leur évolution, étant issues d’un passage significativement plus élevé depuis leur mise en culture. Bien que le contenu des GL soit globalement inchangé, j’ai mis en évidence une augmentation progressive du volume moyen des GL dans les cellules infectées par rapport aux cellules non infectées (\autoref{fig:fig32}\textcolor{blue}{.B}), corrélée à une diminution proportionnelle de leur nombre (\autoref{fig:fig32}\textcolor{blue}{.C}). L'ensemble de ces résultats suggère que l'infection par la souche Jad induit un élargissement local des GL sans changement de leur contenu global.

			\begin{figureth}
	\centering
			\includegraphics[width=\linewidth]{Figure_32.png}
		\caption[Analyse du volume total, du volume moyen et du nombre des gouttelettes lipidiques au cours de l’infection (Partie 2)]{\textbf{Analyse du volume total, du volume moyen et du nombre des gouttelettes lipidiques au cours de l’infection (Partie 2).} Analyse quantitative du (A) volume total des GL (en µm\up{3}); (B) volume moyen des GL (en µm\up{3}); (C) nombre des GL. Les cellules non infectées sont représentées en gris et les cellules infectées en bleu. L’analyse a été conduite sur 40 cellules au cours de deux expériences indépendantes, respectivement illustrées par les points ronds et triangulaires. La valeur moyenne des expériences correspond au symbole encadré en noir et l’écart-type des valeurs moyennes est représenté. Les tests statistiques ont été réalisés sur R à l’aide d’un modèle mixte linéaire. Les \textit{p-values} ont été ajustées par la méthode de Tukey : p < 0,0001 (****); p < 0,001 (***); p <0,05 (*); p >0,05 (ns : non significatif).}
				\label{fig:fig32}
	\end{figureth}
	\FloatBarrier
	
		\subsection{L’infection par Jad induit une redistribution des gouttelettes lipidiques cytosoliques sous la forme de clusters}
		\label{section:cluster}

Les GL peuvent croître par synthèse locale de lipides ou par fusion homotypique, bien qu'elles présentent une activité fusogène constitutive relativement faible dans des conditions physiologiques \citep{RN555}. À mesure que l'infection progresse, nous pouvons observer une redistribution marquée des GL cytosoliques sous la forme d'agrégats compacts (\autoref{fig:fig33}\textcolor{blue}{.A}). Les GL sont généralement considérées comme des organites sphériques individuels de taille, d'abondance et de composition variables selon le type de cellule \citep{RN876}, mais des GL adjacentes sont capables de se regrouper en établissant des contacts entre elles. Dans les adipocytes, les GL sont fortement regroupées avec en moyenne 50 à 65\% des GL constamment en contact dans des conditions normales de croissance \citep{RN555}. Afin de vérifier mes observations et d'évaluer s'il existe une augmentation avérée du regroupement des GL pendant l'infection, un programme innovant a été développé sur Python spécialement pour ce projet par notre collaborateur D. Ershov (Plateforme d'Analyse en Imagerie, Institut Pasteur). En bref, cet algorithme permet de distinguer automatiquement les GL individuelles des GL dites « appariées », situées à une distance inférieure à 200nm. Le BODIPY ne révélant que la composante lipidique des GL, il était indispensable de prendre en compte que la surface protéique occupe un certain espace. L’idéal serait de déterminer directement l’épaisseur de la couche protéique des GL, ce qui est impossible avec la résolution du système de microscopie utilisé, déjà poussée au maximum avec une taille de pixel latéral fixée à 50nm. Ce seuil « théorique » a donc été soigneusement déterminé par le signal de Core présent à la surface des GL, et correspond à l’ampleur moyenne de deux signaux de Core à la limite du chevauchement (voir \autoref{sec:confocal}).   \\ \\
\indent
À l’aide de ce nouvel outil, nous avons évalué un taux de regroupement des GL qui varie entre 35 et 55\% dans les hépatocytes en conditions normales de croissance. Cette fourchette dépend principalement du nombre de passages des cellules depuis leur mise en culture. Comme évoqué lors de la \autoref{section:stabilisation}, lorsque les hépatocytes en culture vieillissent, ils ont tendance à accumuler plus de GL, ce qui pourrait augmenter \textit{a posteriori} les évènements de contacts. Il est intéressant de noter que ce taux est globalement plus faible que dans les adipocytes, peut-être en raison du stock moins abondant de GL dans les hépatocytes par rapport à ces cellules hautement spécialisées dans le stockage des lipides. Suite à l'infection, on observe une augmentation progressive du pourcentage de GL formant des contacts étroits, indiquant que l'infection par Jad favorise en effet la formation de clusters de GL (\autoref{fig:fig33}\textcolor{blue}{.B}). Cette clusterisation « forcée » des GL lors de l’infection pourrait donner lieu à des sites de contact prolongés entre les GL, ce qui pourrait déstabiliser leur tension superficielle et aboutir finalement à la fusion des deux corps lipidiques. Ce mécanisme serait cohérent avec l'émergence de GL plus larges et moins nombreuses, décrites dans la \Autoref{fig:fig32}.

			\begin{figureth}
	\centering
			\includegraphics[width=\linewidth]{Figure_33.png}
		\caption[Analyse de la clusterisation des gouttelettes lipidiques cytosoliques au cours de l’infection]{\textbf{Analyse de la clusterisation des gouttelettes lipidiques cytosoliques au cours de l’infection.} (A) Images 3D représentatives de cellules non infectées (panneaux du haut) et de cellules infectées par Jad (panneaux du bas) acquises à 24h, 48h, 72h et 96h p.i. Les images sont une combinaison de la révélation de la protéine Core par l’anticorps monoclonal ACAP27 (en cyan) et du corps lipidique des GL (en magenta) par incorporation du BODIPY. Le taux de clusterisation moyen des GL pour chaque condition est indiqué en bas à droite. La barre d’échelle indique 5µm. (B) Analyse quantitative du taux de GL appariées aux temps p.i. indiqués. Les cellules non infectées sont représentées en gris et les cellules infectées en bleu. L’analyse a été conduite sur 40 cellules au cours de deux expériences indépendantes. Les tests statistiques ont été réalisés sur R à l’aide d’un modèle mixte linéaire. Les \textit{p-values} ont été ajustées par la méthode de Tukey : p < 0,0001 (****); p <0,01 (**); p >0,05 (ns : non significatif).}
				\label{fig:fig33}
	\end{figureth}
	\FloatBarrier
	
		\subsection{Analyse des gènes impliqués dans la biosynthèse et dans l’expansion des gouttelettes lipidiques}
		\label{section:genes}	

Comme décrit dans la \autoref{section:formationgl}, de nombreux facteurs cellulaires régulent les cycles de formation et de dégradation des GL. Dans l’objectif d’identifier des dérégulations cellulaires qui pourraient sous-tendre les mécanismes de regroupement et d’élargissement des GL induits lors de l’infection, j’ai réalisé une analyse à haut-débit du profil transcriptomique d’hépatocytes infectés par Jad à 96h p.i. Le temps le plus tardif étudié lors de notre analyse par imagerie a été choisi, en raison des processus d’agrégation et d’élargissement très marqués. Bien que les enzymes impliquées dans la régulation des GL diffèrent selon les types cellulaires, la plupart des gènes mis en évidence dans la \Autoref{fig:fig34} constituent des facteurs biologiques décrits comme pertinents dans le système hépatique. Certains facteurs ont été décrits dans d’autres types cellulaires, en particulier dans les adipocytes et leur implication dans les hépatocytes n’a pas encore été démontrée. Toutefois, leur expression étant suffisamment notable en cellules Huh-7.5, il nous a semblé intéressant de les inclure dans l’analyse. De plus, certaines enzymes de biosynthèse des triglycérides ou esters de stérol ont été retrouvés dans le protéome des GL de la lignée hépatocytaire Huh-7.5 lors d’une récente étude \citep{RN641} et ont donc également été inclus dans l’analyse (\textit{LSS, NSDHL, SQLE}). On distinguera les gènes impliqués dans la synthèse \textit{de novo} des triglycérides, des esters de stérol et des phospholipides initiant la formation des GL cytosoliques à partir du RE (\autoref{fig:fig34}\textcolor{blue}{.A}), des gènes impliqués dans l’expansion des GL pré-établies, soit par synthèse locale, soit par fusion (\autoref{fig:fig34}\textcolor{blue}{.B}). \\ \\
\indent
Premièrement, l’expression de la majorité des enzymes responsables de la synthèse \textit{de novo} des GL est régulée négativement lors de l’infection, à l’exception de \textit{GPAT3} et \textit{PLPP3} (\autoref{fig:fig34}\textcolor{blue}{.A}), ce qui est cohérent avec une récente étude menée avec des hépatomes différenciés et des biopsies de foie de patients infectés \citep{RN934}. Nous pourrions donc nous attendre à une réduction du contenu global des GL, ce que nous n'observons pas dans l'analyse par imagerie. Une faible synthèse des GL pourrait être compensée par une dégradation réduite. Par conséquent, comme potentiel effet compensatoire, les gènes hépatiques impliqués dans la dégradation des GL par lipolyse  ou lipophagie ont été examinés et une régulation négative des principales lipases hépatiques PNPLA2, PNPLA3, LIPC, MGL et LIPA a été mise en évidence (\autoref{fig:fig34}\textcolor{blue}{.C}). Ces résultats indiquent que l’infection par Jad atténue les voies de biosynthèse, de lipolyse et de lipophagie des GL sans conséquence directe sur leur contenu global. Ainsi, le renouvellement habituellement dynamique des GL serait affecté par un ralentissement des cycles de formation et de dégradation, résultant en une rétention prolongée des GL pré-existantes.

			\begin{figureth}
	\centering
			\includegraphics[width=\linewidth]{Figure_34.png}
		\caption[Analyse du profil d’expression des gènes hépatiques de la biogenèse, de l’expansion et de la lipolyse des GL au cours de l’infection par Jad]{\textbf{Analyse du profil d’expression des gènes hépatiques de la biogenèse, de l’expansion et de la lipolyse des GL au cours de l’infection par Jad.} Analyse du profil transcriptomique d’hépatocytes infectés, restreinte aux principaux gènes hépatiques impliqués dans la biogénèse, l’expansion et la lipolyse des GL. Des extraits d’ARN totaux collectés à partir de cellules Huh-7.5 non infectées ou infectées par Jad à une MOI de 3 TCID50/cellule pendant 96h p.i. ont été analysés par RNA-Seq au cours de trois expériences indépendantes. Les valeurs d’expression des gènes (\textit{reads}) de la condition infectée ont été normalisées par rapport à la condition non infectée contrôle et sont indiquées en Log2FC (\textit{FoldChange}). Les gènes régulés positivement et négativement sont respectivement indiqués en rouge et bleu. (A) Principaux gènes hépatiques impliqués dans la synthèse \textit{de novo} des triglycéride, esters de stérol et phospholipides initiant la formation les GL cytosoliques ; (B) Principaux gènes hépatiques impliqués dans la synthèse locale des lipides neutres ou régulant la clusterisation et la fusion des GL ; (C) Principaux gènes hépatiques impliqués dans la lipolyse et la lipophagie des GL.}
				\label{fig:fig34}
	\end{figureth}
	\FloatBarrier

Deuxièmement, l’expression des principales enzymes responsables de l’expansion des GL par synthèse locale  des TAG ou des esters de stérol est également réduite lors de l’infection (\autoref{fig:fig34}\textcolor{blue}{.B}). En parallèle, l’expression de \textit{PLIN2} est hautement induite au cours de l’infection. Comme évoqué lors de la \autoref{section:recrutement}, les périlipines, en particulier PLIN2 dans le tissu hépatique, régulent le recrutement des protéines à la surface des GL par encombrement stérique. L’augmentation de l’expression de \textit{PLIN2} pourrait contribuer à stabiliser les GL en réduisant davantage la surface des GL accessible aux lipases ou aux protéines de synthèse locale des lipides neutres. Dans ce contexte, l’hypothèse d’un élargissement des GL par induction d’une synthèse locale des lipides neutres paraît donc peu probable. \\ \\
\indent
Par l’intermédiaire de cette analyse transcriptomique, j’ai confirmé que les gènes responsables de la synthèse des protéines CIDEA et CIDEC ne sont pas exprimés en cellules Huh-7.5, comme il a été montré dans d’autres lignées hépatocytaires, telle que HepG2 \citep{RN558}. Ces données ainsi que celles de la littérature soulignent le fait que CIDEB représente l’unique membre du complexe CIDE exprimé dans le tissu hépatique. Toutefois, nous pouvons constater que l’expression de \textit{CIDEB} est fortement réduite lors de l’infection, ce qui suggérerait que l’élargissement des GL résulterait d’un mécanisme indépendant de CIDEB. On peut noter par exemple une augmentation modérée de l’expression de \textit{LDAH}, qui code une enzyme régulant la synthèse des triglycérides, récemment impliquée dans la coalescence des GL dans des cellules HEK293 \citep{RN560}. De même, des complexes protéiques alternatifs, tels que les homodimères AUP1, responsables d’un pontage entre GL adjacentes, ont été identifiés dans des types cellulaires n’exprimant pas les protéines CIDE \citep{RN596}. Bien que le rôle de AUP1 dans la formation des clusters de GL soit avéré, sa potentielle implication dans un mécanisme de fusion des GL n’a pas encore été mis en évidence. \textit{AUP1} est exprimé dans le modèle cellulaire Huh-7.5 et son expression est significativement induite lors de l’infection. Ces résultats suggèrent que ces facteurs cellulaires pourraient avoir une implication potentielle d’une part dans l’agrégation et d’autre part dans l’élargissement des GL par fusion à mesure que l’infection progresse. Toutefois, il n’est pas exclu que ces processus soient issus d’un mécanisme directement médié par les protéines virales, indépendamment ou conjointement aux facteurs cellulaires.
	
		\subsection{Analyse de l’interface entre la protéine Core et les gouttelettes lipidiques cytosoliques}
		\label{section:coregl}		

\subsubsection{La protéine Core s’accumule progressivement à la surface des gouttelettes lipidiques cytosoliques au cours de l’infection}

Après maturation, l’association de la protéine Core du VHC à la surface des GL est une étape indispensable pour initier la morphogénèse virale \citep{RN453,RN489}. Dans le cas du virus JFH-1, Core est majoritairement localisée à la surface des GL, induisant un parfait enveloppement de ces organites \citep{RN486,RN930}. En raison de cette interaction particulière, Core pourrait représenter un facteur viral majeur dans la perturbation de la morphologie et de la dynamique des GL. Au cours d’anciennes études, Core a déjà été incriminée dans la clusterisation et l’élargissement des GL dans des systèmes d’expression ectopique \citep{RN943,RN947} ou dans des modèles d’infection \citep{RN930} sans que les mécanismes et le rôle biologique de ces processus soient entièrement élucidés. Dans la perspective d’apporter de nouvelles informations sur les mécanismes viraux à l’origine de l’agrégation et de l’élargissement des GL et de comprendre l’implication biologique de ces processus dans le cycle viral, j’ai analysé en détails l’interface entre la protéine Core et les GL au cours de l’infection d’hépatocytes par la souche Jad. \\ \\
\indent
Tout d'abord, Core présente des taux d'enveloppement hétérogènes sur les GL au cours de l'infection, sous la forme de petits puncta lors des temps précoces, et d'un aspect annulaire prédominant aux temps plus tardifs (\autoref{fig:fig35}\textcolor{blue}{.A}). En parallèle des observations, l’analyse quantitative révèle que ~70\% des GL sont mobilisées par Core dès 24h p.i., suggérant que la majorité du contenu en GL est rapidement colonisée par Core après l'infection (\autoref{fig:fig35}\textcolor{blue}{.B}). La proportion des GL associées à Core augmente à 48h p.i. pour se stabiliser à ~85\%, ce qui montre qu’une grande partie des GL est constamment mobilisée par le virus. Cette interaction persistante avec un facteur viral pourrait être responsable des effets adverses sur leur morphologie ou leur dynamique, décrits précédemment lors des sections \ref{section:stabilisation} et \ref{section:cluster}. De plus, le volume de Core présent à la surface des GL augmente drastiquement entre 24 et 48h p.i., puis avec une légère tendance à la hausse, non significative, pour les points plus tardifs (\autoref{fig:fig35}\textcolor{blue}{.C}). Bien que la majorité des GL soit rapidement colonisée par Core dès 24h p.i., ces résultats indiquent que Core s'accumule progressivement à la surface des GL au cours de l'infection, d’où l’évolution du profil vers un « anneau » plus complet. Lors d’une analyse quantitative complémentaire (non montrée), j’ai mis en évidence que le recouvrement de la surface des GL par Core n’atteignait jamais un taux de 100\%, ce qui est cohérent avec le profil annulaire illustré par les images représentatives, plutôt qu’un recouvrement total de la surface. Ces données suggèrent que Core pourrait être préférentiellement recrutée dans certaines zones de la surface des GL. Cela pourrait être attribuable à la présence initiale d'un ou plusieurs facteur(s) d’hôte recrutant Core à la surface des GL ou d'une affinité pour des phospholipides spécifiques présents sur la monocouche des GL, impliquant que la composition des GL soit hétérogène. \\ \\
\indent

				\begin{figureth}
	\centering
			\includegraphics[width=0.85\linewidth]{Figure_35.png}
		\caption[Interaction spatiale et temporelle entre la protéine Core et les GL cytosoliques au cours de l’infection]{\textbf{Interaction spatiale et temporelle entre la protéine Core et les GL cytosoliques au cours de l’infection.} \textit{La légende de la figure est décrite sur la page suivante \footnotemark[1].}}
				\label{fig:fig35}
	\end{figureth}
	\FloatBarrier

\subsubsection{La population des gouttelettes lipidiques recouvertes par la protéine Core est sujette à l’élargissement local}

\footnotetext[1]{(A) Images 3D représentatives de cellules infectées par Jad acquises aux temps p.i. indiqués. Dans les panneaux du haut figurent la révélation de la protéine Core par l’anticorps monoclonal ACAP27 (en cyan). Dans les panneaux en-dessous figurent la révélation du corps lipidique des GL (en magenta) par incorporation du BODIPY et une combinaison des marquages de Core et des GL. Les panneaux du bas représentent un agrandissement des zones indiquées par les cadres en pointillés blancs, mettant en valeur l’aspect de Core à la surface des GL. Les barres d’échelle indiquent 5µm. (B) Analyse quantitative du taux de GL mobilisées par Core aux temps indiqués (en \% par rapport au nombre total de GL). (C) Analyse quantitative du volume de Core associé aux GL aux temps indiqués (en µm\up{3}). L’analyse a été conduite sur 40 cellules au cours de deux expériences indépendantes. Les tests statistiques ont été réalisés sur R à l’aide d’un modèle mixte linéaire. Les \textit{p-values} ont été ajustées par la méthode de Tukey : p <0,0001 (****); p >0,05 (ns : non significatif).}

Le taux d’enveloppement des GL par Core est très hétérogène parmi les cellules infectées, d’où l’importance d’acquérir un grand nombre d'images tridimensionnelles de cellules uniques afin d’aboutir à une analyse statistiquement fiable. Ce taux varie également au sein d’une même cellule, y compris à des temps tardifs, comme on peut le constater sur cet exemple d’une cellule infectée acquise à 96h p.i. (\autoref{fig:fig36}\textcolor{blue}{.A}). Les GL ont été décrites comme des organites hautement dynamiques qui se renouvellement constamment en réponse à la présence élevée d'acides gras dans la bicouche de la membrane du RE \citep{RN645}. Les GL pourraient ainsi subir des cycles de recolonisation par Core, ce qui expliquerait l’existence de GL encore peu, voire pas du tout recouvertes à des temps tardifs d’infection. Toutefois, si on se base sur notre précédente hypothèse, l’infection par Jad semble résulter en une stabilisation des GL pré-existantes, ce qui pourrait avoir comme implication biologique d’assurer et de préserver des plateformes de stockage pour la protéine de capside.

				\begin{figureth}
	\centering
			\includegraphics[width=\linewidth]{Figure_36.png}
		\caption[Analyse des populations de gouttelettes lipidiques cytosoliques recouvertes ou non par la protéine Core lors de l’infection]{\textbf{Analyse des populations de gouttelettes lipidiques cytosoliques recouvertes ou non par la protéine Core lors de l’infection.} (A) Image 3D représentative d’une cellule infectée par Jad acquise à 96h p.i. La protéine Core est révélée par l’anticorps monoclonal ACAP27 (en cyan) et le corps lipidique des GL est révélé par incorporation du BODIPY (en magenta). L’agrandissement des zones indiquées par les cadres en pointillés blancs à droite, met en évidence l’hétérogénéité du recouvrement des GL cytosoliques par la protéine Core. La barre d’échelle indique 5µm. (B) Analyse quantitative du volume moyen (en µm\up{3}) des GL « nues » (en rose) et des GL associées à Core (en bleu) aux temps p.i. indiqués. L’analyse a été conduite sur 40 cellules au cours de deux expériences indépendantes.}
				\label{fig:fig36}
	\end{figureth}
	\FloatBarrier
\clearpage
	
Au sein d'une même cellule infectée peuvent donc co-exister des GL associées à Core, ainsi qu'une population minoritaire de GL « nues ». En sachant que ces deux populations de GL co-habitent, une analyse quantitative du volume moyen des GL en discriminant les GL portant ou non Core à leur surface a été effectuée (\autoref{fig:fig36}\textcolor{blue}{.B}). Cette étude révèle que Core est remarquablement associée aux GL les plus larges et que seule la population de GL colonisées par Core est sujette à un élargissement progressif au cours du temps. Ainsi, ces nouvelles données soutiennent l’hypothèse que Core participe aux processus d’agrégation et d’élargissement des GL observés dans les hépatocytes infectés.

\subsubsection{La protéine Core est particulièrement localisée au niveau des contacts formés entre les gouttelettes lipidiques}

À l’aide de notre algorithme personnalisé sur Python, il est possible de comptabiliser le nombre des contacts formés entre les GL dites « appariées » situées à une distance inférieure à 200nm (\autoref{fig:fig37}\textcolor{blue}{.A}). Au sein des cellules infectées, le nombre de contacts entre les GL augmente progressivement au cours du temps, avec 0,68 contact en moyenne par GL à 24h p.i. et 0,92 contact(s) en moyenne par GL à 96h p.i., ce qui est parfaitement cohérent avec la hausse du taux de GL appariées décrite dans la \autoref{section:cluster}. Lorsque nous examinons les paires ou les clusters de GL plus en détails, nous pouvons remarquer que Core a tendance à se localiser au niveau des zones de contact entre les GL appariées. Sur l’exemple d’une image représentant le cytoplasme d’une cellule acquise à 24h p.i. (\autoref{fig:fig37}\textcolor{blue}{.B}), les puncta de Core semblent coïncider fréquemment avec les zones de contact formées entre GL voisines. Ce profil est moins évident sur la cellule acquise à 96h p.i., en raison de l'accumulation progressive de Core à la surface de l’ensemble de la GL, jusqu'à cette forme particulière en « anneau » que nous avons précédemment décrite dans la \Autoref{fig:fig35}.

				\begin{figureth}
	\centering
			\includegraphics[width=\linewidth]{Figure_37.png}
		\caption[Analyse des zones de contact entre les gouttelettes lipidiques appariées lors de l’infection (Partie 1)]{\textbf{Analyse des zones de contact entre les gouttelettes lipidiques appariées lors de l’infection (Partie 1).} (A) Analyse quantitative du nombre de contacts formés par les GL au cours de l’infection. Les cellules non infectées sont représentées par des boîtes grises et les cellules infectées par des boîtes bleues. Les tests statistiques ont été réalisés sur R à l’aide d’un modèle mixte linéaire. Les \textit{p-values} ont été ajustées par la méthode de Tukey : p <0,05 (*); p <0,001 (***); p < 0,0001 (****); p >0,05 (ns : non significatif). (B) Images représentatives du cytoplasme des cellules infectées par Jad acquises à 24h (à gauche) ou à 96h p.i. (à droite), mettant en évidence la présence de la protéine Core au sein des contacts entre les paires ou clusters de GL cytosoliques. La protéine Core est révélée par l’anticorps monoclonal ACAP27 (en cyan) et le corps lipidique des GL est révélé par incorporation du BODIPY (en magenta). Les paires ou clusters de GL sont mises en valeur avec une aire en pointillés roses et les zones de contact sont indiquées par des flèches bleues.}
				\label{fig:fig37}
	\end{figureth}
	\FloatBarrier

Afin de confirmer ces observations, le signal de Core contenu au sein des zones de contact a été extrait et quantifié indépendamment du signal de Core localisé hors de ces contacts, sur le reste de la surface des GL (\autoref{fig:fig38}\textcolor{blue}{.A}). Étant donné que les zones de contact représentent généralement de très petites régions par rapport à la surface globale des GL, cette discrimination est pertinente uniquement lorsque l’on compare la moyenne de la valeur des pixels et non la somme de la valeur des pixels. Ce paramètre a donc servi d'indication directe de l'accumulation de Core dans une zone donnée. Grâce à cette méthode, nous avons pu mettre en évidence que le signal moyen de Core au sein des zones de contact est significativement supérieur à celui du reste de la surface des GL à chaque temps p.i. étudié (\autoref{fig:fig38}\textcolor{blue}{.B}). L’écart entre le signal Core à l’intérieur et hors des sites de contact est particulièrement marqué à 24h p.i., ce qui valide nos observations précédentes et confirme que Core serait initialement concentrée au sein des contacts entre GL. Cette différence significative s’amenuise aux temps les plus tardifs, en raison de la colonisation progressive de l’ensemble de la surface des GL par Core.

				\begin{figureth}
	\centering
			\includegraphics[width=0.50\linewidth]{Figure_38.png}
		\caption[Analyse des zones de contact entre les gouttelettes lipidiques appariées lors de l’infection (Partie 2)]{\textbf{Analyse des zones de contact entre les gouttelettes lipidiques appariées lors de l’infection (Partie 2).} (A) Illustration du procédé pour discriminer le signal de la protéine Core au sein et hors des zones de contact formés entre deux GL cytosoliques appariées (situées à une distance inférieure à 200nm), par l’algorithme développé sur Python. (B) Intensité du signal de Core à l’intérieur et hors des zones de contact entre GL appariées aux différents temps p.i. indiqués sous forme de \textit{heatmap}. L’analyse a été conduite sur 40 cellules au cours de deux expériences indépendantes. Les tests statistiques ont été réalisés sur R à l’aide d’un modèle mixte linéaire. Les \textit{p-values} sont ajustées par la méthode de Tukey : p <0,01 (**); p <0,001 (***); p < 0,0001 (****).}
				\label{fig:fig38}
	\end{figureth}
	\FloatBarrier

\subsubsection{Les clusters des gouttelettes lipidiques cytosoliques sont redistribués à proximité des usines de réplication virale}

Des études complémentaires de microscopie électronique à transmission (MET) effectuées par notre collaborateur P. Roingeard (Inserm U1259, Tours) montrent une redistribution des clusters de GL à proximité des réseaux membranaires dérivés du RE hébergeant les usines de réplication virale (\autoref{fig:fig39}). Cette redistribution des GL pourrait avoir plusieurs implications biologiques, comme par exemple faciliter la rencontre de Core, stockée à leur surface, et des génomes néosynthétisés pour l’assemblage des particules virales.

				\begin{figureth}
	\centering
			\includegraphics[width=\linewidth]{Figure_39.png}
		\caption[Étude de la redistribution des clusters de gouttelettes lipidiques cytosoliques à proximité des remodelages membranaires du réticulum endoplasmique induits par l’infection]{\textbf{Étude de la redistribution des clusters de gouttelettes lipidiques cytosoliques à proximité des remodelages membranaires du réticulum endoplasmique induits par l’infection.} Micrographies d’hépatocytes non infectés (à gauche) et infectés par la souche Jad (à droite) acquises par MET à 120h p.i. Les gouttelettes lipidiques cytosoliques des images agrandies sur la droite ont été colorisées en rose. Les flèches jaunes mettent en valeur les DMV issues des remodelages membranaires du RE par l’infection. Les barres d'échelle indiquent 2µm.}
				\label{fig:fig39}
	\end{figureth}
	\FloatBarrier

Dans l’objectif d'obtenir des données quantitatives de l’interface entre les populations de GL cytosoliques recouvertes par Core et les usines de réplication virale, j'ai réalisé un co-marquage de la protéine Core, des GL et de l'ARN viral en utilisant des sondes fluorescentes spécifiques de l'ARN de polarité positive (ARN(+)) ou négative (ARN(-)) par méthode d'hybridation \textit{in situ} (\autoref{fig:fig40}\textcolor{blue}{.A}). En guise de marqueur des usines de réplication virale, l’ARN (-) qui sert d'intermédiaire lors de la réplication génomique a été privilégié, plutôt que l’ARN (+), qui ne permet pas de distinguer les génomes destinés à être encapsidés des molécules synthétisées pour la traduction des précurseurs polyprotéiques. Comme contrôle interne de la spécificité des sondes virales, nous avons inclus une sonde arbitraire ciblant l’ARNm de l’actine ß. Les signaux de l’ARN viral (+) et de l’ARN viral (-) se présentent sous la forme de puncta plus ou moins intenses diffus dans le cytoplasme, bien que le contenu en ARN  (+) soit plus abondant qu'en ARN (-), en raison du besoin important en molécules servant à la traduction ribosomale et à l'encapsidation (\autoref{fig:fig40}\textcolor{blue}{.B}). En revanche, nous pouvons constater que le signal de la protéine Core n’est pas identique à celui des précédentes images, et se présente sous la forme de puncta ou d’agrégats dissociés des GL. Pour vérifier si seul le marquage de Core était affecté par cette méthode, j’ai révélé indépendamment PLIN2, une protéine résidente de la surface des GL (données non montrées). Le marquage de PLIN2 était également chaotique en présence de sondes ARN virales, indiquant que la surface des GL est endommagée par cette technique. Les réactifs nécessaires à l’hybridation et à l’amplification des sondes contiennent en effet des détergents qui peuvent déstabiliser irrémédiablement la membrane des GL composée d’une monocouche de phospholipides très sensible aux molécules tensioactives, rendant cette méthode incompatible avec le marquage de Core. Malgré les nombreuses tentatives pour améliorer les conditions expérimentales en réduisant les temps de traitement ou en remplaçant certains produits détergents par des solutions de lavage neutres sans perdre le signal des sondes, je ne suis pas parvenue à rétablir les marquages corrects de Core ou de PLIN2, qui signeraient l’intégrité de la membrane externe des GL. Bien que la couche superficielle soit endommagée, le corps lipidique des GL révélé par incorporation du BODIPY semble conserver sa morphologie. Si on se focalise uniquement sur la localisation du signal des sondes et des GL, nous pouvons constater que certains puncta correspondant à l’ARN viral (-) ou (+) co-localisent de façon notable avec des paires ou des clusters de GL (\autoref{fig:fig40}\textcolor{blue}{.C, flèches jaunes}). Toutefois, en l’absence du marquage de Core, nous ne pouvons pas identifier la nature de la population des GL concernée par cette co-localisation. Dans l’objectif de répondre à cette question, nous tentons à l’heure actuelle une méthode alternative à l’hybridation in situ, basée sur un anticorps non spécifique ciblant l’ARN double brin \citep{RN451,RN946} ce qui permettrait de révéler les intermédiaires de réplication et ainsi de servir de marqueur subsidiaire des usines de réplication virale.

				\begin{figureth}
	\centering
			\includegraphics[width=\linewidth]{Figure_40.png}
		\caption[Tentative d’étude de l’interface entre la protéine Core, les gouttelettes lipidiques cytosoliques et l’ARN viral par hybridation \textit{in situ}.]{\textbf{Tentative d’étude de l’interface entre la protéine Core, les gouttelettes lipidiques cytosoliques et l’ARN viral par hybridation \textit{in situ}.} \textit{La légende de la figure est décrite sur la page suivante \footnotemark[2].}}
				\label{fig:fig40}
	\end{figureth}
	\FloatBarrier

		\subsection{Étude des corrélations entre les paramètres relatifs à Core et aux gouttelettes lipidiques cytosoliques}
		\label{section:correlation}

\footnotetext[2]{(A) Schéma de l’approche expérimentale pour analyser l’interface entre les gouttelettes lipidiques cytosoliques associées à la protéine Core et l’ARN viral. Des cellules Huh-7.5 ont été infectées par la souche parentale Jad à une MOI de 1 TCID50/cellule. Les GL ont été révélées par incorporation du BODIPY en cellules vivantes et la protéine Core a été révélée par immunomarquage à l’aide de l’anticorps monoclonal ACAP27. Les ARN ont été révélés par hybridation in situ en utilisant des sondes fluorescentes spécifiques des régions de l’ARN viral négatif (-) ou positif (+) de la souche Jad ou de l’ARNm cellulaire de l’actine ß (ACTB). (B) Images 3D représentatives de cellules acquises à 72h p.i. Dans les panneaux de gauche à droite figurent respectivement les révélations individuelles de la protéine Core (en cyan), des GL (en magenta), des ARN viraux ou cellulaire (en jaune) puis la combinaison des trois co-marquages. Les agrandissements à droite des zones indiquées par les cadres en pointillés blancs mettent en valeur l’anomalie du signal de la protéine Core, en présence de sondes. Les barres d’échelle indiquent 10µm. (C) Co-marquages des GL et des ARN viraux (-) (à gauche) ou (+) (à droite) uniquement. Les agrandissements à droite des zones indiquées par les cadres en pointillés blancs mettent en valeur une co-localisation partielle entre l’ARN viral (-) ou (+) avec les clusters de GL.}

En parallèle, j’ai effectué une analyse de corrélation entre tous les paramètres relatifs à Core et aux GL quantifiés au cours de cette étude (\autoref{fig:fig41}\textcolor{blue}{.A}). Ce diagramme de corrélation nous permet d’objectiver certaines hypothèses soulevées lors des analyses précédentes. Par exemple, nous pouvons constater qu’il existe un effet modeste mais significatif de l’expérience (EXPT) sur le volume moyen, le diamètre et le volume total des GL. Comme évoqué précédemment, il s’agirait d’un effet aléatoire inhérent à tout système biologique vivant, qui dépendrait directement de la physiologie des cellules au moment de l’expérience. Celle-ci serait influencée par le nombre de passages depuis la mise en culture. Nous pouvons également confirmer que la progression de l’infection (TIME) est corrélée à une diminution du nombre de GL (NUM) et à une augmentation de leur volume moyen (MOY) et de leur diamètre (DIAM), du taux de GL appariées (PAIR) et du nombre de zones de contact entre GL (ZONE). De même, la progression de l’infection corrèle avec une augmentation globale des paramètres relatifs à Core, tels que le volume total de Core (CORE), le volume de Core associé aux GL (INTR) et le volume de Core présent hors et dans les sites de contact (OUT, INS). Enfin, nous pouvons remarquer une corrélation négative entre le volume moyen des GL (MOY) et leur nombre (NUM), ce qui est cohérent avec les résultats décrits dans la \autoref{section:stabilisation}. Il est important de préciser que de nombreux paramètres relatifs aux GL sont mathématiquement liés et présentent donc des corrélations parfaites, telles que le taux de GL appariées (PAIR) avec le nombre de zones de contact (ZONE) et le signal de Core hors (OUT) et dans (INS) les zones de contact. \\ \\
\indent
Ayant conscience de l’existence de ces paramètres analogues, je me suis penchée sur l’étude de corrélation directe entre les paramètres les plus indépendants possibles, c’est-à-dire les paramètres relatifs aux GL par rapport à ceux de Core. En premier, nous pouvons remarquer une corrélation très positive entre le volume total des GL (TOT) et le volume total de Core (CORE), à tous temps p.i. confondus. Cela signifierait que les hépatocytes ayant un contenu plus élevé en GL seraient généralement les cellules avec la plus grande quantité de protéine Core (\autoref{fig:fig41}\textcolor{blue}{.B}). Sachant que la biogenèse des GL n’est pas induite par l’infection, j’émets donc l’hypothèse que la protéine Core serait stabilisée lorsque la teneur en GL est initialement grande. Ces populations d'hépatocytes « gras », avec un contenu initial élevé en GL, pourraient potentiellement représenter les cellules produisant davantage de particules virales. Ensuite, nous pouvons constater une corrélation très positive entre le volume de Core recouvrant les GL (INTR) et le nombre de zones de contact formées entre les GL (ZONE) (\autoref{fig:fig41}\textcolor{blue}{.C}). Cette relation confirme que la protéine Core serait directement liée à la formation des clusters de GL. En effet, plus la protéine Core s’accumule à la surface des GL, plus l’interface disponible pour initier un contact avec une autre GL adjacente augmente.

				\begin{figureth}
	\centering
			\includegraphics[width=\linewidth]{Figure_41.png}
		\caption[Analyse de corrélation entre les paramètres relatifs à Core et aux gouttelettes lipidiques]{\textbf{Analyse de corrélation entre les paramètres relatifs à Core et aux gouttelettes lipidiques.} (A) Diagramme de corrélation entre tous les paramètres relatifs à Core, aux GL et à l’expérience. Les corrélations significativement positives (en bleues) ou négatives (en rouges) entre les paramètres apparaissent colorées selon l’échelle de teinte représentée sur la droite du diagramme. Les corrélations non significatives laissent un carré vide. La légende de chaque paramètre étudié est décrite dans un cadre en pointillé à droite du diagramme. (B) Histogrammes (à gauche) et régressions linéaires (à droite) du volume total de Core (CORE) par rapport au volume total des GL (TOT). (C) Histogrammes (à gauche) et régressions linéaires (à droite) du volume de Core recouvrant les GL (INTR) par rapport au nombre de zones de contact formée entre les GL (ZONE). La matrice de corrélation, les histogrammes et les régressions linéaires ont été produits sur R.}
				\label{fig:fig41}
	\end{figureth}
	\FloatBarrier

		\subsection{La protéine Core est responsable de l’agrégation, de l’élargissement et de la redistribution des gouttelettes lipidiques cytosoliques}
		\label{section:mutants}
		
Dans l’objectif de valider le rôle instrumental de Core dans l’agrégation et l’élargissement des GL, l'étude suivante s’est basée sur des mutants viraux codant une protéine Core incapable de s'associer aux GL. Cette stratégie consiste à perturber l'étape clé que nous soupçonnons être critique pour initier la formation des clusters. Par mutagenèse dirigée, j’ai donc cherché à introduire dans l’ADNc viral des mutations ciblant des régions codantes de la protéine Core précédemment décrites dans la littérature comme étant importantes pour l’adressage de cette protéine aux GL. Une première option était de cibler le site de clivage par la SPP à la jonction entre les domaines D2 et D3 afin d’inactiver la maturation de la protéine Core et de contraindre sa rétention dans le RE. Cette approche devrait permettre de maintenir la stabilité de la protéine Core, alors retenue au sein des membranes du RE et protégée de l’activité protéasomale. Une deuxième option était de cibler les hélices amphipathiques du domaine D2 de la protéine Core, responsables de son ancrage dans la couche phospholipidique des GL \citep{RN304}. Le risque potentiel de cette seconde stratégie était de produire une protéine Core mature libérée dans le cytoplasme et qui serait sensible à la dégradation par la machinerie protéasomale.

	\subsubsection{Construction et caractérisation d’ADNc viraux codant des mutations dans le site de clivage par la SPP}

Notre équipe a montré pour la première fois que le clivage par la SPP impliquant le résidu Ser149 était essentiel à la production de particules du GBV-B, un autre membre du genre Hepacivirus \citep{RN951}. Pour la protéine Core du VHC (souche JFH-1), le site du clivage de maturation a plus tard été identifié au niveau du résidu Phe177 \citep{RN955}. En revanche, aucun motif consensus précis du peptide signal reconnu par la SPP n’a été mis en évidence au sein de la protéine Core à ce jour. Le peptide signal serait globalement caractérisé par une région polaire de 7 à 13 acides aminés, suivi d’une région hydrophobe de 6 à 15 acides aminés importante pour induire une « cassure » optimale de l’hélice \citep{RN956,RN959}. Pour la protéine Core du VHC de génotype 1a, plusieurs études ont mis en évidence l’importance des résidus A180, S183 et C184 pour le clivage par la SPP \citep{RN303,RN958,RN962}. Plus tard, il a été montré que les substitutions de ces résidus vers des résidus Val/Leu ou Ala dans la souche JFH-1 n’étaient pas suffisantes pour abolir le clivage de la SPP \citep{RN953,RN306}. Cependant, une mutation supplémentaire au niveau du résidu T186 affectait la maturation de la protéine Core du JFH-1 par la SPP et son adressage aux GL \citep{RN306}. Dans ce contexte, j’ai généré un triple mutant Jad/C_3aa_SPP codant les mutations A180V, S183L et C184V au sein de Core, ainsi qu’un quadruple mutant Jad/C_4aa_SPP qui porte les mutations A180V, S183L, C184V et T186L (\autoref{fig:fig42}).

				\begin{figureth}
	\centering
			\includegraphics[width=0.85\linewidth]{Figure_42.png}
		\caption[Construction des génomes viraux codant des mutations dans le domaine prédictif de clivage par la SPP]{\textbf{Construction des génomes viraux codant des mutations dans le domaine prédictif de clivage par la SPP.} (A) Représentation schématique du génome de la souche Jad et des sous-domaines structuraux de la protéine Core. Les domaines D1 et D2 (acides aminés 1-177) forment la protéine mature et le domaine D3 (acides aminés 178-191) est clivé par la SPP lors de la maturation. Les hélices amphipathiques HI (119-136), HII (148-164) et la boucle hydrophobe HL (137-147) sont responsables de l’ancrage aux GL. Les séquences en acides aminés des régions ciblées par les substitutions et la nature des mutations (en bleu) à l’origine des génomes Jad/C_3aa_SPP et Jad/C_4aa_SPP sont indiquées.}
				\label{fig:fig42}
	\end{figureth}
	\FloatBarrier

Pour étudier le phénotype de ces mutants, des cellules Huh-7.5 ont été transfectées avec les ARN synthétiques transcrits \textit{in vitro}. Le titre infectieux et l'expression des protéines Core mutantes ont été déterminés à partir des surnageants de culture et des lysats cellulaires récoltés à 48 et 72h post-transfection (p.tf.) (\autoref{fig:fig43}\textcolor{blue}{.A}). Les titres infectieux obtenus pour le mutant Jad/C_3aa_SPP ne sont pas significativement différents de ceux du virus parental à 48h et 72h p.tf., indiquant que la production de particules virales n’est pas affectée par les trois mutations introduites dans le peptide signal (\autoref{fig:fig43}\textcolor{blue}{.B}). En revanche, le mutant Jad/C_4aa_SPP génère un titre infectieux ~1 log inférieur à celui du virus parental à 48h p.tf., mais s’en rapproche à 72h p.tf., suggérant que l’introduction supplémentaire de la mutation T186L ralentit la production de particules virales. Pour déterminer si le niveau d’expression et la stabilité de Core étaient affectés par les mutations introduites, l'expression relative des protéines Core mutantes par rapport à la protéine native a été quantifiée dans les extraits protéiques récoltés à 48h et 72h p.tf. (\autoref{fig:fig43}\textcolor{blue}{.C}). La protéine mutante C_3aa_SPP conserve une bonne stabilité avec une abondance relative de l’ordre de 30\% à 48h p.tf. et 40\% à 72h p.tf. La protéine mutante C_4aa_SPP  a une abondance relative de l’ordre de 40-50\%, indiquant que la mutation T186L n’a pas d’impact sur la stabilité de la protéine. Ces résultats suggèrent que le délai d’assemblage des particules virales infectieuses ne serait pas lié à une diminution de la stabilité de la protéine mutante C_4aa_SPP, mais probablement à un retard du clivage de maturation.

				\begin{figureth}
	\centering
			\includegraphics[width=\linewidth]{Figure_43.png}
		\caption[Caractérisation du phénotype des génomes viraux exprimant les protéines Core mutées dans le domaine prédictif de clivage par la SPP]{\textbf{Caractérisation du phénotype des génomes viraux exprimant les protéines Core mutées dans le domaine prédictif de clivage par la SPP.} (A) Schéma de l’approche expérimentale pour analyser le phénotype des génomes mutés dans le peptide signal de la protéine Core. Des cellules Huh-7.5 ont été transfectées par les ARN synthétiques transcrits \textit{in vitro} indiqués à droite ou par une solution de PBS (Mock). Les surnageants de culture et les extraits protéiques totaux ont été collectés à 48h et 72h p.tf. (B) Détermination du titre viral à partir des surnageants de culture récupérés à 48h p.tf. (à gauche) et à 72h p.tf. (à droite). Le test statistique est un t-test paramétrique réalisé sur PRISM avec les \textit{-values} : p <0,05 (*); p >0,05 (ns : non significatif). (C) Analyse quantitative de l’expression relative des protéines Core mutées par rapport à la protéine native du Jad par immuno-blot. Pour exemple, l’immuno-blot présenté correspond aux extraits protéiques récoltés à 48h p.tf. avec les anticorps ciblant les protéines virales (Core et NS5B) et cellulaire (actine ß) indiquées à droite. Le signal de Core a été normalisé par rapport aux protéines totales déposées et est exprimé relativement à celui obtenu pour la protéine native du Jad, établi à 100\%. Le graphique représente les moyennes et écart-types obtenus à partir de deux quantifications indépendantes.}
				\label{fig:fig43}
	\end{figureth}
	\FloatBarrier

Afin d'évaluer l’impact des mutations introduites sur le clivage de Core par la SPP, la localisation intracellulaire des protéines Core mutantes par rapport aux GL (\autoref{fig:fig44}\textcolor{blue}{.A, B}) a été contrôlée. La protéine mutante C_3aa_SPP est majoritairement localisée au niveau des GL dès 48h p.tf., sous la forme de petits puncta \textcolor{blue}{(flèche a)}, de façon similaire à la protéine Core native de la souche Jad lors des temps très précoces d’infection. Cette observation confirme que ces trois mutations ne sont pas suffisantes pour abolir le clivage de Core par la SPP, comme décrit dans l’étude de \citet{RN306}. Toutefois, nous avons vu précédemment que ces mutations déstabilisent partiellement la protéine, ce qui expliquerait la plus faible accumulation de Core à la surface des GL par rapport à la protéine native. À 72h p.tf., nous pouvons constater que le taux de recouvrement des GL par la protéine mutante C_3aa_SPP augmente marginalement. En revanche, la protéine mutante C_4aa_SPP présente quelques rares signaux d’association de Core aux GL à 48h p.tf., \textcolor{blue}{(flèche b)}, ce qui indique que seul un clivage très partiel par la SPP a lieu. Le transit de Core par les GL étant une étape critique pour la morphogénèse virale, cette observation est rationnelle avec le fait que ce mutant aboutisse à une réplication productive réduite à 48h p.tf. En revanche, nous pouvons remarquer que la protéine mutante C_4aa_SPP forme majoritairement des agrégats cytoplasmiques plutôt éloignés des GL à 48h p.tf. \textcolor{blue}{(flèche c)}. L’association de la protéine mutante 4aa_SPP aux GL augmente notablement à 72h p.tf. (\textcolor{blue}{(flèche d)} et devient le profil majoritaire, bien que la présence d’agrégats cytoplasmiques demeure \textcolor{blue}{(flèche e)}. L’ensemble de ces observations suggère que la protéine mutante C_4aa_SPP s’accumule au sein du RE et que le clivage par la SPP bien que partiel, soit ralenti en présence de la mutation T186L. Pour conclure, le délai de clivage résultant des mutations A180V, S183L, C184V et T186L n’est pas suffisant pour retenir la majorité des protéines Core synthétisées au sein du RE, en particulier aux temps tardifs. Ces mutants se sont donc avérés impropres pour valider le rôle de Core sur la redistribution dynamique des GL.

				\begin{figureth}
	\centering
			\includegraphics[width=\linewidth]{Figure_44.png}
		\caption[Étude de la localisation des protéines Core mutées dans le peptide signal vis-à-vis des gouttelettes lipidiques]{\textbf{Étude de la localisation des protéines Core mutées dans le peptide signal vis-à-vis des gouttelettes lipidiques.} Images représentatives de cellules transfectées par l’ARN viral parental Jad, les ARN mutants Jad/C_3aa_SPP et Jad/C_4aa_SPP acquises à 48h p.tf (A) et à 72h p.tf. (B). Dans les colonnes de gauche figure la révélation de Core par l’anticorps monoclonal ACAP27 (en cyan). Dans les colonnes du milieu figure la combinaison des marquages de Core et des GL (en magenta) par incorporation du BODIPY. Dans les panneaux de droite figure l’agrandissement des zones indiquées par les cadres en pointillés blancs. Les flèches blanches mettent en évidence le signal de Core localisé au niveau des GL et les flèches rouges soulignent une localisation alternative hors GL. Les barres d’échelle indiquent 5µm.}
				\label{fig:fig44}
	\end{figureth}
	\FloatBarrier
	\clearpage
	\subsubsection{Construction et caractérisation d’ADNc viraux codant des mutations dans le domaine D2 de Core}

Mes tentatives pour construire un mutant viral codant pour une protéine Core incapable de s’associer aux GL passant par une stratégie d’inactivation du clivage SPP n’ayant pas été un succès, je me suis ensuite concentrée sur la deuxième option, qui consistait à cibler le domaine D2 de la protéine Core, responsable de l’ancrage aux GL. Dans ce but, j’ai généré en premier le double mutant Jad/C_DP dans lequel deux résidus proline localisés à l'intérieur de la boucle hydrophobe ont été convertis en alanine (\autoref{fig:fig45}). Ce mutant, initialement décrit par \citet{RN487}, constitue un des premiers mutants de Core fonctionnels, ayant une association réduite aux GL tout en conservant une stabilité correcte. Cependant, les données disponibles sur ce mutant dans les études qui ont suivi son identification n’établissaient pas de façon définitive une dissociation complète de Core des GL. De ce fait, j’ai également construit le triple mutant Jad/C_DP_SATG portant une mutation supplémentaire en position 164, au niveau d’un résidu appartenant à un motif hautement conservé, retrouvé dans les oléosines végétales et dans la protéine Core du GBV-B \citep{RN957}.

				\begin{figureth}
	\centering
			\includegraphics[width=0.85\linewidth]{Figure_45.png}
		\caption[Construction des génomes viraux codant des mutations dans le domaine D2 de la protéine Core]{\textbf{Construction des génomes viraux codant des mutations dans le domaine D2 de la protéine Core.} (A) Représentation schématique du génome de la souche Jad et des sous-domaines structuraux de la protéine Core. Les domaines D1 et D2 (acides aminés 1-177) forment la protéine mature et le domaine D3 (acides aminés 178-191) est clivé par la peptidase du signal peptide (SPP) lors de la maturation. Les hélices amphipathiques HI (119-136), HII (148-164) et la boucle hydrophobe HL (137-147) sont responsables de l’ancrage aux GL. Les séquences en acides aminés des régions ciblées par les substitutions et la nature des mutations (en orange et magenta) à l’origine des génomes Jad/C_DP  et Jad/C_DP_SATG sont indiquées.}
				\label{fig:fig45}
	\end{figureth}
	\FloatBarrier

En suivant une approche expérimentale similaire à celle des premiers mutants (\autoref{fig:fig46}\textcolor{blue}{.A}), j’ai d’abord caractérisé les niveaux de réplication de ces nouveaux génomes synthétiques en quantifiant l’ARN viral intracellulaire dans les lysats cellulaires récoltés à 72h p.tf. L’abondance des génomes viraux dans les cellules transfectées avec les ARN Jad/C_DP et Jad/C_DP_SATG est diminuée de ~0,4 à 72h p.tf. log par rapport à celle des cellules transfectées avec l’ARN parental (\autoref{fig:fig46}\textcolor{blue}{.B}). Cette diminution modeste résulte probablement du fait que ces deux génomes mutants n’aboutissent pas à une production de particules virales, empêchant toute ré-infection cellulaire. Ces résultats indiquent donc que les mutations P138A+P143A et P138+P143+Y164S dans le domaine D2 de Core n’ont pas d’incidence sur la réplication génomique mais abolissent la morphogénèse et/ou la sécrétion de particules virales infectieuses. La protéine mutante C_DP est exprimée à un niveau relatif de l’ordre de 30\% à 48h p.tf. et de 60\% à 72h p.tf. par rapport à la protéine Core native de Jad, indiquant une bonne stabilité constitutive de ce variant (\autoref{fig:fig46}\textcolor{blue}{.C}). En revanche, l’expression de la protéine mutante C_DP_SATG apparaît sévèrement affectée avec une abondance relative réduite à hauteur de 5\% à 72h p.tf. Le fait qu’une autre protéine virale, NS5B, soit retrouvée en quantité similaire à celle du Jad, confirme qu’il s’agit d’un problème de stabilité intrinsèque de la protéine Core mutée et non d’un déficit dans la traduction du précurseur polyprotéique. Dans l’espoir d’améliorer la stabilité intrinsèque de la protéine Core triplement mutée aux positions 138, 143 et 164, des ADNc complémentaires ont été construits en convertissant les résidus proline en glycine et le résidu tyrosine en alanine, générant les mutants Jad/C_DG et Jad/C_DG_AATG, afin de tenter de conserver les propriétés conformationnelles de la protéine, sur les conseils d’un collaborateur structuraliste F. Pénin. Pour des raisons inconnues, ces substitutions se sont avérées encore plus dommageables, aboutissant à une instabilité relative encore plus faible pour la protéine mutante C_DG que pour C_DP et à une dégradation quasi-complète de la protéine mutante C_DG_AATG (données non montrées).

				\begin{figureth}
	\centering
			\includegraphics[width=\linewidth]{Figure_46.png}
		\caption[Caractérisation du phénotype des génomes viraux exprimant les protéines Core mutées dans le domaine D2]{\textbf{Caractérisation du phénotype des génomes viraux exprimant les protéines Core mutées dans le domaine D2.} (A) Schéma de l’approche expérimentale pour analyser le phénotype des génomes mutés dans le domaine D2 de la protéine Core. Des cellules Huh-7.5 ont été transfectées par les ARN synthétiques transcrits \textit{in vitro} indiqués à droite ou par une solution de PBS (Mock). Les surnageants de culture, les lysats protéiques et les extraits d’ARN total ont été collectés à 48h et 72h p.tf. (B) Détermination de la quantité d’ARN viral intracellulaire (panneau de gauche) et du titre infectieux à partir d’extraits cellulaires et des surnageants de culture collectés à 72h p.tf. Le test statistique est un t-test paramétrique réalisé sur PRISM avec les \textit{p-values} : p <0,0001 (****); p >0,05 (ns : non significatif). (C) Analyse quantitative de l’expression relative des protéines Core mutées par rapport à la protéine native du Jad par immuno-blot sur les extraits protéiques récoltés à 48h et 72h p.tf. Pour exemple, l’immuno-blot présenté correspond aux extraits protéiques récoltés à 48h p.tf. avec les anticorps ciblant les protéines virales (Core et NS5B) et cellulaire (actine ß) indiqués à droite. Le signal de Core a été normalisé par rapport aux protéines totales déposées et est exprimé relativement à celui obtenu pour la protéine native du Jad, établi à 100\%. Le graphique représente les moyennes et écart-types obtenues à partir de deux quantifications indépendantes.}
				\label{fig:fig46}
	\end{figureth}
	\FloatBarrier

Afin de vérifier quelle voie de dégradation cellulaire est impliquée dans la réduction de la stabilité des protéines Core mutantes, j’ai reproduit une expérience similaire, en incubant les cellules transfectées pendant 14h en présence de 10µM de MG132, un inhibiteur connu du protéasome (\autoref{fig:fig47}\textcolor{blue}{.A}). L’accumulation de conjugués ubiquitinylés permet de confirmer l’efficacité de l’inhibition du protéasome (\autoref{fig:fig47}\textcolor{blue}{.B}). L’analyse des extraits protéiques par immuno-blot montre que ce traitement permet de rétablir une quantité accrue de la protéine mutante C_DP_SATG \textcolor{blue}{(flèche bleue)}. Par ailleurs, ce traitement ne semble pas améliorer les niveaux d’expression de la protéine mutante C_DP. En sachant que cette méthode d’analyse ne permet pas de voir les fines différences à l’oeil, ceci n’exclut pas qu’il pourrait y avoir une faible restauration. Sur un gel indépendant permettant de discriminer les formes matures et immatures de la protéine, on observe que la forme mature de la protéine mutante C_DP_SATG est substantiellement stabilisée par l’inhibition du protéasome (\autoref{fig:fig47}\textcolor{blue}{.B, flèche rose}). Ces résultats montrent que l’introduction de la mutation Y164S entraîne une dégradation de Core, qui pourrait être attribuable soit à une perte de localisation aux membranes du RE, rendant la protéine accessible à la machinerie protéasomale, soit à un mauvais repliement de la protéine après la dissociation du RE. Il est intéressant de noter qu’une protéine Core mutée uniquement au niveau du résidu Y164 (C_SATG) a une stabilité comparable à la protéine native (données non montrées), indiquant que cette seule substitution ne résulte pas en la dégradation irrémédiable de la protéine par la machinerie protéasomale.

				\begin{figureth}
	\centering
			\includegraphics[width=\linewidth]{Figure_47.png}
		\caption[Analyse de la stabilité des protéines Core mutées dans le domaine D2 en présence d’un inhibiteur du protéasome]{\textbf{Analyse de la stabilité des protéines Core mutées dans le domaine D2 en présence d’un inhibiteur du protéasome.} (A) Schéma de l’approche expérimentale pour analyser la stabilité des protéines Core mutantes. Des cellules Huh-7.5 ont été transfectées par les ARN synthétiques transcrits \textit{in vitro} indiqués à droite ou par une solution de PBS (Mock). Un traitement avec 10µM de MG132 a été appliqué aux cultures d’hépatocytes transfectés pendant 14h préalablement à la récolte des extraits protéiques à 72h p.tf. (B) Immuno-blot des extraits protéiques traités ou non par le MG132 récoltés à 72h p.tf. avec les anticorps ciblant les conjugués ubiquitynilés (Ub), la protéine cellulaire actine ß et la protéine virale Core (deux immuno-blots indépendants) indiqués à gauche. Les flèches bleues pointent la forme mature de la protéine Core.}
				\label{fig:fig47}
	\end{figureth}
	\FloatBarrier
	
Bien que la protéine mutante C_DP_SATG soit hautement instable, un signal significatif est détecté par immunofluorescence dans un faible pourcentage de cellules transfectées (~ 1\%) à 72h p.tf. à l’aide d’un anticorps monoclonal ciblant Core (\autoref{fig:fig48}). La protéine mutante C_DP est détectée dans ~25-30\% des cellules transfectées. Pour le virus parental, ~50\% des cellules transfectées présentent un signal positif pour Core à ce temps post-transfection. Nous pouvons constater que les protéines mutantes C_DP et C_DP_SATG ne se présentent pas sous la forme « d’anneau » habituellement observée pour la protéine Core native. En sachant que la protéine C_DP_SATG est dégradée par le protéasome, j’ai étudié la localisation de son signal vis-à-vis de PBSM5, une des sous-unités protéasomales. Le signal principal de PBSM5 se présente sous la forme de larges puncta diffus dans le cytoplasme, avec d’un léger bruit de fond nucléaire lié à l’anticorps. Dans les cellules détectées avec un signal positif pour Core, nous pouvons voir que la protéine C_DP_SATG ne co-localise globalement pas avec la sous-unité protéasomale, indiquant qu’il ne s’agit pas d’amas protéiques en cours de dégradation.

				\begin{figureth}
	\centering
			\includegraphics[width=\linewidth]{Figure_48.png}
		\caption[Analyse de la localisation des protéines Core mutées vis-à-vis d’une sous-unité protéasomale]{\textbf{Analyse de la localisation des protéines Core mutées vis-à-vis d’une sous-unité protéasomale.} Images représentatives de cellules transfectées par l’ARN viral parental Jad, les ARN mutants Jad/C_DP et Jad/C_DP_SATG ou par une solution de PBS (Mock) acquises à 72h p.tf. Dans la colonne de gauche figure la révélation de la sous-unité bêta du protéasome de type 5 (PBSM5) (en jaune). Dans la colonne du milieu figure la combinaison des marquages de PSMB5 et de la protéine Core (en cyan). Dans la colonne de droite figure l’agrandissement des zones indiquées par les cadres en pointillés blancs, mettant en valeur la localisation de Core par rapport à la sous-unité PSMB5. Les barres d’échelle indiquent 5µm. Les panneaux inférieurs soulignent quelques rares proximités entre les signaux de Core et de PSMB5 indiqués par les flèches (a, b, c, d) dans les zones agrandies. Les histogrammes présentés sur la droite représentent l’intensité du signal de Core et de PBSM5 sur une ligne tracée (non représentée) dans les zones agrandies respectives, montrant l’absence global de signaux recouvrants.}
				\label{fig:fig48}
	\end{figureth}
	\FloatBarrier

En raison de la toxicité cellulaire induite par le traitement MG132 et des potentielles conséquences que l’inhibition du protéasome pourrait avoir sur les GL, l’étude des mutants a été poursuivie dans des conditions physiologiques, à défaut d’avoir une faible proportion de cellules transfectées par le mutant Jad/C_DP_SATG montrant un signal détectable de Core par immunofluorescence. Afin d'd'évaluer l’impact des mutations introduites dans Core sur son ancrage aux GL, j’ai contrôlé la localisation des protéines Core mutantes vis-à-vis des GL. En accord avec les résultats de \citet{RN487}, la protéine mutante C_DP forme de petits puncta situés à proximité des GL (\autoref{fig:fig49}\textcolor{blue}{.A}). À partir d'un nombre significatif de cellules observées, nous avons constaté que la protéine mutante C_DP_SATG forme principalement de larges agrégats cytoplasmiques ne se localisant pas à proximité des GL. Une analyse quantitative réalisée sur un nombre de cellules identiques où Core est clairement détectée pour le virus parental et les deux mutants révèle que le contenu cellulaire en protéine mutantes C_DP et C_DP_SATG est en moyenne significativement plus faible que celui en protéine Core native (\autoref{fig:fig49}\textcolor{blue}{.B}), ce qui est cohérent avec leur relative instabilité. En revanche, il n’y a pas de différence significative de contenu en protéines mutantes C_DP et C_DP_SATG dans les cellules analysées. La quantification de la co-localisation du signal de Core et des GL révèle une réduction de 40\% du volume des GL recouvertes par la protéine mutante C_DP par rapport à la protéine native, démontrant que les résidus P138 et P143 sont effectivement déterminants pour l’ancrage de Core à la surface des GL (\autoref{fig:fig49}\textcolor{blue}{.C}). La proportion du volume de GL recouvertes par la protéine mutante C_DP_SATG est significativement plus faible que pour celle de C_DP, pour un contenu similaire en protéine, indiquant que la mutation Y164S supplémentaire abolit très majoritairement l’ancrage aux GL. La protéine C_SATG, mutée uniquement au niveau du résidu Y164, présente un aspect annulaire entourant les GL identique à la protéine native, indiquant que cette seule substitution n’impacte pas le trafic de Core à la surface des GL (données non montrées). En revanche, le génome mutant Jad/C_SATG n’aboutit pas à une production de particules virales indiquant que la mutation Y164S pourrait affecter le repliement structural de la protéine sans contraindre l’ancrage aux GL, ou bloquer l’interaction avec un autre partenaire viral essentiel à la morphogénèse et/ou la sécrétion des particules virales.

				\begin{figureth}
	\centering
			\includegraphics[width=\linewidth]{Figure_49.png}
		\caption[Analyse de la localisation des protéines Core mutées dans le domaine D2 vis-à-vis des gouttelettes lipidiques.]{\textbf{Analyse de la localisation des protéines Core mutées dans le domaine D2 vis-à-vis des gouttelettes lipidiques.} (A) Images 3D représentatives de cellules transfectées par l’ARN viral parental Jad, les ARN mutants Jad/C_DP et Jad/C_DP_SATG ou par une solution de PBS acquises à 72h p.tf. Dans la colonne de gauche figure la révélation de la protéine Core (en cyan). Dans la colonne du milieu figure la combinaison des marquages de Core et des GL par incorporation du BODIPY (en magenta). Les barres d’échelle indiquent 5µm. Les panneaux de droite sont un agrandissement des zones indiquées par les cadres en pointillés blancs (a, b, c, d, e ou f). Les flèches blanches mettent en valeur la localisation des protéines Core par rapport aux GL. Les histogrammes de droite représentent l’intensité du signal de Core et des GL sur une ligne tracée (non représentée) dans les zones agrandies respectives. (B) Analyse quantitative du volume total de Core (en µm\up{3}) à 72h p.tf. (C) Analyse quantitative de la proportion du volume des GL recouvertes par Core (en \%) à 72h p.tf. Les valeurs moyennes des paramètres mesurés sont indiquées dans le tableau de droite. L’analyse a été conduite sur 60 cellules au cours de trois expériences indépendantes. Les tests statistiques ont été réalisés sur R à l’aide d’un modèle mixte linéaire. Les \textit{p-values} ont été ajustées par la méthode de Tukey : p <0,0001 (****); p >0,05 (ns : non significatif).}
				\label{fig:fig49}
	\end{figureth}
	\FloatBarrier

	\subsubsection{Bloquer l’ancrage de Core à la surface des GL abolit les processus d’agrégation, d’élargissement et de redistribution de ces organites}

Pour finir, l’incidence des mutations introduites dans le domaine D2 de la protéine Core sur la dynamique et la redistribution des GL a été analysée, en utilisant une méthode similaire à celle présentée dans les sections \ref{section:stabilisation} et \ref{section:cluster}. Nous pouvons constater que le mutant Jad/C_DP_SATG n’induit pas d’augmentation du taux de GL appariées, contrairement à la souche parentale et au mutant Jad/C_DP (\autoref{fig:fig50}\textcolor{blue}{.A}). De plus, les deux mutants Jad/C_DP et Jad/C_DP_SATG n’entraînent pas d’augmentation du volume moyen des GL (\autoref{fig:fig50}\textcolor{blue}{.B}). Ces résultats indiquent que le mutant Jad/C_DP est capable de conduire à l’agrégation des GL, contrairement au mutant Jad/C_DP_SATG, qui résulterait potentiellement de son association partielle aux GL. En revanche, la transfection par le mutant Jad/C_DP n’induit pas d’élargissement significatif des GL, suggérant que les mutations DP abolissent une propriété de Core indispensable à l’enclenchement du processus de fusion présumé des GL.

				\begin{figureth}
	\centering
			\includegraphics[width=\linewidth]{Figure_50.png}
		\caption[Analyse des paramètres de clusterisation et d’élargissement des GL dans les hépatocytes par les ARN viraux exprimant les protéines Core mutées dans le domaine D2]{\textbf{Analyse des paramètres de clusterisation et d’élargissement des GL dans les hépatocytes transfectés par les ARN viraux exprimant les protéines Core mutées dans le domaine D2.} (A) Analyse quantitative du taux de GL appariées (en \%) à 72h p.tf. (B) Analyse quantitative du volume moyen des GL (en µm\up{3}) à 72h p.tf. L’analyse a été conduite sur 60 cellules au cours de trois expériences indépendantes. Les tests statistiques ont été réalisés sur R à l’aide d’un modèle mixte linéaire. Les \textit{p-values} ont été ajustées par la méthode de Tukey : p <0,0001 (****); p >0,05 (ns : non significatif).}
				\label{fig:fig50}
	\end{figureth}
	\FloatBarrier

Pour vérifier si la redistribution des GL à proximité des usines de réplication virale a lieu en présence des protéines Core mutées au sein du domaine D2, nous avons effectué des études complémentaires par MET de cellules transfectées par l’ARN parental Jad, par les ARN mutants Jad/C_DP et Jad/C_DP_SATG ou par un ARN ne codant pas pour les régions de E1 à p7 (Jad/$\Delta$Ep7), rendant ce génome viral déficient pour l’étape de morphogénèse virale. En premier, nous pouvons constater que tous les hépatocytes transfectés avec un ARN viral présentent des remodelages membranaires au sein du RE à 72h p.tf. (\autoref{fig:fig50}\textcolor{blue}{.A, cadres noirs}), confirmant que le génome délété et les génomes mutés mettent effectivement en place les usines de réplication virale. Le mutant Jad/$\Delta$Ep7 induit une redistribution des GL cytosoliques à proximité des remodelages membranaires (\autoref{fig:fig50}\textcolor{blue}{.B, flèches jaunes}), dans des proportions similaires au virus parental. Les cellules transfectées par le mutant Jad/C_DP présentent quelques GL proches des remodelages membranaires bien qu’une proportion significative reste dispersée dans le cytoplasme, indiquant que ce mutant induit une redistribution partielle des GL à proximité des usines de réplication virale. Dans le cas du mutant Jad/C_DP_SATG, nous n’avons pas observé de redistribution particulière des en GL, qui restent majoritairement disséminées dans le cytoplasme.

				\begin{figureth}
	\centering
			\includegraphics[width=0.80\linewidth]{Figure_51.png}
		\caption[Analyse de la redistribution des GL dans les hépatocytes transfectés par les ARN viraux exprimant les protéines Core mutées dans le domaine D2]{\textbf{Analyse de la redistribution des GL dans les hépatocytes transfectés par les ARN viraux exprimant les protéines Core mutées dans le domaine D2.} (A) Micrographies d’hépatocytes transfectés par une solution de PBS (Mock), l’ARN parental Jad, l’ARN Jad/$\Delta$Ep7 ou par les ARN mutants Jad/C_DP et Jad/C_DP_SATG, acquises par microscopie électronique à transmission à 72h p.tf. Les barres d'échelle indiquent 2 ou 5 µm. (B) Agrandissements des régions indiquées par les cadres en pointillés noirs (a, b, c ou d). Les flèches jaunes mettent en valeur les remodelages membranaires du RE induits par l’infection.}
				\label{fig:fig51}
	\end{figureth}
	\FloatBarrier
	\clearpage

Malgré la difficulté à travailler avec des protéines Core très instables, ces analyses fines par imagerie quantitative, microscopie électronique et par l’utilisation de variants de la protéine Core exprimés dans le contexte réplicatif de l'ARN viral ont permis de montrer que la protéine Core est l’acteur clé des processus d’agrégation, d’élargissement et de redistribution des GL à proximité des usines de réplication virale, pour lesquels l’étape d’ancrage à la surface des GL est essentielle. L’implication biologique de ces processus dans le cycle viral et en particulier dans la morphogénèse des particules infectieuses reste à définir. Il n’est pas exclu que le mutant Jad/$\Delta$Ep7, qui exprime la protéine Core native soit capable de mener la première étape de la morphogénèse, c’est-à-dire, la formation d’une nucléocapside suite à l’interaction des protéines Core et de l’ARN génomique, potentiellement favorisée par les processus d’agrégation et de redistribution des GL cytosoliques à proximité des usines de réplication virale.

\clearpage
%%%%%%%%%%%%%%%%%%%%%%%%%%%%%%%%%%%%%%%%%%%%%%%%%%%%%%%%%%%%%%%%%%%%%%%%%%%%%%%%%%%%%%%%%%%%%%%%%%%%%%%%%%%%%%%%%%%%%%%%%%%%%%%%%%%%%%%%%%%%%%%%%%%%%%%%%%%%%%%%%%%%%%%%

% 3 - PROJET RECOMBINANT
			
	\section{Étude de l’influence des génotypes ou des polymorphismes de Core sur la régulation des GL et des voies de signalisation hépatiques}

		\subsection{Préambule}
		\label{section:preambule2}

Je me suis basée sur des techniques bien établies dans le laboratoire afin de produire et de caractériser un nouveau genre de virus intergénotypiques exprimant les protéines Core dérivées de souches cliniques obtenues à partir de patients ayant développé des degrés variables de stéatose micro- et macro-vésiculaire hépatique. D’autres virus intergénotypiques exprimant des protéines Core dérivées de souches cliniques ou prototypiques précédemment produits par notre groupe ont été inclus dans cette étude. En raison de la quantité d'expériences et d'analyses à réaliser avec ces nouveaux modèles, j'ai dû limiter mon travail de thèse à une sélection de 2 souches prototypiques et de 9 souches cliniques représentatives de différents génotypes, dont 7 associées à différents degrés de stéatose hépatique.
			
		\subsection{Sélection et caractérisation des séquences Core issues de souches cliniques associées à différents degrés de stéatose hépatique}
		\label{section:souches}

Notre groupe a caractérisé des séquences consensus codant Core issues de souches cliniques du VHC de sous-types 1a et 3a isolées en France et associées à des degrés variables de stéatose hépatique chez les patients infectés (coll. C. Gondeau, Hôpital Saint-Eloi, Montpellier). Ces souches proviennent de prélèvements sanguins collectés chez des individus de sexe masculin, âgés de 38 à 60 ans et présentant des charges virales, des co-morbidités et des habitudes personnelles variées, notamment en termes de surcharge pondérale et de consommation d'alcool, connus pour être des facteurs aggravants de la pathologie hépatique (voir la \autoref{section:etiologie}). Je me suis concentrée sur les souches cliniques de sous-types 1a et 3a pour étudier les propriétés stéatogènes putatives des génotypes les plus répandus dans le monde (voir la \autoref{section:epidemiologie}). Ces séquences de Core ont été divisées en trois groupes en fonction du degré de stéatose développé par le patient d’origine (\autoref{fig:fig52}). Cette assignation se base sur les analyses histologiques qui estiment la proportion d’hépatocytes comportant des micro- ou des macro-vésicules graisseuses. Dans ce travail, j’ai également inclus des séquences Core de sous-type 4a et 4f provenant de souches cliniques isolées en Roumanie et au Cameroun, avec lesquelles notre groupe a mené une précédente étude \citep{RN808}. Les données cliniques associées à ces souches n’ont pas été communiquées, toutefois la prise en compte de ces variants de Core a permis d'étendre cette étude à des souches émergentes de génotype 4 qui circulent dans les pays d’Afrique centrale. Enfin, cette étude inclut également les séquences Core des souches prototypiques H77 de sous-type 1a et J6 de sous-type 2a.

				\begin{figureth}
	\centering
			\includegraphics[width=\linewidth]{Figure_52.png}
		\caption[Données virologiques et cliniques relatives aux patients originaires de France, Roumanie et Cameroun]{\textbf{Données virologiques et cliniques relatives aux patients originaires de France, Roumanie et Cameroun.} Le code donné au virus sérique provenant des patients est indiqué à gauche. Parmi les données disponibles sont renseignées l’année de naissance du patient; le sous-type viral infectant : 1a (violet), 3a (vert), 4a ou 4f (bleu); la charge virale sérique (en UI/mL), la date de la biopsie et les degrés de stéatose hépatique micro ou macro-vésiculaire. Les prélèvements sanguins et les analyses du tissu hépatique ont été réalisés avec un intervalle de 1 à 28 mois d’écart. Les profils NS ou « pas de stéatose » (en rouge clair), MS ou « stéatose modérée » (en rouge) et HS ou « stéatose élevée » (en rouge foncé) correspondent respectivement à $\leq$5\%, 10 à 40\%, et $\geq$50\% de micro- ou macro-vésicules hépatiques. Les éventuelles co-morbidités connues, données relatives aux thérapies antivirales, et habitudes personnelles d’intérêt clinique pour chaque individu sont décrites à droite. N. C. : non communiqué.}
				\label{fig:fig52}
	\end{figureth}
	\FloatBarrier

L’ensemble des séquences codantes consensus de Core ont été alignées sur la séquence prototypique de JFH-1, identique à celle du variant Jad (\autoref{fig:fig53}). Cet alignement révèle que la séquence en acides aminés de la protéine Core du VHC est hautement conservée, avec moins de 20\% de divergence entre les génotypes. Le groupe de séquences de génotype 3 diffère de 8 à 10\% avec les séquences de génotype 1, de 5 à 8\% avec les séquences du génotype 4 et représente le groupe le plus distant des séquences prototypiques de génotype 2, avec 12 à 18\% de différences. Les séquences de génotypes 1 et 4 présentent le plus haut degré d'homologie, avec une divergence n’excédant pas 5\%. Parmi les séquences de génotype 3, la séquence 3a-376 est la moins conservée et porte plusieurs résidus uniques (20L, 68S, 69T, 110S, 143A, 150F, 158V). Il est intéressant de remarquer que parmi les séquences prototypiques de génotype 2, la séquence Core de J6 présente une homologie plus élevée avec les séquences des souches cliniques (8 à 13\% de différences) que la séquence Core de JFH-1 (10 à 18\% de différences). La séquence Core de JFH-1 contient plusieurs caractéristiques uniques par rapport à toutes les autres séquences de Core, notamment une série singulière d’acides aminés dans le domaine 1 (dans la région 68-81), un motif LSGA particulier dans le domaine 2 (aux positions 144-147), des résidus F et P uniques aux positions respectives 172 et 173, et une séquence C-terminale unique du peptide signal, faisant de la séquence Core de JFH-1 la plus singulière de tout le panel de cette étude.
	\clearpage

Cet alignement permet de mettre en évidence certains motifs communs au sein des génotypes : le résidu 71P présent dans les deux séquences de génotype 1; les résidus 68D, 72T, 74K, 77G, 78K, 91L, 114H, 119V, 139V et 185I présents uniquement dans les deux séquences de génotype 2; les résidus 3L, 16I, 22I, 36V, 182F, 186I, 187N/H, 190A et 191S/G présents uniquement dans les séquences de génotype 3; le résidu 20M présent uniquement dans les deux isolats du génotype 4. Certaines séquences de Core portent par ailleurs des résidus naturels décrits précédemment comme étant des facteurs de risque potentiels de la stéatose \citep{RN990,RN991} ou du CHC \citep{RN988,RN989}. Par exemple, notre groupe a récemment montré que le résidu T71 dans la séquence de Core de l'isolat 4fC était responsable d'une activation significativement supérieure de la voie Wnt/ß-caténine \citep{RN808}.

				\begin{figureth}
	\centering
			\includegraphics[width=0.70\linewidth]{Figure_53.png}
		\caption[Alignement des séquences consensus de la protéine Core des souches cliniques et prototypiques]{\textbf{Alignement des séquences consensus de la protéine Core des souches cliniques et prototypiques.} \textit{La légende de la figure est décrite sur la page suivante \footnotemark[3].}}
				\label{fig:fig53}
	\end{figureth}
	\FloatBarrier
	
\footnotetext[3]{Les séquences consensus en acides aminés, déduites du séquençage nucléotidique de l’ARN des virus sériques, sont classées selon le génotype de la souche infectante. Les acides aminés identiques sont représentés par des points noirs ou rouges. Les acides aminés les plus communs sont de couleur noire et les résidus singuliers sont de couleur rouge. Les résidus naturels décrits comme facteurs de risque pour la stéatose ou le CHC sont encadrés et leurs positions pointées par des flèches selon le code couleur indiqué. La position du peptide signal (178-191) pour le JFH-1 est indiqué par un cadre en pointillés noirs. Un diagramme indiquant le pourcentage de conservation des séquences est inclus en bas de l’alignement.}

Les séquences de Core étudiées se regroupent naturellement par génotype sur un arbre phylogénétique (\autoref{fig:fig54}). Il est intéressant de noter que certains sous-groupes phylogénétiques présentent des signatures cliniques communes, comme le couple 3a-389/3a-401 associé à une absence de stéatose, et le couple 3a-311/3a-390 issu de patients ayant développé une stéatose élevée. En revanche, les séquences de génotype 3a associées à une stéatose modérée (3a-376, 3a-395) sont distribuées uniformément au sein de la phylogénie, sans groupement particulier. De même, la séquence unique de génotype 1a associée à une stéatose modérée (1a-411) ne converge pas notablement vers les séquences 3a associés à des degrés de stéatose équivalents. En raison du petit nombre de séquences considérées pour chaque signature clinique dans cette étude, il sera délicat de conclure sur une corrélation potentielle entre un cluster particulier et une issue de stéatose sévère. Néanmoins, les liens entre sous-groupes phylogénétiques montrant des caractéristiques cliniques communes mettent en évidence une proximité significative de séquences, qui sera étudiée plus en détail au cours de ce travail.

				\begin{figureth}
	\centering
			\includegraphics[width=0.60\linewidth]{Figure_54.png}
		\caption[Phylogramme circulaire des séquences protéiques de Core de l’étude]{\textbf{Phylogramme circulaire des séquences protéiques de Core de l’étude.} L’arbre phylogénétique a été construit en utilisant la méthode du maximum de vraisemblance selon le modèle de substitution de Jukes et Cantor. Un ré-échantillonnage à partir de 100 répliques a été effectué afin d’évaluer la fiabilité du regroupement. Les valeurs significatives > 60\% sont indiquées en gris clair. L’arbre est dessiné à l’échelle, la longueur des branches étant proportionnelle au nombre moyen de substitutions d’acides aminés par site, comme indiqué par la barre d’échelle.}
				\label{fig:fig54}
	\end{figureth}
	\FloatBarrier


		\subsection{Développement de virus intergénotypiques dérivés d’une souche hautement réplicative en lignée cellulaire d'hépatome exprimant des protéines Core hétérologues}
		\label{section:virus}

Compte tenu de l’impossibilité de cultiver des souches cliniques \textit{in vitro} (voir la \autoref{section:modele}), des virus chimériques ont été développés en se basant sur les stratégies précédemment décrites, plutôt que des systèmes d’expression transitoire, qui pourraient conduire à une surexpression protéique et déclencher des dérèglements cellulaires non pertinents. Pour cela, je me suis appuyée sur la souche Jad hautement réplicative dans des cellules d'hépatome humain en culture, sur la base d'une approche déjà validée dans une précédente étude de notre groupe \citep{RN808}. Le but de cette approche est de produire des ADNc recombinants dérivant du Jad, en substituant la séquence codant Core par les séquences correspondantes provenant des autres souches prototypiques ou des isolats cliniques. L'introduction d'une séquence de Core hétérologue peut avoir trois incidences délétères potentielles sur le cycle viral : (i) un défaut de traduction : quelques codons N-terminaux de Core participent à la structure et à l’activité de l’IRES, par conséquent, la production d’un IRES chimérique pourrait donc compromettre l’initiation de la traduction du génome et moduler en cascade la réplication génomique ; (ii) un défaut de clivage à la jonction intergénotypique Core/E1 par les SP cellulaires ; (iii) une altération de la synthèse des particules virales, puisque l’interaction de Core avec les autres protéines virales et avec l'ARN est nécessaire à la morphogénèse des virions et que ces interactions pourraient être génotype-spécifiques. Or il est essentiel de disposer de modèles d’infection performants et comparables pour mener à bien les différents axes de ce projet. Les ADNc recombinants codant les protéines Core de la souche prototypique H77 et des souches cliniques 4aR et 4fC ont été précédemment construits et caractérisés, confirmant la faisabilité de cette approche \citep{RN808}. Pour mon projet, la plupart des séquences étudiées appartiennent au génotype 3, qui présente le plus faible degré d’homologie avec la séquence prototypique du Jad. Cette dissimilitude entre les séquences qui visent à être substituées pouvait rendre plus hasardeuse la production des virus intergénotypiques fonctionnels codant Core de génotype 3 dans le contexte du Jad.

\subsubsection{Utilisation d’une activité rapportrice pour évaluer les capacités réplicatives et d’assemblage des génomes recombinants exprimant les protéines Core hétérologues}

Dans l’objectif de vérifier rapidement les capacités de réplication génomique et de morphogenèse des génomes recombinants, les séquences Core des souches prototypiques et des isolats cliniques ont été insérées dans un clone moléculaire bicistronique de la souche Jad (Jad2-EIL3)\citep{RN810} qui contient le gène de la luciférase de luciole (FLuc) (\autoref{fig:fig55}). J’ai ainsi ajouté au panel du laboratoire les nouveaux ADNc recombinants codant les protéines Core de la souche prototypique J6 et des isolats cliniques 1a-411, 3a-311, 3a-376, 3a-389, 3a-390, 3a-395 et 3a-401.

				\begin{figureth}
	\centering
			\includegraphics[width=\linewidth]{Figure_55.png}
		\caption[Représentation schématique des ADNc dérivés du contexte génomique Jad-2EIL3]{\textbf{Représentation schématique des ADNc dérivés du contexte génomique Jad-2EIL3.} Les génomes Jad-2EIL3 et dérivés sont bicistroniques avec un premier cistron codant les protéines C à NS2 (en noir) sous le contrôle de l'IRES du VHC et un deuxième cistron en aval de l'IRES de l'EMCV codant FLuc (en jaune), la protéine 2A du FMDV (en gris), l’ubiquitine (Ubi, en gris), et les protéines NS3-NS5B (en bleu). Des génomes défectifs pour la réplication génomique et la morphogénèse virale ont été produits : J-2EIL3/GAA, muté dans le site actif de NS5B par la substitution des codons G-D-D en G-A-A, et Jad-2EIL3/$\Delta$Ep7, délété des séquences codant E1-E2-p7. Les génomes recombinants dont la séquence Core a été substituée par les séquences correspondantes des souches de génotype 1 (en violet), de génotype 2 (en orange), de génotype 3 (en vert) et de génotype 4 (en bleu) sont représentés en bas. Les mutations adaptatives conférant à la souche Jad une infectiosité élevée en lignée hépatocytaire Huh-7.5 sont indiquées : deux substitutions dans la séquence codante de NS5A (V2153A, V2440L) et une substitution dans la séquence codante de NS5B (V2941M).}
				\label{fig:fig55}
	\end{figureth}
	\FloatBarrier

Dans la construction Jad-2EIL3, les gènes codant les protéines Core à NS2 sont placés dans un premier cistron en aval de la région 5’NC sous le contrôle de l'IRES du Jad. Un deuxième cistron, sous la dépendance traductionnelle de l'IRES du virus de l’encéphalomyocardite murine (EMCV) code les protéines non structurales NS3 à NS5B. Le gène FLuc, fusionné à son extrémité C-terminale à la séquence codant le peptide 2A du virus de la fièvre aphteuse (FMDV) et suivi de la séquence codant un monomère d'ubiquitine est inséré en aval de l'AUG initiateur de l'IRES de l'EMCV de manière à être en phase avec la séquence codant NS3. L'activité du peptide 2A entraîne la libération de la protéine FLuc-2A et l'activité de l'ubiquitine libère la protéine NS3 mature, un évènement essentiel pour assurer l'activité protéasique de NS3 et son activité dans la réplication du génome. L'activité FLuc est directement proportionnelle au nombre de copies d'ARN viral répliquées dans la cellule. Pour jauger les mesures obtenues, des ADNc contrôles ont été utilisés : J-2EIL3/GAA, dans lequel les codons du site actif de l’ARN polymérase NS5B (Gly-Asp-Asp) ont été substitués en codons non fonctionnels (Gly-Ala-Ala), et Jad-2EIL3/$\Delta$Ep7, dans lequel les séquences codant les glycoprotéines d’enveloppe E1 et E2 et la viroporine p7 ont été délétées. Le génome J-2EIL3/GAA est défectif pour la réplication, tandis que le génome Jad-2EIL3/$\Delta$Ep7 est défectif pour l'assemblage des particules virales. Les résultats relatifs aux ARN Jad-2EIL3/C1aH77, Jad-2EIL3/C4aR et Jad-2EIL3/C4fC, précédemment obtenus par notre groupe, ont été inclus dans ces représentations à titre comparatif. \\ \\
\indent
Afin d'évaluer la capacité réplicative des nouveaux génomes recombinants, des cellules Huh-7.5 ont été transfectées avec les différents ARN viraux bicistroniques générés par transcription \textit{in vitro}, puis l’activité FLuc a été mesurée dans les lysats de cellules extraits à 4h et à 72h p.tf., Pour chaque ARN, un rapport a été calculé entre la valeur de l'activité FLuc intracellulaire mesurée à 72h p.tf qui correspond au plateau de réplication génomique, et celle mesurée à 4h p.tf, correspondant uniquement à la traduction de l'ARN introduit par électroporation (\autoref{fig:fig56}\textcolor{blue}{.A}). Cette opération permet de s'affranchir de la variabilité potentielle liée à des différences mineures d'ARN introduits lors de la transfection. L’ARN J-2EIL3/GAA présente un ratio négatif d'activité FLuc relative, démontrant que le niveau basal d'activité observé à 4h p.tf. suite à la traduction est plus élevé que l'activité mesurée à 72h p.tf. Ce résultat s'explique par la dégradation progressive de ces ARN non réplicatifs au cours du temps. À l'inverse, on observe une augmentation importante de l'activité FLuc entre 4h et 72h p.tf. pour la majorité des ARN recombinants, du même ordre de grandeur que celle observée pour l'ARN Jad-2EIL3/$\Delta$Ep7 et l'ARN parental Jad-2EIL3, traduisant une réplication efficace des génomes viraux. Les ARN recombinants Jad-2EIL3/C3a-389 et Jad/C3a-401 présentent néanmoins une activité FLuc significativement plus faible que l’ARN parental, signe d’une réplication génomique moins robuste de ces ARNs. Ainsi, la substitution de la séquence codant Core par des séquences équivalentes appartenant à différents génotypes du VHC affecte marginalement la réplication des ARN recombinants dans le contexte bicistronique.

				\begin{figureth}
	\centering
			\includegraphics[width=0.65\linewidth]{Figure_56.png}
		\caption[Analyse de la capacité réplicative et productive des ARN recombinants dans le contexte bicistronique Jad-2EIL3]{\textbf{Analyse de la capacité réplicative et productive des ARN recombinants dans le contexte bicistronique Jad-2EIL3.} (A) Des lysats de cellules Huh-7.5 transfectées avec les ARN Jad-2EIL3 recombinants (barres colorées) ou contrôles (barres blanche, grise et grise foncée) ont permis d'établir la mesure relative entre l'activité FLuc à 72h p.tf. et à 4h p.tf. (B) Les surnageants des cultures transfectées avec les différents ARN recombinants et contrôles ont été prélevés à 72h p.tf. et mis en contact avec des cellules Huh-7.5 naïves. Les valeurs moyennes de l'activité FLuc mesurée à 72h p.i. dans des lysats de ces cellules sont représentées. Les résultats proviennent de 3 transfections indépendantes, chacune réalisée en duplicats. Les valeurs sont calculées relativement à la valeur du génome parental de chaque transfection et exprimées par rapport à la valeur moyenne obtenue pour le parent au cours des trois expériences. La ligne en pointillée représente le seuil de détection. Le test statistique est une ANOVA univariée réalisée sur les valeurs normalisées, avec PRISM. Les \textit{p-values} ont été ajustées par la méthode de Dunnett : p <0,0001 (****); p <0,05 (*); p >0,05 (ns : non significatif).}
				\label{fig:fig56}
	\end{figureth}
	
	\FloatBarrier

Pour déterminer si la transfection des ARN recombinants conduisait à la sécrétion de particules virales infectieuses, des cellules Huh-7.5 naïves ont été infectées avec les surnageants de culture des cellules transfectées, récoltés à 72h p.tf. Les cellules ont été lysées 72h après la mise en contact avec les surnageants. L'activité FLuc mesurée dans les lysats de cellules infectées par les différents virus recombinants est représentée dans la \Autoref{fig:fig56}\textcolor{blue}{.B}. Pour l’ARN Jad-2EIL3/$\Delta$Ep7 déficient pour l'assemblage, on observe une activité FLuc d'environ 2 log RLU, seuil sous lequel l'ARN viral est incapable de conduire à la production de particules virales. Pour l'ARN parental Jad-2EIL3, on observe une valeur de l'activité FLuc supérieure à ce seuil d'environ 4,5 log RLU, ce qui témoigne d'une sécrétion robuste de particules virales infectieuses dans le surnageant de culture des cellules transfectées. Pour la majorité des ARN recombinants intergénotypiques, on obtient une activité FLuc moyenne d’une valeur supérieure de 4,4 à 5 log RLU par rapport au seuil défini. Ce résultat indique que la présence de ces séquences Core hétérologues dans le contexte Jad-2EIL3 n’altère pas la production de particules virales infectieuses. De façon notable, l’activité FLuc moyenne mesurée pour l’ARN Jad-2EIL3/C3a-389 est supérieure au seuil d'environ 4 log RLU mais significativement inférieure à l'activité FLuc observée pour l'ARN parental Jad-2EIL3. Cette légère diminution de la sécrétion de particules virales infectieuses pourrait être indirectement causée par la réplication génomique affaiblie de cet ARN recombinant. En revanche, l’activité FLuc moyenne mesurée pour l’ARN Jad-2EIL3/C3a-376 est supérieure au seuil d'environ 3 log RLU seulement, ce qui conduit à une production de particules virales infectieuses à des titres intermédiaires. Cet ARN recombinant ayant une réplication génomique intacte, ce résultat indique que le remplacement de la séquence C3a-376 dans le contexte bicistronique affecte substantiellement la morphogénèse virale. Comme décrit précédemment, la séquence C3a-376 est la plus singulière des séquences de génotype 3 étudiées, avec plusieurs résidus naturels uniques dans les régions D1 et D2 (voir \autoref{section:souches}). Cette singularité de séquence pourrait perturber l’interaction de Core 376 avec les autres facteurs viraux de génotype 2 et diminuer l’efficacité d’assemblage des particules virales.

\subsubsection{Amélioration de la robustesse du virus intergénotypique codant la séquence Core 3a-376 par maintien à long terme des cultures hépatocytaires transfectées}

La substitution de la séquence Core du Jad par des séquences de génotype 3 n’affectant globalement pas la réplication génomique ni la morphogénèse des virions dans le contexte génomique contenant un rapporteur FLuc, les séquences Core hétérologues ont été transférées au sein de l’ADNc monocistronique du Jad (\autoref{fig:fig57}).

				\begin{figureth}
	\centering
			\includegraphics[width=0.80\linewidth]{Figure_57.png}
		\caption[Représentation schématique des ADNc recombinants dérivés du contexte génomique Jad]{\textbf{Représentation schématique des ADNc recombinants dérivés du contexte génomique Jad.} Structure du génome de la souche Jad et des génomes recombinants dont la séquence Core a été substituée par la séquence correspondante des souches de génotypes 1 (en violet), 2 (en orange), 3 (en vert) et 4 (en bleu). Les mutations adaptatrices conférant à la souche Jad une infectiosité élevée en lignée hépatocytaire Huh-7.5 sont indiquées : deux substitutions dans la séquence codante de NS5A (V2153A, V2440L) et une substitution dans la séquence codante de NS5B (V2941M).}
				\label{fig:fig57}
	\end{figureth}
	\FloatBarrier

La majorité des ARN recombinants générés par transcription \textit{in vitro} des ADNc plasmidiques, conduisent à une synthèse robuste de particules virales dans le surnageant de culture 3 jours après transfection de cellules Huh-7.5, de l’ordre de 4 à 8x10\up{4} TCID50/mL, équivalente à celle de la souche parentale. Seule la production du virus intergénotypique Jad/C3a-376 est modeste, avec un titre infectieux de l’ordre de 1 à 5x10\up{2} TCID50/mL 4j p.tf., ce qui concorde avec les résultats du suivi de l’activité FLuc obtenus dans le contexte bicistronique. L’ARN Jad/C3a-376 conduit à une production virale non négligeable, mais pourrait être améliorée pour atteindre les performances de la souche Jad et des autres recombinants intergénotypiques. Dans ce but, des passages en séries de cellules Huh-7.5 transfectées avec l’ARN intergénotypique correspondant ont été effectuées, en parallèle de cellules transfectées avec l’ARN parental Jad pour comparer l’évolution des titres viraux. Compte tenu de l’activité mutagène naturelle de l'ARN polymérase peu fidèle du VHC, il était attendu que cette opération permette la sélection d'une ou de plusieurs substitution(s) nucléotidique(s) au sein du génome viral recombinant, apte(s) à compenser le déséquilibre induit par l’insertion d’une séquence Core hétérologue. Le résultat du titrage des particules infectieuses dans le surnageant au cours des différents passages cellulaires est représenté dans la \Autoref{fig:fig57}. Quatre tentatives indépendantes d’adaptation du virus Jad/C3a-376 ont été réalisées au cours de trois transfections avec des ARN issus de deux réactions de transcription distinctes (représentées par les lettres A, B et C). Au cours de l’expérience B, le contrôle Jad n’a pas été inclus.

				\begin{figureth}
	\centering
			\includegraphics[width=0.80\linewidth]{Figure_58.png}
		\caption[Suivi de l'adaptation potentielle du virus recombinant intergénotypique Jad/C3a-376 par passages successifs en culture cellulaire]{\textbf{Suivi de l'adaptation potentielle du virus recombinant intergénotypique Jad/C3a-376 par passages successifs en culture cellulaire.} Des cellules Huh-7.5 ont été transfectées avec les ARN transcrits \textit{in vitro} Jad/C3a-376 (en vert) ou Jad (en noir). Les titres viraux infectieux ont été déterminés à chaque passage cellulaire. L’ARN intracellulaire du Jad/C3a-376 a été séquencé (SEQ) au niveau de la région codant Core-E1 ou de la région codante complète (ORF) aux temps indiqués par les flèches rouges ou bleues. Quatre expériences indépendantes d’adaptation ont été réalisées, indiquées par les lettres A, B et C avec un ou deux ARN (\#).}
				\label{fig:fig58}
	\end{figureth}
	\FloatBarrier
	
Au fil des passages, on distingue une différence notable de morphologie entre les cultures transfectées avec l'ARN Jad, qui montrent une souffrance cellulaire significative dès les premiers passages, et les culture transfectées avec l'ARN Jad/C3a-376, qui ne montrent pas ou peu d’effet cytopathogène (ECP). En raison de l’ECP trop marqué, les cellules transfectées par l’ARN parental Jad n’ont plus été maintenues en culture après 18 ou 19j p.tf. Les titres infectieux du parent Jad sont de l’ordre de 4,5 à 5,0 log TCID50/mL à 4j p.tf., alors que ceux du virus Jad/C3a-376 sont inférieurs d’environ 2 log. Au cours des passages, les titres infectieux du Jad sont stables, voire diminuent à hauteur de 3,8 à 4,5 log à 19j p.tf., principalement en raison de la mortalité cellulaire liée à l’infection persistante. En revanche, les titres infectieux du recombinant Jad/C3a-376 augmentent progressivement au cours des passages et atteignent un niveau identique aux titres du Jad à partir de 17j p.tf. Les cellules transfectées avec l’ARN Jad/C3a-376 ont ainsi été maintenues jusqu’à 28j p.tf. (pour les expériences A et C) afin de continuer à suivre l’évolution du titre infectieux. L'augmentation nette du titre viral du Jad/C3a-376 entre 4 et 14j p.tf. au cours des quatre expériences indépendantes démontre une robustesse virale accrue, liée potentiellement à la sélection de mutations adaptatives. Un premier séquençage de la région codant la protéine Core du virus recombinant Jad/C3a-376 à 4j p.tf. a permis de vérifier l’intégrité de la séquence Core hétérologue. D'après les mesures des titres infectieux, l'adaptation du virus Jad/C3a-376 semble avoir lieu dès 14j p.tf. J’ai choisi néanmoins de rechercher les éventuelles substitutions nucléotidiques à partir d’ARN viraux collectés plus tardivement, afin de laisser le temps aux éventuelles mutations de se stabiliser au sein des génomes circulants. Dans ce but, les ARN viraux isolés à 18, 22 ou 27j. p.tf. à partir des quatre transfections ont été rétro-transcrits  et les ADNc résultants ont été amplifiés par PCR pour obtenir des fragments chevauchants et couvrant l’ensemble de l’ORF, ainsi qu’une partie des régions non codantes (nucléotides 200-9480 du génome viral). Aucune substitution notable n’a été détectée, ni dans la région codant Core, ni dans le reste de l’ORF, suggérant l’absence de mutations de compensation dans les régions génomiques séquencées. Des mutations de compensation auraient pu être attendues dans NS5A compte tenu des interactions étroites connues entre les protéines Core et NS5A \citep{RN491}. Cependant, cette stratégie de séquençage n’a pas permis d’obtenir d'information concernant les séquences des 200 nucléotides 5’ terminaux et du domaine X de la région 3'NC. On peut raisonnablement formuler l'hypothèse qu'une mutation dite « de second site » dans la région X très conservée de 98 nucléotides puisse être apparue pour optimiser l'interaction de Core avec la région 3’NC, décrit comme importante durant le processus d'encapsidation \citep{RN986}. Il conviendrait de compléter le séquençage des régions non codantes pour identifier des preuves moléculaires de l’existence d’interaction(s) de Core avec les régions NC et de l’importance de telle(s) interaction(s) pour l’assemblage de particules infectieuses. Il est important de noter que la robustesse du virus Jad/C3a-376 ainsi obtenu a été vérifiée en utilisant le surnageant collecté à 22j p.tf. pour infecter des cellules naïves en parallèle du virus parental. La production virale post-infection s’est révélée similaire pour les deux virus de 24 à 96h p.i. (données non montrées), indiquant que le virus Jad/C3a-376 a été adapté avec succès et de façon stable à la culture cellulaire.

\subsubsection{Analyse des cinétiques de réplication des virus intergénotypiques exprimant les protéines Core hétérologues dans le contexte génomique Jad}

Des stocks viraux ont ensuite été produits pour tous les virus intergénotypiques, à partir du surnageant de culture des cellules transfectées, collecté à 25j p.tf. pour le recombinant Jad/C3a-376 « adapté » ou collecté à 3j p.tf. pour les autres virus recombinants. Les titres viraux des stocks collectés à 5 ou 7 jours suivant l’infection sont de l’ordre de 1 à 5x10\up{5} TCID50/mL. Dans l’objectif de comparer en détail la cinétique de réplication des virus intergénotypiques avec celle de la souche parentale, trois paramètres ont été suivis au cours de l’infection : (i) abondance de l’ARN viral intracellulaire, (ii) titrage des particules virales sécrétées par quantification du génome viral, (iii) titrage des particules infectieuses sécrétées. Pour cela, des cellules Huh-7.5 ont été infectées à haute MOI (3 TCID50/cellule) avec les différents virus intergénotypiques ou la souche parentale Jad. Au terme de la période d’adsorption virale à 4h p.i., les cellules ont été extensivement lavées afin d’éliminer l’inoculum viral. Les surnageants ainsi que les lysats cellulaires ont été collectées à différents temps post-infection : 4, 12, 24, 48, 72, 96h et 120h p.i. (\autoref{fig:fig59}). \\

				\begin{figureth}
	\centering
			\includegraphics[width=0.85\linewidth]{Figure_59.png}
		\caption[Schéma de la méthode expérimentale pour évaluer la cinétique de réplication des virus intergénotypiques]{\textbf{Schéma de la méthode expérimentale pour évaluer la cinétique de réplication des virus intergénotypiques.} Des cellules Huh-7.5 ont été infectées par les différents virus intergénotypiques ou par la souche parentale Jad à haute MOI (3 TCID50/cellule). Les surnageants de culture et les lysats cellulaires ont été collectés aux temps p.i. indiqués. Les titres infectieux des surnageants ont été déterminés par titrage en dilutions limites (TCID50). La quantité des ARN viraux extracellulaires, issus des particules virales sécrétés, et des ARN viraux intracellulaires, issus des lysats cellulaires, a été mesurée par rétro-transcription et PCR quantitative en une étape (RT-qPCR).}
				\label{fig:fig59}
	\end{figureth}
	\FloatBarrier

Au cours de l’expérience de cinétique, on distingue une différence notable de morphologie entre les cellules contrôles non infectées et les cultures infectées avec la souche parentale ou les virus recombinants (\autoref{fig:fig60}). Dans les cultures infectées par la souche parentale ou par les virus recombinants, un ralentissement de la croissance cellulaire est visible dès 48h p.i. A la fin de l’expérience, les cultures de cellules infectées n’atteignent pas la confluence. À l’inverse, les cultures contrôles naïves atteignent la confluence au point 96h. La souche parentale engendre une souffrance cellulaire marquée, particulièrement notable à 96h et 120h p.i., qui est remarquablement beaucoup moins visible pour l’ensemble des virus recombinants.

			\begin{figureth}
	\centering
			\includegraphics[width=\linewidth]{Figure_60.png}
		\caption[Suivi du profil morphologique des cultures d’hépatocytes naïfs et d’hépatocytes infectés par la souche parentale Jad ou par les virus intergénotypiques]{\textbf{Suivi du profil morphologique des cultures d’hépatocytes naïfs et d’hépatocytes infectés par la souche parentale Jad ou par les virus intergénotypiques.} Images des tapis cellulaires au cours d’une expérience de cinétique, de 24 à 120h p.i., obtenues par microscopie à champ lumineux ou « bright field » au grossissement 10X. Les images des cultures naïves (-) sont situées dans le panneau du haut et les images des cultures infectées par la souche parentale Jad ou par les virus recombinants indiqués sont montrés en dessous.}
				\label{fig:fig60}
	\end{figureth}
	\FloatBarrier

Les analyses quantitatives de l’étude cinétique sont représentées dans la \Autoref{fig:fig61}, et séparées arbitrairement en deux groupes de virus, afin de faciliter la lecture des représentations graphiques. Les données proviennent de six expériences indépendantes réalisées au fil de l’obtention des différents virus recombinants et ont été normalisées par rapport au contrôle parental inclus au sein de chaque expérience. Cette opération permet de s’affranchir de la variabilité entre expériences provenant principalement de la permissivité des lots de cellules utilisés pour l’infection et \textit{in fine} de comparer les paramètres de l’ensemble des virus recombinants. Les ARN viraux intracellulaires et extracellulaires ont été quantifiés après rétro-transcription et PCR quantitative en temps réel (RT-qPCR), avec un seuil de 50 à 100 copies selon les lectures. Pour s’affranchir des variations de dépôt, les ARN viraux intracellulaires quantifiés par µg d'ARN total ont été normalisés à partir de la quantité d’ARN 18S mesurée au sein du même échantillon. \\ \\
\indent
Les valeurs des ARN intracellulaires mesurés à 4h p.i. se situent autour de 10\up{6} à 10\up{7} copies/µg ARN total pour tous les virus, indiquant que l’efficacité d’entrée est similaire pour les virus recombinants et la souche parentale (\autoref{fig:fig61}\textcolor{blue}{.A}). On note une réduction globale de 0,5 à 1 log du niveau d'ARN viral intracellulaire à 12h p.i., indiquant que la réplication génomique n’a pas encore pris le pas sur la dégradation des génomes introduits à ce temps p.i. À partir de 24h p.i., on détecte une nette augmentation du niveau d’ARN intracellulaire accompagnée d’une synthèse de particules virales infectieuses pour l’ensemble des virus, indiquant que les étapes de réplication génomique, de morphogénèse et de sécrétion des virions s’initient entre 12 et 24h p.i. (\autoref{fig:fig61}\textcolor{blue}{.C}). Un cycle viral complet peut donc avoir lieu en 24h, mais le plateau de sécrétion des particules infectieuses est atteint à 72h ou 96h p.i., avec des titres de l’ordre de 1 à 5x10\up{5} TCID50/mL. Les cinétiques de réplication ont un profil relativement similaire entre les virus recombinants et la souche parentale, tant au niveau de l’ARN intracellulaire, de l’ARN extracellulaire que du titre viral. Toutefois, on note une production virale légèrement plus rapide et plus importante pour la plupart des recombinants que pour le parent Jad, à l’exception notable des virus Jad/C1aH77 et Jad/C1a-411. Ces résultats indiquent une efficacité d’assemblage et/ou de sécrétion de particules virales optimisée avec les séquences Core de génotype 3a ou 4, à l’inverse de celles de génotype 1a. On note également que le virus Jad/C3a-376 issus des passages tardifs des expériences d’adaptation montre une cinétique de réplication et un titre viral similaires aux autres virus recombinants codant les séquences de génotype 3a. Ces résultats soulignent le succès des tentatives d’adaptation de ce virus intergénotypique, initialement peu robuste.

			\begin{figureth}
	\centering
			\includegraphics[width=0.80\linewidth]{Figure_61.png}
		\caption[Cinétique de réplication des virus intergénotypiques dérivés du contexte génomique Jad]{\textbf{Cinétique de réplication des virus intergénotypiques dérivés du contexte génomique Jad.} \textit{La légende de la figure est décrite sur la page suivante \footnotemark[4].}}
	\label{fig:fig61}
	\end{figureth}
	\FloatBarrier
	
\footnotetext[4]{Mesure de (A) l’ARN viral intracellulaire par µg d’ARN total évalué par rapport à la quantification en temps réel d’ARN 18S (en log copies/µg d’ARN total), (B) l’ARN viral extracellulaire (en log copies/mL), (C) le titre viral infectieux (en log TCID50/mL) au cours de l’infection. Chaque virus a été analysé au cours de deux expériences indépendantes, chacune en duplicats. Les valeurs représentées sur les graphiques sont calculées relativement aux valeurs obtenues par Jad dans chaque expérience et exprimées par rapport aux valeurs moyennes obtenues sur l’ensemble des expériences. La souche de référence Jad est représentée par des courbes en pointillés noirs. Les tests statistiques ont été réalisés sur R à l’aide d’un modèle mixte linéaire, à partir des données brutes afin d’éviter les biais de la normalisation. Les \textit{p-values} situées dans les cadres en bas à droite des différents graphiques, indiquent les différences significatives de l’évolution des virus intergénotypiques par rapport au contrôle parental : significativement supérieur (en rouge) et significativement inférieur (en bleu). Les \textit{p-values} ont été ajustées par la méthode de Tukey : p <0,05 (*); p <0,01 (**); p <0,001 (***); p <0,0001 (****).}

Pour finir, j’ai comparé l’infectiosité spécifique de tous les virus recombinants et de la souche parentale Jad en calculant le ratio entre les particules virales infectieuses (TCID50) et les particules virales totales sécrétées (copies) (\autoref{fig:fig62}). Les ratios à 4h p.i. et 12h p.i. ne sont pas représentés car pas biologiquement pertinents en l’absence de néosynthèse de particules virales. On ne discerne pas de profil particulier en matière d’infectiosité spécifique entre les différents virus, quel que soit le temps p.i. On note toutefois une infectiosité spécifique légèrement supérieure pour les virus recombinants Jad/C3a-311, Jad/C3a-376 et plus faible pour les virus recombinants Jad/C2a-J6 et Jad/C4fC. On évalue à 250 à 1000 fois l’excès de particules non infectieuses selon les virus, ce qui est cohérent avec les ratios observés en culture cellulaire pour les recombinants J6(C-NS2)/JFH-1 décrits dans la littérature \citep{RN343}.

			\begin{figureth}
	\centering
			\includegraphics[width=\linewidth]{Figure_62.png}
		\caption[Comparaison de l’infectiosité spécifique des virus intergénotypiques dérivés du contexte génomique Jad]{\textbf{Comparaison de l’infectiosité spécifique des virus intergénotypiques dérivés du contexte génomique Jad.} L’infectiosité spécifique représente le ratio de particules virales infectieuses (TCID50) et de particules totales présentes dans le surnageant de culture (copies d’ARN).}
				\label{fig:fig62}
	\end{figureth}
	\FloatBarrier

\subsubsection{Analyse de la propagation des virus intergénotypiques exprimant les protéines Core hétérologues dans le contexte génomique Jad}

Afin de compléter la caractérisation des virus intergénotypiques, j’ai comparé leur propagation au sein des cultures cellulaires infectées par rapport à la souche parentale par microscopie confocale. Pour ce faire, la proportion de cellules positives pour une protéine virale 24, 48, 72 et 96h après infection avec les virus correspondants à 1 TCID50/cellule a été évaluée manuellement par microscopie confocale (\autoref{fig:fig63}\textcolor{blue}{.A}). Afin de ne pas biaiser l’évaluation, la protéine NS5A commune à tous les virus a été choisie pour cette étude. Le signal obtenu pour NS5A est très intense et relativement homogène, ce qui rend facilement identifiable les cellules infectées (\autoref{fig:fig63}\textcolor{blue}{.B}). Au sein des cellules infectées, NS5A est uniformément distribuée dans le cytoplasme, ce qui est compatible avec une localisation au niveau du RE. On observe parfois des puncta intenses à proximité des GL à partir de 48h p.i., signant une concentration transitoire de la protéine dans des régions précises. Ces structures pourraient être attribuables aux sites de réplication génomique ou d’assemblage des particules virales, comme il a été suggéré lors d’une étude par microscopie électronique basée sur un réplicon sous-génomique codant la protéine NS5A fusionnée à la GFP \citep{RN985} ou de travaux sur la dynamique de NS5A-GFP en cellules vivantes infectées par la souches Jc1 \citep{RN983}. Pour estimer la proportion de cellules infectées par rapport au nombre total de cellules, le noyau a été révélé à l’aide du DAPI. Cette estimation a été réalisée à partir de 5 à 7 champs de cellules, comptabilisant en moyenne 500 à 700 cellules au cours de deux expériences indépendantes. Les proportions représentées ont été normalisées par rapport au contrôle parental respectif.

			\begin{figureth}
	\centering
			\includegraphics[width=\linewidth]{Figure_63.png}
		\caption[Propagation des virus intergénotypiques dérivés du contexte génomique Jad par microscopie confocale (Partie 1)]{\textbf{Propagation des virus intergénotypiques dérivés du contexte génomique Jad par microscopie confocale (Partie 1).} (A) Schéma simplifié de l’approche expérimentale pour analyser la propagation des virus intergénotypiques au cours de l’infection. Des cellules Huh-7.5 ont été infectées par les différents virus intergénotypiques ou la souche parentale Jad à une MOI de 1 TCID50/cellule et fixées à 24, 48, 72 ou 96h p.i. (B) Champs représentatifs de cellules Huh-7.5 infectées par différents virus aux temps p.i. indiqués, obtenus par projection d’intensité maximale. La protéine virale NS5A a été révélée par immunomarquage (en jaune) et les noyaux cellulaires par incorporation du DAPI (en bleu). Les cellules comportant un signal positif pour NS5A sont mises en évidence par des cadres jaunes pointillés.}
				\label{fig:fig63}
	\end{figureth}
	\FloatBarrier

Le taux de cellules positives pour NS5A augmente progressivement au cours de l’infection, de manière similaire pour le virus parental et la plupart des virus recombinants, à l'exception notable de Jad/C4aR et Jad/C4fC qui se propagent plus rapidement (\autoref{fig:fig64}). À 24h p.i, entre 5 et 15\% des cellules sont positives pour NS5A pour la souche parentale et les virus intergénotypiques codant les protéines Core 1aH77, 2a-J6, 3a-311, 3a-376 et 3a-395, alors que cette proportion est de 30 à 35\% pour les virus intergénotypiques codant les protéines Core 4aR et 4fC. Ces derniers ont colonisé environ 75\% des cellules dès 48h p.i., puis atteint 95\% des cellules dès 72h p.i. La souche parentale ainsi que les autres virus recombinants atteignent un taux d’infection à hauteur de 70 à 80\% à 72h p.i., qui augmente légèrement jusqu’à atteindre 80 à 95\% de cellules infectées à 96h p.i. On peut noter que la quasi-totalité des cellules sont infectées par tous les virus parental et intergénotypiques à 96h p.i. dans des conditions d’infection initiales à 1 TCID50 / cellule. Ainsi, ces résultats démontrent que les virus intergénotypiques ont une efficacité de dissémination similaire, voire supérieure (Jad/C4aR et Jad/C4fC) à celle de la souche parentale. La diffusion plus dynamique des virus recombinants Jad/C4aR et Jad/C4fC entre 24h et 72h p.i. est cohérente avec les données issues de la cinétique de réplication, soulignant encore une fois la performance des virus intergénotypiques codant les protéines Core de génotype 4. Cette analyse n’a pas encore été étendue à tous les virus intergénotypiques et il serait intéressant de la compléter pour les virus recombinants codant core de génotype 3a qui ont également montré une cinétique de réplication légèrement plus rapide (Jad/C3a-389, Jad/C3a-390 et Jad/C3a-401). Par ailleurs, la proportion entre la distribution de NS5A strictement « réticulée » et la distribution en puncta est du même ordre de grandeur pour les différents virus, montrant que la substitution de Core n’a pas d’impact significatif sur la localisation de la protéine virale NS5A.
	\vfill
			\begin{figureth}
	\centering
			\includegraphics[width=\linewidth]{Figure_64.png}
		\caption[Propagation des virus intergénotypiques dérivés du contexte génomique Jad par microscopie confocale (Partie 2)]{\textbf{Propagation des virus intergénotypiques dérivés du contexte génomique Jad par microscopie confocale (Partie 2).} Estimation des proportions de cellules positives pour NS5A au sein du tapis cellulaire infecté par la souche parentale Jad ou par les différents virus intergénotypiques testés. L’analyse a été conduite sur 5 à 7 champs de cellules, dans deux expériences indépendantes. Les proportions sont calculées relativement aux valeurs obtenues pour la souche parentale au sein de chaque expérience et exprimées par rapport aux valeurs moyennes inter-expériences du parent.}
				\label{fig:fig64}
	\end{figureth}
	\FloatBarrier
	\vfill
	\clearpage

Pour conclure ce chapitre, l’ensemble de ces résultats montre que la substitution de la protéine Core du JFH-1 par des protéines Core hétérologues de divers génotypes n’affecte pas de façon majeure les étapes de traduction, réplication génomique et morphogénèse virale, à l’exception d’une seule séquence de génotype 3a. Un virus intergénotypique résultant de l’insertion de cette dernière séquence (Jad/C3a-376) aussi robuste que la souche parentale et sans altération de la séquence Core initiale a néanmoins pu être sélectionné. Nous avons donc généré un large éventail de virus intergénotypiques uniques qui se propagent et se répliquent à un niveau et selon une cinétique similaires à ceux de la souche prototypique parentale, ce qui permet d’effectuer des analyses comparatives dans les cellules d'hépatome infectées.

		\subsection{Analyse du profil et de la localisation intracellulaire des protéines Core hétérologues exprimées par les virus intergénotypiques}
		\label{section:localisation}

Pour réaliser l’étude du profil et de la localisation des protéines Core hétérologues vis-à-vis des GL, il était nécessaire d’avoir un anticorps monoclonal dit « pan-génotypique », c’est-à-dire capable de reconnaître avec la même affinité les variants de Core, indépendamment de leur origine génotypique. Plusieurs anticorps monoclonaux ciblant la protéine Core du VHC ont été commercialisés, la plupart contre les protéines Core des souches prototypiques disponibles, telles que JFH-1 de génotype 2a, H77 de génotype 1a ou encore Con1 de génotype 1b. En sachant que les séquences de Core de génotype 3a sont relativement distantes des séquences de génotypes 2a et 1a (voir \autoref{section:souches}) et que ces protéines pouvaient présenter des différences conformationnelles, l’efficacité de reconnaissance de ces protéines par ces anticorps commerciaux pouvait être affectée.

	\subsubsection{Evaluation de la capacité d’anticorps monoclonaux à reconnaître les protéines Core hétérologues de différents génotypes}
	
Dans un premier temps, j’ai testé plusieurs anticorps monoclonaux commerciaux produits chez la souris : l’anticorps 1851 ciblant la protéine Core de la souche H77, les anticorps 4F5 et 3D11 ciblant la protéine Core de la souche JFH-1 et l’anticorps C7-50 ciblant la protéine Core de la souche Con1 (voir \autoref{tab:tabM1}). En parallèle, notre laboratoire a obtenu deux solutions d’anticorps non commercialisés : un anticorps monoclonal ACAP27 produit chez la souris contre la protéine Core de la souche H77 donné généreusement par A. Budkowska (Institut Pasteur) et des anticorps polyclonaux (FL) produits chez le lapin immunisé avec la protéine Core de cette même souche, donnés généreusement par A. Kakkanas (\textit{Hellenic Pasteur Institute, Athens, Greece}). En théorie, les anticorps polyclonaux FL devraient reconnaître efficacement tous les variants de Core, compte tenu de la présence de régions très conservées entre les séquences, et donc de la grande probabilité d’avoir des épitopes en commun. Ce sérum était donc précieux pour contrôler et comparer l’expression et la stabilité des protéines Core hétérologues produites par les virus intergénotypiques. \\ \\
\indent
Cette panoplie d’anticorps a été testée sur des lysats protéiques collectés à partir de cellules Huh-7.5 infectées par les différents virus intergénotypiques ou par la souche parentale à 96h p.i. Cette échéance tardive a été choisie afin de limiter les différences en production de protéines virales, qui pourraient survenir lorsque les cinétiques de réplication virale ne sont pas parfaitement synchrones. Des immunoblots révélés par fluorescence dans le proche infrarouge ont permis de quantifier l’expression de la protéine virale Core. Ce système fluorescent permet en effet d’effectuer une quantification fiable et linéaire du signal jusqu’à un seuil de saturation du détecteur \citep{RN981} et de renseigner sur les différences fines d’abondance entre les différentes protéines Core. Les variations de dépôts étant fréquentes pour ce type d’approche expérimentale, il est nécessaire de normaliser systématiquement le signal de Core par rapport à une référence cellulaire invariable pour lisser les variations de dépôts des protéines totales. C’est d’autant plus important que l’infection ralentit la croissance cellulaire et que seule a souche parentale induit une mortalité cellulaire très élevée à 96h p.i. (voir \autoref{section:virus}). Les protéines virales n’atteignaient jamais le seuil de saturation du détecteur et leur signal figurait toujours dans la gamme linéaire de détection. En revanche, l’actine ß étant trop abondante, le signal a tendance à atteindre le plateau de détection. Ainsi, l’actine ß ne constitue pas un bon candidat pour la normalisation quantitative des dépôts et fera seulement office de contrôle visuel des extraits. À la place, les protéines totales ont été quantifiées après révélation avec une solution REVERT (Li-Cor), compatible avec ce système de détection par fluorescence \citep{RN980,RN979}. La protéine virale NS5B, commune à l’ensemble des virus, a été révélée en guise de contrôle visuel du niveau d’infection.

			\begin{figureth}
	\centering
			\includegraphics[width=\linewidth]{Figure_65.png}
		\caption[Evaluation de l’efficacité de reconnaissance des protéines Core exprimées par les virus intergénotypiques par différents anticorps monoclonaux et polyclonaux]{\textbf{Evaluation de l’efficacité de reconnaissance des protéines Core exprimées par les virus intergénotypiques par différents anticorps monoclonaux et polyclonaux.} (A, C, E) Les protéines totales extraites à 96h p.i. des cellules Huh7.5 infectées avec les virus intergénotypiques ou avec la souche parentale ont été séparées par électrophorèse SDS-PAGE. La détection des protéines virales NS5B et Core et de l’actine ß sont montrées séparément. L’anticorps utilisé pour révéler la protéine Core est indiqué en bas de chaque panneau. Les marqueurs de masses moléculaires (kDa) sont indiqués à droite des images. (B, D, F) Analyse quantitative de l’abondance relative des protéines Core hétérologues (barres colorées) par rapport à la protéine native (barre noire). Les valeurs d’intensité du signal de Core ont été normalisées par rapport aux protéines totales révélées par la solution REVERT. Les graphiques représentent des moyennes et écart-types issus de 2 à 3 séries d’immunoblots indépendantes. Le seuil de 100\% (ligne pointillée) est établi à la valeur moyenne de Jad/C1aH77, protéine contre laquelle les anticorps 1851, FL et ACAP27 ont été produits.}
				\label{fig:fig65}
	\end{figureth}
	\FloatBarrier

Comme attendu, le signal détecté par l’anticorps polyclonal est élevé pour toutes les protéines Core exprimées par les virus intergénotypiques, confirmant un spectre élargi à tous les génotypes (\autoref{fig:fig65}\textcolor{blue}{.A}). L’analyse quantitative révèle toutefois une abondance relative de 40 à 60\% par rapport à la protéine Core de la souche parentale pour toutes les protéines Core exprimées par les virus intergénotypiques, suggérant que les variants de Core seraient moins stables que la protéine native à 96h p.i. (\autoref{fig:fig65}\textcolor{blue}{.B}). Le signal détecté par l’anticorps 1851 est élevé pour la protéine Core 1a-H77 contre laquelle il a été produit, y compris pour pour l’autre protéine Core de génotype 1a (1a-411) et pour les protéines Core de génotype 2 (JFH-1, J6) (\autoref{fig:fig65}\textcolor{blue}{.C-D}). En revanche, le signal détecté pour les protéines Core de génotype 3a est diminué de ~80 à 90\%, et dans une moindre mesure, de ~50\% pour les protéines Core de génotype 4. Ce résultat démontre que l’anticorps monoclonal 1851 ne reconnaît pas efficacement les protéines Core de génotype 3a. Les autres anticorps monoclonaux commerciaux 4F5, 3D11 et C7-50 ne reconnaissent pas non plus les protéines Core de génotype 3a et l’anticorps 3D11 s’est avéré de plus incapable de reconnaître les protéines Core de génotype 4 (résultats non montrés). En revanche, l’anticorps monoclonal ACAP27 montre un profil étendu à l’ensemble des protéines Core hétérologues, y compris les protéines Core de génotype 3a (\autoref{fig:fig65}\textcolor{blue}{.E-F}). \\ \\
\indent
Ces résultats peuvent s’expliquer par les variations de séquences au sein des épitopes cibles des anticorps, lorsque ceux-ci sont connus. Cette recherche n’a pas été totalement fructueuse et seuls les épitopes des anticorps 1851, C7-50 et ACAP27 m’ont été renseignés (\autoref{fig:fig66}). L'épitope ciblé par les anticorps 1851 et C7-50 correspondant aux résidus 30 à 40 est identique pour les souches considérées de génotypes 1a, 1b, 2a ou 4 mais contient une valine en position 36 au sein de toutes les séquences de génotype 3. Ce résidu unique au génotype 3 pourrait être la cause de la faible affinité de ces anticorps pour les protéines Core appartenant à ce génotype.  L’épitope ciblé par ACAP27 est plus large (résidus 39 à 72) et bien conservé pour la grande majorité des souches. Les quelques mutations naturelles présentes dans certaines séquences de Core, notamment de génotype 2a, n’apparaissent pas gêner l’efficacité de reconnaissance par l’anticorps ACAP27 par immunoblot. Il a ensuite été vérifié que ces conversions n'affectent pas non plus l’efficacité de reconnaissance de l'épitope conformationnel par l’anticorps ACAP27 en conditions natives, lors des analyses d’imagerie par immunofluorescence.

			\begin{figureth}
	\centering
			\includegraphics[width=\linewidth]{Figure_66.png}
		\caption[Alignement partiel des séquences protéiques de Core des souches prototypiques et des souches cliniques]{\textbf{Alignement partiel des séquences protéiques de Core des souches prototypiques et des souches cliniques.} Seuls les acides aminés en positions 1 à 80 de Core sont affichés. Les séquences sont classées selon le génotype de leur souche d’origine. Les nucléotides identiques sont représentés par des points noirs. Les nucléotides les plus communs sont de couleur noire et les résidus les moins communs sont de couleur rouge. Des cadres pointillés indiquent l’épitope cible connu des anticorps monoclonaux.}
				\label{fig:fig66}
	\end{figureth}
	\FloatBarrier

En conclusion, ces analyses comparatives démontrent qu’il n’est pas judicieux d’utiliser les anticorps monoclonaux commerciaux produits contre les protéines Core de JFH-1 ou Con1 pour étudier notre panel de virus intergénotypiques, en raison de leur manque d’efficacité à reconnaître les protéines Core de génotype 3. En revanche, l’anticorps ACAP27 présente une spécificité croisée satisfaisante pour les protéines Core de génotypes 1, 2, 3 et 4. Cet anticorps a donc été sélectionné pour la suite de l’étude. De plus, grâce à des anticorps polyclonaux, j’ai mis en évidence que les protéines Core hétérologues ont une stabilité inférieure à la protéine native, sans affecter de façon notable l’efficacité de réplication et de production des virus intergénotypiques (voir \autoref{section:virus}).

	\subsubsection{Analyse du profil et de la localisation intracellulaire des différentes protéines Core hétérologues au cours de l’infection}

Ayant à disposition deux anticorps « pan-génotypiques », l’anticorps monoclonal ACAP27 et les anticorps polyclonaux FL, le profil et la localisation des différentes protéines Core hétérologues vis-à-vis des GL on été comparés par microscopie confocale. Pour effectuer cette étude, des cellules Huh-7.5 ont été infectées par la souche parentale et par les différents virus intergénotypiques puis mises en contact pendant 15h avec du BODIPY avant d’être fixées à 24h, 48h, 72h et 96h p.i, selon l’approche expérimentale schématisée dans la \Autoref{fig:fig67}. La protéine Core a été révélée par immunofluorescence indirecte en utilisant les anticorps polyclonaux FL produits par le lapin et la protéine virale NS5A à l'aide d'un anticorps monoclonal produit chez la souris.

			\begin{figureth}
	\centering
			\includegraphics[width=\linewidth]{Figure_67.png}
		\caption[Schéma simplifié de l’approche expérimentale pour analyser le profil et la localisation intracellulaire des différents variants de Core]{\textbf{Schéma simplifié de l’approche expérimentale pour analyser le profil et la localisation intracellulaire des différents variants de Core.} Des cellules Huh-7.5 ont été infectées par la souche parentale Jad ou par les virus intergénotypiques à une MOI de 1 TCID50/cellule. Les GL ont été révélées par incorporation du BODIPY en cellules vivantes et les protéines virales NS5A et Core ont été révélées par immunomarquage à l’aide d’un anticorps monoclonal de souris et de l’anticorps polyclonal FL produit chez le lapin après fixation des cellules aux différents temps p.i indiqués.}
				\label{fig:fig67}
	\end{figureth}
	\FloatBarrier

Pour la souche parentale Jad, la proportion de cellules positives pour NS5A augmente graduellement à mesure que l’infection progresse, jusqu’à atteindre la quasi-totalité de la culture (\autoref{fig:fig68}\textcolor{blue}{.A}), comme observé précédemment dans le cadre de l’étude de la dissémination virale à la même MOI (voir \autoref{section:virus}). Au sein des cellules infectées, on retrouve l’aspect de NS5A majoritairement réticulée dans le cytoplasme avec une intensité de signal homogène. En revanche, à ce grossissement de champs d’une quarantaine de cellules, on s’aperçoit que le signal de Core est très hétérogène d’une cellule infectée à une autre, y compris aux temps tardifs, en dépit de la forte stabilité de la protéine native. On distingue des cellules avec un signal global de Core « intense », « modéré » ou « faible » voire inexistant, bien que le co-marquage de NS5A confirme l’infection de ces cellules (\autoref{fig:fig68}\textcolor{blue}{.B}). Pour évaluer l’hétérogénéité du marquage de Core au cours de l’infection, j’ai comptabilisé et catégorisé manuellement les cellules présentant un signal positif pour NS5A (définies comme infectées) en trois groupes, sur la base de cette estimation visuelle de l’intensité de Core (\autoref{fig:fig68}\textcolor{blue}{.C}). La majorité des cellules infectées présente un signal de Core « modéré » à 24h p.i., tandis qu’environ 8\% des cellules montrant un signal de Core « intense ». À partir de 48h p.i., on observe une forte augmentation de la proportion en cellules infectées présentant un signal « intense » de Core, jusqu’à environ 50\%. Ces proportions restent ensuite globalement stables à 72h p.i. et 96h p.i., indiquant qu’il n’y a pas de progression linéaire de l’intensité du signal de Core, contrairement à celle de NS5A. On peut d’ailleurs constater qu’une part non négligeable de cellules infectées présentent un signal faible voire inexistant pour Core, de l’ordre de 10 à 18\% à mesure que l’infection progresse.

			\begin{figureth}
	\centering
			\includegraphics[width=\linewidth]{Figure_68.png}
		\caption[Analyse de l’évolution des marquages de NS5A et de Core au cours du temps après infection par la souche parentale Jad]{\textbf{Analyse de l’évolution des marquages de NS5A et de Core au cours du temps après infection par la souche parentale Jad.} (A) Images représentatives de champs de cellules infectées par la souche parentale Jad aux temps p.i. indiqués. De haut en bas, les protéines virales NS5A et Core et les GL ont été révélées respectivement par l’anticorps monoclonal 4F5, les anticorps polyclonaux FL et le BODIPY. En dessous, figure la combinaison des marquages de NS5A, Core et des GL. Les barres d’échelle indiquent 50µm. (B) Zoom sur les images prises à 96h p.i., établissant les trois groupes d’intensité du signal de Core « Intense », « Modéré », « Faible / à peine visible » mis en évidence par des cadres en pointillés cyans, bleus ou magentas respectifs. (C) Estimation de la proportion en cellules infectées sur la base d’un signal positif pour NS5A, selon l’intensité du signal de Core aux différents temps p.i. indiqués. Les proportions ont été évaluées sur 500 à 700 cellules.}
				\label{fig:fig68}
	\end{figureth}
	\FloatBarrier

Le profil de Core dans les cellules infectées par les virus intergénotypiques a ensuite été comparé avec la souche de référence. La proportion de cellules détectées avec un signal de Core défini comme « intense » sur la base du parent est significativement plus faible pour l’ensemble des virus recombinants et atteint une valeur plafond de l’ordre de 10 à 15\% à 96h p.i. (\autoref{fig:fig69}), bien qu’à cette étape, un plateau de production virale similaire à la souche parentale soit atteint. La majorité des cellules infectées par les virus recombinants présentent un signal de Core « modéré », ainsi qu’une proportion plus importante de cellules avec un signal « faible » que la souche parentale, de l’ordre de 30\%. Ces proportions ont été précisément déterminées pour quelques virus intergénotypiques codant les protéines Core de génotype 1, 2 ou 3 (\autoref{fig:fig70}) et l’étude pourrait être élargie à l’ensemble des virus recombinants, en particulier aux virus codant pour les protéines Core de génotype 4. La faiblesse du signal de Core n’est vraisemblablement pas attribuable à des affinités différentes de l’anticorps polyclonal pour les variants de Core ou à des différences de dissémination virale, comme le démontre le co-marquage de la protéine NS5A et les analyses précédentes (voir \autoref{section:virus}). Ainsi, le faible signal de Core dans les cellules infectées par les virus intergénotypiques est probablement lié à une stabilité moindre des protéines Core hétérologues par rapport à la protéine native, ce qui est cohérent avec les analyses par immunoblot.

		\begin{figureth}
	\centering
			\includegraphics[width=0.80\linewidth]{Figure_69.png}
		\caption[Analyse de l’hétérogénéité du marquage de Core dans les cellules infectées par les virus intergénotypiques (Partie 1)]{\textbf{Analyse de l’hétérogénéité du marquage de Core dans les cellules infectées par les virus intergénotypiques (Partie 1).} Images représentatives 3D obtenues par MIP (ou \textit{Maximum Intensity Projection}) à 96h p.i. de champs de cellules non infectées ou infectées par la souche parentale Jad ou par quelques virus intergénotypiques. Dans le panneau de gauche figure la révélation de la protéine NS5A par l’anticorps monoclonal 4F5 (en jaune) et la proportion de cellules positives est indiquée dans l’encart en bas à droite. Dans le deuxième panneau figure la révélation de la protéine Core par l’anticorps polyclonal FL (en cyan). Les cellules présentant un signal de Core défini comme « intense » sont mises en évidence par un cadre en pointillés et l’estimation de leur proportion par rapport au nombre de cellules infectées positives pour NS5A est indiquée. Dans le troisième panneau figure la révélation du corps lipidique des GL (en magenta) par incorporation du BODIPY. Dans le panneau de droite figure la combinaison des marquages de Core et des GL. Les barres d’échelle indiquent 50µm.}
				\label{fig:fig69}
	\end{figureth}
	\FloatBarrier

		\begin{figureth}
	\centering
			\includegraphics[width=0.80\linewidth]{Figure_70.png}
		\caption[Analyse de l’hétérogénéité du marquage de Core dans les cellules infectées par les virus intergénotypiques (Partie 2)]{\textbf{Analyse de l’hétérogénéité du marquage de Core dans les cellules infectées par les virus intergénotypiques (Partie 2).} Estimation de la proportion en cellules infectées à 96h p.i. sur la base d’un signal positif pour NS5A, selon l’intensité du signal de Core. Les proportions ont été évaluées sur 500 à 700 cellules.}
				\label{fig:fig70}
	\end{figureth}
	\FloatBarrier

L’anticorps monoclonal ACAP27, qui a été testé en parallèle, a révélé des profils similaires d’hétérogénéité de Core à ceux observés avec l’anticorps polyclonal pour l’ensemble des virus intergénotypiques et la souche parentale. Ainsi, l’anticorps ACAP27 ne présente pas de différences d’affinité notable pour les protéines Core de génotype 2, 3 ou 4 dans leur conformation native. En l’absence de bruit de fond cytoplasmique et nucléaire, contrairement à l’anticorps polyclonal, l’utilisation de cet anticorps a été privilégié pour étudier précisément la localisation des protéines Core hétérologues vis-à-vis des GL. Toutes les protéines Core produites par les virus intergénotypiques sont localisées à la surface des GL, comme le montre le signal de Core qui coïncide avec celui du BODIPY (\autoref{fig:fig71}). Elles présentent toutefois des taux de recouvrement des GL globalement plus faibles que la protéine native, probablement en raison d’une consommation plus rapide pour la synthèse des nucléocapsides, en accord avec le signal majoritairement « modéré ». Il est intéressant de noter que cette différence présumée de stabilité de Core n’a pas d’impact sur l’efficacité de production de particules infectieuses. Ces observations suggèrent que la protéine Core de la souche parentale est en « excès » sur les GL et qu’un autre facteur, viral ou cellulaire, serait limitant dans la production de particules virales. Par ailleurs, cette accumulation excessive de Core pourrait entraîner un effet cytotoxique plus important, du fait de son implication dans la dérégulation de nombreuses voies cellulaires (voir \autoref{section:core}), ce qui expliquerait la souffrance cellulaire massive que nous observons dans le contexte de la souche parentale.

		\begin{figureth}
	\centering
			\includegraphics[width=\linewidth]{Figure_71.png}
		\caption[Analyse de la localisation des protéines Core hétérologues vis-à-vis des gouttelettes lipidiques]{\textbf{Analyse de la localisation des protéines Core hétérologues vis-à-vis des gouttelettes lipidiques.} \textit{La légende de la figure est décrite sur la page suivante \footnotemark[5].}}
				\label{fig:fig71}
	\end{figureth}
	\FloatBarrier
	
	
		\subsection{L’élargissement des gouttelettes lipidiques varie selon les protéines Core hétérologues, indépendamment du génotype ou du tableau clinique}
		\label{section:gouttelettes}

\footnotetext[5]{Images représentatives de l’aspect de la protéine Core (en cyan) et des GL (en magenta) dans une cellule naïve ou dans une cellule infectée par la souche parentale Jad ou par les différents virus intergénotypiques à 96h p.i. La protéine Core est révélée par l’anticorps monoclonal ACAP27 et le corps lipidique des GL est révélé par le BODIPY. Dans les panneaux de droite figurent les agrandissements des zones indiquées. Les histogrammes représentent l’intensité du signal de Core et des GL sur une ligne tracée (non représentée) dans les zones agrandies respectives. Les barres d’échelle indiquent 5µm.}

Cette étude s’est basée sur l’hypothèse que la dérégulation de la biogenèse ou de la dynamique des GL dans les cellules infectées par le VHC pourrait représenter un événement préliminaire lié à la progression de la stéatose. Au cours du premier projet, j’ai montré que par l’intermédiaire de la protéine Core, l’infection par la souche parentale induisait un élargissement local des GL sans engager leur biosynthèse. Dans l’objectif d’identifier un éventuel effet différentiel de la biosynthèse et l’élargissement des GL en fonction de l’origine génotypique de Core ou du tableau clinique des souches, j’ai comparé le volume total et le volume moyen des GL dans des hépatocytes naïfs et dans des hépatocytes infectés par la souche parentale Jad ou par les 11 différents virus intergénotypiques, par imagerie quantitative, en suivant la même approche que dans l'étude précédente (décrite dans la \autoref{section:stabilisation}). En raison du caractère chronophage de cette analyse, cette comparaison n’a été effectuée que sur des cellules acquises à 96h p.i., temps auquel le processus d’élargissement est le plus marqué dans le contexte de la souche parentale. \\ \\
\indent
En sachant que les protéines Core hétérologues présentent des stabilités variables, j’ai pris soin de sélectionner les cellules qui présentent un signal « intense », pour effectuer cette analyse dans des conditions d’expression comparables de Core et dans l’objectif d’attribuer tout effet observé au niveau des GL à la séquence de Core et non pas à une différence dans l’abondance relative de cette protéine. Afin de vérifier que les cellules acquises par imagerie respectaient les critères souhaités, le volume total de Core a d’abord été comparé dans les infections par la souche parentale et par les virus intergénotypiques (\autoref{fig:fig72}\textcolor{blue}{.A}). Les niveaux d’expression de Core restent très hétérogènes entre les cellules sélectionnées, comme on peut le voir par l’étendue des violons dans les représentations graphiques, mais ils n’apparaissent pas significativement différents pour le virus parental ou les virus intergénotypiques. La proportion du volume de Core qui co-localise avec les GL a ensuite été comparé et on constate qu’en moyenne 43\% du volume de Core est associé aux GL pour le virus parental à 96h p.i. (\autoref{fig:fig72}\textcolor{blue}{.B}). Cette valeur peut paraître faible par rapport à l'appréciation visuelle de la distribution de la protéine Core, qui semble se retrouver exclusivement à la surface des GL. Il faut prendre en compte que cet algorithme mesure la proportion de pixels en co-localisation parfaite, alors que la méthode de marquage par le BODIPY révèle uniquement la partie lipidique centrale des GL. La protéine Core révélée par l’anticorps est quant à elle, localisée au niveau du manteau protéique supérieur de la GL, d’où le chevauchement « partiel » entre les signaux obtenus pour Core et le BODIPY. Une majorité des virus recombinants ne présente pas de différence significative dans le taux d'association de Core aux GL par rapport à Jad, à l'exception notable de Jad/C2a-J6, Jad/C1a-411, Jad/C3a-390, Jad/C3a-401 et Jad/C4fC, qui présentent un taux d'association modestement, mais significativement inférieur par le biais de cette méthode de marquage, de l’ordre de 31 à 37\%. Ces résultats valident nos observations précédentes : l’ensemble des protéines Core hétérologues sont essentiellement localisées à la surface des GL, bien qu’il y a des légères variations selon la séquence protéique.

		\begin{figureth}
	\centering
			\includegraphics[width=0.70\linewidth]{Figure_72.png}
		\caption[Analyses des paramètres relatifs à Core dans les cellules infectées par la souche parentale et par les virus intergénotypiques]{\textbf{Analyses des paramètres relatifs à Core dans les cellules infectées par la souche parentale et par les virus intergénotypiques.} (A) Analyse quantitative du volume total de Core (en µm\up{3}) à 96h p.i. dans les cellules infectées par la souche parentale (violon noir) ou par les différents virus intergénotypiques (violons colorés). (B) Analyse quantitative de la proportion du volume de Core co-localisant avec les GL (\%) dans les cellules infectées par les différents virus à 96h p.i. L’analyse a été conduite sur 40 cellules au cours de deux expériences indépendantes par virus. Les valeurs pour chaque virus recombinant sont exprimées relativement à la valeur moyenne obtenue pour Jad dans la même expérience puis rapportées à la valeur moyenne obtenue pour Jad sur l’ensemble des expériences. Les tests statistiques ont été réalisés sur R à l’aide d’un modèle mixte linéaire. Les \textit{p-values} sont ajustées par la méthode de Tukey : p <0,05 (*); p <0,01 (**); p <0,001 (***); p <0,0001 (****); p >0,05 (ns : non significatif).}
				\label{fig:fig72}
	\end{figureth}
	\FloatBarrier
	\vfill
	\clearpage

En ce qui concerne les paramètres liés aux GL, aucune différence significative du volume total n’a été observée par rapport aux cellules naïves tant dans les cellules infectées par la souche parentale que par virus les recombinants, à l’exception d’une légère augmentation pour le virus Jad/C3a-389 (\autoref{fig:fig73}\textcolor{blue}{.A}). Ces résultats sont cohérents avec ma première étude portant sur la souche parentale et montrent que les protéines Core de différents génotypes, y compris du génotype 3, n’induisent pas la biosynthèse des GL. En revanche, ces analyses montrent clairement une différence d’élargissement des GL dans les cellules infectées par les virus intergénotypiques (\autoref{fig:fig73}\textcolor{blue}{.B}). Les virus recombinants Jad/C2a-J6, Jad/C1aH77 et Jad/C3a-376 induisent une augmentation de la taille des GL relativement similaire à celle du parent Jad. L’élargissement maximal des GL cytosoliques est observé dans les cellules infectées par les virus recombinants Jad/C3a-311 et Jad/C3a-389, qui n’apparaît pas significativement supérieur à celui des cellules infectées par la souche parentale. Les protéines C3a-311 et C3a-389 sont issues de souches cliniques associées respectivement à une stéatose élevée et à une absence de stéatose. Le virus recombinant Jad/C1a-411, codant une protéine Core issue d’une souche clinique associée à une stéatose modérée, n’induit qu’une légère augmentation du volume moyen des GL. Enfin, les virus recombinants Jad/C3a-390, Jad/C3a-395, Jad/C3a-401, Jad/C4aR et Jad/C4fC n’induisent pas d’augmentation du volume des GL. Ces résultats démontrent qu’il n’existe pas de corrélation entre l’ampleur de l’élargissement des GL cytosoliques observé lors de l’infection et d’une part, l’origine génotypique de Core, ou d’autre part, le degré de stéatose développé par le patient. Il est intéressant de noter que la plupart des recombinants qui induisent peu ou pas d'augmentation du volume des GL présentent un taux d’association de Core aux GL globalement plus faible que les autres virus. Cette corrélation est cohérente avec notre étude précédente, qui suggère que l’accumulation de Core à la surface des GL serait directement liée à l’élargissement local des GL. Ainsi, les polymorphismes de séquence de Core semblent affecter son taux de recrutement aux GL ainsi que l’effet d’élargissement des GL cytosoliques, mais indépendamment de son origine génotypique. Par ailleurs, l’existence de telles souches du VHC hautement réplicatives n’induisant pas d’élargissement des GL, suggère que ce processus est dispensable à la production de particules virales et qu’il s’agirait d’une conséquence biologique liée à la nature de la protéine Core. Afin de compléter ces analyses, il serait intéressant d’évaluer si les virus intergénotypiques induisent la formation de clusters de GL et une redistribution à proximité des usines de réplication virale, comme la souche parentale.

		\begin{figureth}
	\centering
			\includegraphics[width=0.70\linewidth]{Figure_73.png}
		\caption[Analyse des paramètres relatifs aux GL dans les cellules infectées par la souche parentale et par les virus intergénotypiques.]{\textbf{Analyse des paramètres relatifs aux GL dans les cellules infectées par la souche parentale et par les virus intergénotypiques.} Analyse quantitative du volume total (A) et du volume moyen (B) des GL (en µm\up{3}) dans les cellules naïves (violin gris) et dans les cellules infectées par la souche parentale (violon noir) ou par les différents virus intergénotypiques (violons colorées) à 96h p.i. L’analyse a été conduite sur 40 cellules au cours de deux expériences indépendantes par virus. Les valeurs pour chaque virus recombinant sont exprimées relativement à la valeur moyenne obtenue pour Jad dans la même expérience puis rapportées à la valeur moyenne obtenue pour Jad sur l’ensemble des expériences. Les tests statistiques ont été réalisés sur R à l’aide d’un modèle mixte linéaire. Les \textit{p-values} sont ajustées par la méthode de Tukey : p <0,05 (*); p <0,01 (**); p <0,001 (***); p <0,0001 (****); p >0,05 (ns : non significatif).}
				\label{fig:fig73}
	\end{figureth}
	\FloatBarrier

		\subsection{Impact de l’origine génotypique de Core sur les dérégulations du transcriptome hépatique}
		\label{section:transcriptomique}
		
Une analyse comparative des transcriptomes de cellules infectées par les différents virus intergénotypiques a été entreprise afin de mettre en évidence d’éventuelles propriétés stéatogènes directes des séquences de Core. À cette fin, un séquençage ARN à haut débit (RNA-Seq) a été réalisé (\autoref{fig:fig74}\textcolor{blue}{.A}). Les données du séquençage ont été obtenues récemment et n’ont pas pu être analysées extensivement. En conséquence, les résultats présentés au cours de ce chapitre sont préliminaires. L’examen de l’abondance des ARNm cellulaires a été réalisé à partir d’extraits d’ARN collectés à 120h p.i., une échéance tardive à laquelle un taux d’infection proche de 100\% a été observé pour l’ensemble des virus. Par ailleurs, il a été montré qu’à partir de 4-5j p.i., le profil de dérégulations transcriptionnelles est établi et stable dans des cellules Huh-7.5 partiellement différenciées infectées par une souche de sous-type 2a \citep{RN934}. Quatres réplicats biologiques de cultures non infectées ou infectées avec notre panel de 11 virus intergénotypiques (6 virus recombinants codant Core de sous-type 3a, 2 virus recombinants codant Core de sous-type 1a, 2 virus recombinants codant Core de génotype 4, et un virus recombinant codant Core de sous-type 2a) ou avec la souche parentale Jad ont été préparés. Les lectures (ou \textit{reads}) ont été alignées sur le transcriptome humain ce qui représente un total de 58.735 transcrits (\autoref{fig:fig74}\textcolor{blue}{.B}). En raison de la faible expression d'une large part de gènes humains dans les cellules Huh-7.5, les transcrits donnant une valeur moyenne inférieure à 100 \textit{reads} sur l’ensemble des réplicats biologiques ont été retirés de l’analyse, ce qui a réduit la taille du crible à 10.671 transcrits. Ce critère permet de se restreindre aux gènes hautement exprimés par le tissu hépatique. \\ \\
\indent
Tout d’abord, nous pouvons remarquer que ~67\% des ARNm sont modulés par l’infection avec l’ensemble des virus recombinants, ce qui indique que le VHC altère sévèrement le transcriptome hépatocytaire. Cette proportion est cohérente avec les données récemment publiées par Lupberger et al. dans un modèle de cellules Huh-7.5 partiellement différenciées et infectées par la souche Jc1 \citep{RN934}. Sur ces 67\%, 770 ARNm (~11\%) sont fortement dérégulés par l’infection avec un rapport supérieur à 2 par rapport aux cellules naïves (exprimé en log2(foldchange) = log2FC > 1 pour une régulation positive ou < -1 pour une régulation négative). Une analyse GSEA (\textit{Gene Set Enrichment Data}) complémentaire met en évidence que ces gènes hautement dérégulés appartiennent à de nombreuses voies de signalisation telles que l’hypoxie, l’apoptose, le peroxisome, la voie de signalisation PPAR, la transition épithélio-mésenchymateuse qui est impliquée dans la cicatrisation et la fibrogénèse, les voies inflammatoires TNF$\alpha$ et TGFß, la réponse au stress, l’adipogénèse et le métabolisme des lipides, du cholestérol et des carbohydrates. L’inflammation, l’hypoxie, l’apoptose et la réponse au stress sont des voies cellulaires hautement induites par l’infection, tandis que le métabolisme lipidique ou la voie de signalisation du perixosome sont significativement ralenties (données non montrées).

		\begin{figureth}
	\centering
			\includegraphics[width=\linewidth]{Figure_74.png}
		\caption[Méthode expérimentale pour étudier la régulation des gènes hépatiques dans les cellules infectées par les différents virus intergénotypiques]{\textbf{Méthode expérimentale pour étudier la régulation des gènes hépatiques dans les cellules infectées par les différents virus intergénotypiques.} (A) Des cellules Huh-7.5 ont été infectées par les virus recombinants à une MOI de 3 TCID50/cellule. Les extraits d’ARN ont été collectés à 120h p.i., purifiés et traités à la DNAse afin d’éliminer toute trace d’ADN génomique et analysés par RNA-Seq. L’analyse a été conduite à partir de 4 réplicats biologiques. (B) Les lectures (\textit{reads}) ont été alignées sur le transcriptome humain. Le nombre et le pourcentage des transcrits > 100 lectures (ou \textit{reads}) modulés par l’infection avec une \textit{p-value} ajustée < 0.05 pour au moins un des virus intergénotypiques sont indiqués.}
				\label{fig:fig74}
	\end{figureth}
	\FloatBarrier
	
Dans l’objectif de voir s’il existe un impact différentiel ou une spécificité dans la dérégulation de ces gènes ou des voies de signalisation en fonction de l’origine génotypique de Core, les données relatives à chaque virus recombinant ont été groupées par génotype de Core (\autoref{fig:fig75}\textcolor{blue}{.A}). La souche parentale Jad a été exclue de cette analyse pour ne se concentrer que sur des virus recombinants produits de façon similaire. Le nombre de gènes modulés par l’infection avec au moins un virus d’un groupe donné est relativement proche entre les différents génotypes, avec au minimum 5173 gènes pour le génotype 2 et au maximum 5996 gènes pour le génotype 3, sans appliquer de filtre sur le ratio de dérégulation par rapport aux cellules non infectées (\autoref{fig:fig75}\textcolor{blue}{.B}). Le génotype 3 présente un nombre légèrement supérieur de gènes significativement dérégulés, probablement en raison du plus grand nombre de virus associés à ce groupe. Le nombre de gènes hautement dérégulé par l’infection (log2FC > 1 ou < -1) est également voisin entre les différents groupes, avec 496 gènes pour les virus de génotype 1, 596 pour le génotype 2 (représenté par l’unique virus Jad/C2a-J6), 643 pour les virus de génotype 3, et 547 pour les virus de génotype 4. Il est intéressant de remarquer un fort contraste entre les proportions de gènes qui sont régulés positivement ou négativement selon le degré de modulation. En effet, parmi les gènes simplement dérégulés (graphique de gauche), cette proportion est plutôt équivalente, tandis que les gènes fortement affectés (graphique de droite) sont majoritairement régulés positivement. Par ailleurs, nous pouvons voir un fort impact différentiel de l’origine génotypique de Core sur les dérégulations transcriptionnelles, avec seulement ~56\% d'ARNm (3989 gènes) communément modulés par au moins un virus de chaque groupe codant Core de génotype 1, 2, 3 et 4 (\autoref{fig:fig75}\textcolor{blue}{.C, graphique de gauche}). De plus, un nombre significatif de gènes dans cette banque de données est spécifiquement dérégulé par au moins un virus d’un génotype, sans l’être par aucun virus des autres génotypes : virus codant pour Core de génotype 1 (305 gènes = ~4.2\%), Core de génotype 2 (200 gènes = ~2.8\%), Core de génotype 3 (505 gènes = ~7\%) et Core de génotype 4 (257 gènes = ~3.5\%). \\
\indent L’analyse par GSEA de ces différents ARNm spécifiques de génotype met en évidence que les virus recombinants codant Core de génotype 3 modulent spécifiquement certains facteurs d’hôtes intervenant dans le métabolisme des acides gras neutres et des phospholipides. À titre d’exemple, l’expression des gènes APOB, LPIN2, LPCAT3 et PCYT1A impliqués dans ces voies est spécifiquement inhibée par certains virus recombinants codant Core de génotype 3. Il est intéressant de pointer l’existence d’ARNm hautement dérégulés par l’un des quatre génotypes de Core en particulier (log2FC > 1 ou < -1) : 19 gènes dérégulés fortement par au moins un des virus codant pour Core de génotype 1, 32 pour Core de génotype 2, 66 pour Core de génotype 3, et 41 pour Core de génotype 4 (\autoref{fig:fig75}\textcolor{blue}{.C, graphique de droite}). Ces facteurs hautement modulés par l’infection, pourraient constituer un niveau d’importance supérieur dans l’identification des dérégulations spécifiques au génotype de Core et il sera nécessaire de caractériser extensivement leurs implications dans les différentes voies biologiques.

		\begin{figureth}
	\centering
			\includegraphics[width=0.80\linewidth]{Figure_75.png}
		\caption[Analyse de la régulation du transcriptome hépatique dans les cellules infectées par les virus recombinants selon l’origine génotypique de Core]{\textbf{Analyse de la régulation du transcriptome hépatique dans les cellules infectées par les virus recombinants selon l’origine génotypique de Core.} \textit{La légende de la figure est décrite sur la page suivante \footnotemark[6].}}
				\label{fig:fig75}
	\end{figureth}
	\FloatBarrier

\footnotetext[6]{(A) Tableau des groupes par génotype. (B) Analyse quantitative du nombre de gènes significativement modulés (\textit{p-value} ajustée < 0.05, graphique de gauche) ou fortement modulés (\textit{p-value} ajustée < 0.05, log2FC > 1 ou -1, graphique de droite) par les différents groupes de virus recombinants codant Core de génotype 1, 2, 3 ou 4. Les gènes régulés positivement ou négativement sont respectivement représentés par des barres solides ou des barres hachurées. (C) Diagrammes de Venn montrant les gènes spécifiquement ou communément modulés (\textit{p-value ajustée} < 0.05, diagramme de gauche) ou fortement modulés (\textit{p-value} ajustée < 0.05, log2FC > 1 ou -1, diagramme de droite) par au moins un virus des différents groupes codant Core de génotype 1, 2, 3 ou 4.}

Dans un deuxième temps, je me suis concentrée sur les ARNm codant des facteurs d’hôte pertinents pour le développement de la stéatose hépatique, à partir de la base de données STRING relative aux pathologies humaines. Cette base de données regroupe les réseaux d’interactions protéine-protéine connus ou prédits dans plus de 14.000 organismes vivants, obtenus à partir de données expérimentales provenant d’études génomiques et protéomiques à grande échelle ou de méthodes de prédiction fonctionnelle \citep{RN978}. À partir de cette ressource, j’ai extrait la liste de 100 facteurs d’hôte qui présentent le plus haut score de confiance dans le développement de la stéatose hépatique. Ce score est un indicateur de la fiabilité avec laquelle une protéine  appartient à un réseau donné, qui dépend de la masse de données expérimentales disponibles. Il est important de noter que parmi ces 100 facteurs, seuls 52 ARNm correspondants excèdent le critère imposé (> 100 reads) lors de l’analyse pour se restreindre au système cellulaire Huh-7.5. Parmi eux, on retrouve des facteurs impliqués dans le métabolisme lipidique (DGAT1, DGAT2, PLIN2, PNPLA2, PNPLA3), dans l’adipogénèse (ADIPOR1, ADIPOR2), dans la voie de signalisation de l’insuline (FOXO1, IRS1), dans l’homéostasie du cholestérol (LDLR, SCD, HMGCR) et dans le péronisme (ACOX1, PPARA, PPARG, PPARGC1A, ALB) (\autoref{fig:fig76}). Ces voies cellulaires apparaissent pertinentes avec le développement de cette pathologie, qui est fortement liée à un dérèglement du métabolisme énergétique. Parmi les 48 facteurs non considérés, on retrouve des senseurs de l’immunité innée (TLR2, TLR4, TLR9) ou des cytokines pro- et anti-inflammatoires (IL10, IL6, TNF), ce qui est cohérent avec le fait que les cellules Huh-7.5 ne sont pas représentatives d’un système d’immunité innée intègre. Néanmoins, ce crible met en évidence que 83\% des 52 facteurs pro-stéatogènes retenus par notre modèle sont dérégulés par l’infection, indiquant que le VHC impacte sévèrement les facteurs de risque associés à la stéatose hépatique. La majorité de ces ARNm dérégulés sont communs à tous les virus indépendamment du génotype de Core, à l’exception de \textit{PPARG} qui est spécifique au génotype 1, \textit{APOB} qui est spécifique au génotype 3 et \textit{SIRT1} qui est spécifique au génotype 4. \\ \\
\indent
Cette étude préliminaire montre pour la première fois une modulation différentielle de facteurs pro-stéatogéniques clés en lien avec l’origine génotypique de Core. Toutefois, ces données ne sont pas suffisantes pour établir un lien entre le génotype 3 et la stéatose hépatique. En effet, en groupant les données des virus recombinants par génotype, l’analyse est restreinte à une dimension qualitative. Il sera primordial de creuser le degré de modulation de ces ARNm entre les différents virus intergénotypiques. De plus, il serait intéressant d’étudier en détail la liste des ARNm spécifiquement dérégulés par les quatre génotypes de Core illustrés dans la \Autoref{fig:fig75}\textcolor{blue}{.C}, sans se limiter aux facteurs de la stéatose. Enfin, une analyse qualitative similaire devra être effectuée en groupant les virus recombinants selon le tableau clinique associé aux souches dont les séquences Core ont été extraites et non par génotype, ce qui permettra d’étudier les potentielles dérégulations différentielles en fonction du degré d’avancement de la stéatose et de confirmer le rôle de la protéine Core dans cette pathologie.

		\begin{figureth}
	\centering
			\includegraphics[width=\linewidth]{Figure_76.png}
		\caption[Cartographie du réseau de facteurs d’hôte impliqués dans le développement de la stéatose hépatique]{\textbf{Cartographie du réseau de facteurs d’hôte impliqués dans le développement de la stéatose hépatique.} Les protéines illustrées par des cercles représentent des facteurs pro-stéatogéniques connus ou prédits, obtenus à partir de la base de données STRING et restreints aux cellules Huh-7.5. Les interactions protéiques connues entre ces facteurs sont représentées par des lignes noires et la taille du cercle est proportionnelle au score de confiance du facteur. Les facteurs d’hôte participant à la régulation du métabolisme lipidique, de l’adipogénèse, de l’homéostasie du cholestérol et du péroxisome sont mis en évidence par des cadres en pointillés noirs. Les dérégulations au niveau transcriptionnel de ces facteurs par les différents groupes de virus recombinants codant Core de génotype 1, 2, 3 ou 4 sont respectivement illustrées en violet, orange, vert ou bleu et sont mises en évidence par un cercle noir plus épais. Les facteurs sur fond blanc ne sont pas significativement dérégulés au niveau transcriptionnel.}
				\label{fig:fig76}
	\end{figureth}
	\FloatBarrier

	%% Copyright (C) 2017-2021 Emeline Simon
%%
%% The current owner of this work is Emeline Simon
%% <contact at emeline.simon@gmail.com>.
%%
%% This is 03_discussion.tex the third chapter for my PhD Thesis.
%%
%%%%%%%%%%%%%%%%%%%%%%%%%%%%%%%%%%%%%%%%%%

\chapter{Discussion et perspectives}
	\minitoc
	\newpage

%%%%%%%%%%%%%%%%%%%%%%%%%%%%%%%%%%%%%%%%%%%%%%%%%%%%%%%%%%%%%%%%%%%%%%%%%%%%%%%%%%%%%%%%%%%%%%%%%%%%%%%%%%%%%%%%%%%%%%%%%%%%%%%%%%%%%%%%%%%%%%%%%%%%%%%%%%%%%%%%%%%%%%%%

\section{Discussion}

Globalement, nos recherches approfondies combinant analyse quantitative par imagerie confocale et utilisation de variants de Core obtenus par mutagènese dirigée démontrent le rôle instrumental jusque là présumé de la protéine Core dans la clusterisation, la redistribution et l’élargissement des GL induits lors de l'infection par le VHC. En outre, il a été démontré que les fonctions de la protéine Core ne se limitent pas à l’assemblage de la capside, mais que celle-ci a également un rôle central dans la pathogénèse. Le développement de modèles pionniers dans le domaine d’étude du VHC, \textit{i.e} les virus intergénotypiques codant des protéines Core hétérologues, ont permis d’explorer le lien direct potentiel entre l’origine génotypique de Core et la stéatose hépatique par imagerie confocale et par crible transcriptomique à haut débit.

\subsection{Analyse combinée par microscopie confocale quantitative et mutagénèse dirigée pour étudier le rôle instrumental de la protéine Core dans la dérégulation des GL au cours de l’infection par le VHC}
\label{section:discussion1}

Les premières observations microscopiques de la localisation de la protéine Core du VHC ont été réalisées avec différentes lignées cellulaires (CHO, HepG2, BHK-21, Huh-7, U-2 OS) dans lesquelles les protéines Core de génotype 1a (souche \textit{Glasgow}) ou 1b (souche japonaise) étaient exprimées de façon stable, transitoire ou inductible \citep{RN485,RN1045,RN1055}. Dès lors, ces études mettent en évidence pour la première fois une interaction singulière de la protéine Core du VHC avec les GL, unique à l’époque chez les \textit{Flaviviridae} et les virus humains, jusqu’à la description d’une interaction de la protéine de capside du DENV avec les GL \citep{RN1037}. Cette localisation, ne serait-ce que transitoire, a été confirmée plus tard avec des systèmes infectieux pour la souche clinique JFH-1 de génotype 2a se répliquant spontanément en culture cellulaire [\citep{RN487}, la souche hybride Jc1 exprimant Core de la souche J6 de génotype 2a \citep{RN495} et plus récemment pour des virus hyper-adaptés de génotypes 1a, 2a, 2b, 2c et 3a \citep{RN1060}, suggérant un mécanisme universel à tous les génotypes. Le recrutement de la protéine Core à la surface des GL est une étape indispensable à la morphogenèse du VHC \citep{RN487,RN488,RN489}. \\ \\
\indent
Au démarrage de cette étude, la protéine Core a déjà été incriminée dans plusieurs processus de dérégulations des GL à des degrés variables selon les modèles. Toutefois, les différents travaux menant à ces découvertes n’apportaient ni de dimension quantitative précise, ni de mécanisme viral concret pouvant être à l’origine de ces dérégulations. Grâce aux récentes avancées dans les approches bio-informatiques applicables en imagerie, et en développant un algorithme novateur sur Python, nous sommes parvenus à quantifier pour la première fois l'évolution au cours de l'infection des évènements d'association de Core avec les GL et de leurs conséquences sur la dynamique des GL. Ainsi, nous avons mis en évidence que l’infection par la souche Jad augmente de 20\% la proportion de GL formant des clusters et double quasiment le nombre de contacts entre elles, au temps le plus tardif étudié. En parallèle, la taille moyenne des GL double, voire triple sans changement de leur contenu global, tandis que leur nombre est proportionnellement réduit. Ces effets apparaissent progressivement au cours de l’infection et sont particulièrement marqués à partir de 3 et 4j p.i. et sont cohérents avec l’étude de Boulant et al., montrant que la souche JFH-1 induit un regroupement, mais pas d’accumulation visible des GL au bout de 3j p.i. \citep{RN930}. Dans les modèles d’expression ectopique de Core de génotype 1a ou 1b, la redistribution et la clusterisation des GL s’effectue en moins de 16h et s’accompagne d’une augmentation du contenu en GL \citep{RN1040,RN947}. Notre étude de corrélation entre les paramètres relatifs à Core et aux GL souligne une variation proportionnelle entre le taux de clusterisation des GL et la quantité de Core recouvrant leur surface, ce qui explique probablement les divergences relevées dans l’ampleur du regroupement et de la redistribution des GL entre les systèmes de forte expression ectopique de Core et d’infection. Par ailleurs, contrairement à ce qui a été noté dans ces systèmes d’expression transitoire \citep{RN1041,RN947}, nous ne détectons pas la protéine Core dans le noyau. Nous observons parfois de rares GL qui transitent au sein du noyau dans la lignée Huh-7.5, mais elles apparaissent totalement dénuées de Core, soulignant qu’une surexpression isolée de la protéine peut mener à des effets biologiques différents de l’infection. \\ \\
\indent
Nos analyses fines par imagerie ont mis en évidence que la protéine Core est localisée préférentiellement dans les zones de contact entre les GL appariées, et que seule la population de GL mobilisée par Core est soumise à cette clusterisation et à cet élargissement progressif, pointant une action potentiellement directe de la protéine Core dans ces dérégulations cellulaires. La réduction ou l’absence respective de ces effets avec des mutants viraux déficients pour l’assemblage, pour lesquels Core est partiellement ou totalement délocalisée de la surface des GL, confirme le rôle instrumental de cette protéine virale dans la dérégulation des GL au cours de l’infection par le VHC. À l’heure actuelle, le mécanisme et le rôle biologique du recrutement de la protéine Core à la surface des GL, ainsi que l’implication de ces organites dans l’assemblage des particules virales ne sont pas encore précisément établis. Notre étude a permis d’établir certaines hypothèses en rapport avec ces différents points, qui seront discutées lors des paragraphes ci-dessous.
\clearpage

	\subsubsection{Recrutement de la protéine Core à la surface des gouttelettes lipidiques cytosoliques : un mécanisme identifiable à celui des protéines de classe I ?}

La protéine Core contient des éléments fréquemment retrouvés dans les protéines de classe I associées aux GL comme les oléosines végétales, tels que des hélices amphipatiques, un noeud de résidus proline (formé par les résidus P138 et P143) et un motif Y/FATG \citep{RN304,RN957}. D’après nos données de mutagénèse dirigée, empêcher le recrutement de la protéine Core mature à la surface des GL par l’insertion de mutations dans le noeud de résidus proline et dans le motif YATG du domaine D2 (Core DP_SATG) déstabilise fortement la protéine. Ce résultat est en accord avec des stratégies visant à empêcher le recrutement de la protéine Core mature aux GL en mutant les hélices amphipatiques, qui aboutissaient à une dégradation protéolytique quasi-complète de la protéine \citep{RN304}. Une première hypothèse est que la protéine Core mature serait libérée transitoirement dans le cytoplasme, car les hélices amphipatiques présentes dans le D2 de Core (de l’ordre de 20aa) ne seraient pas suffisamment longues pour maintenir la protéine mature dans la bicouche du RE, contrairement aux hélices amphipatiques des oléosines végétales (de l’ordre de 72aa) \citep{RN609}, qui restent stablement ancrées dans le RE après la substitution des résidus proline \citep{RN1069}. Une autre hypothèse est que les mutations combinées dans le noeud de résidus proline et du motif Y/FATG perturberaient la topologie correcte de la protéine dans la bicouche du RE, ce qui induirait un stress membranaire et le déclenchement de l’UPR. Quelle qu'en soit la raison, la protéine C_DP_SATG est ciblée et dégradée par la machinerie protéasomale. Seules les mutations du noeud de proline dans la protéine Core (C_DP) ont permis de produire une protéine mutante relativement stable dans le contexte de la souche Jad, pour laquelle nous détectons une co-localisation modeste de l’ordre de 20\% de C_DP avec les GL. Il est difficile de conclure s’il s’agit d’un ancrage fonctionnel à la surface des GL pour cette faible proportion de protéine C_DP, ou d’une rétention à des sites de transfert entre RE et GL, comme évoqué par \citet{RN930}. Cependant, nous avons montré une prédisposition de C_DP à induire une clusterisation et une redistribution partielle des GL à proximité des usines de réplication virale, contrairement à la protéine Core_DP_SATG, ce qui favorise la première hypothèse. La localisation partielle de la protéine C_DP à la surface des GL ou sa rétention dans les sites de transfert du RE protègerait la protéine de la protéolyse. À l’inverse, la protéine C_DP_SATG, dépourvue de distribution ponctuée, serait incapable d’atteindre les sites de transfert entre le RE et les GL et serait rapidement engagée dans la dégradation protéasomale. Les quelques agrégats visibles de C_DP_SATG dispersés dans le cytoplasme se comportent peut-être comme des corps d'inclusion à court terme. L’ensemble de ces résultats et des données disponibles dans la littérature soutiennent que la protéine Core se comporte comme une protéine de classe I, diffusant à travers la bicouche du RE vers les GL par un recrutement actif, après le clivage protéolytique du peptide signal par la SPP. Le recrutement actif de Core vers les GL en cours de biosynthèse serait médié par la protéine cellulaire DGAT1, une enzyme constitutive du RE impliquée dans la biosynthèse \textit{de novo} des lipides neutres \citep{RN944,RN945}. \\ \\
\indent
À ce jour, la question du « timing » du recrutement de Core dans le cycle de vie des GL reste en suspens : s’effectuerait-il uniquement au moment de la biosynthèse \textit{de novo} des GL ou les GL pré-existantes pourraient-elles acquérir Core lors d’échanges ultérieurs avec des zones enrichies en protéine Core sur la membrane externe du RE ? Premièrement, notre étude du profil transcriptomique montre une réduction globale de l’expression des enzymes impliquées dans la synthèse \textit{de novo} de lipides neutres dans les hépatocytes infectées, la première étape qui déclenche la formation de nouvelles GL, ce qui semblerait contre-productif pour distribuer les protéines Core synthétisées en continu à leur organite cible. Deuxièmement, une étude pionnière dont l’objectif était de quantifier l’interactome entre les organites intracellulaires par imagerie spectrale, a révélé que 85\% des GL sont en contact avec le RE à un temps donné, en raison de l’importance des échanges entre ces deux organites pour réguler l’homéostasie lipidique et la lipotoxicité cellulaire \citep{RN580}. Cette caractéristique augmenterait les probabilités de transfert de Core des sites de traduction vers les GL. Dans notre étude, nous avons évalué que 75\% de la population de GL est mobilisée par Core dès 24h p.i. Les GL alternent des cycles de formation et de dégradation plus ou moins dynamiques selon l’état métabolique des cellules et le type cellulaire, rendant le calcul précis de la demi-vie moyenne d’une GL difficile. Toutefois, étant donné que leur fonction principale est le stockage des lipides à long terme, il paraît peu probable que 75\% des GL se soient complètement renouvelées en 24h p.i. Au vu de ces résultats et des informations complémentaires de la littérature, nous pouvons émettre l’hypothèse que la distribution de Core à la surface des GL peut survenir à la fois lors de la formation d’une GL naissante à partir des membranes du RE enrichies en Core et lors des contacts entre les GL pré-existantes et les membranes du RE.

	\subsubsection{Rétention de la protéine Core à la surface des gouttelettes lipidiques cytosoliques : un mécanisme contre-productif pour l’assemblage des particules virales ?}

Bien que l’association de la protéine Core aux GL cytosoliques soit une étape essentielle pour la morphogénèse virale, le passage de Core à la surface des GL peut être transitoire, comme c’est le cas pour la souche hybride Jc1, dont la protéine Core (J6) est retrouvée majoritairement sous forme de puncta intenses localisés à proximité de la membrane du RE \citep{RN489}. Dans le cas de la souche JFH-1 sauvage, Core est majoritairement localisée à la surface des GL, induisant un parfait enveloppement de ces organites \citep{RN486}. En raison du titre infectieux 50 à 100 fois supérieur de la souche Jc1, il a été évoqué que la protéine Core de Jc1 serait plus rapidement mobilisée vers les sites putatifs d’assemblage, contrairement à celle de JFH-1 et que, par conséquent, l’association durable de Core aux GL serait contre-productive pour la morphogénèse virale \citep{RN495}. 

Dans notre étude avec la souche Jad adaptée, ainsi que celle de Kaul et al. qui l’a décrite pour la première fois \citep{RN342}, les titres infectieux sont de l’ordre de 1 à 5x10\up{5} similaires à ceux du virus Jc1, car les mutations d’adaptation situées dans les séquences des protéines NS5A et NS5B améliorent la transition entre la réplication génomique et l’assemblage \citep{RN1051}. Nos résultats montrent que la protéine Core de Jad se comporte comme celle de JFH-1, en s’accumulant progressivement à la surface des GL au cours l’infection, jusqu’à coloniser 85\% de la population de ces organites et envelopper en moyenne 70\% de leur surface au temps le plus tardif de notre étude. Ainsi, nos données indiquent que l’association stable de Core aux GL ne constitue pas une dimension contre-productive à l’assemblage et à la synthèse de particules infectieuses, contrairement au dogme établi. Le virus JFH-1, étant la seule souche clinique aboutissant d’emblée à un cycle viral complet dans la lignée Huh-7.5, pourrait simplement être sous-optimal pour la production de particules virales. De plus, dans notre étude, le virus chimérique Jad/C2a-J6, exprimant la protéine Core de J6 dans le contexte du Jad, montre un profil d’association aux GL similaire à celui du virus Jad, confirmant que les protéines p7 et NS2 de la souche Jc1 seraient nécessaires à la re-mobilisation précoce de Core aux sites putatifs d’assemblage dans les membranes du RE adjacentes observée précédemment \citep{RN495}. Par ailleurs, les protéines Core produites par l’ensemble des virus intergénotypiques sont moins stables que la protéine Core native de la souche Jad et couvrent parfois moins extensivement la surface des GL, sans que cela affecte la capacité de production de particules infectieuses. Ces résultats suggèrent que Jad conduit à une accumulation en « excès » de Core, qui n’est pas entièrement consommée pour produire des particules virales, et que les titres infectieux seraient plafonnés dans les cellules Huh7.5 en raison d'un autre facteur viral ou d’hôte limitant pour l’assemblage de toutes les souches de VHC.

	\subsubsection{Redistribution et clusterisation des gouttelettes lipidiques à proximité des usines de réplication virale : un mécanisme contribuant à l’assemblage des particules virales ?}

Nos observations par microscopie électronique mettent en évidence une redistribution des GL à proximité des remodelages membranaires induits par l’infection, et notamment des DMV, structures qui ont été décrites comme le siège d'une réplication génomique active \citep{RN452}. Ces remodelages du RE et re-distribution des GL prennent place de façon localisée dans le cytoplasme, mais ne sont pas limités à la zone péri-nucléaire, contrairement à ce qui a été précédemment décrit dans des cellules hébergeant des réplicons sous-génomiques sans assemblage de virions \citep{RN1044,RN329}, ou dans les systèmes d'expression transitoire de Core sans MV \citep{RN1040,RN947}. En revanche, nos résultats sont en accord avec ceux visualisant récemment une co-localisation des protéines non structurales (notamment NS5A) responsables du recrutement des génomes viraux néosynthétisés et de la glycoprotéine d’enveloppe E2 au niveau du MV entourant les GL associées à Core, par une approche de microscopie corrélative (CLEM) combinant l’observation de la localisation des protéines virales par fluorescence et des structures intracellulaires par microscopie électronique \citep{RN490}. Selon ces auteurs, les sites putatifs d’assemblage du VHC constitueraient alors les sites de transfert formés par les contacts membranaires entre le RE et les GL, et les GL serviraient de plateforme de stockage essentielles pour délivrer Core à proximité des autres composants de la particule virale. Ce rapprochement permettrait \textit{in fine} d’augmenter la probabilité d’interaction entre la protéine Core du VHC et l’ARN nouvellement synthétisé, favorisant ainsi l’encapsidation du génome qui constitue la première étape de l’assemblage du virus. Nos tentatives d’identifier la sous-population de GL concernée par cette redistribution, \textit{i.e.} déterminer si elle est majoritairement associée à Core, par triple co-marquage des GL, de Core et des ARN viraux par hybridation \textit{in situ}, ce qui n'a jusqu’à présent jamais été exploré dans la littérature, n'ont malheureusement pas abouti à ce jour, en raison de problèmes techniques (sensibilité de la membrane phospholipidique aux détergents inclus dans les réactifs d’hybridation, résultant en une désinsertion des protéines de surface des GL.

	\subsubsection{Élargissement des gouttelettes lipidiques par un mécanisme présumé de fusion homotypique : un effet secondaire de l’assemblage des particules virales ?}

En l’absence de biosynthèse \textit{de novo} accrue de GL et compte tenu de la réduction de l’expression d’enzymes impliquées dans la synthèse locale de lipides neutres dans les cellules infectées, nous postulons que l’élargissement des GL est médié par un mécanisme de fusion homotypique, et serait une conséquence directe de la redistribution et de la clusterisation des GL. En effet, Core pourrait établir un pont physique reliant les GL limitrophes, ce qui pourrait déséquilibrer leur tension de surface et favoriser \textit{a posteriori} leur fusion. Core présente une forte plasticité conformationnelle permettant de s’oligomériser, une étape essentielle pour la formation de la nucléocapside \citep{RN1043,RN1056}. Par conséquent, un mécanisme possible qui sous-tend la liaison des GL adjacentes pourrait être attribué à la capacité de Core à former des homodimères \textit{en trans}, en pontant les protéines Core d'une gouttelette à une autre, selon un mécanisme similaire à celui impliquant les complexes protéiques cellulaires CIDEB et AUP1 \citep{RN596,RN558}. Cette hypothèse assez ambitieuse nécessiterait que le domaine D1 responsable de la dimérisation \citep{RN1062} soit accessible à ce stade de la conformation de Core. Des informations sur la structure tridimensionnelle de Core aideraient à légitimer cette hypothèse, mais seule une structure très partielle de la région du peptide signal résolue à ce jour par RMN est disponible \citep{RN1085}. Un autre mécanisme indirect concevable serait que Core mobilise les complexes cellulaires tels que AUP1 ou CIDEB. Alternativement, la fusion des GL appariées pourrait être indirectement facilitée par la dé-mobilisation de Core vers les sites putatifs d’assemblage situés sur le RE. Celle-ci aurait pour conséquence d'augmenter la tension superficielle des GL (puisque les protéines associées à la surface des GL contribuent à préserver « l’émulsion »), et de déclencher la fusion entre les GL appariées \citep{RN608}. Cette hypothèse serait cohérente avec le fait que la protéine Core_DP, déficiente pour l’assemblage, induit une clusterisation et une redistribution partielle sans élargissement des GL, en raison d’une topologie altérée qui ne permet pas de mobiliser cette protéine mutée pour former la nucléocapside. \\ \\
\indent
Il est intéressant de noter que nos résultats préliminaires avec le virus Jc1 ne documentent pas d’élargissement des GL au cours de l’infection (non montrés), ce qui est cohérent avec des données récentes de la littérature \citep{RN1057}. En ligne avec la discussion ci-dessus, nous pouvons émettre l'hypothèse que le passage de Core de la souche J6 à la surface des GL dans le contexte du virus Jc1 n’est pas suffisamment long pour mettre en place les mécanismes de dérégulations marqués, tels que quantifiés aux temps tardifs de l’infection par la souche Jad. Dans ce contexte, la re-mobilisation rapide de la protéine Core mature par les protéines p7 et NS2 de la souche J6 au niveau des sites putatifs d’assemblage du RE représenteraient un mécanisme alternatif pour délivrer Core auprès des autres éléments viraux pour former les particules virales. Pour confirmer cette hypothèse, il conviendrait de vérifier si Jc1 induit une clusterisation et une redistribution des GL à proximité des usines de réplication virale. \\ \\
\indent
Globalement, l’ensemble de nos résultats a permis de compléter ceux de la littérature et d'apporter des éléments nouveaux avec une dimension quantitative, qui supportent le modèle illustré dans la \Autoref{fig:fig77} : (1) la protéine Core du VHC a une topologie et un mécanisme de recrutement similaire aux protéines de classe I : elle diffuse préférentiellement à la surface des GL selon un recrutement actif médié par la protéine DGAT1 juste après le clivage du peptide signal par la SPP. La monocouche lipidique des GL accommoderait plus stablement les courtes hélices amphipatiques de Core que la bicouche du RE. Les GL peuvent acquérir les protéines Core à leur surface au cours de leur synthèse ou lorsqu’elles établissent des contacts ultérieurs avec des zones riches en protéines Core dans la membrane externe du RE ; (2) la protéine Core induit une redistribution des GL à proximité des DMV qui hébergent les usines de réplication génomique virale, concentrant localement la protéine de capside au plus proche des génomes viraux néosynthétisés pour optimiser la formation des nucléocapsides ; (3) Cette redistribution des GL induit secondairement des évènements d’appariement, soit par un mécanisme directement médié par la capacité de la protéine Core à former des oligomères \textit{en trans}, soit par le recrutement d’un autre facteur d’hôte capable de dimérisation, comme CIDEB ou AUP1 ; (4) Cette clusterisation « forcée », en parallèle d’une re-mobilisation des protéines Core recouvrant la surface des GL vers les sites d’assemblage pour former les nucléocapsides peut déstabiliser la tension superficielle des GL et déclencher leur fusion homotypique.

		\begin{figureth}
	\centering
			\includegraphics[width=\linewidth]{Figure_77.png}
		\caption[Modèle de la dérégulation et de l’implication des gouttelettes lipidiques dans l’étape de morphogénèse du VHC privilégié par notre étude]{\textbf{Modèle de la dérégulation et de l’implication des gouttelettes lipidiques dans l’étape de morphogénèse du VHC privilégié par notre étude.}}
				\label{fig:fig77}
	\end{figureth}
	\FloatBarrier

\subsection{Production de virus chimériques exprimant des protéines Core hétérologues comme nouveaux outils pour évaluer le lien direct potentiel entre l’origine génotypique de Core et la stéatose hépatique}

La recherche visant à décrire et à comprendre les mécanismes moléculaires et physiopathologiques liés à l'infection par le VHC a été freinée par le manque de système de culture approprié et de modèle animal permissif à l’infection. En particulier, les mécanismes liés aux perturbations induites spécifiquement par l’infection par le VHC de génotype 3 demeurent mal compris. En effet, les méta-données cliniques et épidémiologiques obtenues à partir de cohortes de patients révèlent une prévalence de 80\% des cas de stéatose sévère dans les infections chroniques par les souches de génotype 3 \citep{RN359,RN360}, qui a été interprétée comme le résultat d’un effet « stéatogène » direct des virus de ce génotype \citep{RN1225}. À la suite de ces enquêtes, plusieurs évidences expérimentales reposant principalement sur des systèmes d’expression transitoire de protéines virales suggéraient que les désordres du métabolisme lipidique seraient induits de façon génotype-spécifique \citep{RN1206,RN1209,RN1207}. En ce sens, un rôle central et génotype-spécifique dans les dysfonctionnements cellulaires associés au virus, incluant la lipogénèse, la stéatose et la ß-oxydation a été attribué à la protéine Core \citep[pour revue,][]{RN1123}. À l’inverse, d’autres études basées sur des analyses histologiques de biopsies de foie de patients divergent largement à ce sujet, et remettent en cause l’association préférentielle du génotype 3 du VHC avec la stéatose hépatique \citep[pour revue,][]{RN1201}. Ces témoignages contradictoires renforçaient la nécessité d’utiliser des systèmes d'infection pertinents pour de telles études de corrélats physio-pathologiques. \\ \\
\indent
La souche JFH-1 du VHC de génotype 2a est le seul isolat naturel qui récapitule spontanément l'ensemble du cycle viral menant à la production de particules du VHC en lignée cellulaire d'hépatome humain Huh-7.5 \citep{RN124}. Plusieurs équipes sont parvenues à adapter avec succès des souches de divers génotypes du VHC en lignées hépatocytaires : les souches H77-S, TN et HCV1 de génotype 1a \citep{RN353,RN1193,RN1202}, les souches J6 et T9 de génotype 2a \citep{RN354}, les souches DH8 et J6 de génotype 2b \citep{RN355}, la souche clinique S83 de génotype 2c \citep{RN1186}, et les souches cliniques DBN3a, S52, S310 de génotype 3a \citep{RN357,RN356,RN1060}. Néanmoins, non seulement ces souches hyper-adaptées ne reflètent pas la séquence native des souches cliniques puisqu’elles contiennent de multiples mutations d’adaptation, mais leur multiplication reste très variable, ce qui rend leur utilisation difficile. Dans ce contexte, pour aborder la question du lien entre le génotype 3 du VHC et la stéatose hépatique, notre étude s'est initialement concentrée sur la production de virus chimériques codant une série de protéines Core hétérologues.

	\subsubsection{L’interaction entre les protéines Core hétérologues de génotype 3 et les autres éléments viraux est permissive pour l'assemblage des particules virales}

La série de virus chimériques intergénotypiques que nous avons produite avec succès dans le contexte de la souche Jad \citep{RN342,RN811} exprime en lieu et place de la protéine Core native, la protéine équivalente issue de six souches cliniques de sous-type 3a, ainsi que d’une souche clinique de sous-type 1a, qui ont l’avantage d’être associées à un tableau clinique connu, établi à partir des biopsies hépatiques prélevées chez les patients d’origine. Ces patients ont développé différents degrés de stéatose macro- et micro-vésiculaire, avec une accumulation variable de vésicules graisseuses dans les hépatocytes. \\ \\
\indent
À ce jour, aucun VHC recombinant naturel dont le génome résulterait de la substitution exacte de la séquence codante d’une seule protéine virale n’a été détecté sur le terrain. Les « points chauds » des recombinaisons intergénotypiques naturelles identifiées se situent au sein de la séquence de la protéine NS2 ou dans la région N-terminale de NS3, générant ainsi des protéines hybrides, ou encore à la jonction NS2/NS3, séparant le supposé « module d’assemblage » de Core à NS2 du « module de réplication » de NS3 à NS5B \citep[pour revue,][]{RN94}. Cette jonction fréquente pouvait suggérer l’importance de préserver des interactions homologues entre les éléments propres à chaque module, afin de ne pas nuire aux étapes de réplication génomique et d’assemblage des particules virales. C'est pourquoi la plupart des génomes intergénotypiques fonctionnels générés par génie génétique afin d’étudier les propriétés physiopathologiques des autres génotypes viraux et de disposer de modèles pour tester les réponses humorales pan-neutralisantes, portent la séquence codante des protéines Core-E1-E2-p7-NS2 hétérologues, représentant un « module d’assemblage » minimal \citep{RN352,RN351}. Ainsi, aucune des données disponibles dans la littérature, qu’elles soient relatives aux recombinants naturels ou synthétiques, ne pouvait donc prédire si la seule substitution de la séquence complète de Core perturberait la fonctionnalité du génome intergénotypique résultant. Comme preuve de concept, une progénie virale robuste a été obtenue par notre groupe pour la première fois à partir d’un panel de génomes chimériques construits dans le contexte du Jad et codant Core provenant de souches prototypiques ou cliniques de sous-types 1a, 1b, 4a, 4f et 4p du VHC \citep[][et résultats non publiés.]{RN808}. Les génomes chimériques nouvellement produits pour cette étude, exprimant différents variants de la protéine Core de sous-type 3a se sont avérés également aboutir à une synthèse robuste de particules virales infectieuses, bien que ces séquences de Core soient toutes phylogénétiquement plus distantes de la séquence initiale de sous-type 2a que les séquences des autres génotypes testés. Seul un virus intergénotypique codant la protéine Core 376 de sous-type 3a s’est avéré sous-optimal. Il est notable que la séquence consensus de la protéine Core de cet isolat soit la plus divergente parmi les séquences 3a considérées, avec des polymorphismes uniques au niveau des résidus en positions 20, 66, 67, 110, 144, 151, 158. Une adaptation suite à cinq passages successifs des cellules Huh-7.5 transfectées par l’ARN viral correspondant transcrit \textit{in vitro} a permis d’engendrer et de stabiliser à quatres reprises une variante de ce virus aussi robuste que le virus parental. L’examen de la séquence nucléotidique majoritaire de ces variants n’a pas révélé de mutation potentiellement compensatrice au sein de la séquence codante de la polyprotéine, ni dans la partie variable de la région 3'NC ou dans les ~150 nucléotides 3'-terminaux de la région 5'NC. Il serait intéressant de déterminer les séquences des régions 5’ et 3' NC manquantes de ces variants, afin d'identifier si une mutation dans les régions non codantes optimiserait l’étape de morphogénèse virale, ce qui, le cas échéant renforcerait l'hypothèse qu'une interaction directe entre une région NC de l'ARN et la protéine Core serait critique pour l'encapsidation de l'ARN. \\ \\
\indent
Globalement, ces études conjointes de notre groupe démontrent pour la première fois l’existence d’interactions largement permissives entre les protéines Core hétérologues de sous-types 1a, 1b, 3a, 4a, 4f et 4p et les éléments natifs de sous-type 2a, qui conduisent efficacement à la morphogénèse des particules virales. L'intérêt de ce nouveau type de virus intergénotypiques réside dans le fait qu'ils expriment des protéines E1, E2, p7, NS2, NS3, NS4A, NS4B, NS5A et NS5B identiques et que seule la séquence de Core est déclinée. De plus, contrairement aux souches hyper-adaptées, ces virus intergénotypiques présentent des cinétiques de réplication et des taux d’expression protéique comparables. Ainsi, il a été possible d'inférer à la seule protéine Core hétérologue les phénotypes et les mécanismes physiopathologiques différentiels induits par l’infection, comme ceux observés vis à vis des GL et des voies métaboliques impliquées dans la stéatose hépatique décrits ci-dessous.

	\subsubsection{Absence d’effet « stéatogène » spécifique aux protéines Core hétérologues de génotype 3 ou au tableau clinique des patients à l’échelle des gouttelettes lipidiques}

Notre étude reposait sur l’hypothèse que l’observation et la comparaison du contenu et de la taille des GL dans des cellules infectées par les virus intergénotypiques pouvait être un indicateur initial de l’effet « stéatogène ». Toutes les protéines Core hétérologues de génotype 1, 2, 3 ou 4 exprimées par les virus intergénotypiques considérés sont retrouvées en majorité à la surface des GL, ce qui étend les études basées sur des systèmes d’expression transitoire des génotypes 1 et 3 de Core \citep{RN990,RN991,RN1041}. Toutefois, le degré d’association de Core à la surface des GL peut varier de 30 à 45\% selon la séquence de Core, indépendamment de son origine génotypique. Une telle variabilité a été décrite dans l’étude très récemment publiée recourant aux virus de différents génotypes hyper-adaptés à la culture cellulaire \citep{RN1060}. Ces auteurs ont montré que le degré de co-localisation directe de Core aux GL s’étend entre 40 et 60\% selon la souche virale, une fourchette relativement similaire à celle que nous avons observée avec nos modèles de virus chimériques de Core. Ces données combinées confirment que la localisation des protéines Core hétérologues ne semble pas être perturbée par les éléments viraux de la souche Jad. \\ \\
\indent
Notre étude n’a pas révélé de différence significative du volume total des GL par cellule dans les cultures infectées par les différents virus intergénotypiques, indiquant que les protéines Core de génotype 3 ne sont pas particulièrement promptes à induire une biosynthèse des GL, par rapport à celles de génotypes 1, 2 ou 4. Une augmentation modeste est toutefois obtenue pour le virus chimérique exprimant la protéine 389 de sous-type 3a. Pour confirmer qu’elle n’est pas issue d’un dépassement du seuil de significativité lié au hasard de la sélection des cellules dans ces expériences, il serait utile d’élargir l’étude à un nombre plus important de cellules. L'absence générale de régulation du contenu global de GL par cellules se démarque des résultats obtenus dans des systèmes d'expression transitoire \citep{RN990,RN991,RN1223} ou avec les souches hyper-adaptées \citep{RN356,RN1060} qui révélaient une accumulation plus importante de GL cytosoliques pour certaines protéines Core ou pour la souche S310 de sous-type 3a. Les raisons que l'on peut invoquer pour expliquer ces résultats divergents relèvent probablement des différences dans les modèles d'étude ou dans les temps post-infection retenu, les comparaisons entre souches ayant parfois été réalisées après plusieurs dizaines de passages des cellules infectées \citep{RN356}. En revanche, notre étude met en évidence un élargissement différentiel des GL dans les cellules infectées par les virus intergénotypiques. Cet élargissement, déjà noté pour la souche Jad, est plus ou moins marqué selon les séquences de Core, mais ne semble pas particulièrement plus élevé pour les séquences Core de génotype 3 contrairement aux données obtenues dans les systèmes d’expression transitoire \citep{RN1041,RN1208}. De plus, les virus intergénotypiques qui portent le résidu Phe en position 164 ou la combinaison Phe/Ile des résidus aux positions 182 et 186 du D3 dans la séquence Core hétérologue ne conduisent pas systématiquement à un élargissement des GL (Jad/C3a-390, Jad/C3a-395, Jad/C3a-401), ce qui est en désaccord avec les études par expression transitoire de \citet{RN990} et de \citet{RN991}. L’ampleur de l’élargissement semble plutôt corréler avec le taux de recrutement de Core à la surface des GL, confirmant nos conclusions précédentes sur le rôle essentiel de Core dans ce mécanisme physiopathologique (voir \autoref{section:discussion1}). \\ \\
\indent
La biosynthèse des GL pourrait être un effet relatif à la stéatose micro-vésiculaire, caractérisée par une accumulation de nombreuses petites vésicules graisseuses, tandis que l’élargissement des GL pourrait être attribuable à la stéatose macro-vésiculaire, qui se définit par la présence de peu voire d’une unique GL volumineuse dans le cytoplasme des hépatocytes. Cette hypothèse reste toutefois à moduler, étant donné l’absence d’élargissement des GL pour les virus intergénotypiques exprimant les protéines Core des souches cliniques 390 et 395, qui proviennent pourtant de souches cliniques associées à une stéatose sévère. En effet, les biopsies hépatiques prélevées à partir des patients infectés par ces deux souches contiennent respectivement 60\% d’hépatocytes avec des micro ou macro-vésicules graisseuses ou 40\% de macro-vésicules graisseuses. Une étude réalisée sur biopsies de foie de patients infectés par des virus de génotype 1 ou 3 a révélé l’absence de corrélation entre génotype infectant et proportion d’hépatocytes présentant des vésicules graisseuses \citep{RN1188}. Les auteurs soulignent toutefois l’existence d’une corrélation entre les virus de génotype 3 et la taille des vésicules. De plus, ces auteurs mettent en évidence l’existence d’une part significative d’hépatocytes sains qui contiennent de larges GL, suggérant qu’il existerait, en parallèle des effets pathogéniques direct du virus, un mécanisme indirect de l’induction de la stéatose. Le développement de la stéatose micro et macro-vésiculaire hépatique des patients infectés par les souches virales 390 et 395 pourrait être davantage attribuable à des facteurs confondants, comme le surpoids et la consommation excessive d’alcool. L’importance des facteurs d’hôte dans le développement de la stéatose hépatique sévère avait déjà été mis en lumière par les travaux de \citet{RN1228}, qui supportent l’absence d’une relation entre la stéatose hépatique et l’origine génotypique du virus. En effet, les études épidémiologiques en Europe de l’Ouest ont mis en évidence que l’excès d’alcool est fréquemment observé chez les utilisateurs de drogues injectables, et qu’il s’agit d’une population à risque pour la transmission du VHC de génotype 3 \citep{RN1226}. \\ \\
\indent
Globalement, notre étude n’a donc pas permis de confirmer le lien établi par les études cliniques et épidémiologiques entre les protéines Core de génotype 3 et les dérégulations sévères des GL qui pourraient sous-tendre l’apparition des signes histologiques de la stéatose, malgré le recours à des modèles d’infection. La force mais également la limite de notre modèle est qu’il se restreint uniquement à la nature génotypique de Core. Or, il a été montré que d’autres protéines virales, telles que NS5A, pourraient être conjointement impliquées dans le développement de la stéatose hépatique \citep{RN1210,RN1204,RN1187}. Cela impliquerait de développer des virus doublement chimériques de Core et de NS5A pour inclure les éventuels effets causés par les protéines NS5A de génotype 3. Toutefois, cette même conclusion a été atteinte dans le cadre des souches virales adaptées de différents génotypes dont le génotype 3 \citep{RN1060}, confirmant l’absence de lien entre les souches de génotype 3 et les dérégulations sévères des GL dans les systèmes d’infection hépatocytaires.

	\subsubsection{Le transcriptome hépatique est modulé différentiellement selon l’origine génotypique de Core}

Notre étude comparative des transcriptomes d’hépatomes infectés par les différents virus intergénotypiques révèle pour la première fois dans un système infection une dérégulation différentielle du profil d’expression des gènes hépatiques selon l’origine génotypique de Core. Une telle modulation différentielle a été décrite précédemment dans des systèmes d’expression de la protéine Core isolée de génotype 1b, 2a, 3a ou 4d \citep{RN1222,RN1221,RN1211} puis dans des biopsies hépatiques de patients infectés par une souche de génotype 1 ou 3 \citep{RN1219,RN1208,RN1196,RN1192}. Les conclusions de ces auteurs relatives à un effet avéré du génotype 3 sur le métabolisme lipidique ou sur les voies pro-inflammatoires divergent selon les modèles. Une précédente étude de notre groupe s’est focalisée sur le lien entre les polymorphismes naturels de Core présents dans les souches de génotype 4 émergentes en Afrique centrale et la voie de signalisation de la Wnt/ß-caténine qui contribue au développement du CHC \citep{RN808}. La conclusion principale de cette étude révèle que le virus Jad/C4fC induisait une activation significativement plus marquée de la voie Wnt/ß-caténine que la protéine Jad/C4aR, attribuable au résidu Thr en position 71 présent uniquement dans la protéine 4fC. À l’heure actuelle, les données du crible transcriptomique effectuées avec 11 virus intergénotypiques sont en cours d’analyse afin d’identifier si des polymorphismes naturels de Core spécifiques aux souches de génotype 3 sont impliqués dans la dérégulation de certaines voies biologiques représentant des facteurs de risque pour la stéatose. Ces approches constituent un travail pionnier important, qui permettra d'obtenir des informations sur les déterminants viraux impliqués dans la progression de la maladie hépatique.

\clearpage

%%%%%%%%%%%%%%%%%%%%%%%%%%%%%%%%%%%%%%%%%%%%%%%%%%%%%%%%%%%%%%%%%%%%%%%%%%%%%%%%%%%%%%%%%%%%%%%%%%%%%%%%%%%%%%%%%%%%%%%%%%%%%%%%%%%%%%%%%%%%%%%%%%%%%%%%%%%%%%%%%%%%

\section{Perspectives}

Depuis le début du 21ème siècle, le domaine de la recherche biomédicale est en pleine révolution technologique : d’une part, avec l’évolution considérable du domaine de l’imagerie depuis l’invention des premiers prototypes de microscope optique au début du 17ème siècle, qui ont révolutionné l’étude du vivant, jusqu’à la conception de la cryo-microscopie électronique qui révèle maintenant des informations structurelles détaillées presque au niveau atomique ou fournit une résolution de l’ordre de quelques nanomètres dans la cellule vivante \citep{RN1026,RN1028} ; d’autre part, avec l’essor des méthodes d’analyses moléculaires à haut débit dites « omiques » couplées au développement des méthodes d’analyses bio-informatiques qui facilitent le stockage et l’interprétation d’importants volumes de données ou « big data ». Aujourd’hui, les avancées dans le domaine de l’imagerie et dans les technologies « omiques » permettent aux biologistes de regarder dans les deux directions : à l’échelle micro- ou nanoscopique, afin d’observer toujours plus en détail les structures cellulaires et moléculaires, ou à  l’échelle de la cellule unique ou de l’organisme afin d’avoir une image globale plus large et fonctionnelle d’un système biologique.

\subsubsection{Applications de l’imagerie en cellules vivantes et de la cryo-microscopie électronique pour valider les mécanismes viraux présumés dans la clusterisation et dans la fusion des gouttelettes lipidiques induites par l’infection}

Plusieurs lacunes liées à la limitation des systèmes exploités dans cette étude bénéficieront des progrès rapides dans le domaine de la microscopie. Une première direction pour compléter cette étude serait de suivre par microscopie en cellules infectées vivantes la clusterisation et la fusion des GL, à l’aide d’une souche Jad exprimant une molécule fluorescente, telle que la GFP. La séquence codant NS5A s’avère flexible pour l’insertion d’une protéine rapportrice de grande taille, sans affecter de manière excessive la capacité de production virale \citep{RN985,RN983}. La protéine NS5A fusionnée à la GFP dans le contexte de la souche Jad pourrait servir de marqueur subsidiaire des complexes de réplication génomique, ce qui permettrait également de suivre l’évolution de la redistribution des GL à proximité de ces structures. La quantification du phénomène de fusion est difficile en raison de la rareté et de l’imprévisibilité de ce processus. Cependant, un groupe a publié une méthode de quantification de la fusion des GL par récupération de la fluorescence après photoblanchiment (FRAP) \citep{RN1023}, mesurant le taux d’échanges de lipides neutres entre des GL adjacentes, marquées métaboliquement. Il serait intéressant de voir si cette approche est applicable dans notre modèle pour confirmer que l’élargissement local des GL s’effectue par un mécanisme de fusion homotypique. La vitesse d’acquisition des images doit correspondre à la dynamique du processus étudié, par conséquent, la microscopie confocale à balayage est inappropriée pour l’étude en cellules vivantes. En effet, elle recueille le signal d’un seul point focal en balayant le faisceau laser sur l’échantillon, générant lentement une image pixel par pixel. Une alternative serait d’utiliser un microscope confocal à disque rotatif qui éclaire l'échantillon à l'aide d'un réseau d’ouvertures disposées sur un disque, créant des centaines de faisceaux focalisés simultanément pour obtenir une image de l'ensemble du champ de vision. Cependant, le marquage fluorescent des GL par le BODIPY est particulièrement sensible à l’illumination prolongée par les lasers, et la phototoxicité réduit la viabilité des échantillons. La microscopie par diffraction optique tridimensionnelle combinée avec des lampes à épifluorescence, qui permet de visualiser les structures intracellulaires et de s’abstraire de l’utilisation de lasers, serait donc un outil idéal pour cette perspective d’étude \citep{RN1027,RN1030}. Enfin, pour vérifier notre première hypothèse vis-à-vis d’un mécanisme direct de Core par dimérisation en trans dans la formation des clusters de GL, il serait intéressant d’étudier à l’échelle quasi-atomique la structure de la protéine Core associée aux GL (microscope TITAN Krios, Institut Pasteur, Paris). Ce travail permettrait d’obtenir d’une part des données sur la topologie de la protéine Core mature et d’autre part, de vérifier si Core est capable de se multimériser à la surface des GL, impliquant la possibilité que la première étape de l’assemblage des particules virales puisse s’effectuer à la surface des GL et non après la remobilisation de Core vers les sites de transfert localisés sur la membrane externe du RE.

\subsubsection{Applications des technologies « omiques » pour étudier les propriétés pathogènes spécifiques au génotype et aux polymorphismes de Core}

Une deuxième direction pour compléter cette étude serait d’analyser et de comparer le protéome et le lipidome des GL dans les cellules infectées par les différents virus recombinants intergénotypiques, afin de réaliser une étude intégrative combinant plusieurs technologies « omiques » à haut débit. Un protocole de purification des GL a été récemment développé par le groupe de E. Herker et sera mis à disposition pour notre groupe dans le cadre d’un projet collaboratif \citep{RN1024}. En parallèle, il serait intéressant d’évaluer les éventuelles différences de compositions protéique et lipidique des GL associées ou non à Core dans les cellules infectées, afin d’identifier les potentielles fonctions biologiques retirées ou attribuées à cette sous-population de GL détournée pour l’assemblage des particules virales. Enfin, pour vérifier notre deuxième hypothèse vis-à-vis d’un mécanisme indirect par le recrutement de facteurs d’hôte dans la formation des clusters de GL, il serait intéressant d’identifier dans un premier temps les partenaires cellulaires de la protéine Core par une analyse interactomique à haut débit. Ce travail impliquerait préalablement d’aboutir à la production d’une protéine Core portant une étiquette pour la purification par affinité en tandem applicable à une analyse par spectrométrie de masse, sans déstabiliser excessivement la protéine, une approche actuellement en cours de développement dans notre groupe. Dans un second temps, le « silencing » par le biais d’ARN interférent court (siARN) ou la construction de lignées Huh-7.5 KO par CRISPR/Cas9 pour les partenaires cellulaires mis en évidence avec un rôle potentiel dans le but la redistribution, la clusterisation ou la fusion des GL permettrait d’apporter une dimension fonctionnelle à notre étude.

\subsubsection{Évolution vers des systèmes plus physiologiques}

Dans le cadre de l’étude d’agents pathogènes, il est important de prendre en compte les limitations des modèles. Un grand nombre de pathogènes humains ne sont toujours pas cultivables \textit{in vitro}, que ce soit dans le règne procaryote, qui nécessitent des milieux très stricts en terme de pH ou de température, ou dans le cas des parasites ou des virus, qui dépendent obligatoirement d’une cellule hôte. Avec le développement des lignées cellulaires immortelles faciles à entretenir depuis l’avènement des cellules HeLa en 1953, il a été rendu possible de cultiver de nombreux virus humains \textit{in vitro}. Néanmoins, ces modèles cellulaires ne sont pas toujours physiologiquement proches de l’infection naturelle. Des essais préliminaires d’infection de PHH par les virus recombinants produits ont été réalisés en collaboration avec C. Gondeau (Institut de Recherche en Biothérapie, Montpellier, France) mais nous ne sommes pas parvenus à montrer une infection robuste et reproductible d’un donneur à l’autre. Nos analyses complémentaires dans ce modèle ont suggéré qu’il faudrait utiliser des virus intergénotypiques construits dans le contexte natif JFH-1, car les mutations d’adaptation de la souche Jad confèrent un phénotype d’atténuation de la réplication dans ces cultures. Un modèle de cellules Huh-7.5 partiellement différenciées et maintenues sous hypoxie a récemment été développé \citep{RN1031}, reproduisant plus physiologiquement l’environnement hépatique, les caractéristiques des hépatocytes primaires ainsi que la structure native des particules du VHC. Ce nouveau modèle cellulaire plus pertinent pourrait donc représenter un modèle de choix pour les futures directions de nos projets.
	%% Copyright (C) 2017-2021 Emeline Simon
%%
%% The current owner of this work is Emeline Simon
%% <contact at emeline.simon@gmail.com>.
%%
%% This is 04_material.tex the fourth chapter for my PhD Thesis.
%%
%%%%%%%%%%%%%%%%%%%%%%%%%%%%%%%%%%%%%%%%%%

\chapter{Matériels}
	\minitoc
	\newpage

%%%%%%%%%%%%%%%%%%%%%%%%%%%%%%%%%%%%%%%%%%%%%%%%%%%%%%%%%%%%%%%%%%%%%%%%%%%%%%%%%%%%%%%%%%%%%%%%%%%%%%%%%%%%%%%%%%%%%%%%%%%%%%%%%%%%%%%%%%%%%%%%%%%%%%%%%%%%%%%%%%%%%%%%

% 1 - ETHIQUE
			
	\section{Déclaration d'éthique}

Cette étude s'est conformée aux directives éthiques de la déclaration d'Helsinki de 1975 et a été approuvée par l'un des comités et instituts suivants selon les souches virales cliniques concernées : le Comité consultatif national d'éthique pour les sciences de la vie et de la santé en France (déclaration numéro DC-2008-531, C. Gondeau) ; le Comité national d'éthique du Cameroun (numéro 199/CNE/SE/2011, R. Njouom) et le Ministère de la Santé du Cameroun (numéro 631- 01.12) ; le \textit{Bioethics Committee of the Cantacuzino National Medical-Military Institute of Research and Development} de Roumanie (numéro 16/CEE, G. Oprisan). Un consentement éclairé écrit a été obtenu de tous les patients ou de leurs familles. Aucune information sur les patients n'est disponible au laboratoire, hormis leur sexe, leur âge et la raison de la résection chirurgicale.


% 2 - ISOLATS CLINIQUES
			
	\section{Isolats cliniques et souches prototypiques du VHC}

Les ARN viraux extraits du sérum de patients souffrant d'hépatite C chronique et atteints à des degrés variables de micro et macro-stéatose ont été fournis par le Dr. C. Gondeau (Institut de Recherche en Biothérapie, Montpellier, France) \citep{RN812} à l’exception d’une souche 3a (GenBank \
\#LN897692.1) obtenue du Dr. Penelope Mavromara (\textit{Hellenic Pasteur Institute, Athens, Greece}). Les degrés de stéatose ont été évalués par microscopie optique sur coupes colorées par estimation semi-quantitative d'un pourcentage d'hépatocytes contenant des vacuoles graisseuses par rapport à l'ensemble des hépatocytes. La séquence de la protéine Core mature de l'isolat S311 de sous-type 3a s'est avérée identique à celle de la souche grecque que le laboratoire avait précédemment utilisée pour produire un virus intergénotypique Jad/C3aG. Le résidu en position 178 (Leu), au sein du peptide signal C-terminal de la protéine Core immature exprimée par Jad/C3aG est une Phe dans l'isolat 311, différence que nous avons considéré négligeable puisque le peptide signal est clivé dans la protéine Core mature fonctionnelle. Ce virus, nouvellement désigné Jad/C3a-311, a été utilisé dans ce projet pour refléter les caractéristiques de Core de l'isolat S311. La séquence codant Core de l’isolat de sous-type 4a (désignée 4aR, ENA \#ERZ655054) a été isolée d’un donneur de sang séro-positif pour le VHC et fournie par le Dr. Gabriela Oprisan (\textit{Cantacuzino National Medical-Military Institute of Research and Development, Titu Maiorescu University,} Bucharest, \textit{Romania}). La séquence codant Core de l’isolat de sous-type 4f (désignée 4fC, ENA \#ERZ672786) a été isolée d’un patient souffrant d’un CHC et fournie par le Dr. Richard Njouom (Centre Pasteur du Cameroun, Yaoundé, \textit{Cameroon}). Les séquences 4aR et 4fC ont été déterminées et publiées dans une étude précédente de notre laboratoire \citep{RN808}. La séquence codant Core de la souche prototypique H77 de sous-type 1a est issue de l’ADNc de longueur génomique (1aH77, GenBank \#AF011751) \citep{RN310}, généreusement fourni par le Dr. Robert Purcell (\textit{National Institute of Health}, Bethesda, \textit{United States}). Les séquences codant Core des souches prototypiques JFH-1 (GenBank \#AB047639) et J6 (GenBank \#D00944) de sous-type 2a proviennent des ADNc de longueur génomique, respectivement pJFH1 et pJc1-2EI3, généreusement fournis par les Drs. T. Wakita (\textit{National Institute of Infectious Diseases,} Tokyo, \textit{Japan}) \citep{RN124} et R. Bartenschlager (\textit{University of Heidelberg,} Heidelberg, \textit{Germany}) \citep{RN813}.


% 3 - PLASMIDES

	\section{Plasmides}
			
		\subsubsection{Les plasmides pJFH1 et pJad codant les ADNc viraux}

Le plasmide pJFH1 contient l’ADNc de la souche JFH-1 de sous-type 2a \citep{RN124} cloné en aval du promoteur de l’ARN polymérase du phage T7. L’ADNc d'un dérivé du virus JFH-1 hautement adapté à la culture cellulaire Huh-7.5 a été produit par le laboratoire du Dr. Annette Martin (Groupe Hepacivirus, Institut Pasteur, Paris) à partir des plasmides pJFH1 et pJFH1-2EI3 adapt (don du Dr. Ralf Bartenshlager) et nommé pJad \citep{RN811} (\autoref{fig:figM1}). Il contient deux substitutions nucléotidiques dans la séquence codant NS5A (codant les conversions d'acides aminés V2153A et V2240L) et une substitution dans la région codant NS5B (conversion V2941M) qui confèrent à la progénie virale un titre infectieux atteignant ~3 x 10\up{5} TCID50/mL \citep{RN342}.

	\begin{figureth}
	\centering
			\includegraphics[width=0.75\linewidth]{Figure_78.png}
		\caption[Carte du plasmide pJad.]{\textbf{Carte du plasmide pJad.} L'ADNc de longueur génomique de la souche Jad est inséré en aval du promoteur de l'ARN polymérase du phage T7, générant le plasmide pJad de 12 360 paires de bases (pb). Les mutations adaptatrices (Mut. adapt.) sont indiquées, ainsi que le site de restriction unique (\textit{Xba}I) utilisé à des fins de linéarisation de l'ADN avant la transcription \textit{in vitro}. Les sites de restriction (\textit{Age}I et \textit{Bsi}WI) utilisés pour la construction des ADNc recombinants intergénotypiques sont également indiqués. La carte a été conçue par S. Aicher à l’aide du logiciel CLC Main Workbench.}
				\label{fig:figM1}
	\end{figureth}
		\FloatBarrier

		\subsubsection{Les plasmides rapporteurs pJFH1-2EIL3, pJad-2EIL3 et dérivés}

Les plasmides pJad-2EIL3, pJFH1-2EIL3/GAA, pJad-2EIL3ΔEp7 étaient disponibles dans le laboratoire du Dr. Annette Martin (Groupe Hepacivirus, Institut Pasteur, Paris). Le plasmide pJad-2EIL3 est dérivé du pJad et contient un ADNc bicistronique du Jad, pour lequel les séquences codant C-NS2 sont insérées en aval de la région 5'NC du VHC et les séquences NS3-NS5B sont placées en aval d'une insertion de l’IRES de l'EMCV \citep{RN810}. De plus, l'IRES de l'EMCV est immédiatement suivi de la séquence rapportrice de la FLuc, de la séquence codant le peptide 2A du FMDV et de la séquence codant pour un monomère d’ubiquitine qui permet la libération de NS3 (\autoref{fig:figM2}). Le plasmide pJFH1-2EIL3/GAA code la substitution des codons du site actif de l’ARN polymérase NS5B (Gly-Asp-Asp) en codons non fonctionnels (Gly-Ala-Ala), et le plasmide pJad-2EIL3ΔEp7 contient une délétion des séquences codant les glycoprotéines d’enveloppe E1 et E2 et la viroporine p7.

	\begin{figureth}
	\centering
			\includegraphics[width=0.85\linewidth]{Figure_79.png}
		\caption[Carte du plasmide pJad-2EIL3.]{\textbf{Carte du plasmide pJad-2EIL3.} L'ADNc du génome du VHC est inséré en aval du promoteur de l'ARN polymérase T7. Les séquences codantes de FLuc, du peptide 2A du FMDV et du monomère d’ubiquitine sont insérées entre les séquences codantes de NS2 et NS3, générant un vecteur pJad-2EIL3 de 15 443 pb. Les sites de restriction (\textit{Mlu}I, \textit{Age}I et \textit{Not}I) utilisés pour la linéarisation de l’ADN avant la transcription \textit{in vitro} et le clonage sont indiqués. La carte a été conçue par S. Aicher à l’aide du logiciel CLC Main Workbench.}
				\label{fig:figM2}
	\end{figureth}
		\FloatBarrier


% 4 - Culture cellulaire

\section{Culture cellulaire}	

La lignée cellulaire d’hépatome humain Huh-7.5 (aimablement fournie par le Dr. Charles Rice, \textit{The Rockefeller University,} New York, \textit{United States}), hautement permissive à la réplication du VHC \citep{RN322}, a été utilisée comme modèle expérimental principal. Les cellules Huh-7.5 ont été cultivées en monocouche en milieu complet, \textit{i.e} milieu DMEM (\textit{Dulbecco’s Modified Eagle Medium}, Gibco, \#11574486) supplémenté avec 10\% de sérum de veau foetal inactivé (FBS, Gibco, \#10270-106), 1mM de pyruvate de sodium (Gibco, \#11360039), 100 U/mL de pénicilline, 100µg/mL de streptomycine (Gibco, \#11548876) et des acides aminés non essentiels (Gibco, \#11140035) et maintenues à 37°C sous une atmosphère de 5\% de CO\textsubscript{2}.

% 5 - ANTICORPS

\section{Anticorps}	

Les anticorps utilisés lors de cette étude sont répertoriés dans Tableaux \ref{tab:tabM1} et \ref{tab:tabM2}.

\begin{flushleft}
\begin{tabular}{|m{5cm}|M{5cm}|M{1.75cm}|M{2cm}|}
\hline
Anticorps &  Société & Référence & Dilution \\
\hline
Anticorps monoclonal anti-actine ß de souris (AC-15)\up{a} & Abcam (Cambridge, \textit{United Kingdom})    &ab6276&   1:5000\\
Anticorps monoclonal anti-ubiquitine de souris (P4D1)\up{a} &   Affymetrix eBioscience (San Diego, \textit{United States})  & 15217707   &1:2000\\
Anticorps polyclonal anti-PSMB5 de lapin\up{b} &Thermo FisherScientific (Waltham, \textit{United States}) & PA1-977&  1:200\\
Anticorps polyclonal anti-PLIN2 de mouton\up{b} &Don du Dr. John McLauchlan (\textit{MRC}, Glasgow, \textit{United Kingdom}) & -&  1:1000\\
 Anticorps monoclonal anti-Core JFH-1 de souris (1851)\up{a} &Santa Cruz Biotechnology (Dallas, \textit{United States}) & SC-58144&  1:1000\\
Anticorps monoclonal anti-Core Con1b de souris (C7-50)\up{a}&Abcam 
(Cambridge, \textit{United Kingdom})  & Ab2740&1:1000\\
Anticorps monoclonal anti-Core JFH-1 de souris (4F5)\up{a}& BioFront Technologies (\textit{Tallahassee, United States})  & HCV-4F5 &1:1000\\
 Anticorps monoclonal anti-Core JFH-1 de souris (3D11)\up{a}& BioFront Technologies (\textit{Tallahassee, United States})  & HCV-3D11&1:1000\\
Anticorps monoclonal anti-Core 1aH77 de souris (ACAP27)\up{a,b}& Don du Dr. Agata Budkowska (Institut Pasteur, Paris, France)
& -& WB : 1:3000   IF : 1:1000\\
 Anticorps polyclonal anti-Core de lapin (FL)\up{a,b}& Don du Dr. Athanasios Kakkanas (\textit{HPI, Athens, Greece})
& -& WB : 1:2000   IF : 1:2000\\
Anticorps monoclonal anti-NS3 JFH-1 de souris (2E3)\up{c} & BioFront Technologies (\textit{Tallahassee, United States})& HCV-2E3& 1:3000\\
Anticorps monoclonal anti-NS5A JFH-1 de souris (2F6)\up{b}& BioFront Technologies (\textit{Tallahassee, United States})& HCV-2F6& 1:2000\\
Anticorps monoclonal anti-NS5B JFH-1 souris (4B8)\up{a}& BioFront Technologies (\textit{Tallahassee, United States})& HCV-4B8 & 1:2000 \\
\hline
\end{tabular}
\end{flushleft}

\begin{tableth}
\caption[Liste des anticorps commerciaux et issus de dons utilisés lors de cette étude.]{\textbf{Liste des anticorps commerciaux et issus de dons utilisés lors de cette étude.} Les méthodes expérimentales qui reposent sur l’utilisation des anticorps sont précisées : révélation de protéines cellulaires ou virales par immuno-blot (a), immunofluorescence (b) ou détection chromogène des foyers infectieux pour déterminer le titre viral (c).}.
			\label{tab:tabM1}
\end{tableth}

\begin{flushleft}
\begin{tabular}{ |m{5cm}|M{5cm}|M{1.75cm}|M{2cm}|  }
\hline
Anticorps &  Société & Référence & Dilution\\
\hline
Anticorps anti-souris IgG (H\&L) de chèvre conjugué Dylight 680\up{a}& Li-Cor Biosciences 
(Lincoln, \textit{United States})& 926-68070& 1:10.000\\
Anticorps anti-souris IgG (H\&L) de chèvre conjugué Dylight 800\up{a}& Thermo Fisher Scientific 
(Waltham, \textit{United States})& SA5-35521& 1:10.000\\
Anticorps anti-lapin IgG (H\&L) de chèvre conjugué Dylight 800\up{a}& Thermo Fisher Scientific 
(Waltham, \textit{United States})& SA5-35571& 1:10.000\\
Anticorps anti-souris IgG (H\&L) de chèvre conjugué Alexa Fluor 488\up{b}& Thermo Fisher Scientific (Waltham, \textit{United States})& A-11029& 1:500\\
Anticorps anti-lapin IgG (H\&L) d’âne conjugué Alexa Fluor 555\up{b}& Thermo Fisher Scientific (Waltham, \textit{United States})& A-31572& 1:500\\
Anticorps anti-lapin IgG (H\&L) d’âne conjugué Alexa Fluor 647\up{b}& Thermo Fisher Scientific (Waltham, \textit{United States})& A-31673& 1:500 \\
Anticorps anti-mouton IgG (H\&L) d'âne conjugué Alexa Fluor 647\up{b}& Thermo Fisher Scientific (Waltham, \textit{United States})& A-21448& 1:500 \\
 \hline
\end{tabular}
\end{flushleft}

\begin{tableth}
\caption[Liste des anticorps commerciaux et issus de dons utilisés lors de cette étude (Suite).]{\textbf{Liste des anticorps commerciaux et issus de dons utilisés lors de cette étude (Suite).} Les méthodes expérimentales qui reposent sur l’utilisation des anticorps sont précisées : révélation de protéines cellulaires ou virales par immuno-blot (a), immunofluorescence (b) ou détection chromogène des foyers infectieux pour déterminer le titre viral (c).}.
			\label{tab:tabM2}
\end{tableth}
\clearpage

% 6 - KITS ET REACTIFS

\section{Enzymes, réactifs et kits commerciaux}	

Les enzymes, réactifs et les kits commerciaux utilisés lors de cette étude sont répertoriés dans les Tableaux \ref{tab:tabM3} et \ref{tab:tabM4}.

\begin{flushleft}
\begin{tabular}{ |m{8.5cm}|M{4cm}|M{2.5cm}|  }
\hline
Nom & Société & Référence \\
\hline
Countess Cell Counting Chamber Slides & Thermo Fisher Scientific & C10228 \\
MassRuler Express Forward DNA Ladder Mix & Fermentas & SM1283 \\
RNA Millenium Markers-Formamide & Thermo Fisher Scientific & 10351375 \\
Pwo Super Yield DNA Polymerase & Roche & 04743750001 \\
One Taq 2X Master Mix Polymerase & New England BioLabs & M0482S \\ 
TOPO TA cloning for sequencing & Thermo Fisher Scientific & 450071 \\
Zero Blunt TOPO PCR cloning for sequencing & Thermo Fisher Scientific & 450159 \\
Mung Bean Nuclease & New England BioLabs & M0250S \\
NucleoBond Xtra Midi Plus & MACHEREY-NAGEL & 740412.10 \\
QIAGEN Plasmid Mini Kit & Qiagen & 12123 \\
HiPure Plasmid Filter Midi Purification Kit & Thermo Fisher Scientific & K210014 \\
QIAquick PCR Purification Kit & Qiagen & 28104 \\
Shrimp Alkaline Phosphatase rSAP & New England BioLabs & M0371S \\
T7 RiboMAX Express Large Scale RNA Production System & Promega & P1320 \\
RQ1 RNase-Free DNase & Promega & M6101 \\
TaqMan Fast-virus 1 Step Master Mix & Applied Biosystems & 13498446 \\
Eukaryotic 18S rRNA Endogenous Control & Applied Biosystems & 10321085 \\
Oligo(dT)18 Primer 100μM & Thermo Fisher Scientific & SO131 \\
Random primers d(pN)6 1mM & ROCHE & 1103473100 \\
Mix dNTPs 100 mM & Eurobio & GAEPCR11-5C \\
SUPERSCRIPT II Reverse Transcriptase & Thermo Fisher Scientific & 1806401 \\
RNasin Ribonuclease inhibitor 20-40 u/μL, 10000U & Promega & N2115 \\
Rapid DNA Ligation Kit & ROCHE & 11635379001 \\
Phosphatase Inhibitor Cocktail Tablets & ROCHE & 04906845001 \\
NUPAGE NOVEX 4-12\% BIS TRIS GELS 1,5 mm 12 Puits & Thermo Fisher Scientific & NP0322BOX \\
NUPAGE NOVEX 4-12\% BIS TRIS GELS 1,5 mm 15 Puits & Thermo Fisher Scientific & NP0323BOX \\
NUPAGE NOVEX 12\% BIS TRIS GELS 1,5 mm 12 Puits & Thermo Fisher Scientific & NP0342BOX \\
NuPAGE LDS Sample Buffer (4X) & Thermo Fisher Scientific & NP0007 \\
NuPAGE MOPS SDS Running Buffer (20X) & Thermo Fisher Scientific & NP0001 \\
NuPAGE MES SDS Running Buffer (20X) & Thermo Fisher Scientific & NP0002 \\
NuPAGE Transfer Buffer (20X) & Thermo Fisher Scientific & NP0006 \\
NuPAGE Antioxidant & Thermo Fisher Scientific & NP0005 \\
Stripping Buffer & Euromedex & ST010 \\
Nitrocellulose Premium, 0.45μm & Amersham & 10600003 \\
SeeBlue Pre-Stained Standard & Thermo Fisher Scientific & LC5625 \\
Revert 700 Total Protein Stain & Li-Cor Biosciences & 926-11011 \\
\hline
\end{tabular}
\end{flushleft}

\begin{tableth}
\caption[Liste des enzymes, kits et réactifs commerciaux utilisés lors de cette étude.]{\textbf{Liste des enzymes, kits et réactifs commerciaux utilisés lors de cette étude.}}.
			\label{tab:tabM3}
\end{tableth}

\begin{flushleft}
\begin{tabular}{ |m{8.5cm}|M{4cm}|M{2.5cm}|  }
\hline
Nom & Société & Référence \\
\hline
Big Dye v.1.1 & Applied Biosystems & 4336799 \\
Big Dye v.3.1 & Applied Biosystems & 4337455 \\
Big Dye Terminator Buffer 5X & Applied Biosystems & 4336097 \\
Reporter Lysis 5X Buffer & Promega & E3971 \\
Luciferase Assay Reagent & Promega & E1500 \\
MG132 Proteasomal Inhibitor & Sigma & C-2211 \\
RNAzol RT & Sigma & R4533 \\
RNeasy Plus Mini & Qiagen & 74134 \\
RNase-free DNase I Set & Qiagen & 79254 \\
QiaAmp Viral RNA Mini & Qiagen & 52904 \\
Dynabeads mRNA Purification Kit & Thermo Fisher Scientific & 61006 \\
HS NGS Fragment Kit & Agilent & DNF-474-0500 \\
µ-Slide VI 0.5 Glass Bottom & Ibidi & 80607 \\
Nunc Optical MicroWell 96-well plates Coverglass Bottom & Thermo Fisher Scientific & 160376 \\
Ibidi Mounting Medium & Ibidi & 50001 \\
Goat Serum & Sigma & G9023 \\
Donkey Serum & Sigma & D9663 \\
ViewRNA Cell Plus Assay & Thermo Fisher Scientific & 88-19000 \\
BODIPY 558/568 C12 & Thermo Fisher Scientific & D3835 \\
LipidTOX Red Neutral Lipid Stain & Thermo Fisher Scientific & H34476\\
DAPI & Thermo Fisher Scientific & D1306 \\
ImmPRESS HRP Anti-Mouse IgG Polymer Detection Kit & Vector Laboratories & MP-7402 \\
DAB Substrate Kit & Vector Laboratories & SK-4100 \\
\hline
\end{tabular}
\end{flushleft}

\begin{tableth}
\caption[Liste des enzymes, kits et réactifs commerciaux utilisés lors de cette étude (Suite).]{\textbf{Liste des enzymes, kits et réactifs commerciaux utilisés lors de cette étude (Suite).}}.
\label{tab:tabM4}
\end{tableth}

\clearpage


% 7 - INSTRUMENTS

\section{Instruments}	

\begin{itemize}
  \item[$\bullet$] \textit{Countess II Automated Cell Counter}, Thermo Fisher Scientific (Groupe Hepacivirus, Institut Pasteur, Paris)
  \item[$\bullet$] \textit{7500 Fast Real-Time PCR System}, Applied Biosystems (Département de Virologie, Institut Pasteur, Paris)
  \item[$\bullet$] \textit{QuantStudio™ 7 Flex Real-Time PCR System}, Thermo Fisher Scientific (U5 Signalisation Antivirale, Institut Pasteur, Paris)
  \item[$\bullet$] \textit{LSM 700 Inverted Confocal Microscope}, Zeiss (Plateforme de Bio-imagerie Photonique, Institut Pasteur, Paris)
  \item[$\bullet$] \textit{JEM-1011 Transmission Electron Microscope}, JEOL (Faculté de Médecine, Université de Tours, Tours)
  \item[$\bullet$] \textit{EM UC7 Ultramicrotome}, Leica (Faculté de Médecine, Université de Tours, Tours)
  \item[$\bullet$] \textit{TriStar LB 942 Multimode Microplate Reader}, Berthold Technologies (Groupe Hepacivirus, Institut Pasteur, Paris)
  \item[$\bullet$] \textit{Odyssey CLx Imaging System}, Li-Cor (Département de Virologie, Institut Pasteur Paris)
  \item[$\bullet$] \textit{QIAcube HT}, Qiagen (Groupe Hepacivirus, Institut Pasteur, Paris)
  \item[$\bullet$] \textit{GenePulser XCell}, Bio-Rad (Groupe Hepacivirus, Institut Pasteur, Paris)
  \item[$\bullet$] \textit{Novex NuPAGE SDS-PAGE Gel System} Thermo Fisher Scientific  (Groupe Hepacivirus, Institut Pasteur, Paris)
  \item[$\bullet$] \textit{Veriti 96-Well Fast Thermal Cycler}, Applied Biosystems (Groupe Hepacivirus, Institut Pasteur, Paris)
  \item[$\bullet$] \textit{Eppendorf Mastercycler 5333 Thermal Cycler}, Eppendorf (Groupe Hepacivirus, Institut Pasteur, Paris)
  \item[$\bullet$] \textit{Nanodrop MySPEC Micro-volume Spectrophotometer}, Ozyme (Groupe Hepacivirus, Institut Pasteur, Paris)
  \item[$\bullet$] \textit{2100 Bioanalyzer}, Agilent (Plateforme Biomics, Institut Pasteur, Paris)
  \item[$\bullet$] DNB-G400, MGI (Plateforme Biomics, Institut Pasteur, Paris)
\end{itemize}
	%%%%%%%%%%%%% Copyright (C) 2017-2021 Emeline Simon
%%
%% The current owner of this work is Emeline Simon
%% <contact at emeline.simon@gmail.com>.
%%
%% This is 05_methods.tex the fifth chapter for my PhD Thesis.
%%
%%%%%%%%%%%%%%%%%%%%%%%%%%%%%%%%%%%%%%%%%%

\chapter{Méthodes Expérimentales}
	\minitoc
	\newpage

%%%%%%%%%%%%%%%%%%%%%%%%%%%%%%%%%%%%%%%%%%%%%%%%%%%%%%%%%%%%%%%%%%%%%%%%%%%%%%%%%%%%%%%%%%%%%%%%%%%%%%%%%%%%%%%%%%%%%%%%%%%%%%%%%%%%%%%%%%%%%%%%%%%%%%%%%%%%%%%%%%%%%%%%

% 1 - SOUCHES CLINIQUES
			
	\section{Identification et clonage des séquences de Core des souches cliniques}
	\label{sec:souches}
	
	Les séquences codant Core provenant des souches cliniques S411, S311, S376, S389, S390, S395, S401 ont été obtenues après transcription inverse (RT) des ARN viraux à l'aide d'une amorce non spécifique (pdN(6), Thermo Fisher Scientific) et de \textit{reverse transcriptase SuperScript II} (Thermo Fisher Scientific), suivie d'une amplification par PCR semi-nichée ou nichée à l'aide de \textit{One Taq 2X Master Mix with Standard Buffer} (Thermo Fisher Scientific) et de paires d'amorces conçues pour s’hybrider avec des séquences consensus pour chaque sous-type au niveau de la région 5' NC et de la partie codante N-terminale de E1 (voir \autoref{tab:tabM5}). Les fragments d'ADN amplifiés et purifiés ont été, d'une part, séquencés directement par technique Sanger et électrophorèse capillaire (Eurofins). Les séquences consensus prédominantes codant Core ont été établies à partir de ces analyses. D'autre part, les fragments d'ADN ont été clonés dans un vecteur navette à l'aide du kit \textit{TOPO TA cloning for sequencing} ou \textit{Zero Blunt TOPO PCR cloning for sequencing} (Thermo Fisher Scientific), afin de disposer d'une séquence clonale à substituer à celle de Core du JFH-1 et d'apprécier la présence et la fréquence d'éventuelles quasi-espèces. Pour cette seconde étape, les ADN plasmidiques de 10 à 28 clones issus des produits d'une ou de deux réactions de RT-PCR indépendantes ont été séquencés (Eurofins). \\
	
\begin{tabular}{ |m{6.5cm}|m{6.5cm}|  }
\hline
Nom de l'amorce & Séquence (5' -- 3')\\
\hline
HCV3a_243FW1 & GATCACTAGCCGAGTAGTGTTGG \\
HCV3a_276FW2 (semi-nichée) & GCCTTGTGGTACTGCCTGATAG \\
HCV3a_945RV1 & GCTATTGGAACAGTCGTTGGTAAGG \\
PCR-5'-gt1_4-FW & CTTGTGGTACTGCCTGATAGG \\
Nested-5'-gt1_4-FW (semi-nichée) & GATAGGGTGCTTGCGAGTGC \\
PCR-E1-gt1-RV & GACCAGTTCATCATCATATCCC \\
\hline
\end{tabular}

	\begin{tableth}
\caption[Liste des amorces conçues pour amplifier et séquencer la séquence codante de Core des souches cliniques.]{\textbf{Liste des amorces conçues pour amplifier et séquencer la séquence codante de Core des souches cliniques.} La spécificité génotypique (HCV/gt 3a, 1, 1a) et la polarité (FW : génomique, RV : anti-génomique) des amorces, ainsi que l'utilisation pour la PCR semi-nichée sont indiquées dans la colonne de gauche.}.
			\label{tab:tabM5}
\end{tableth}

Globalement, pour tous les virus, la plupart des mutations silencieuses ou des mutations codantes (donnant lieu à une modification d'acide aminé dans la séquence de Core) a été retrouvée dans une proportion faible de clones (< 5\% si $\geq$24 clones analysés ou < 10\% si 10-12 clones analysés) et peut représenter soit des quasi-espèces mineures circulant chez les patients, soit des erreurs liées aux enzymes durant les étapes de RT ou PCR. Il est à noter qu'un nombre plus important de positions de Core cibles de substitutions minoritaires a été retrouvé pour l'isolat S401 de sous-type 3a, suggérant la présence de quasi-espèces virales variées chez ce patient. Par ailleurs, pour cet isolat, une mutation codante fréquente a été identifiée au niveau du codon en position 15 par rapport à la séquence considérée comme consensus (Ile -->Thr pour 10/24 clones) dans les séries de clones issues des deux réactions RT-PCR indépendantes, représentant ainsi une quasi-espèce non minoritaire. La situation s'est avérée du même ordre pour l'isolat S395 de sous-type 3a, avec 4/28 clones présentant un résidu Leu au lieu de Pro en position 4 et 4/28 clones présentant un résidu Ile au lieu de Val en position 162, ainsi que d'autres quasi-espèces plus minoritaires. Il en est de même pour l'isolat S411 de sous-type 1a, avec un nombre élevé de positions cibles de quasi-espèces silencieuses équilibrées (25-50\%), et le codon en position 75 (Ser) cible de mutations codantes (Thr ou Ala, pour 4 clones sur 24). Ces résultats dénotent une co-évolution plus importante du virus infectant chez ces trois patients par rapport aux autres. 

% 2 - CONSTRUCTION DES PLASMIDES
			
	\section{Construction des plasmides}
	
		\subsubsection{Construction des plasmides pJad et pJad-2EIL3 chimériques}
		
La construction des plasmides contenant les ADNc chimériques repose sur une stratégie de PCR de recouvrement conçue pour échanger uniquement la séquence codant Core. En bref, trois segments d'ADNc viral sont amplifiés dans une première étape en utilisant des amorces dotées d'extensions complémentaires des segments à fusionner, qui permettent le recouvrement des produits et la création des points de jonction exacts. Le produit de fusion est amplifié dans une seconde étape à partir des 3 produits PCR issus de la première étape, puis digéré par les enzymes dont les sites sont présents à ses extrémités, et cloné dans le vecteur pJad digéré par ces mêmes enzymes. Les plasmides pJad/C1aH77, /C4aR et /C4fC sont décrits dans un travail précédent du laboratoire \citep{RN808}. Le plasmide pJad/C3a-311, incluant une séquence codant Core mature identique à celle de l’isolat S311, a été précédemment produit dans le laboratoire. Les plasmides pJad/C1a-411, /C2a-J6, /C3a-376, /C3a-389, /C3a-390, /C3a-395, et /C3a-401 ont été construits pour cette étude. Un premier fragment d'ADN a été amplifié par PCR en utilisant pJad comme matrice avec une amorce de polarité positive s'hybridant en amont du site de restriction unique \textit{Age}I dans la région 5’NC et une amorce de polarité négative complémentaire de l'extrémité 3' de la région 5’NC et dotée en 5' d'une extension correspondant à la partie 5' de la séquence codant Core hétérologue à introduire. Un second fragment a été produit en utilisant le plasmide pCR4-TOPO hébergeant la séquence consensus codant Core d'une souche clinique  ou le plasmide pJc1-2EI3 (Core 2a-J6) comme matrice, avec des amorces s'hybridant respectivement sur les séquences 5’- et 3'- terminales de Core hétérologue et étendues en 5', de l'extrémité 3' de la région 5’NC et en 3', de l'extrémité 5' de la séquence de E1. Un troisième produit de PCR a été généré en utilisant pJad comme matrice, couvrant la séquence codante de E1 depuis son extrémité 5' jusqu'au site de restriction unique \textit{Bsi}WI, chevauchant ainsi l'amorce utilisée pour la deuxième PCR. Les PCR ont été réalisées avec 2,5U de polymérase \textit{Pwo Super Yield DNA Polymerase} (Roche) selon les instructions du fabricant. Une PCR de recouvrement dans laquelle les trois produits de PCR on été mélangés dans des rapports équimolaires a été réalisée en utilisant les deux amorces externes. Le produit PCR résultant couvre un segment de la région 5’NC, les séquences hétérologues codant Core et un segment de la séquence codant E1, encadré des sites de restriction \textit{Age}I et \textit{Bsi}WI. Après purification (\textit{QIAquick PCR Purification Kit}, Qiagen), ce produit PCR a été digéré avec les enzymes de restriction \textit{Age}IHF et \textit{Bsi}WI (New England Biolabs) et cloné dans le plasmide pJad préalablement digéré par ces mêmes enzymes de restriction et déphosphorylé à l’aide de 1U de phosphatase alcaline de crevette recombinante (\textit{Shrimp Alkaline Phosphatase rSAP}, NewEngland Biolabs). La ligation est réalisée avec le kit \textit{Rapid DNA Ligation} (Roche). Des cellules compétentes XL1-Blue ont été transformées avec ce mélange de ligation. Les colonies ont été sélectionnées dans des boîtes de Pétri LB contenant de l’ampicilline, à partir desquelles l’ADN plasmidique a été isolé (\textit{Plasmid Mini Kit}, Qiagen). Les clones ont été criblés avec les enzymes de restriction \textit{Kpn}I, \textit{Blp}I, \textit{Eco}RV et/ou \textit{Cla}I, présents dans les séquences Core hérétologues mais pas dans la séquence Core native du Jad. Les clones positifs ont été amplifiés pour la préparation à large échelle d’ADN (\textit{NucleoBond Xtra Midi Plus}, Macherey-Nagel). Toutes les séquences Core hétérologues de souches cliniques et prototypiques ont été insérées dans le plasmide pJad-2EIL3 en remplaçant le fragment de restriction \textit{Age}I-\textit{Not}I (situé dans la séquence codante de NS2) ou \textit{Age}I-\textit{Bsi}WI par les fragments correspondants isolés des plasmides chimériques dans le contexte pJad, générant ainsi les plasmides chimériques pJad-2EIL3/C1aH77, /C1a-411, /C2a-J6, /C3a-311, /C3a-376, /C3a-389, /C3a-390, /C3a-395, /C3a-401, /C4aR et /C4fC.

		\subsubsection{Construction des plasmides pJad/C_DP, pJad/C_DP_SATG, pJad/C_3aa_SPP et pJad/C_4aa_SPP par mutagenèse dirigée}

Un plasmide dérivé de pJad portant les mutations ponctuelles des codons GCA et CGC induisant les modifications d'acides aminés Y138A et Y143A dans la séquence de Core a été construit selon le même type de stratégie par PCR de recouvrement, mais reposant sur deux fragments qui se chevauchent au niveau de la région à muter et s'étendent en amont du site \textit{Age}I et en aval du site \textit{Bsi}WI. Deux oligonucléotides complémentaires portant les substitutions nucléotidiques d'intérêt ont été utilisés. Les clonages ont été réalisés entre les sites \textit{Age}I et \textit{Bsi}WI de pJad, générant ainsi le plasmide pJad/C_DP. La même stratégie a été employée pour produire (i) à partir de la matrice pJad/C_DP, le plasmide pJad/C_DP_SATG portant la substitution nucléotidique additionnelle TCA altérant le résidu Y164 en S164, (ii) à partir de la matrice pJad, plasmide pJad/C_3aa_SPP portant les substitutions nucléotidiques GTC, CTC et GTC induisant les modifications d’acides aminés A180V, S183L et C184V et (iv) à partir de la matrice pJad/C_3aa_SPP, le plasmide pJad/C_4aa_SPP portant la substitution nucléotidique additionnelle CCT altérant le résidu T186 en L186. Les amorces conçues pour la mutagenèse dirigée par PCR sont listées dans le \autoref{tab:tabM6}. Deux clones indépendants de chaque construction ont été retenus pour chaque construction mutante afin de confirmer par la suite les phénotypes des ARN viraux correspondants. L'intégrité et la séquence de tous les clones a été confirmée par une analyse de la séquence nucléotidique (voir \autoref{sec:sequencing}).

\begin{flushleft}
	\begin{tabular}{ |m{3.25cm}|m{7.75cm}|m{3cm}|  }
 \hline
Nom de l'amorce & Séquence (5' -- 3') & Substitution \\
\hline
DP_CoreJFH- & CCACTAAGC\textbf{G\textcolor{red}{C}G}GCGCCTACGAC\textbf{T\textcolor{red}{G}C}GATGTACCC  & Position 138 : Y -> A   Position 143 : Y -> A \\
CoreJFH_DP+ & GGGTACATC\textbf{G\textcolor{red}{C}A}GTCGTAGGCGC\textbf{C\textcolor{red}{G}C}GCTTAGTGG  & Position 138 : Y -> A   Position 143 : Y -> A \\
Y164S_CoreJFH- & GGTTCCCTGTTGC\textbf{T\textcolor{red}{G}A}ATTAACCCCGTCC & Position 164 : Y -> S \\
CoreJFH_Y164S+ & GGACGGGGTTAAT\textbf{T\textcolor{red}{C}A}GCAACAGGGAACC & Position 164 : Y -> S \\
Mut_SPP_CoreJFH- & GGTGAT\textbf{G\textcolor{red}{AC}G\textcolor{red}{AG}}CAACAG\textbf{G\textcolor{red}{A}C}CAGCAAGAAGATAG & Position 180 : A -> V   Position 183 : S -> L  Position 184 : C -> V \\
CoreJFH_Mut_SPP+ & TGCTG\textbf{G\textcolor{red}{T}C}CCTGTTG\textbf{\textcolor{red}{CT}C\textcolor{red}{GT}C}ATCACCGTTCCGGTC  & Position 180 : A -> V   Position 183 : S -> L   Position 184 : C -> V \\
Mut_T186L_CoreSPP- & GAGACCGGAACG\textbf{\textcolor{red}{AG}G}ATGACGAGCAAC & Position 186 : T -> L \\
CoreSPP_Mut_T86L+ & GTTGCTCGTCAT\textbf{C\textcolor{red}{CT}}CCGTTCCGGTCTC & Position 186 : T -> L \\
\hline
\end{tabular}
\end{flushleft}

	\begin{tableth}
\caption[Liste des amorces conçues pour générer les inserts d’ADN à l’origine des plasmides pJad/C_DP, pJad/C_DP_SATG, pJad/C_3aa_SPP et pJad/C_4aa_SPP.]{\textbf{Liste des amorces conçues pour générer les inserts d’ADN à l’origine des plasmides pJad/C_DP, pJad/C_DP_SATG, pJad/C_3aa_SPP et pJad/C_4aa_SPP.} Les résidus nucléotidiques qui diffèrent de la séquence génomique complémentaire sont indiqués en rouge et les codons sont mis en valeur en gras.}.
			\label{tab:tabM6}
\end{tableth}

% 3 - TRANSCRIPTION
			
	\section{Transcription de l’ARN \textit{in vitro}}
		\label{sec:transcription}
	
Pour préparer les ARN viraux de longueur génomique \textit{in vitro}, le plasmide pJad et les plasmides dérivés codant les ADNc recombinants pJad/C(X) ont été linéarisés par l'enzyme de restriction \textit{Xba}I qui va générer une extrémité 3' protubérante, éliminée par la suite à l’aide d’un traitement à la nucléase \textit{Mung Bean} (New England Biolabs). Le plasmide pJad-2EIL3 et les plasmides dérivés codant les ADNc recombinants pJad-2EIL3/C(X) ou les ADNc contrôles ont été linéarisés par l’enzyme de restriction \textit{Mlu}I et l’extrémité 3’ exacte des ARN synthétiques est obtenue par l’action du ribozyme du VHD dont la séquence est insérée immédiatement en aval de la région 3'NC du VHC et en amont du site \textit{Mlu}I. Les ADN linéarisés ont été purifiés par extraction au phénol/chloroforme, et quantifiés sur des gels d'agarose. Un microgramme d’ADN linéarisé sert ensuite de matrices pour la transcription \textit{in vitro} par l’ARN polymérase du phage T7 (Promega). La matrice d’ADN est ensuite éliminée après un traitement avec 1U de DNase (Promega), puis les ARN sont purifiés par extraction au phénol/chloroforme et précipités avec un volume d’isopropanol complémenté avec de l’acétate de sodium 0,3M. La qualité et la quantité des ARN resuspendus en H\textsubscript{2}O sont évaluées par électrophorèse sur gel d’agarose et par mesure de l’absorbance à 260nm.

% 4 - TRANSFECTION
			
	\section{Transfection}
	\label{sec:transfection}	
	
	Après détachement des monocouches par trypsination, les cellules Huh-7.5 sont lavées en tampon phosphate salin sans Ca\up{2+} ni Mg\up{2+} ou DPBS (\textit{Dulbecco’s Phosphate Buffered Saline}, Gibco) et resuspendues dans du milieu Opti-MEM (Gibco) à une concentration de 5x10\up{6} cellules/mL. Deux millions de cellules sont électroporées en présence de 5µg d'ARN transcrits \textit{in vitro} ou de PBS à l'aide d'un système \textit{Gene Pulse} (Biorad, Munich, \textit{Germany}) à 240V et 900µF, dans une cuvette de 0,4cm de large (Eurogentec). Immédiatement après le choc électrique, les cellules sont transférées dans du milieu complet puis réparties dans (i) des flacons de 75cm\up{2} (1,6x10\up{6} cellules) pour la production de stocks viraux, ou (2x10\up{6} cellules) pour les observations par microscopie électronique, (ii) des flacons de 25cm\up{2} (8x10\up{5} cellules) pour les expériences d’adaptation, (iii) des plaques 6 puits (3x10\up{5} cellules par puits) pour le suivi de la réplication virale par dosage de l’activité FLuc, ou (4x10\up{5} cellules) pour la préparation d’extraits protéiques et sur (iv) des lames en verre (1x10\up{4} cellules par canal) pour l’imagerie (Ibidi). Les cellules transfectées sont placées pendant 4h à 37°C sous 5\% de CO\textsubscript{2}, puis le milieu de culture est remplacé par du milieu complet frais afin d'éliminer des débris issu du choc électrique. Les cellules sont maintenues à 37°C sous 5\% de CO\textsubscript{2} pour une durée dépendant des expériences. À des fins d’inhibition du protéasome, les cellules transfectées ont été traitées avec 0,1\% de DMSO contenant 10µM de MG132 (Sigma) durant 14h avant la récolte. Le surnageant de culture et les extraits protéiques sont collectés à 3j p.tf. pour la production de stocks viraux (voir \autoref{sec:stocks}) et pour le suivi de la réplication virale par dosage de l’activité FLuc (voir \autoref{sec:fluc}). Les extraits protéiques sont récoltés 2 ou 3j p.tf. pour l’analyse de l’abondance des protéines virales par immunoblot (voir \autoref{sec:immunoblot}). Les cellules transfectées sont fixées à 2 ou 3j p.tf. pour les observations par microscopie confocale (voir \autoref{sec:confocal}) et à 3j p.tf. en vue des observations par microscopie électronique (voir \autoref{sec:electronique}).

% 5 - INFECTION
			
	\section{Infection}
	\label{sec:infection}
	
La veille de l’infection, des cellules Huh-7.5 sont ensemencées (i) en plaque 24 puits (3x10\up{4} cellules/puits) pour la quantification des ARN et titres infectieux viraux et pour la préparation des extraits protéiques, ou (1,5x10\up{4} cellules/puits) pour la préparation d’extraits d’ARN total, (ii) en plaques 12 puits (4,5x10\up{4} cellules/puits) pour le suivi de l’infectiosité par dosage de l’activité FLuc, (iii) sur des lames en verre (5x10\up{3} cellules/canal) pour les observations par microscopie confocale, et (iv) dans des flacons de 25cm\up{2} (1,25x10\up{5} cellules) en vue des analyses par microscopie électronique. Les cellules sont infectées avec une MOI de 1, 3 ou 10 TCID50/cellule. Pour le suivi de l'infectiosité des ARN viraux par dosage de l'activité FLuc, les cellules naïves sont mises en contact avec 0,5mL du surnageant récolté à 72h p.tf. Après incubation en présence de l'inoculum viral pendant 4h à 37°C sous 5\% de CO\textsubscript{2}, l'inoculum est remplacé par du milieu complet frais et les cellules maintenues à 37°C pour une durée dépendant des expériences. Des stocks viraux indépendants ont été utilisés pour les réplicats biologiques. Le surnageant de culture et les ARN totaux sont collectés à 1, 2, 3, 4 ou 5j p.i. pour la quantification des titres infectieux (voir \autoref{sec:titrage}) et des ARN viraux intra- et extra-cellulaires (voir \autoref{sec:genome}). Les extraits protéiques sont préparés à 4j p.i. pour les analyses par immunoblot (voir \autoref{sec:immunoblot}). Les ARN totaux sont préparés à 4 ou 5j p.i. pour l’analyse à haut débit par RNA-Seq (voir \autoref{sec:transcriptomique}). Les cellules infectées sont fixées 1, 2, 3 ou 4j p.i. pour les observations par microscopie confocale (voir \autoref{sec:confocal}) et à 5j p.i. pour les analyses par microscopie électronique (voir \autoref{sec:electronique}).

% 6 - STOCKS

	\section{Production des stocks viraux}
	\label{sec:stocks}

La progénie virale des génomes parentaux et recombinants a été obtenue par génétique inverse suite à l’électroporation d'ARN viraux transcrits \textit{in vitro }(décrit dans la \autoref{sec:transfection}) en collectant les surnageants de culture des cellules Huh-7.5 transfectées à 3j p.tf., ou à 25j p.tf. dans le cadre des expériences d’adaptation, puis en les clarifiant par centrifugation à 3000g pendant 10min. Les surnageants clarifiés ont été utilisés pour infecter des cellules Huh-7.5 naïves à une MOI de 0,01 TCID50/cellule. Les cellules infectées sont amplifiées lorsqu'elles atteignent une confluence de 80-100\%, généralement tous les 2 à 3j, par transfert dans des flacons de plus grande surface en présence d'une fraction du surnageant de culture collecté (~20\%) ajouté à du milieu complet frais. Les effets cytopathiques associés à l’infection ont été surveillés quotidiennement pour ajuster le protocole de passage des cellules et la collecte finale d'un large volume de surnageant de culture constituant le stock viral, conservé en mono-doses à -80°C. Avant utilisation des stocks, le titre infectieux est établi suite à un minimum de trois titrages indépendants (voir  \autoref{sec:titrage}).

% 7 - TITRAGE

	\section{Détermination du titre viral}
	\label{sec:titrage}
	
Le titre infectieux des stocks viraux a été déterminé par un test en dilutions limites. Des dilutions sériées des surnageants ont été utilisées pour infecter 2,5x10\up{3} cellules Huh-7.5 par puits de plaques à 96 puits à raison de 8 puits par dilution. Cinq jours après l’infection, les cellules sont fixées par ajout de 200µL de méthanol et incubées avec une solution de DPBS supplémenté de 0,3\% de peroxyde d'hydrogène (Sigma) pendant 5min à température ambiante afin de bloquer l'activité peroxydase endogène. Les sites antigéniques non spécifiques sont saturés à l'aide d'une solution de blocage contenant 1,25\% de sérum de cheval, puis les cellules sont incubées avec 0,4µg/mL d’un anticorps monoclonal de souris anti-NS3 du VHC (BioFront) durant la nuit à 4°C. Le lendemain, les cellules sont incubées avec un conjugué anti-IgG souris couplé à la peroxydase de raifort (\textit{ImmPRESS HRP}, Vector Laboratories) pendant 30min à température ambiante. L'activité peroxydase est révélée par contact avec le substrat chromogénique 3,3'-diaminobenzidine (\textit{DAB Substrate Kit, Vector Laboratories}) pendant 20 minutes à température ambiante, ce qui génère une coloration brune sur le site de l’antigène cible. Les foyers infectieux ont été dénombrés manuellement au microscope optique. Le titre viral est exprimé en doses infectieuses 50\% en culture tissulaire par mL (TCID50/mL) et est calculé par la méthode de Reed et Muench.

% 8 - QUANTIFICATION DES GENOMES VIRAUX

	\section{Quantification des génomes viraux}
	\label{sec:genome}
	
Pour extraire les ARN intracellulaires, le tapis cellulaire des puits de plaques 6 ou 24 puits est mis en contact avec respectivement 200 ou 100µL de RNAzol (\textit{RNAzol RT}, Sigma) et les lysats sont transférés en tubes et congelés à -20°C. Après décongélation, l'ARN total est extrait selon les indications du fournisseur. Le culot d'ARN est ensuite resuspendu dans 15 à 30µL d'H\textsubscript{2}O certifiée sans ribonucléases et conservé à -80°C. La concentration d'ARN intracellulaire total est déterminée par mesure de la densité optique à l'aide d'un nano-spectrophotomètre (\textit{MySpec}, Ozyme). L’ARN viral est extrait de 140µL de surnageant de culture à l’aide d’un tampon de lyse virale supplémenté avec 10ng/µL d’ARN porteur (\textit{QiaAmp Viral RNA Mini}, Qiagen), purifié selon les indications du fournisseur, élué dans 55µL d’H\textsubscript{2}O certifiée sans ribonucléases et conservé à -80°C. Les génomes viraux sont quantifiés à partir de 20ng d’ARN intracellulaire total ou de 5µL d’ARN viral particulaire par RT-PCR quantitative en une étape (\textit{TaqMan Fast-virus 1 Step Master Mix}, Applied Biosystems) à l’aide d’une paire d’amorces qui ciblent la région 5’NC du VHC et de sondes fluorescentes FAM (6-carboxyfluorescéine) et TAMRA (6-carboxytétraméthylrho- damine). Les quantités d'ARN intracellulaires introduites dans le test RT-qPCR sont normalisées par rapport à la quantité d’ARN ribosomal mesurée à l’aide d’amorces et d'une sonde fluorescente VIC (2$'$-chloro-7$'$phenyl-1,4-dichloro-6-carboxy-fluorescéine) spécifiques de l’ARN 18S (\textit{Eukaryotic 18S rRNA Endogenous Control}, Applied Biosystems). Les échantillons d’ARN ont été analysés sur un appareil de PCR à temps réel (\textit{7500 Fast Real-Time PCR System}, Applied Biosystems). L’ARN a été soumis à une RT à 50°C pendant 5min, puis après dénaturation de l’enzyme à 95°C pendant 20s, les étapes de la qPCR sont réalisées sur 40 cycles de 15s  à 95°C et de 1min à 60°C. La quantification absolue des génomes viraux est obtenue à l’aide d’une courbe standard établie à partir d’ARN du VHC transcrit \textit{in vitro} dont la concentration en équivalents génomes par µL (copies/µL) a été déterminée par mesure de l’absorbante à 260nm. Les valeurs finales sont exprimées en copies/mL et en copies/µg d’ARN total.

% 9 - SEQUENCAGE

	\section{Séquençage du génome viral}
	\label{sec:sequencing}
	
Pour séquencer le génome viral des expériences d’adaptation (voir section \autoref{sec:transfection}) ou des stocks (voir \autoref{sec:infection}), les ARN sont extraits des cellules transfectées avec 1mL de RNAzol (\textit{RNAzol RT}, Sigma) ou de 140µL de surnageant à l’aide d’un tampon de lyse virale supplémenté avec 10ng/µL d’ARN porteur (\textit{QiaAmp Viral RNA Mini}, Qiagen) selon les indications des fournisseurs. Les ARN sont resuspendus respectivement dans 30µL ou 55µL d'H\textsubscript{2}O certifiée sans ribonucléases et conservés à -80°C. La concentration d'ARN total est déterminée par mesure de la densité optique à l'aide d'un nano-spectrophotomètre (\textit{MySpec}, Ozyme). Un microgramme d’ARN total ou 10µL d’ARN viral particulaire est additionné de 1µM d’amorces de séquence aléatoire (\textit{hexamer pdN(6)}, Roche) et d’un mélange de 1mM de chaque dNTP (\textit{dNTP set}, Eurobio). Afin d'éliminer les structures secondaires de l'ARN, le mélange est incubé 5min à 65°C dans un thermocycleur (\textit{Eppendorf Mastercycler 5333 Thermal Cycler}, Eppendorf), puis immédiatement transféré dans de la glace. Après refroidissement, le mélange réactionnel est supplémenté de 10mM de dithiothréitol (DTT), 500U d'inhibiteur de ribonucléases (\textit{RNasin}, Promega) et 200U de transcriptase inverse (\textit{SuperScript II Reverse Transcriptase}, Thermo Fisher Scientific) dans un volume final de 20µL, puis incubé dans un thermocycleur pendant 10min à 25°C, 50min à 42°C pour la RT, et 15min à 70°C. Deux microlitres de l’ADNc issu de la réaction de RT est amplifié par PCR en présence du réactif \textit{One Taq 2X Master Mix Polymerase} (New England Biolabs) et de 0,2µM de couples d'amorces amplifiant spécifiquement des segments de 1000 à 1500 pb couvrant les régions génomiques au terme de 30 cycles à 94°C pendant 30s, 55°C pendant 30s et 68°C pendant 1min. Les produits de PCR résultants sont vérifiés par migration sur gel d’agarose et purifiés par l'intermédiaire du Kit \textit{QIAquick PCR Purification} (Qiagen). Les réactions de séquençage sont assemblées selon la méthode de Sanger avec 15ng du produit PCR purifié, 0,32µM d’amorce s'hybridant spécifiquement sur le brin sens ou anti-sens de l'ADN et un mélange constitué de quatres sondes fluorescentes marquant les ddNTPs (\textit{Big Dye Terminator v1.1 or V3.1 Cycle Sequencing}, Applied Biosystems)  et incubées au thermocycleur pour 25 cycles à 95°C pendant 45s, à 50°C pendant 30s et à 60°C pendant 4min. Les réactions de séquençage sont analysées par électrophorèse capillaire (Eurofins).

% 10 - MESURE DE L'ACTIVITE FLUC

	\section{Mesure de l’activité luciférase}
	\label{sec:fluc}
	
Les cellules Huh-7.5 transfectées (voir \autoref{sec:transfection}) ou infectées (voir \autoref{sec:infection}) destinées au suivi de l’activité rapportrice FLuc sont lysées dans 200 ou 100µL de tampon de lyse (\textit{Reporter Lysis Buffer}, Promega) selon la taille de la plaque de culture et congelées à -20°C. Après décongélation, les lysats sont centrifugés à 3000g pendant 5min afin d'éliminer les débris cellulaires. L'activité FLuc de 10µL de lysat clarifié est mesurée en plaques 96 puits blanches à l'aide d'un luminomètre (\textit{TriStar LB 942 Multimode Microplate Reader}, Berthold Technologies) après injection automatique de 50µL de substrat dilué au 1:2 dans de l’H\textsubscript{2}O (\textit{Luciferase Assay Reagent}, Promega).

% 11 - TRANSCRIPTOMIQUE

	\section{Analyse du transcriptome hépatique par RNA-Seq}
	\label{sec:transcriptomique}

L’ARN total des cellules Huh-7.5 infectées en vue de l’analyse du transcriptome hépatique par RNA-Seq (voir \autoref{sec:infection}) est extrait avec 350µL d’un tampon de lyse (\textit{RNeasy Plus Mini}, Qiagen) supplémenté par 0,1\% de ß-mercaptoéthanol (Sigma) selon les instructions du fabricant. Pour garantir l’élimination complète de l’ADN génomique, une digestion à l'aide de DNase a été effectuée sur colonne pendant 15min (\textit{RNase-free DNase Set}, Qiagen). La qualité et l’intégrité de l’ARN ont été évaluées par mesure de la densité optique à l'aide d'un nano-spectrophotomètre (\textit{MySpec}, Ozyme) et d’un bioanalyseur (\textit{2100 Bioanalyzer}, Agilent). Tous les échantillons présentaient une excellente pureté (A\textsubscript{260/230} > 1,8 et A\textsubscript{260/280} > 2,0) et ne présentaient aucun signe visible de dégradation (Score RIN > 9). Les ARNm poly(A) ont été enrichis à partir de 200ng d’ARN totaux avec le kit \textit{Dynabeads mRNA Purification Kit} (Thermo Fisher Scientific) selon le protocole \textit{RNA Direction Library prep set user Manual} de MGI (MGI Tech, Shenzen Shi, Chine). Les ARNm sont clivés en fragments de 150pb, rétro-transcrits et le brin complémentaire synthétisé selon le même protocole. Les adaptateurs MGI sont ensuite ajoutés par ligation avant une étape d'amplification des fragments d'ADN de 14 cycles. La qualité des banques d'ADN est contrôlée par électrophorèse capillaire (\textit{Fragment Analyzer}, Agilent) en utilisant un kit pour fragments d’ADN de 100 à 6000 pb (\textit{HS NGS Fragment Kit}, Agilent). Les banques d'ADN sont ensuite assemblées et préparées selon les protocoles \textit{RNA Directional Library prep set user Manual} et \textit{High-throughput (Rapid) Sequencing Set User Manual} de MGI (MGI Tech, Shenzen Shi, Chine), puis séquencées sur une cellule à haut débit en simple lecture de 100pb (DNBSEQ-G400, MGI Tech, Shenzen Shi, Chine).

% 12 - IMMUNOBLOT

	\section{Analyse quantitative des protéines par immunoblot}
	\label{sec:immunoblot}
	
Les extraits protéiques sont préparés par lyse des cellules Huh-7.5 transfectées (voir \autoref{sec:transfection}) et infectées (voir \autoref{sec:infection}) dans respectivement 200 ou 100µL de tampon dénaturant (\textit{Lithium Dodecyl Sulfate Sample Buffer}, Thermo Fisher Scientific) contenant 5\% de ß-mercaptoéthanol (Sigma). Les lysats protéiques sont systématiquement dénaturés par incubation à 95°C pendant 15min et conservés à -20°C. Les protéines sont séparées par électrophorèse sur gel de polyacrylamide à 4-12\% ou 12\% d’acrylamide (\textit{NuPAGE Bis-Tris Gels}, Thermo Fisher Scientific) en tampon de migration (\textit{NuPAGE MES SDS Running Buffer} ou \textit{NuPAGE MOPS SDS Running Buffer}, Thermo Fisher Scientific) et transférées sur une membrane de nitrocellulose (\textit{Nitrocellulose Premium, 0.45μm}, Amersham) en système liquide (\textit{NuPAGE Transfer Buffer}, Thermo Fisher Scientific). Les membranes sont saturées pendant 1h dans une solution de DPBS contenant 0,1\% de Tween-20 (DPBS-T) et 5\% de lait écrémé déshydraté et incubées durant la nuit à 4°C en présence d’anticorps primaires dilués en DPBS-T contenant 1\% de lait. Le lendemain, les membranes sont extensivement rincées au DPBS puis incubées pendant 1h avec des anticorps secondaires anti-IgG de souris ou de lapin couplés à un fluorochrome DyLight 680 ou 800 (Thermo Fisher Scientific) dilués en DPBS-T contenant 1\% de lait. L’ensemble des anticorps primaires et secondaires utilisés pour cette étude sont listés dans les Tableaux \ref{tab:tabM1} et \ref{tab:tabM2}. Après élimination des anticorps par un traitement de 15min avec un tampon de dissociation (\textit{Stripping Buffer}, Euromedex), les protéines totales sont révélées par une incubation de 5min dans un colorant protéique (\textit{Revert 700 Total Protein Stain}, Li-Cor Biosciences), suivie d'une fixation dans une solution aqueuse contenant 30\% d’éthanol et 6,7\% d’acide acétique glacial. Les signaux de fluorescence aux deux longueurs d'onde dans le proche infrarouge sont détectés à l'aide du système d'imagerie laser Odyssey CLx (Li-Cor Biosciences) et quantifiés à l’aide du logiciel ImageStudioLite (Li-Cor Biosciences).

% 13 - IMMUNOMARQUAGE

	\section{Marquages fluorescents des lipides, des protéines et de l’ARN viral}
	\label{sec:immunomarquage}
	
Pour réaliser le marquage des GL, les cellules Huh-7.5 vivantes sont mises en contact avec une sonde fluorescente dérivée de l’acide dodécanoïque (BODIPY® 558/568 C12, Thermo Fisher Scientific) diluée dans du milieu sans sérum pendant 15h à 37°C préalablement à la fixation. Les cellules Huh-7.5 transfectées (voir \autoref{sec:transfection}) et infectées (voir \autoref{sec:infection}) destinées aux visualisations par imagerie confocale sont fixées pendant 20min avec une solution de paraformaldéhyde à 4\%, puis perméabilisées pendant 30min avec une solution de digitonine à 40µg/mL pour préserver la structure des GL. Comme méthode alternative de marquage des GL, les cellules fixées ont été incubées pendant 30min avec un colorant lipidique (\textit{LipidTox Red Neutral Lipid Stain}, Thermo Scientific). La révélation des ARN viraux de polarité positive et négative est réalisée par une approche d’hybridation \textit{in situ} (\textit{ViewRNA Cell Plus Assay}, Thermo Fisher Scientific) à l’aide de sondes qui s’hybrident respectivement aux bases 3733-4870 du brin positif (VF1-10121) ou 4904-5911 du brin négatif (VF6-11102) de la souche JFH-1, conçues par Shulla et al. \citep{RN818}. Ces régions sont situées dans les séquences codantes de NS5A et NS5B mais ne couvrent pas les mutations d’adaptation de la souche Jad, par conséquent, ces sondes s’hybrident parfaitement à l’ARN de Jad. En parallèle, des sondes contrôles ciblant les ARNm cellulaires du gène de l'\textit{ACTB} conçues par le fabricant ont été employées. Les étapes d’incubation avec les sondes spécifiques des brins (+) et (-) du VHC ou des transcrits de l'\textit{ACTB}, d’amplification et de révélation du signal ont été effectuées selon les instructions du fabricant. Pour les lames dédiées à révéler l’ARN viral, l’ensemble des solutions précédant l’hybridation ont été complémentés avec des inhibiteurs de ribonucléases. Afin de limiter l’action des détergents sur l’intégrité des GL, la solution de lavage incluse dans le kit a été remplacée par du DPBS. Pour le marquage fluorescent des protéines virales ou cellulaires, une saturation des sites antigéniques non spécifiques a été réalisée pendant 30min avec une solution de DPBS contenant du sérum de chèvre ou d’âne à 5\% selon l’espèce de l’anticorps secondaire employé. Les cellules ont ensuite été incubées pendant 1h avec l’anticorps primaire spécifique des protéines virales ou cellulaires dilué dans du DPBS contenant 1\% de sérum de chèvre ou d’âne. Les complexes immuns ont ensuite été révélés par incubation pendant 1h à l’obscurité en présence d’anticorps secondaires conjugués à un fluorochrome Alexa Fluor 488, 555 ou 647 (Thermo Fisher Scientific). L’ensemble des anticorps primaires et secondaires utilisés pour cette étude sont listés dans les Tableaux \ref{tab:tabM1} et \ref{tab:tabM2}. Les noyaux ont été révélés avec une solution à 1µg/mL de 4-,6-diamidino-2-phénylindole (DAPI) pendant 10min (Thermo Fisher Scientific), puis les lames sont conservées dans un milieu de montage non polymérisant (\textit{Ibidi mounting medium}, Ibidi).

% 14 - MICROSCOPIE CONFOCALE

	\section{Acquisition, traitement et analyse d’images par microscopie confocale}
	\label{sec:confocal}
	
Les lames sont observées avec un microscope confocal inversé (LSM 700, Zeiss) contenant quatre diodes laser de longueur d’émission de 405, 488, 555 et 639nm. Les images ont été acquises avec des objectifs 40X ou 63X en immersion à huile NA 1.4 à l’aide du logiciel de l’instrument (ZEN, Zeiss). Les filtres d'émission sont configurés de façon à détecter les signaux de manière séquentielle avec une restriction $\leq$460nm pour les noyaux colorés au DAPI, 490 à 555nm pour le fluorochrome Alexa Fluor 488, 550 à 620nm pour le fluorochrome Alexa Fluor 555, et $\geq$650nm pour le fluorochrome Alexa Fluor 647 ou les sondes d’hybridation \textit{in situ}. Le diaphragme confocal est ajusté à 1 unité Airy (UA) de manière à empêcher la lumière hors du plan focal d'atteindre le détecteur. La résolution des images lors de l’acquisition a été adaptée afin d’obtenir une taille de pixels de 0,05µm, pour se placer dans les paramètres optimaux de déconvolution (voir \autoref{section:preambule1}). Les images tridimensionnelles ont été automatiquement empilées en utilisant un intervalle vertical de 0,15µm sur le plan z entre chaque section acquise. Après l'acquisition, les images sont restaurées par l'outil de déconvolution du logiciel Huygens Professional (Scientific Volume Imaging), une procédure itérative qui permet de résoudre les biais induits par la limite de résolution de l'objectif. L'analyse volumétrique a été réalisée à l'aide de l'outil \textit{Huygens Object Analyzer}. Sur la base d’un seuil défini automatiquement par le logiciel, le signal de chaque canal a été segmenté en « objets » volumétriques permettant de calculer le nombre, le volume individuel de chaque objet et l’intersection spatial entre les objets de de chaque canal en voxels. Pour réaliser l’étude sur l’interaction des GL dites « appariées » et les zones de contact, un programme innovant sur Python a été développé spécialement pour ce projet par D. Ershov (\textit{Image Analysis Hub}, Institut Pasteur, Paris). En bref, les GL sont détectées à l’aide d’un détecteur multi-échelle et converties en un masque sphérique. La région représentant la couche protéique en surface a été calculée autour de chaque masque sphérique. L’épaisseur moyenne de l’enveloppe protéique a été fixée à 200nm sur la base du signal de Core mesuré dans les cellules infectées. Étant une estimation limitée par la diffraction, l’enveloppe est calculée de telle sorte à être immergée dans le masque sphérique des GL sur 100nm et à s’étendre à l’extérieur de celui-ci sur 100nm. Pour vérifier la précision de la segmentation, nous avons contrôlé que l’extraction des masques correspondant aux GL et à l’enveloppe protéique chevauchaient parfaitement les signaux sous-jacents respectifs. Les masques sphériques situés à une distance surface-à-surface inférieure à 200nm définissent les GL « appariées ». Les sites de contact représentent la zone de chevauchement des deux masques de l’enveloppe protéique appartenant aux GL « appariées ». L’intensité du signal extrait hors et à l’intérieur des zones de contact a été corrigé à partir du signal cytoplasmique et normalisé par rapport à une valeur moyenne du bruit de fond calculé sur les cellules naïves de la même expérience. Les analyses ont été effectuées sur 40 à 60 images 3D individuelles provenant de deux à trois expériences indépendantes. Les reconstructions tridimensionnelles utilisées comme illustrations ont été générées à l'aide d'Imaris (ImarisViewer 9.5.1, Oxford Instruments).

% 15 - MICROSCOPIE ELECTRONIQUE

	\section{Acquisition d’images par microscopie électronique}
	\label{sec:electronique}
	
Les cellules Huh-7.5 transfectées (voir \autoref{sec:transfection}) ou infectées (voir \autoref{sec:infection}) destinées aux études de microscopie électronique ont été fixées à froid dans un tampon à pH neutre contenant 4\% de paraformaldéhyde, 1\% de glutaraldéhyde et 0,1M de phosphate monosodique. Après 24h d’incubation à 4°C, les cellules ont été brièvement resuspendues en DPBS puis placées dans une solution à 2\% de tétrode d’osmium (AgarScientific, Stansted, \textit{United Kingdom}). Les échantillons ont été entièrement déshydratés par incubation successive avec des solutions graduelles d’éthanol et d'oxyde de propylène, puis laissés pendant une nuit dans de la résine Eson pure (Sigma). La polymérisation de la résine a été faite par incubation à 60°C pendant 48h. Des sections ultrafines (90nm) de ces blocs de résine ont été obtenues avec un ultramicrotome Leica EM UC7 (Wetzlar, \textit{Germany}) puis colorées avec 2\% d'acétate d'uranyle (Agar Scientific) et 5\% de citrate de plomb (Sigma). Les observations ont été effectuées avec un microscope électronique à transmission (JEOL 1011, Tokyo, \textit{Japan}).

% 13 - BIOINFORMATIQUES ET STATISTIQUES

	\section{Approches bioinformatiques et statistiques}
	\label{sec:bioinfo}

Les représentations graphiques des résultats ont été générées à l'aide des logiciel Prism 9 (GraphPad, San Diego, \textit{United States}), RStudio (Boston, \textit{United States}), Python (Fredericksburg, \textit{United States}) et Cytoscape (Seattle, \textit{United States}) \citep{RN822,RN823} puis assemblées dans Illustrator (Adobe, \textit{Denmark}) pour la préparation des figures  \citep{RN824}. L'analyse statistique des études volumétriques des GL et de Core (voir \autoref{section:stabilisation}, \ref{section:cluster}, \ref{section:coregl}, \ref{section:mutants} et \ref{section:gouttelettes}) a été réalisée sous R par H. Varet (Plateforme de Bioinformatique et Biostatistique, Institut Pasteur, Paris) à l'aide d'un modèle linéaire à effet mixte issu du paquet R lme4 version 1.1-23. Les variables liées à l’infection ont été considérées comme fixes, tandis que l’effet lié aux dates des expériences a été inclus comme aléatoire. Les cellules infectées et non infectées ont été comparées à chaque temps à l'aide du package R emmeans version 1.4.7 et les \textit{p-values} ont été ajustées à l'aide de la méthode de Tukey. L’analyse statistique des cinétiques de réplication (voir \ref{section:virus}) a été réalisée sous R par E. Jacquemet (Plateforme de Bioinformatique et Biostatistique, Institut Pasteur, Paris) à l'aide d'un modèle linéaire à effet mixte. Les trois paramètres mesurés ont été ajustés à l’aide d'un modèle similaire, en considérant les variables liées à l’infection comme la souche virale et le temps post-infection comme effets fixes. Les réplicats et la répétition des expériences qui font office de source de variabilité expérimentale ont été inclus comme effet aléatoire. L’analyse statistique des autres résultats quantitatifs a été effectuée sur Prism 9 à l’aide d’un t-test paramétrique ou d’une ANOVA univariée à comparaisons multiples selon la forme des données. Pour l’analyse de corrélation entre les variables relatives aux GL et à Core (voir \autoref{section:correlation}), les coefficients de corrélation r compris entre 1 et -1 ont été calculées par la méthode de Spearman à partir de l’algorithme standard inclus dans RStudio. La significance statistique des coefficients de corrélation a été évaluée à l’aide du paquet R corrplot version 0.84.

	\cleardoublepage

			\newpage
			\phantomsection\addcontentsline{toc}{chapter}{Références}
			\markboth{}{}
		\bibliographystyle{apalike}
		\bibliography{bibliography}
	\cleardoublepage
	
	\coverpage
\end{document}