%% Copyright (C) 2017-2021 Emeline Simon
%%
%% The current owner of this work is Emeline Simon
%% <contact at emeline.simon@gmail.com>.
%%
%% This is 02_objectives.tex the second chapter for my PhD Thesis.
%%
%%%%%%%%%%%%%%%%%%%%%%%%%%%%%%%%%%%%%%%%%%

\chapter{Contexte et objectifs du projet}
	\minitoc
	\newpage

%%%%%%%%%%%%%%%%%%%%%%%%%%%%%%%%%%%%%%%%%%%%%%%%%%%%%%%%%%%%%%%%%%%%%%%%%%%%%%%%%%%%%%%%%%%%%%%%%%%%%%%%%%%%%%%%%%%%%%%%%%%%%%%%%%%%%%%%%%%%%%%%%%%%%%%%%%%%%%%%%%%%%%%%

L’infection chronique par le VHC est une maladie progressive, longtemps asymptomatique qui peut mener au développement de la stéatose, une accumulation anormale de lipides neutres sous la forme de GL dans les hépatocytes. En lien avec la grande diversité génétique du VHC, distribuée en 8 génotypes et >90 sous-types (voir \autoref{section:epidemiologie}), des études cliniques ont rapporté une corrélation entre les infections chroniques par les souches du VHC de génotype 3, fréquentes chez les usagers de drogues en Europe, et une prévalence élevée de stéatose, de l’ordre de 80\%. Cette pathologie hépatique est une évolution fréquente chez les patients souffrant de syndromes métaboliques « modernes » comme la NAFLD et AALD (voir \autoref{section:steatose}). Autrefois considérée comme bénigne, la stéatose s’est avérée être une prédisposition importante accélérant la progression vers des complications hépatiques graves telles que la cirrhose et le CHC, responsables de 2 millions de décès annuels. Dans un contexte de croissance des maladies métaboliques tant dans les pays riches que dans les pays en développement, l’OMS a déclaré en 2010 qu’elles constituaient un problème majeur de santé publique à l’échelon mondial. Pour l’hépatite C chronique, le développement d’AAD permettant de guérir aujourd'hui $\geq$95\% des patients infectés est une révolution thérapeutique sans précédent (voir \autoref{section:elimination}). Cependant, en raison de leur production coûteuse, les AAD sont actuellement limités aux pays occidentaux et la guérison n’élimine pas toujours le risque de développer des complications hépatiques. \\ \\ 
\indent
Les manifestations cliniques qui surviennent au cours de l’infection chronique par le VHC résultent d'une combinaison complexe de facteurs indirects tels que l'inflammation chronique, la susceptibilité génétique et les habitudes individuelles et de facteurs directs, i.e. induits par les protéines virales. Comme il a été décrit au cours de l’introduction, de nombreux travaux mettent en évidence l’implication de la protéine Core du VHC dans le développement de la stéatose hépatique (voir \autoref{section:modele}). Premièrement, Core présente une association singulière à la surface des GL cytosoliques, dont la fonction essentielle à l’homéostasie lipidique est altérée au profit de la morphogénèse des virions. Le rôle de cette interaction cruciale dans l’assemblage des particules virales n’a toujours pas été élucidé et seule une vision incomplète de la façon dont le VHC détourne les GL est disponible à ce jour. De plus, l’implication du dérèglement des GL par l’infection, qui contribue probablement à l’apparition des signes histologiques, à savoir, l’accumulation anormale des vacuoles lipidiques dans les hépatocytes, n’est pas encore connue. Deuxièmement, Core semble jouer un rôle clé dans la dérégulation des voies de signalisation pro-inflammatoires et métaboliques des hépatocytes, qui représentent des facteurs de risque importants pour la stéatose hépatique. Certains polymorphismes naturels des acides aminés de Core retrouvés dans les souches de génotype 3 comme la Phe en position 164 ou la combinaison Phe et Iso en positions 182 et 186 respectives, ont été associés à un fort élargissement des GL et à une lipogénèse de novo accrue, ce qui appuie les études cliniques. Cependant, en l’absence de modèles animaux immunocompétents permissifs à l’infection pour étudier la pathogénèse du VHC et de systèmes de culture cellulaire permettant la multiplication de souches cliniques du VHC (voir \autoref{section:modele}), les résultats relatifs à une association entre les déterminants spécifiques du génotype de Core et les troubles du métabolisme lipidique reposent principalement sur des systèmes d'expression transitoire in vitro. Par conséquent, on ne dispose à ce jour que d'une vue incomplète ou potentiellement biaisée de l'implication directe de Core dans la stéatose hépatique, et les connaissances mécanistiques dans des systèmes d'infection hépatique appropriés font défaut. Ainsi, la question de savoir si les génotypes spécifiques de Core modulent différemment les GL et les voies biologiques clés qui peuvent expliquer l'aggravation des résultats cliniques reste ouverte. \\ \\
\indent
Notre équipe s'intéresse d’une part, à approfondir les connaissances fondamentales du mécanisme d’assemblage du VHC et d’autre part, à déchiffrer le rôle de Core dans la pathobiologie de l'hépatite C et à identifier les mécanismes génotype-spécifiques et les déterminants viraux impliqués dans les réponses pathogènes de l'hôte, en utilisant des systèmes d'infection physiologiquement pertinents. Dans ce contexte, les objectifs de mon projet de thèse sont de : \\

\begin{itemize}
  \item[$\bullet$] Étudier l’interface spatio-temporelle entre la protéine Core et les GL afin d’identifier les mécanismes de dérégulation de ces organites au cours de l’infection et leur rôle précis dans le mécanisme d’assemblage des particules virales.

  \item[$\bullet$] Développer de nouveaux modèles virologiques réplicatifs en cellules d’hépatome humain et codant pour des protéines Core de souches cliniques de différents génotypes isolées à partir de patients présentant divers degrés de stéatose hépatique, afin de répondre aux questions suivantes.

  \item[$\bullet$] Déterminer si l'origine génotypique de Core a un impact différentiel sur la biogénèse et l’élargissement des GL, en lien avec les manifestations cliniques de la stéatose hépatique.

  \item[$\bullet$] Étudier si et dans quelle mesure l'origine génotypique de Core module différentiellement le transcriptome hépatique, afin de mettre en évidence des propriétés pathogènes spécifiques au génotype viral. \\
\end{itemize}
\indent Ce projet s'inscrit dans le cadre d'un travail collaboratif soutenu par l'ANRS et permettra, à terme, une meilleure compréhension des mécanismes biologiques impliqués dans la pathogénèse du VHC et de l’importance des facteurs viraux directs et des polymorphismes génotypiques du VHC dans le développement de la stéatose hépatique.
