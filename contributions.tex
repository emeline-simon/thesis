
%% Copyright (C) 2017-2021 Emeline Simon
%%
%% The current owner of this work is Emeline Simon
%% <contact at emeline.simon@gmail.com>.
%%
%% This is contribution.tex for my PhD Thesis.
%%
%%%%%%%%%%%%%%%%%%%%%%%%%%%%%%%%%%%%%%%%%%

\cleardoublepage
\section*{Contribution}

\noindent
L'idée originale du projet de thèse a été conçue par le Dr. Annette Martin. La supervision, le financement et l'orientation générale du projet a été menée par le Dr. Annette Martin. \\

\noindent
Emeline Simon a participé à la conception et au financement du projet et a mené les échanges avec les principaux collaborateurs et les supports techniques des fournisseurs scientifiques. Emeline Simon a conçu et mis et point les détails techniques du travail expérimental, les méthodes d'analyse, la représentation graphique et l'interprétation des résultats. Emeline Simon a présenté les résultats du projet à l'occasion de plusieurs congrès nationaux et internationaux. \\

\noindent
Les membres du groupe du Dr. Annette Martin (Unité GMVR, Institut Pasteur, Paris), les Dr. Stéphanie Aicher et Mathieu Fritz (ex-doctorants), Brigitte Blumen (Technicienne, Institut Pasteur), Damien Batalie (Technicien, Institut Pasteur) et Angeliki Anna Beka (ex-stagiaire Erasmus+) ont contribué significativement à la réussite de ce projet, en fournissant le matériel initial issu de la biologie moléculaire et en aidant à la collecte des données expérimentales. Tous les membres du groupe du Dr. Annette Martin, avec Houda Tabbal (Post-doctorante, Institut Pasteur), ont participé et alimenté la discussion des résultats lors des réunions d'équipe et l'ensemble de leurs commentaires critiques a contribué à façonner le déroulement et les perspectives du projet. \\

\noindent
Le Pr. Philippe Roingeard (Faculté de Médecine, Université de Tours, Tours) a préparé et analysé les lames dédiées aux observations par microscopie électronique. \\

\noindent
Le Dr. Dmitry Ershov (Plateforme d'Analyse d'Images, Institut Pasteur, Paris)  a conçu et mis au point l'algorithme Python à partir des indications d'Emeline Simon, et généré les données servant à l'étude des contacts entre les gouttelettes lipidiques. \\

\noindent
Didier Simon a conçu un algorithme Excel afin d'automatiser l'extraction des données servant à l'analyse volumétrique des gouttelettes lipidiques à partir des indications d'Emeline Simon. \\ 

\noindent
Juliana Pipoli Da Fonseca et Thomas Cokelaer (Plateforme Biomics, Institut Pasteur, Paris) ont réalisé le séquençage et les analyses bioinformatiques du crible transcriptomique à haut-débit. \\

\noindent
Hugo Varet et Elise Jacquemet (Plateforme de Bioinformatique et Biostatistique, Institut Pasteur, Paris) ont pris en charge les analyses statistiques des principales données du projet. \\