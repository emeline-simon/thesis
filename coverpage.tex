%% Copyright (C) 2017-2021 Emeline Simon
%%
%% The current owner of this work is Emeline Simon
%% <contact at emeline.simon@gmail.com>.
%%
%% This is coverpage.tex for my PhD Thesis.
%%
%%%%%%%%%%%%%%%%%%%%%%%%%%%%%%%%%%%%%%%%%%

\makeatletter

\newcommand{\coverpage}{
\newgeometry{top=4cm, bottom=2cm, left=2.75cm, right=2.75cm}
  \begin{titlepage}
    {\LARGE \textbf{Résumé}} \\
    \vspace{0.5cm}
\begin{singlespace}
L'infection chronique par le virus de l'hépatite C (VHC) est une maladie du foie progressive qui peut mener à une stéatose, une accumulation anormale de gouttelettes lipidiques (GL) dans les hépatocytes. Des études cliniques ont mis en évidence une corrélation entre l'infection chronique par les VHC de génotype 3 et une prévalence élevée de stéatose, ainsi qu'un taux de progression accru vers la cirrhose et le carcinome hépatocellulaire. Parmi les facteurs viraux, la protéine de capside (Core) présente une association singulière et critique à la surface des GL et joue un rôle clé dans la dérégulation des voies de signalisation hépatiques, représentant des facteurs de risque potentiels pour la stéatose. Dans ce contexte, les objectifs de mon projet de thèse étaient (i) d'évaluer l'impact de l'infection par le VHC sur la biogénèse et la dynamique des GL, et (ii) d'identifier si les génotypes ou les polymorphismes de Core sont impliqués dans la dérégulation de l'homéostasie des GL et des voies métaboliques hépatocytaires clés qui pourraient sous-tendre les aggravations cliniques. \\ \\
\indent
En lien avec le premier objectif, des analyses cinétiques par imagerie quantitative d'hépatocytes infectés par une souche prototypique du VHC ont révélé que le contenu global des GL est inchangé, ce qui suggère que l'infection n'a pas d'impact sur la biogenèse globale des GL. Cependant, au fur et à mesure que l’infection progresse, les GL montrent un élargissement et une redistribution marquée sous la forme d’agrégats compacts, portant Core à leurs sites de contact et à leur surface. En utilisant des ARN du VHC codant une protéine Core mutée incapable de s'associer aux GL, aucun élargissement ou regroupement de GL n'a été observé. Ces résultats indiquent que Core est le principal moteur de l'agrégation des GL, potentiellement en reliant physiquement les GL limitrophes, en déstabilisant leur tension de surface et en favorisant leur fusion. Le rôle fonctionnel de l'agrégation et de la redistribution des GL pourrait être de délivrer la protéine de capside aux sites d'assemblage du VHC. \\ \\
\indent
En lien avec le second objectif, de nouveaux virus intergénotypiques ont été produits, exprimant des protéines Core hétérologues dérivées d'isolats cliniques de sous-types 1a, 2a, 3a, 4a et 4f et associés à divers degrés de stéatose. L'ampleur de l'élargissement des GL dans les cellules infectées par ces différents virus varie selon les séquences de Core, mais indépendamment des génotypes ou des degrés de stéatose. En parallèle, des analyses transcriptomiques comparatives à haut débit d'hépatocytes infectés par les virus intergénotypiques ont mis en évidence des dérégulations hépatiques différentielles marquées. En outre, plusieurs facteurs prédictifs de la stéatose se sont avérés spécifiquement modulés par les protéines Core de génotype 1a, 3a ou 4. Ces résultats révèlent une modulation différentielle de facteurs pro-stéatogènes clés en lien avec l'origine génotypique de la protéine Core du VHC, ce qui pourrait \textit{in fine} fournir des marqueurs de progression de cette pathologie hépatique.
\end{singlespace}
  \end{titlepage}

\restoregeometry
}